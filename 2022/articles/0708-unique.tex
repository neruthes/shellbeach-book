\chapter{“并不相信他人眼中的自己是特别的、独一无二的”}

\ardate{2022-07-08}{emEYWlk0qcSwuUTZEYkltg}




前几天,一个很久没见的朋友问我要不要一起过夜,然后我就开始揣测他的意图:你是想找个人陪伴你吗?你是想约吗?而我会反感自己被物化的感觉。后来和他见上面并聊到这一点时,我说:因为你提出过不过夜的时候,并没有说清楚意图是什么,所以我会有不安全感,这种不安全感驱使着我去揣测你是否想要过夜背后的意图,因为我无法从他人那获得安全感。

同时我也说到我从十六岁开始就在软件上约陌生人见面,从那时候起我就需要通过从他人的言行举止看出对方可能的欲求是什么来保护自己。所以当我无法看出你提过是否过夜背后的欲求是什么的时候,我下意识的第一反应是:危险。

他说:我们都认识那么多年了,你就不能从我身上获得安全感吗?我说:不能,我不能从任何人身上获得安全感。他问为什么我要从不稳定的他人那获得安全感,而不是从自己身上获得安全感?然后他提到温尼科特的理论概念,提到婴儿对安全感的来源从一开始母亲的乳房到奶嘴到玩偶之类的过渡性客体再到自身。我说:如果没有这样的过程呢?他说:肯定有这样的过程……我打断他说:不然这个个体就存活不下来了吗?

后来他说,好像无论他做什么,我都会认为他把我物化了\pozhehao{}如果他不提出过夜的目的,我会猜他是为了要陪伴或性,这两者都会让我感到被物化;如果他提出来过夜的目的是为了要陪伴或性,那我也会感到被物化。我说是的,因为我并不相信他人眼中的自己是特别的、独一无二的,而这样的被物化的猜想能保护我自己免受失望\pozhehao{}期望在他人眼中的我是独特的、独一无二的,但到头来自己却被对方用另一个人轻而易举地替换掉。

当我回溯这种不重要感时,我能回想起很多事情\pozhehao{}有对象的前任;在认识前任之后的几个隐藏着一些事物的男生;读高中时在校外兴趣班结交的几个曾经的朋友,但因为自己说错了一些话而被他们孤立、排挤、敌对,后来我也就脱离了他们;小中学时期遇到的一些期望我能考高分但在我考低分时就甩开我的几个老师;还有读小学时得知亲戚(一对夫妻)有了他们自己的孩子而我只能和自己的生父生母继续生活下去。

每一次的关系断裂、期望落空,我丧失的都不仅仅是那个重要他人本身,我同时也丧失了那个重要他人眼中的自己。而我想要拒绝丧失自己的任何一部分,因为剩下来的还属于自己的部分本身就没有多少。每一次和重要他人的关系断裂后,我都会自问:一个人究竟要丧失多少,才能完全丧失掉自己?至少当完全丧失掉自己的时候,也就什么也不会感受到了。

不过,我也意识到这种自我保护的方式同样隔绝了其他人能够给予我的帮助、其他人能够视我为独特的、独一无二的人的机会。这就像是一个固化的图式,这个图式能很好地保护自己,能让自己不进一步地脆弱、丧失、受损甚至是崩溃,但这并不会让事情变得更好、并不能让自己变得没那么脆弱,因为我拒绝了来自于其他人的任何可能性。就像是小时候的我为了不被神经质般的母亲而伤害到内心时,有意识地封闭掉自己的所有情绪,但这样的自我保护方式也杜绝了来自外界的任何可能存在的帮助,杜绝了很多本能够让自己感觉到开心、快乐的令人难忘的时刻的回忆。


