\chapter{“你还不如死了算了,生块叉烧都好过生你”}

\ardate{2022-03-12}{HVA28pqnLSasVQdk52ghQw}


在上面那篇写作里,我收到的留言的其中一段话是:「对此我也会疑惑,所谓对内在世界无止境的探索会不会让你更无力?这样的学习到底能在多大程度上成为你抵御现实中所体验到的无能?」我对此的回复是:“嗯,对内在世界的探索很大程度上抵御现实中所体验到的无能。”

留言里的那句话让我感到很反感和抵触之处是“抵御”一词。当看完那句话的时候,我内心感到很难受。那种难受的感觉是一种脆弱感,或许就是留言者所说的那种无能的感觉,感觉自己很脆弱、很无能。就像是在过去好几年里,时时刻刻都能感觉到的那种无能感,那种被困在原地的感觉,只想摆脱痛苦。毕竟如果自己真的除了自杀就什么事情也改变不了、谁也改变不了的话,为什么不去自杀呢?自杀至少是我唯一能做到的事情,唯一能让自己不感到那么无助、那么无力、那么无能的事情。

现在的我之所以会防御着那种无能感,很大程度上是不想再感受到以前身处于那种孤独的、无力的境地所感受到的痛苦。但我也知道那种无力感依然“在那里”,那种无力感一直在说:“没有人懂我,没有人懂我的痛苦,他们只会合理化我的想法、自己的感受、自己的心情。没有人能帮到自己。There's nothing left and no one left.”

而且在留言里,我也看到了一个朋友试图合理化我的想法和感受,比如说:「当一个东西和任何其他东西都不会产生互动,那它存在不存在似乎都没有区别,所以“我不去推动事情发生,一切就都不会发生”,在我看来倒是很自然而然的事情。」当然我知道他的本意可能并不是想合理化我的想法和感受,而可能只是在表达关于他自己的事情。但这依然让我感到很孤独和无力。就像是:“嗯,没有人能懂你的痛苦的。”

那一自杀意图背后的动机,一直以来都是为了躲避痛苦\pozhehao{}现实经历的受挫的痛苦;他人不理解自己的痛苦;无力改变现实的痛苦;被困在原地的痛苦;没有人真正看见自己的痛苦;明明自己已经付出了那么多努力,还被人说自己只不过是在“抵御现实中所体验到的无能”的痛苦。

不过现在的我不会像以前那样将别人“一棍子打死”,不会觉得别人的本性就是坏的、邪恶的,现在的自己也不像以前那么愤世嫉俗。有时候他人就是不愿或无能做到理解他人的痛苦。愤世嫉俗总是很容易的,攻击他人也总是很容易的,特别是小时候的我就已经十分“嘴毒”。对我来说,真正困难的是,去接受他人的不理解,去接受我并不是时刻都那么全能,也不是时刻都那么脆弱和无力,去接受任何一面的自己,以及接受自己在任何一个时刻所感受到的情感都不是自己的全部。

我很讨厌“防御”、“抵御”这一类词,也很讨厌很多心理学的术语,而我在接热线的时候也不会主动说术语,更不会说:“噢,你就是在投射”、“噢,你就是在防御”、“噢,你就是在退行”。我会觉得很多心理学术语(特别是传统精神分析流派和精神动力学流派的术语)把人给异化了,就像是对一个情绪陷入低谷的人说:“所以说,你之前所感觉到的对生活的热爱,是否只是一种对自杀意图的防御?”好像一切美好的事物都只是一种防御,仅此而已。在使用心理学术语者眼中,事物似乎并不存在什么存在性价值,而只有工具性价值,并且将除了工具性之外的部分统统抹去。人只是一部机器,总是在防御着什么、平衡着什么。

当我再回头看那段留言的时候,我觉得留言者或许只是在关心这样的防御是否有效、是否足以防御生活里的无能感。但在我第一眼看到留言时,我对此只有唯一的解读:对方在指责我在防御些什么。

这种指责就像是在读小学时,家里人总是指责我很懒很蠢,就好像是我选择了变懒和变蠢。他们会说我没有把字写工整是我懒,我算错数是我故意的,我做作业很慢也是我特意在拖延时间,我不想和他们说话是内向。一切都是我的错,我不应该存在在这个世界上,我活着对他们而言只是个累赘罢了,一个被唾弃的累赘。“你还不如死了算了,生块叉烧都好过生你。”

是啊,这样的指责是如此的熟悉,以至于直到现在,我还能“听到”这样的指责,能从不同人的不同话语里“听到”同样的指责。这可能也是那段留言能轻易地触发到我的无能感、脆弱感的原因,也是现在的我会去极力防御着这份无能感、脆弱感的原因。但我对生活的热爱并不只是为了防御些什么,还有热爱这份热爱本身。我也不是只能在同一段话里“听到”与以前相似的指责,或者说不是只能“听到”指责的声音。

不过我也可以借此合理化一下自己的自杀意图:遇到这样的父母谁不想自杀!
