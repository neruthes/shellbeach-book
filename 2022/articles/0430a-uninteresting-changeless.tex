\chapter{无趣,不足够好,改变不了什么}

\ardate{2022-04-30}{-6iZujX9A8UIEWsz-denjg}



到了咨询室门口的我发现大门紧闭,我开始怀疑咨询师是不是忘记了我今早会去咨询。我向咨询师发了条信息:

\useimg{aimg/2022-0430-1.jpg}

\dialoguelist{咨询师}{
	\dialoguesepline{咨询师}{进入咨询室后}
	\dialogue{我}{不如我们说一下上次咨询后在电梯里的碰面?当时那个场景会让我想起我和办公室同事的互动,也是在电梯里不会打招呼,甚至还回避着目光交流。那时候我会有一种被拒绝感和被遗弃感,但我也知道这种感觉不仅仅有属于你的部分,也有属于我自己的部分、我自己的经历。我上周末的完型课程作业就是要写一个羞愧的时刻,我就写了在电梯里的时刻。在完型看来,羞愧是一种对人际关系的拒绝。\\
		刚刚我催你快来开门的时候,你的句子后面是感叹号而不是句号或省略,所以我也会好奇那个感叹号背后会有怎样的情感,可能会是一种绝望?(我笑了笑)\\
		我也会留意到在上次咨询结束时,你的双腿不像是以前的咨询从头到尾都翘着腿,而是在临近结束的几分钟里双腿并着放在地上,好像很想往前走,但双手又叠在双腿上,好像在安抚着双腿说:要呆在当下。那时候我的感觉是以为你急着想去洗手间,但后来在电梯里看见你的时候,我就发现你好像是赶着要去什么地方,而我就是那个路边的人,在你赶着要去的地方的路边的人。}
	\dialogue{咨询师}{那那时候的你会有怎样的感受吗?}
	\dialogue{我}{我会感觉自己好像不应该在咨询室外和咨询师有太多的互动。}
	\dialogue{咨询师}{但好像你也期待在咨询室外能有更多的互动?}
	\dialogue{我}{嗯,会期待至少是一些基本的互动,比如说点点头。我在之前的咨询前在走廊等的时候,我会看见隔壁咨询室的其他咨询师和来访的互动。比如说有一个咨询师会敞开着咨询室的门,而ta有时候也会坐在走廊的沙发上和来访聊天。虽然在那个当下我感觉那个来访已经迫不及待想要倾诉了,而没有听咨询师自己想要说的话,就已经提前进入了那种模式。还有另一个咨询师也是敞开门,然后有个男来访者很快地走了进去,然后那个咨询师在纸上写东西,说ta在写一些东西,待会儿会用得上。我感觉前者的关系就像是倾诉和倾听的关系,后者的关系就像是朋友的关系,两人一起想要达成某个目标。但我和你的关系,如果用一个不恰当的比喻,就像是约炮的关系。当我进入了这扇门,然后我们开始做我们要做的事情,然后当我离开了这里后,我们又回到没有关系。
		就像是客体恒存性的丧失,好像我们之间的关系只能在固定的时间和空间里存在着,离开了这个时间和空间,我们之间的关系就不复存在了。}
	\dialogue{咨询师}{好像你期望我们的关系可以在咨询室外有更多的互动和交集?}
	\dialogue{我}{嗯,至少这个边界不是一道墙或一道门,而是一个空间,而那些咨询师可以在来访者靠近边界太近的时候觉得危险,靠得较远的时候觉得安全。但你的边界好像就是咨询室的门,关着就是关着的。\\
		但我也知道出于(咨询)设置,你不想在咨询室外有太多的互动。而且我之前也有看过咨询师在咨询室外遇到来访者的知识,甚至在一本小说里看到那个女咨询师说她在球场里看见来访者的慌张。所以在电梯里看见你的反应的时候,我并没有觉得出乎意料。}
	\dialogue{咨询师}{如果你没有看过这方面的知识呢?你会觉得咨询室外的我像是怎样的?}
	\dialogue{我}{如果是那样的话,我会觉得你在咨询室外不像是一个人。就好像即使离开了咨询室,你还是背着咨询室的门,门还是关着的。当门关着的时候,你就像是一个物体,只是在外面的空间里移动着,只有进入了咨询室你才会变回一个人。就像是在咨询室外你会把自己保护起来、包裹起来,只有进入了咨询室才会脱去防御。}
	\dialogue{咨询师}{我会留意到你在上一次咨询和这次咨询都讨论到了咨访关系的设置,好像这对于你来说真的很重要。}
	\dialogue{我}{不过我也会想到,为什么我要花那么多时间(我看了看桌面上时钟的时间:咨询已经过去20分)来讨论你的边界,我明明可以用这些时间来谈我自己的事情,但我又会觉得谈论我自己的事情会很无趣,所以与其谈论我自己的事情,还不如谈论彼此的互动之间的事情会更有趣。}
	\dialogue{咨询师}{好像现在你越来越多讨论我们之间的关系,而之前你一直都是谈论你自己的内在世界,好像之前的你一直都很自给自足,好像你的内在一直都很丰富。}
	\dialogue{我}{嗯,我的焦点好像转移到了彼此的关系,而不是谈论我自己。我会在想为什么。我好像既在逃避谈论我自己,但又很享受谈论我们彼此之间的事情。}
	\dialogue{咨询师}{这好像真的是一个很有趣的觉察。你会是在回避些关于自己的什么呢?}
	\dialogue{我}{可能就是自己的无趣吧。其实我一直觉得自己是个很无趣的人,但我也知道在别人眼里我并不是一个无趣的人。我可能会把很多关于自己的部分看作是理所当然的,比如说助教的点评里说我能察觉到你的肢体语言、会解读它们是一种优势,而我可能很少会看见这种优势。我会觉得很多事情都是理所当然的,比如说自己会读英文小说,读没有译文的教科书也没有问题;比如说每天会写两三千字的文章。很多在我看来都是理所当然的东西在别人看来都不是理所当然的。}
	\dialogue{咨询师}{(笑了笑)好像你也知道这些关于你的有趣的部分只是在自己看来理所当然,在别人看来是很有趣的。}
	\dialogue{我}{嗯,所以我也在想,为什么我会觉得自己很无趣。我好像是最近才觉得自己很无趣,最近自己还失眠,每天最多只睡六个小时,最少的话只有两个小时,而且最近的性欲也变强了。}
	\dialogue{咨询师}{能问一下是多最近吗?一周?两周?一个月?}
	\dialogue{我}{大概是一个月吧。}
	\dialogue{咨询师}{我会想起之前的咨询里你分享过关于最近的一个男生拒绝了你,会是和那个男生有关吗,在时间线上?}
	\dialogue{我}{嗯,应该是和他有关。}
	\dialogue{咨询师}{那怎样才算是有趣呢?}
	\dialogue{我}{比如说理解他人的能力、读懂他人的能力,任何一方面的能力只要上了一个level都行。}
	\dialogue{咨询师}{听起来好像你希望通过变得更有趣来吸引他人留在你身边。}
	\dialogue{我}{嗯,是这样的。其实我会感觉到悲伤,而不是无趣。我会想起和前任的关系,以及之后认识的那些无故消失的男生的关系。每当他们无故消失后,我都会很自责,自责自己做什么都不够。如果自己足够有趣的话,说不定关系就能朝着另一个方向走了。如果自己能成为对方的样子,说不定我就能挽回关系了。}
	\dialogue{咨询师}{为什么你会觉得你要成为对方的样子?}
	\dialogue{我}{因为和前任相处的时候,他会对我有很多期望,比如说成为一个工作狂、有更好的经济能力、知道自己的方向要怎么走、更加独立。我会感觉,我在那个当下就已经感觉到,他期望的是能遇到一个和他一样的人,能和他有着相同的特质的人,就像是另一个他。但我又做不到,我做不到成为一个和他一样的人,做不到成为一个工作狂。无论我做什么,都不足够,都不够好,都改变不了现实,都改变不了彼此的关系。}
	\dialogue{咨询师}{听起来这真的很无力。}
	\dialogue{我}{(沉默)}
	\dialogue{咨询师}{我会回想起你之前分享过,你小时候有脑积水,我不太确定这个猜想对不对,我会想那时候的你会不会觉得因为自己身体上有缺陷,所以觉得自己不值得有更好的父母?}
	\dialogue{我}{那时候应该没有,只是在经历了前任的关系后自己才有这样的不足够的感觉。虽然小时候的自己也经常被父母说我写作业慢、说我蠢,但那时候我并没有这么觉得,可能因为小时候的自己还是完整的。但在经历了前任的关系后,我觉得自己更加破碎了。}
	\dialogue{咨询师}{那你和最近那个男生的关系里,你会觉得他拒绝你的原因是因为你不足够好吗?}
	\dialogue{我}{不会,他拒绝我是因为他不想让现有的关系受损。这和我是否足够好无关,即使我理解他人的能力、读懂他人的能力再强,也会是同样的结果。这次的关系的结果更多是他的原因。所以我也会在想为什么自己依然感觉不足够,好像如果我足够好、足够优秀的话,我就能改变这段关系了。}
	\dialogue{咨询师}{噢~好像你能留意到这次的现实情况并不是自己不足够好,但在情感上依然觉得自己不足够好。}
	\dialogue{我}{嗯,是的。就是,无论自己做些什么都改变不了现实,无论自己再怎么努力都依然不足够改变现实,不足够改变彼此的关系。\\
		不过现在的我开始看见无趣背后的情感,那并不只是一种无趣,而是一股很强烈的悲伤,觉得自己不足够,怎么都不足够。但起码自己看到无趣背后的情感后,即使无法改变现实,但自己的情感应该会更好起来,起码会流动起来,而不会一直觉得自己很无趣。}
	\dialogue{咨询师}{我们的时间不多了,所以就不展开讲了,但我依然想提一点。我发现这次咨询和上次咨询临近结尾的时候你都会主动地对这次咨询做一个总结。}
	\dialogue{我}{可能因为我在想我的文字稿应该怎么写。写到最后的时候应该wrap it up,打包回生活里,而不是敞开着一个话题,充满着不确定性,起码不会将太多的绝望带入生活里。}
	\dialogue{咨询师}{嗯,我好像也确实能感觉到你在总结的时候会把整个咨询的基调往上抬,往更积极的方向抬。}
}

在结束咨询后,我一直觉得很悲伤,很想大哭一场,很想边哭边说:“这并不足够,但我又能怎么办!”随后我回想起过年时和前任见面时的聊天:

\blockquote{
	但我依然想和他有更深的关系,所以我说我在和他分手后的这两年半以来见过很多人,但都觉得(和他们的相处)蛮无聊,不过和他聊天不无聊。他说,我感到无聊是因为我和他人的互动方式以及我自己都没有怎么变化吧?我回答说:“不是的。我自己一直在变,和他人的互动方式也在变。我觉得和你聊天不无聊是因为我‘看见’了你背后的那个人,那个带着漠然、超然的态度对待身边的事物和人甚至是用这一态度对待你自己的那个你,也看到这一态度在你生活中的各方各面的延伸。”他有点不相信地说,难道不无聊不是因为我没有预料到这次的聊天(的内容和进展)吗?我说:“不是的,我在来之前就设想到现在这一步。我之前的那些隐隐约约的想法和感觉都在这次的聊天里慢慢地展开了。但我没有设想到的是你背后的那个带有着漠然、超然的态度的那个人。看见了这个人让我感觉我们之间的距离拉近了。”

	他问我:“这就足够了吗?”我说:“我会想知道我们之后的关系可以变成怎样?我会对未知充满好奇。”他说我们之间的关系已经没有其他可能性了。两人关系的顶峰就是共同生活,之后就会走向分离和结束,所以还不如在“无聊”之前就停在那里。……

	之后我试过几次试图突破他的自我保护,想要问出他会想要些什么,但每次都问不出来。我感觉他把自己有所欲求的部分隐藏地很深,而且他还不断问我:“这就足够了吗?”同时,我也感觉他在用这个提问促使我走得更深,或结束这次的聊天。我说我脑海里有一幅画面:一个观察者的画面。你在这副身体的背后一直观察着事情的走势。他说我如果这么觉得就这么觉得。

	一方面,我想找到那个他还渴望着我甚至只是对事物和他人有所欲求的那部分自我,但另一方面,我并不想破坏他的自我保护所可能营造出来的平静感。如果他以这种方式获得了平静,为什么我还要为了自己的私欲而破坏那份平静感?

	我回想起一开始和他见面的初衷是为了拉近和他的距离,而在这次的见面里也确实做到了这一点,但我依然有一种并不足够的感觉,距离还不够近。

	\blockquotesource{白色灯塔先生}{“这就足够了吗?”}{2022}
}

我脑海里有设想过很多次,如果下次和前任见面的时候,我会对他说些什么。我会对他说:“这并不足够!和你的距离并不足够!但我又能怎么办!我什么也做不了!”我也会想到,如果自己真的足够好呢?他会不会就愿意和我复合了、愿意再次在一起了?我知道这样的设想并不现实,但我好像依然想去继续抱着这样的幻想,好像这样的幻想能给我最后一丝希望,而不至于彻底地绝望。因为如果是彻底地绝望的话,如果自己无论做什么都无法改变现实、无法改变任何事物和人、无法留住任何人的话,那我还活着干嘛……

在和前任的相处里,这种无论自己做些什么都不足够的感觉“诞生”了,而之后我又希望能在前任的后续关系里攻克这一不足够感,就像是一直试图去完成一个未完成事件。

也许我还需要很长一段时间去哀悼这一彻底的绝望吧,而不是抱着最后一丝希望像个疯子般试图完成这一未完成的、未攻克的不足够感。

