\chapter{零碎的想法 | 33}

\ardate{2022-02-18}{\_IShTTlIg-UExjC3dXJXlw}



\section*{1}

几天前第一次鼓起勇气去接电话热线。本可以接完一个热线就下线缓一缓,但我却一连接了好几个热线,直到热线开放的时间将近结束时才停了下来。挂完电话后,我感觉自己的脸是通红的,好像自己进入了另一种状态\pozhehao{}像是沉迷于美剧的一集又一集剧情。当第二天快睡醒时,我感觉自己在感受着一种很强烈的情感,那种情感从whisper变成了roar,而那个roar把我叫醒了,像是在说:“我再也不要接热线了!”

但当我彻底醒过来后,我开始感受不到也回想不起“我再也不要接热线了!”背后的想法和感受。我回顾不起来的那种强烈的冲击感到底是什么?

在睡梦里,我隐约记得那些来电者的声音(或者更应该说像是鬼魂)徘徊在自己的脑海里。他们好像都唤起了我的一些回忆,甚至是我的自我的一部分。由于热线的保密协议,我不会在此讲述任何关于来电者的故事或信息,但我可以回想起那些自己被唤起的记忆和自我。

\section*{$\blacksquare$}

我会想起大四在书店实习时,有上司在的时候我极度不喜欢待在店里,所以就会经常去洗手间,找个隔间刷手机。但那里的洗手间很破旧,灯光也很暗,还有一股挥之不去的臭味(差不多是恐怖电影的氛围)。因此,后来我就去了洗手间旁边的消防楼梯通道里,坐在台阶上刷手机。我记得那时候我还在玩一个农场养成类的游戏,通常玩个15分钟才回去店里。

所以现在有时候我去商场洗手间后尝试去探新路线时,我会不时在几乎不会有人经过的楼梯间里看见不同的餐饮业店员坐在台阶上刷手机、吸烟,甚至是睡觉。我想我可能在某种程度上能理解为什么他们想在一个off-the-gird的角落呆着里,因为在剩余的时间里他们被迫要呆在一个他们只想逃离的地方。

\section*{$\blacksquare$}

我会想起我的朋友琥珀,他离开这个城市回家生活也快一年了吧。虽然他说他被诊断为双相障碍,但和他相处时,我看到的几乎只有他抑郁的一面,而没有狂躁的一面。在每次见面时,他会想我多陪陪他,比如说在他的住处过个夜或待久一点,不过那时候我因为过往的一些不好的经历而本能性地排斥任何试图“粘着我”的人,所以那时候我并没有陪他久一点。

当他离开了这个城市后,我一直感到蛮后悔的,后悔于自己没有在他需要我的陪伴时多陪一陪他,也后悔自己因对过往经历的抗拒而没有从他那获得更多的陪伴,而现在他也早已不在这座城市生活了。

虽然时间总是有限的,无论是热线的时间还是我和琥珀见面的时间,但有时候依然会想要让某一刻可以一直延续下去。

\section*{$\blacksquare$}

我会回想起去年1月和前任约见面的事情。在那之前,我压根没有想到他会回我信息,毕竟之前我想约他见面的信息已经陆陆续续发了差不多一年(每几个月发一次),但他一次都没有回复过我。那时候当看见他突然回我信息说见一面时,还在吃午饭的我差点把饭喷了出来。

见面前,我的想法是想知道他无故消失的原因。见面时,他说他是因为太忙了才在这一年多里没有回我信息。(我信你个鬼!)我还问了他他觉得我们分手的原因是什么。我记得他当时的回答是我只是想找个人依赖,“找另一个你能依赖的人不就好了”。

见完之后,我感到更加难过,对又一次的分离更加不舍,因为不知道下一次再见会是什么时候,甚至不知道是否还有下一次的见面。只好继续一人独自面对未来的未知,面对那个曾经唯一的依靠、现在唯一的依靠消失在现在的茫茫人海里,消失在过去的回忆里。

\section*{$\blacksquare$}

我想起一开始用社交软件时(大概是在16-17岁的时候),有两个男生同时喜欢我。我后来跟他们说我会选其中一个在一起,后来也确实在一起了,但继续相处后觉得那个男生很少回我信息,并给我一种感觉:我并不是他生活里重要的人。所以后来我提出了分开。

我想那时候的我甚至连喜欢都没有喜欢上对方,而只是因为有这样的选项、这样的人就去试错了,更像是以约炮的态度进入一段情侣关系。

在选择和谁在一起时,那时候的我会比较两人的各种特点和优势,比如说能腾出来的时间、金钱、身高、身材、性格等方面,一些最为表面、标签化的东西。即使在这两个被我所量化的人里选出了更优的那一个,但这依然不是自己所想要的人。和自己不喜欢的人做一些自己设想会感到很亲密但实际上并没有亲密感的事情(例如牵手)时,那种违和感,那种“一切都是错的”的感觉。

\section*{2}

前几天读到《格式塔治疗实录》里皮尔斯说的这段话:

\blockquote{最知名的未完成情境就是我们没有原谅我们的父母。你们都知道,父母总是不对。他们要么太强,要么太弱,要么太聪明,要么太傻。如果他们很严厉的话,他们需要柔软,等等。可是你什么时候见过全对的父母?如果你想玩指责的游戏,你可以一直指责父母,让父母为你所有的问题承担责任。除非你愿意放开你的父母,否则你就继续说服自己你是个孩子。但是画上句号,放开你的父母,说出“现在,我是一个大女孩了”,这是另外一回事。这就是治疗的一部分\pozhehao{}对父母放手,特别是原谅父母,这对大多数人是最困难的事情。}

\useimg{aimg/2022-0219-1.jpg}

我会想到Rick and Morty第五季里,Evil Morty们发现Rick们用Central Finite Curve从无限宇宙中分离出了Rick是全宇宙中最聪明的人的那些无限宇宙,而无数版本的Morty都在这些被分隔起来的无尽宇宙里围绕着Rick转。耗尽一生的无数版本,在一个无尽的摇篮里照顾着一个巨婴。所以Evil Morty最终离开了这些被分离出来的无限宇宙,去了真正的无限宇宙。

原谅不代表放手,放手也不代表原谅。原谅和放手是两码事。放开父母并不代表原谅父母,也更不需要原谅父母,只是不再彼此纠缠在一起,试图获得一些永远不可能得到的事物。

我还会想起当“原生家庭”这一概念开始被大众所凝视时,有不少的声音在说自己之所以那么糟糕是因为原生家庭的错。然后有另一种声音在说:“原生家庭”这一概念不是为了让大家利用父母来为自己摆脱责任,而是要看见父母给自己带来的影响,并承担自身的改变的责任。

文中“全对的父母”也让我想到温尼科特提出的“足够好的(父)母”,而且全对的父母可能也会让个体在成长的过程中无法经历足够的受挫,无法从适当的受挫中发展属于自己的心智化能力和自我安抚能力等内在资源,从而在之后的人生道路中难以承受外界压力。不过,全错的父母也无法给孩子提供心智化能力和自我安抚能力等内在资源的成长土壤\pozhehao{}无法成为一个安全基地,无法contain(涵容)孩子的情绪、无法holding(护持)孩子去探索世界。

\tristarsepline

当我想到自己的个人经历时,我会回想起读大三的一个梦:我在塔上的平台看着一个小朋友走向远处那等着他的父母以及那个巨大的海贼船(游乐设施),背景是一片金黄色的日落。梦里的我感到既开心又伤心\pozhehao{}开心于那个男孩有这样的父母,伤心于自己永远不可能拥有这样的家庭、这样的父母。

在最近一个月所学的释梦书和课程的观点是:梦中的人是自己的子人格。如果梦到了孩子,那很可能是内在孩子;如果是梦到父母,那可能是内在父母。所以如果以另一个角度去看那个塔的梦的话,或许可以看成:我(被我认同的自我)看着自己的内在小孩走向自己的内在父母,但我并不认同我自己就是那个内在小孩,而只是一个旁观者。塔的高处对我而言象征着更遥远的事物,一些我更想要去触碰、去抵达的地方,毕竟在打算坐电梯去塔顶时的我是充满着好奇心和探索心的。背景的那片日落对我而言象征着未来更美好的事物\pozhehao{}正如读小学时几乎要“走出”六楼的窗外时所看见的日落,而他们(那一家三口)正在走向未来更美好的事物,但我还不敢跟上去。我记得那个小孩邀请我,说“要一起去坐那个海贼船吗?”我说我要考虑一下,并叫他先过去吧。可能我的内在小孩在邀请我走向未来,而我的内在父母早已身处更未来(远)的地方。

所以当读到“愿意放开你的父母”时,我会感到蛮伤心的,因为这意味着detachment,与自己的现实父母进行分离,同时接受自己的现实父母永远都不足够好的这一一直以来的现实,以及敞开自己地去感受自己对此感到的悲伤,哀悼对足够好的父母的幻想的丧失,放弃对足够好的父母、对一个家的期望和渴望。不过这份渴望依然还会在,只不过它不再指向现实生活中的任何一个人、任何一处地方\pozhehao{}no place as home。相反,它开始指向内在小孩、内在父母、梦境里的地方等内在(资源)的部分,一些能够永存于自己的内在世界的部分,而不是处于人来人往、世事变迁的外在世界的地方。

对我而言,唯有与现实里的父母相分离(detach),才能开始培养内在的父母;唯有与现实里的孩子相分离(grow out of),才能开始培养内在的孩子。这不代表过去那些糟糕的事情、糟糕的人对自己不再造成影响。我想它们一直都会给自己造成影响,就像背着一个时不时需要卸重的包袱。但我想,我应该继续前行了,走向那一家人,走向那片日落。

