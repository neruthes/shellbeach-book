\chapter{“想对他说CNM的冲动”}

\ardate{2022-07-23}{JfYxgx5hKKVhzHuNYJXnYw}




今天读到了一篇某个心理咨询师写的推文“我选择回避痛苦,来让自己好过一点”:我们如何面对「真实」,其中提到她在之前写过的文章里收到过一个愤怒的评论,大意是:你整天说心理咨询有多困难,既然你觉得困难,你就没资格当咨询师,你干脆别做了。在她看来,这是对心理咨询的全能幻想,把咨询师幻想为理想化的全能照顾者,希望心理咨询可以让自己不费力地变好,就像是被一个全能的好妈妈照顾一样,而当发现对方做不到时,便认为那是坏妈妈。

这让我在情绪上有所感触,后来我才意识到这之所以能激发我的情绪,是因为有人也跟我说过类似的话:“你大可以不去学习这类课程……我之所以会选择指出你的这个问题,只是说,你前阵子看共情的课程,某种意义上是希望自己能够做到这一点,但是你又做的不好。那不如不改变……如果不合适自己,不如放手。”当看到那个人这么说的时候,我感到蛮愤怒的,因为在我看来,对方是在攻击我。那个推文里的咨询师带着同理心地利用理论来概念化对方,这样的防御对她来说或许是可行的,而对我来说这或许也是可行的,但我依然有另一个自由,对他说CNM的自由。

攻击性背后带有着愤怒,愤怒背后可能有着各种各样的东西,例如说我想对他说CNM的愤怒背后代表着一种自己的边界正在被对方所侵入、破坏,但他的攻击性背后的愤怒的背后又会是些什么呢?

在后来的一次群聊里,在我看来,他在攻击另一个人的共情能力差(在他眼中的我也是共情能力很差的),然后我质问他:“去攻击对方的共情能力差真的有意义吗?”他说他并不认为他是在攻击,而是把事实摆出来,希望其他人能通过事实有所行动或改变。

但对方并不会将他自认为的“把事实摆出来”解读为把事实摆出来,而是会有其他解读,例如在我的解读里,他是在攻击对方和我,他是在用他自己的主观现实来盖过其他人的主观现实\pozhehao{}例如他在攻击对方时的话语里,绝大多数的主语都是你:你作为XXX、你对XXX很片面、你本质上并没有XXX。这甚至给我一种心灵等同模式(心智化不恰当使用的其中一种情况)的感觉\pozhehao{}把对他人,对自己想象性的解读等同于外在现实。

如果他所说的“把事实摆出来”是一种攻击,那这种攻击背后的愤怒又会是什么呢?我会想到,如果是在人本主义的理论里,这当中可能会有未被正视的痛苦;如果是在自体心理学的理论里,这当中可能会有自体客体体验的断裂而导致的自体感破损。但同时我也想到,他的两次攻击都是冲着他所认为的不会共情的人所释放的,而且两次都是他自认为对方不会共情,而没有跟对方核实自己的猜测(心灵等同)。

为什么要将“不会共情”投射给对方然后进行攻击?他的(至少是在意识层面的)用意是希望其他人能通过事实有所行动或改变,就像是:我是为了你好。我并不了解他的过往经历,所以我的猜想是,每当他有镜映需求时,他曾经或现在的重要他人或许一直无法做到镜映他的需求,甚至可能会对他说:你就是不会共情别人,我说出来是为了你好,希望你能改变。如果举一个例子,当自己口渴想要问父母要水,但父母却说:“我都已经那么累了,你就是不会体谅别人。但我说出来是为了你好,希望你能有所改变。”这更像是一种角色逆转:本应是父母照顾孩子的感受(口渴),但父母反过来要求孩子要照顾父母的感受(累),即亲职化的亲子关系。

镜映需求的早期受挫被压抑后,会在后续的生活中不断浮现于不同的人际关系当中,一直试图从潜意识整合进意识层面。正是因为过往某些不被共情的重大创伤的存在,现在便一直渴望被共情的经历,但由于当时的某些应对方式(例如“把事实摆出来”地说:你XXX)对于现在的情况而言过于僵化,而一直无法得到被共情的体验,只能一次又一次地压抑进潜意识里,然后又一次又一次地浮现出来。

不过,和那篇推文的咨询师一样,这也只是我的猜测,但这种猜测并没有减轻我对他的愤怒,想对他说CNM的冲动。

