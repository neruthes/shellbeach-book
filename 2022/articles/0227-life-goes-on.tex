\chapter{Life still goes on, I still go on.}

\ardate{2022-02-23}{BO-CdWuOxfPxW97QSjrakg}


\blockquote{
    对于想要离开的来访者,如果治疗师只是指出或暗示:结束治疗无助于她独立解决问题或者她尚且无法依靠“自身力量”发挥作用;这通常会显得既不恰当也不够尊重(有悖于职业伦理)。更为恰当的表达是,“我能够理解你想要离开,但我觉得这个决定有些仓促。不知道你是否愿意和我一起探究一下这个决定背后的原因?”
    \citebook{格式塔咨询与治疗技术}
}

在读完上周读到的这段话后,我问咨询师:“我会想到你是否会对彼此的关系有一个设想?因为在之前的一次咨询里我提到想要结束咨询时,你会提到类似于‘觉得这个决定有些仓促。不知道是否愿意探究一下这背后的原因?’,所以我会在想,那时候你是否会想些什么,或者是对彼此的关系会有怎样的设想吗?”

咨询师说TA并没有对我们之间后续的咨询有任何幻想(留意到了用词从“设想”变为了“幻想”)。TA之所以觉得想要结束咨询的决定有点仓促,是因为我之前说过我的亲密关系都只维持了3、4个月,所以TA会考虑到我决定结束咨询这本身是否是一种人际/亲密关系的重复。

听完咨询师对此的猜疑后,我感受到了一种亲密感,以及在此之后感受到的抵触感。我向咨询师表露了自己的感受变化后,咨询师问:“你能多说一说那种抵触感吗?”我继续说:“这种抵触亲密的感觉会让我想起我和我妈的互动。因为小时候经常被我妈打,所以现在当她试图触碰我(即使是隔着衣服地触碰我的肢体),我也会很本能地弹开。而且我也很反感有陌生人去触碰我的身体。就好像那种抵触感存入了自己的肌肤里。咨询师说:“确实,当你被打时,是你的皮肤承受了疼痛。”我点了点头。

咨询师继续问我关于这种抵触感是否还能想到更多吗?我说我好像不能想到更多的事情了。在沉默了一下后,我说:“我在想其他人会不会不那么无趣,其他人说不定能联想到一些其他的回忆、一些故事的展开。但我好像只是把它(抵触感)看成是情感本身。情感出现了,就看着它出现,然后它说不定什么时候就会消失。”

咨询师回应说:“好像你会想知道其他来访者会不会更有趣。其实你是不是在担心我会感到无趣?”我马上回答说:“不是的。我不是在担心你会不会觉得我无趣,而是我好像已经有一个内心形象在认定自己是无趣的。这可能更像是一种自责的声音吧,在说‘为什么自己那么无趣’。我想很多人都有自责的这一部分。那个自责的声音还会说:‘如果自己在读大学时就知道自己想学的东西是什么、想走的方向是什么,那该有多好。’但又会出现另一个声音:‘自己不可能突然就找到自己想做的事情。这是需要过程的,我走到现在这一步是需要过程。’”咨询师回应道:“好像你在脑海里会两个不同的声音。”我回答道:“嗯。有时候我感觉自己就像是个协调者,协调着内心不同的声音、不同的欲求。比如说有一天下班我在想是回家上网课、还是去约人、还是去想逛的地方逛,但我决定不了主意,然后我就会觉得很累,所以后来决定下班先睡一觉。在睡觉时,我做了一个梦。我在梦中看着我面前的一个小女孩很乖地站着,然后我在犹豫应该往哪个方向走\pozhehao{}一时往左,一时往右,但最终都只是在原地来回走动。因为最近在上释梦的课程,那个课程的理论说梦里的人都是自己的一个人格,而自己所身处的那个人是自己在那个当下所自我认同的自我。我确实觉得梦里的那个犹豫不决、一时往左一时往右的那个我是我在那个当下所认同的自我。而我面前的那个小女孩看似很乖,但当我试图去 role-play(角色扮演)那个小女孩时,我感到很累,我在想为什么我的养育者、照顾者一时决定往左、一时决定往右,我究竟应该往哪个方向走?

所以这样看起来,即使我能在有着不同欲求的自我冲突中抽离出来,站在一个第三者的角度来协调各自的需求,但依然有一个小女孩般的自我会跑出来说:‘为什么他决定不了接下来要干什么、要往哪个方向走。就好像总有一个自我会跑出来和我抗衡。’”

咨询师问:“那你会对这种协调会有怎样的策略或想法吗?”我说:“我好像暂时没有什么策略或想法,现在只是试图将时间分配给有着不同欲求的自我,比如说把今晚的时间分给那个想去上网课、学习的自我,明晚分给那个想到处逛逛的自我。”

咨询师联想到:“好像之前你在我们的咨询里也会承担协调者的身份,比如说当你发现我们在走的方向不一致时,你会去协调我们两个人走的方向。”我回顾了一下那次的咨询,回答说:“嗯,确实是。那时候的协调者也像是站在第三者的角度去看彼此冲突或者说分歧。”在沉默地思考了一会儿后,我继续说:“我也会想到,如果我去不承担那个协调者的角色的话,那我又能成为谁,我又能站在哪里?如果是以前的我的话,我好像就只会被卷入冲突里。这一协调者的角色好像更像是自我认知、自我认同的一部分。如果没有了它,我好像会更加迷茫。……嗯。我想,比起看着自己面前的内心世界里有着不同欲求的自我的相互冲突,以及面对人际关系里的冲突,会让我更加感到更加迷茫的是,如果我不成为一个站在第三者角度的协调者的话,那我又能是谁?这是一种更深层、更弥漫的迷茫。”

咨询师说:“我好像能听到有两个角色:一个是协调者,另一个是协调者另一面的角色。”我有点困惑地问:“所以你想知道那另一面的角色是什么吗?”咨询师点了点头,我想了想后说:“另一面可能是会完全陷入其中的角色吧。如果我真的陷入了冲突里,我可能就会变成一个完全不像是自己的人。”咨询师说:“我会想起之前你说你身处亲密关系里的时候,也会感觉自己像是变成了另一个人。”我回答说:“嗯。在亲密关系里,自己的心智化能力好像会降低。”咨询师回应说:“好像陷入了冲突里,你会感受到一种自我完整感的缺失?”我低下头边感受边思考,并说:“嗯,应该会是的。”

咨询师留意到我开始频繁皱眉,并说:“我会留意到你从刚刚开始就频繁地皱眉。这背后你会是感受到什么情感吗?”我说:“嗯,我还真的是一直在皱眉。Em……我好像感觉自己身处于困境,一个我自己还没能找到出路的困境。这种陷入困境的感觉让我想起以前读大学时的我,那时候的我觉得自己来到了一个新的世界,充满着重新开始的机会。但在参与了一些竞赛后,发现自己并没有多少能力后、在自己受挫后就变得越来越不怎么与人社交了,也不怎么离开校外、离开宿舍,就窝在宿舍里。我想现在的状况可能激发了以前在读大学时感受到的那种困境感,感觉自己被困了。我想在的困境感只是以前读大学时的那种感觉的延续或是加强吧。”

咨询师回应说:“好像在接触到那种感觉后,你又会理性地去分析这种情感。”我笑了笑说:“嗯!好像当自己开始陷入那种情感时,我的理智化就会开始无意识地工作,试图将自己从那种情感里捞起来。我并不是有意识地去理智化这整个过程\pozhehao{}去将过去的感觉和现在的感觉连接起来,来减缓现在的情感的强度。但自己确实无意识地这么做了。我想可能是因为这个过程自己已经做过很多次了:去感受情感,去追溯情感的起源,去理解情感。”

咨询师回应说:“刚刚你会用‘捞起来’这个词,我在想,如果真的沉了下去呢?那会是一种怎样的状态?溺水?”我感受了下,回答说:“会是一种窒息感。我会想起读大学时自己很想去自杀,其中一部分原因是因为自己每一口呼吸都感觉自己没有在吸入氧气。我一直在呼吸,但吸入的每一口气里的氧气都不足够,一直处于缺氧的状态。怎么呼吸都吸不够足够的氧气,每次呼吸都很痛苦,很难受,很想结束这一切。”咨询师回应道:“这听起来真的很痛苦。”(但我怀疑,咨询师很可能并没有类似的体验,可能并不能切身感受到我的感受,而只是说说而已。)我继续说:“我想这种窒息感是因为自己在心理上困境感在身理上体现为了窒息感。”咨询师回应道:“那好像真的是一种很不舒服的感觉,才会想把自己‘捞起来’。”(我留意到咨询师在用词上把“痛苦”转用了一个程度更低的词“很不舒服”,可能是想让我减轻陷入情感的程度吧。因为我记得之前学的一个边缘性人格障碍课程的关于心智化的内容里有提到这一技巧)。我继续说:“虽然一开始进入到那种情感会不舒服,但之后就会完全沉浸于其中,会觉得舒服,因为那种困境感、窒息感是自己所熟悉的事物,会有一份安全感,就像是一份拥抱,像是一层膜将自己包裹起来,可能是用于保护自己当时所处的困境吧。”咨询师简短地回答了:“嗯。”

我好像开始留意到了自己的语气变化,并说:“我的语气好像也越来越低沉、越来越慢。好像在这次咨询的一开始的那个问题开始一直越挖越深,但直到挖到现在也依然找不到一条出路,一个解决办法。越来越多的冲突和矛盾,越来越深陷于其中,找不到出路。”我感到很沮丧、很抑郁,然后咨询师说:“这可能需要我们在下一次的咨询里继续聊这件事。”我回答了个“嗯”,然后起身离开咨询室。

在等电梯时,我试图继续往下挖,我好像能挖到一个更深处的自我,那个自我身处于黑暗和痛苦当中,像是在水下被一层膜裹成了一团。那个自我好像在说:我还依然身处于痛苦当中,LOOK AT ME! 在毕业之后,我就理所当然地认为读大学时的那部分自我消失得差不多了,但好像那部分的自我依然存在着,那个之前被我遗忘的自我,依然身处于痛苦的自我。


\blockquote{
    结束心理咨询后,我去了前任以前住的公寓小区,计划和一个男生逛逛附近的一个有湖的公园。在等那个男生时,我去了以前和前任以及前任曾经养过那只猫Holly在某天晚上一起去的那个草地小公园里坐着,试图回想过去那些越来越模糊、越来越遥远的往事,回想那只猫在草地里乱串、在路边的石基上走的画面。直到现在都还会设想Holly死后被前任埋在了这附近的哪个位置。

    公园旁边有一条流淌在地平面之下的小溪,光圈围绕着落地灯,建在斜坡上的公园有着交错并高低不一的小路,每条小路之间都有花基,小路和花基之间被脚高度的矩形大理石所分隔。

    那只猫不喜欢走在小路上,但总是走在矩形的大理石上,只在换路的时候从大理石上跳下来小路上,之后又跳回到大理石上。与它的同类不同,那只猫很大胆,但与它的同类很像的是,都爱发脾气。

    在这里的一天晚上,我曾经和前任在溜他的猫,那只猫看起来一点都不胆小,还会一直跟着我们走。那时候的自己突然有一种自己拥有了想要的一切\pozhehao{}一个家、一个人、还有一只猫\pozhehao{}的感觉,一种感觉自己不再broken的感觉。

    然后猫去世了,前任逐渐消失了。
    \blockquotesource{独白,对话 |}{Shell Beach}{2021}
}


在想象与现实画面的交杂里,我突然发现自己在看着小公园里有两只狗在草坪和小路之间到处奔跑,它们是走向草地小公园的一对男女情侣养的。看着那两只狗在这个小公园里到处跑的样子,让我更加想念我和前任和Holly曾经在这里的时光,好像这样的场景现在在另一对情侣中延续了下去。This too shall pass. But it was my home, the closest home I could possibly have.

和迟到了一小会儿的男生吃过饭后,我们去了附近的那个有湖的公园。在湖边小路走时,他问我我和朋友落叶的关系,我说,那时候我们彼此最大的慰籍就是能够谈论彼此的自杀意图。那个男生问我们是讨论怎么死会没那么痛苦吗?我说,不是的,只是在讨论比如说今天多么不想起床、活着多么痛苦的事情\pozhehao{}只是谈论自杀意图,还没有到谈论自杀计划的程度。那个男生开始说,他觉得这个世上没有什么值得留恋的,但他还不会计划自杀,只是如果哪天突然离开了这个世界,也不会觉得有什么值得期盼的。我突然联想到他所读的专业,然后问这(无痛地自杀的念头)会不会是和他一开始决定读这一专业有关,他说是的。我回应说:“那你真的蛮早就已经开始在思考这方面的事情了。”我也突然意识到自己好像在做自杀风险评估,所以不打算继续问下去了,毕竟还没有自杀计划,不算是中高风险。同时我也意识到他一开始为什么问的是“怎么死会没那么痛苦”而不是其他方面,因为这是他思索已久的部分、更为看重的部分。

走着走着,我看见了小溪对面的那片我曾经计划自杀的地方。那时候打算哪天半夜觉得自己熬不过去的话,就打车到这里,躺在草坪上看着星星,拿刀刺入脖子下的大动脉。之所以会选择这里,是因为这里离前任公寓很近,离我上一个家很近,the closest home I could possibly have.


晚上,我们找了另一处江边走。在坐在江边的石凳上聊天时,他说出于一些他的个人原因,他不能和我有更深入的关系了,只能止步于此,最多只能停留在现在的亲密程度。我尊重他的选择,并为彼此的关系不能继续深入下去,不能深入到我想要的程度,不能达到我想要的事物而感到悲伤。这应该是我第三次有这样的经历了,或者说是第三次被激起和强化这种感觉,这种无力感、悲伤、窒息感、失败感、受挫感。

我回想起下午一起坐在湖边的场景,想到自己下午身处的那个我曾经计划自杀的地方。Life still goes on, I still go on.

