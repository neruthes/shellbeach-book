\chapter{“那份勇气的来源”}

\ardate{2022-07-18}{bczWVRQDk7zOqmdI7GcDPA}




最近和一个朋友频繁地见面聊天,而他的经历会让我觉得和我自己的经历有不少相似之处,例如他对象在他看来就像是建了一道墙,无法让他跨过去。而我朋友会正因为他对象建了一道墙而更奋力地想要冲破那道墙、拉近彼此的距离、突破对方的界限,而他对象也因此而更加拉远距离、围墙建得更高更结实、界限划得更分明。他不知道他对象是怎么想的、为什么要怎么做、为什么要这么绝情。在我看来,他现在的状态像是越来越失望,甚至接近绝望的状态。他会说:但他又怕死,我回应说:“但你依然想死。”

但我不想表露太多关于他个人隐私的部分,所以我还是更多地说我自己的部分。

我和前任的相处时,两人的关系一开始很美好,甚至是过于美好。后来彼此发生了摩擦、冲突和误会。这些误会越卷越大,直到后来两人都不再想和对方说话、交流。在误会逐渐累积时,我想找他面对面好好聊这些误会,但他并不怎么给我这样的机会,或者是他会给我这样的好好聊的机会,但这样的机会在我看来并不足够多。误会堆积的速度远超过误会解决的速度,而两个人的关系也随之逐渐崩解。

那时候的我不知道他是怎么想的、为什么要这么做、为什么要作出这样的决定。比如说我不知道为什么他认为两个人不合适但分手的决定权放在我手上,不知道为什么他觉得我们不合适,不知道他为什么有时候过夜不回公寓但从来不会告诉我他在哪里过夜了,不知道为什么他即使回到公寓但依然只是个空壳般地魂飞魄散,不知道他为什么叫我离开公寓一段时间再回去。不知道对方在想什么、为什么要这么做的困惑感、不安全感、陌生感等各种各样的情绪驱使着我跟上去,想去知道更多,想去冲破对方设立的边界,拉近彼此的距离。

在他无故消失一年多后,我已经逐渐习惯了由他引发的各种情绪的存在。然后他重新出现了,说他知道我还想知道为什么、还有困惑,便约了喝咖啡聊天。聊完之后,由他引发的那些情绪重新被激活了、程度加强了,我更加感到困惑和迷茫,更加不知道为什么他要这么对待我。(即使是现在,他的表述大致是:任何事物和人在他看来都是一样的。)

所以后来,我开始刷一些亲密关系的书,之后开始看心理咨询的书,然后开启了一段和以前截然不同的经历。那时候的我与其说渴望得到能够解释对方的所作所为的答案,我更渴望的是,能够得到答案背后的那份平静感\pozhehao{}我渴望自己终于能够make peace with the past. 但事实上,这样的平静感并不会持续太久,因为渴望追求答案的这份渴望背后的那些痛苦和绝望依然在那里,直到现在也依然在那里。There will never be peace.

和那个朋友聊天时,他问我:如果改变不了他人,不知道他是怎么想的、不知道他为什么要这么绝情,那好像也就只能改变自己了。我回应说:但好像你也无法改变你自己\pozhehao{}没有办法阻止自己感受到悲伤、难过,也无法阻止自己一直在思考要怎么办、一直在设想不同的情况、一直在猜对方在想什么、为什么要这么做。对你来说,关于自己的那些东西也依然停不下来。

后来我跟那个朋友说,其实我现在的bar(标准)已经很low了\pozhehao{}只要还活着就足够了,现实生活里的各种各样的事物和人我都能放手。工作做不好,吃饭吃不下,睡眠时间乱套,和身边的人的关系处理不好、开始恶化甚至是崩解,这些对我来说都是可以放手的,甚至连自己的生活和生命我都可以放手。当情绪到了最糟糕的状况下,在那时候的我想到:究竟什么才是最重要的?什么才是唯一重要的?什么是我能够去放手但我依然不愿意去放手的?我不想放手未来的可能性,不想放手自己对对方有好感、喜欢的人。除此之外,一切都可以放手。而那些我能够放手但依然不愿意放手的东西和人,我会紧握到最后。

那个朋友说,如果那么多的东西都能放手,那会是有多绝望。我想也是啊,就是一种对现实世界彻底地绝望\pozhehao{}改变不了他人,甚至不知道对方是怎么想的,也改变不了对方的选择,改变不了这个现实世界,也改变不了自己。自己的存在对于这个世界而言是多么的微不足道,什么也改变不了,无论自己做什么、付出多大的努力和代价也影响不了这个世界,也不会有任何反馈。

从小到大的环境,无论是教育环境还是家庭环境,它们灌输的其中一个价值观是:只要努力就有回报。比如说只要努力学习,成绩就会提高。但现实世界往往没有那么简单,并不是努力了、付出了就会有回报、就会有反馈。有时候甚至是绝大多少情况下,自己的努力并没有办法让现实世界变得更好,这个世界也并不会因为自己的存在而变得更好,甚至是更为糟糕。

如果是这样的话,为什么自己还那么努力,为什么自己还要做任何事情,为什么还要往现实世界投注任何精力?但在那种绝望的境地里呆久了,我依然会想去做一些事情,想去尝试一些新的东西,想去促成一些事情的发生。有时候甚至只是和一个朋友见面或者是和自己喜欢的人聊聊天\pozhehao{}一些在我看来微不足道的事情。但那时候的我依然想这么做,因为即使处于彻底绝望的境地里,自己内心依然有一部分存活了下来,依然渴望着更多。

那个朋友说我是多么勇敢才能面对那些东西。当他这么说时,我才意识到,好像之前的我一直无法看见自己有勇气的部分\pozhehao{}有勇气去面对现实世界原本的样子,面对与他人的关系,面对我自己,即使这并不会改变些什么、事情也不会变得更好。这份勇气的来源,可能是因为我知道自己一直以来都能够放弃现实世界里的任何事物和人,但也知道自己内心依然有着那么微弱的一小部分,那个部分依然渴望着更多。那份能够放弃一切但依然有所渴望的部分,便是那份勇气的来源。

今晚在亲戚饭局上,我和以前有所不同\pozhehao{}我开始表达自己的观点,而不是像以前总是在回避或完全不参与他们的话题。表达完自己的观点后,自己马上就被攻击了。在那个当下,我会觉得蛮受伤、蛮伤心、蛮难过的。但同时我也马上思考为什么我会感受到难过和伤心。

我非常渴望有一个好的家庭,但我所身处的环境并不是一个好的家庭\pozhehao{}只要我发表自己的观点就很可能被攻击。我想到我可以像Neruthes那样子通过人际关系撤回将自己保护得很好,而我也曾经有很长一段时间就像他那样将自己保护得很好。但这种自我保护的方式并没有让事情变得更好、并没有让我自己变得更好,而更像是一种止损罢了,毕竟没有冒险就没有回报。

另一方面,我也不可能遗弃掉、放弃掉渴望的部分,不然,我可能会成为一个怎样的人?

如果没有了那份渴望\pozhehao{}无论是渴望另一个人、渴望亲密关系、渴望一个好的家庭\pozhehao{}的话,我很可能在很早之前就选择自杀了,因为并没有任何值得我留下来、留在这个世界的理由。而且,如果我为了现在的这个糟糕的家庭而放弃了自己充满渴望的部分的话,我也就放弃了未来可能会有的任何家庭。

所以我会选择直面现在这个糟糕的家庭,他们就是那么的糟糕,没有理解、没有接纳、没有共情,只有忽视和攻击。

同时我也会想,为什么今晚的我和之前每一次饭局的我有所不同:为什么这次我会去表达自己的观点,而平时都是回避或完全不参与他们的话题。好像当看见自己有勇气的那一面后,那个勇气的部分开始逐渐manifest(呈现)了出来,那个部分开始呈现得更大,更不在乎自己是否会受伤,而只是去追求自己所渴望的事物\pozhehao{}渴望一个好的家庭、一段亲密关系、一个能提供自体客体经验的他人,渴望有所回应、被尊重、被看见、被理解、被镜映\pozhehao{}各种我所渴望的事物。Just not in this fucking family.


