\chapter{聊天 | 与Neruthes, 22 Feb 2022}

\ardate{2022-02-23}{CG6K6SslxcUyUCffKbi8bw}

\midnote{以下是和 Neruthes 的聊天逐字稿(经修订并经当事人同意)}

\section*{照顾者与被照顾者}

\dialoguelist{Neruthes}{
\dialogue{Neruthes}{亲近他人的冲动是在补偿被前任遗弃,亲近前任的冲动是在补偿被原生家庭遗弃(?),然后还会有更深层次的「遗弃」吗?}

\dialoguesepline{Neruthes}{(聊天开始)}
\dialogue{Neruthes}{那么这种关于被养育被拥有的这种心态,我的这个初步理解、我的这种推测是正确的吗?以及,上文也提到了一些关于这个被养育被拥有这件事情上,所以这几条消息就是最近这篇文章我总结出来的,基于最近的一些文章我总结出来的一个归纳吧。这种归纳,你从你的角度看,你怎么看?}
\dialogue{August}{我觉得会有那种感觉,就是那种被遗弃的感觉。但现实情况是,其实我不会记得完全真实,但是那种感觉是一直叠加下去的,就像是在经历了一次又一次亲密关系之后,然后,要么是一些令人失望的,要么是对方是无故消失的,所以这种都会一次又一次加强那种被遗弃感。}
\dialogue{Neruthes}{那这种被遗弃感的感觉的获得,它是以亲密关系为前提的是吗?如果你不认为双方处在一种亲密关系当中,那么两个人之间任何互动都不会让你产生被遗弃感?}
\dialogue{August}{但是如果是简单的朋友之间也会啊,就是,比如说如果你有哪一个朋友交往的很好,但是突然他不联系你了,即使那个朋友在你看来还没有到达亲密关系那么深的程度,但是依然会对你有一定的影响,有一定的冲击力。}
\dialogue{Neruthes}{OK。所以这还是一个比较扩大化的,会把生活中的各种各样的离别都容易往这种感觉上去归纳。}
\dialogue{August}{嗯。因为好像这个东西一直在重复的时候,就开始有点警惕,就是那种感觉,好像越来越深,越来越深,就好像一次又一次的强化就只是一个很小的刺激源(但)可能都会被激发很大的情感。}
\dialogue{Neruthes}{也像是某种警觉性。}
\dialogue{August}{嗯。}
\dialogue{Neruthes}{就是看到白胳膊就想起。就想起nudity,就想起incest,就好像也有点这个意思。生活中的一些离别,作为一种白胳膊,你看到这个白胳膊的时候,你可能会就往远一点,去多脑补一些它的比较极端化的这种状况,作为一种警觉性来使用。那么,关于这种被遗弃感的成因,我的推测就是,或许你在与人交往中,比较更多的字句与被养育者、被养育、被拥有的这样的一个角色。那我的这种推测,你觉得你会怎么看?这种推测的方向是正确合理的吗?还是说你有不一样的解释?}
\dialogue{August}{不只是。在我看来不是被养育,而是被照顾,就是自己的需求会被另一方所满足,就像这种被照顾。然后我感觉这种被照顾的感觉是亲密关系里面都会有,就像是比如说从小朋友、小孩子的那一种被养育者所照顾,然后到之后的亲密关系里面,就是被另一个人所照顾。}
\dialogue{Neruthes}{被照顾。如果这种照顾如果有缺失,那么就会构成某种意义上的遗弃了。}
\dialogue{August}{而且(程度)更强。更着重的是对方的离场,就是对方是完全消失的,就不只是\pozhehao{}比如说一个母婴来说,不只是母亲没有及时照顾到这个婴儿,而是母亲直接消失不见了,只留下这个婴儿一个人。}
\dialogue{Neruthes}{那我能不能理解成,这个事情强调的点不在于分别,而在于对方对于现状的单方面破坏?}
\dialogue{August}{我觉得两者都有,但是单方面的离开会更加具有破坏性。}
\dialogue{Neruthes}{在你具体的经历当中有没有过那种分别,但是不那么单方面的经历?或者你的经历过这种“分别”都是相当单方面的?以至于在你的经历中的这个重量……}
\dialogue{August}{有,有一段是去年。去年有认识一个男生,但是他是后来回老家那边,然后现在也有微信上偶尔聊几句那样子,就不是人完全消失了,是能够聊得上,就能够知道对方会回消息。}
\dialogue{Neruthes}{这种程度的,所以就……消失跟消失或者说离场跟离场也是不一样的,它也是分不同程度的。不同程度的离场,所以也就自然会有不同程度的被遗弃感。已被照顾的这个状态去自觉。为什么会有这样的习惯,这样的模式形成起来呢?这个往我自己的经历里面套,我觉得是不太套得动的。可能我比较会反过来,我更容易往照顾者的那个角色套。至于为什么,或许也有一些我独特的原因。对照顾者的角色的偏好的形成会是一种怎样的过程?或许可以通过两种不同经历的对比,我们可以操作总结。对照顾者和照顾者这种不同的角色偏好。为什么会形成?}
\dialogue{August}{童年被养育者,被父母照顾的那种经历?就是这种经历迁移到了之后的一个又一个的亲密关系里面。}
\dialogue{Neruthes}{我记忆中有一帮发展心理学爱好者,特别喜欢讲的是成年后的行为作为对童年的补偿的意义。所以,能不能从这个角度上去猜测,或许这种被照顾者的这个角色偏好,它象征着某种童年经历、象征着童年的某种缺憾,来自于养育者的一些的所谓的不负责任?\\
这个提示了我,其实我的经历里面有这样的一个部分,关于童年被喂饭的经历。被喂饭是一种非常难受的经历、的体验。我饱了,但是总有人觉得我不饱。儿童的胃容量是比较小的,但儿童的热量消耗并没有比成人明显的小很多,因为会把摄入的食物、各种营养素固化到自己的身体内,用于成长新的身体组织。但胃还很小,消化能力还比较有限,所以成年人一日三餐的这个节奏,它用在儿童,特别是年龄比较小的儿童上,他可能是很不合适的。但是一日三餐爱好者,他就觉得祖宗之法不可变,一定要一日三餐,多一餐不行,少一餐也不行。那这样子就有人会觉得我家孩子这一顿是不是吃太少了,一定要让他多吃点。然后就总有人就想追着我喂,想让我多吃点多吃点,但是“我饱了呀,你不要喂我呀”,我就觉得很烦很难受,“哎呀,不吃。都说了吃饱了,还要喂烦不烦?”再然后就有的时候玩着玩具或者看着动画,精力比较集中,这个时候就会趁我不注意喂到我嘴里,让我吃下去,然后等我反应过来的时候已经吃下去了,“哎呀,就更加烦了,哎呀,来骗来偷袭”。这就是我童年经历中一个关于被养育的这个角色中的很不好的记忆。所以这可能就是养育者的过度负责任。在这个提供养育、提供照顾这件事情上搞得太过火了,就会过犹不及吧。搞成这种样子,可能对后来的这种偏好(照顾者)有一定的影响。\\
因为有这样子被过度喂养的经历,所以才更容易有这样的倾向来……不去希望作为、不去在后续的与他人的关系中复刻那种被照顾的感觉,因为被照顾这件事情,它在我的童年里面有不好的体验。所以其实我也不是很理解它的反面是怎么样的。比如说饭不够吃,在我们这个年代其实不太会吧,或者是其他的一些缺失,比如说什么加班、出差特别忙,没时间照顾孩子。这个倒是有可能,但是我其实不太理解,因为按照我的童年记忆,我觉得我的生活中、我的记忆中,(在)我的童年生活,我是不太需要、不太依赖他人,一个具体的人的。有电视,有玩具,有吃有穿,我觉得就很好了。有各种各样的书和电视节目,那个时代甚至还有广播可以听。我对具体的人没有什么需求,但是我可能会对人的劳动成果有需求,比如说给我做早餐和晚餐。这是人的作为劳动成果的价值。但是,所以说这个人的价值,我是不太重视的。我更不管,我只要把早餐、把饭吃好之后,我剩下的时光,我觉得跟我的玩具、和我的书、和我的电视一起过,我觉得挺开心。所以说那个人,还是做TA的晚餐去吧。}
\dialogue{August}{好像只有物质需求,但是没有了心理需求。}
\dialogue{Neruthes}{就按照成年后的这些年,我总结出来的,接触的这种社会上的话语,就是我在个人的生活中,我在自己独处中,通过这种客观物体上的这种互动的生活,给我自己带来的幸福,和别人、他人互动的过程中得到的幸福最终是一样多的。比如说我最近的社交活动其实做的比较少,除了网上,偶尔水水群、刷刷知乎这种程度,也就是自己写点脚本,玩一玩各种各样的新玩具,比如排版工具玩起来真好玩。对这种这种生活习惯也是贯穿了多年。就在社交活动中,社交活动只是人生的人的幸福感的其中一个比较小的来源,另外很大的一块东西还是在于就探索各种物质的、抽象的东西上,而不是寄托在和他人的关系的增进上。就我对我觉得包括成年后接触这些各种各样的软件(排版软件、平面设计软件)以及自己开发软件的这种过程,这种学习过程,这都是广义上的玩具。我觉得自己用和玩具的互动,这带来的幸福已经是相当充分的了。所以说与他人的互动就有肯定比没有好,但是我觉得我不会很去依赖它。我也不会觉得我一个月不见、不跟人约饭,我就觉得我的生活好像空洞洞。我不这么想。\\
那这里是不是也有着比较大的,比如说经历和思想上的差异。}
\dialogue{August}{我觉得就是刚刚的内容,听起来就像是一种本质性的差异,就像是有的人有更多的社交需求,所以他们会更需要去社交。有的人就像你说的和自己的玩具也能获得很多乐趣,然后也不怎么需要有这么强的社交需求,一个月里面不和别人约饭也可以。可能每个人在社交需求还有其他需求方面的那种程度,本来就因人而异。}
\dialogue{Neruthes}{这个或许里面有很大的先天成分?或许吧,都不好说。}
\dialogue{August}{可能先天和后天相结合?加上你刚刚说的被喂饭的一些经历,可能也促使了你选择另一种方式去发展。}
\dialogue{Neruthes}{所以就是我的童年,关于被照顾这件事情没有什么缺憾,而且那种狭义的照顾,其实我本来就不太需要。在这两个因素的加持之下就更加的不会在亲密关系或其他类似的亲密关系的这种与人的社交互动中,去追求一种被照顾的、这种性质的模式,也不会有这种想要作为被照顾者的这样的一个角色去与他人互动。\\
被照顾者这种角色偏好的成因,我们现在可以稍微窥见一斑了。至于说这种偏好,它要多大程度上要在亲密关系中去寻求满足?好像也是一个挺微妙的问题。\\
在我个人的想法上,我是觉得亲密关系和和心理健康上面的具体问题的处理之间还是不要联系比较好。不要搞得好像配偶是解决自己健康问题的一种工具,这我觉得这样不太好,所以,我的想法、我的观念就是亲密的互动,还是,反正有比没有强。我到底要不要追求,那另说。心理健康上的这种……比如说这个角色偏好问题,人肯定都是有偏好的,但是偏好如果所谓强到一定程度,那会让我……我的一种怀疑吧,我怀疑这会让人在亲密关系当中的这种追求,它会变味。有亲密关系,亲密关系到底要多大程度上被用来服务于被照顾感的需求?感觉是一个很艰难的问题。我不太好做出我的判断,因为我没这方面需求。}
\dialogue{August}{我想,每个人都会在亲密关系里面获得不同的东西,有自己想要获得的……}
\dialogue{Neruthes}{我觉得我的心态是比较开放的,我会去欣赏每个具体的人的美。我不太会去搞这种模板的预设。我希望他是一个能照顾我的人,或者他是一个需要我的照顾的人,或者他是一个乖巧文静的,或者他是一个调皮捣蛋的,我觉得其实都好。喜欢照顾的人和需要被照顾的人,我都能欣赏得来。我会去珍惜和每个人的那种独特的互动的感觉。这个就是可能听起来有点渣男,但是你们都是我的翅膀。和每个人的独特的那种互动的模式都是人生经历中很重要的一部分,我觉得没有必要去把其中的某一种模式,把它推高成为凌驾于其他的模式之上、其他人的关系之上\pozhehao{}尤其重要、尤其特殊、尤其值得,我就算跟这个人不行,将来还会要再找一个人放进来的这种程度。就是说将来和别的人那会有别的互动,无论他跟现在的这个互动长得像或者不像,那都是和那个人的事情,那都是将来和那个人的新的一段旅程了。只要你不去。去追求某一种特别具体的互动模式,那么我们自然不会获得和某个具体的人在某种互动模式上互动不太好的那种遗憾的感觉。只要不去预设,就不会有遗憾,只要不去……可能心态有点逃避,但是我觉得也挺OK的。}
\dialogue{August}{就把自己的期望给抹杀掉,然后也抹杀掉了遗憾吗……}
\dialogue{Neruthes}{五瓣花有五瓣花的美,七瓣花有七瓣花的美,我为什么一定要追求那个六瓣的花呢?如果这个世界上恰好没有六瓣花,那我这一辈子,我就没有机会获得幸福的,我觉得其实挺奇怪的。不见得。花的瓣数,其实几瓣都行了啦,各有各的美。如果能够和其中一朵花长相思守,那就思下去吧。主要还是看哪朵花先……先到先得的原则,去开放地对待每一份……\\
当时好像还聊了一个猜想:(你)在文章中描述过亲近他人的很大程度上是关于前任的某种补偿的措施。比如说我记得你在文章里写过每每次见过前任之后,都会更加的想要亲近点其他的什么人。我就在想这是不是某种补偿的那种意思?假如在前任身上获得的某种东西,见面之后发现没获得上,然后就更加的想换个人来尝试获得一下。}
\dialogue{August}{有可能是亲密的关系吧,就有可能是亲密的关系。}
\dialogue{Neruthes}{就是这种被照顾的感觉?}
\dialogue{August}{不是。而是与另一个人很亲近的那一种关系本身。至于这种亲近能带来一些什么,是另一回事,但是,可能能从中感觉到的是,或者是想要感觉到的是那种亲近感吧,与另一个人很亲近的那种感觉。}
\dialogue{Neruthes}{感觉。这个地方有个细节、技术问题,我也在好奇。亲近感和被理解感,这个东西它是不是有点循环论证的意思。}
\dialogue{August}{不是。你可以和另一个人很亲近,但是那个人不一定能理解你。但是另一个人能理解你,不代表他和你就很亲近。但是被理解的这种经历会拉近两个人之间的距离。}
\dialogue{Neruthes}{那所以作为一个投标单位,虽然在理解这件事情上可能很难做好。但这不并不排斥这个投标单位,在亲近感上面可以是一个供应商。就要这么去理解吗?}
\dialogue{August}{为什么是投标单位和供应商之间的理解……就不能是提供者和接受者吗?}
\dialogue{Neruthes}{一些比较有我的特色的修辞吧。我觉得也比较微妙的就是,亲近感这件事情好像我能提供,但是好像我不能提供。能提供是因为我有一种我能够理解你的幻觉,不能提供是因为我昨天看到的这可能只是幻觉。这可能是我的某种补偿机制在起作用。去解决问题、去辅导、带领他人的这样的一种模式、惯性在发挥作用。所以当看到一个好像能套的上去的人的时候,就是想往上会套。但是我也知道这只是我的一种习惯的互动模式,这不代表我和这个具体的人之间,这种可能性或者说适合程度真的能够足够高。\\
这个模式的形成是怎么回事?不太清楚。好像在年龄比较小的时候,十五六岁的时候就开始有这个倾向。这种辅导、带领的这样的互动模式。}
\dialogue{August}{你好像也说这种模式好像会影响你理解他人。}
\dialogue{Neruthes}{记得有一帮人喜欢讲假性亲密关系这个概念。我觉得我这个情况其实也是某种假性亲近,假性的亲近他人,假性的……}
\dialogue{August}{看似很亲近。}
\dialogue{Neruthes}{这种模式会促使着我把看着能套的人往里套。至于说对方到底需不需要?这个事情其实没有得到太多的尊重。最近几年来我的一个变化就是,还是要多去提前去分析分析,这到底是不是别人用的上。别人是否真的太需要这种东西?这可能会有点成就感吧。这个过程就只是在把我自己的这种我只在关心我自己,我没在关心别人这样的一个状态。这个这种状态拿到一些活动中是不好的。\\
顺带从这展开一下,从我的视角上怎么解读我们的这种互动的发展呢?第一件最重要的事情是我在利用你的文章做借题发挥。(……)然后另外一部分是一种好奇心,去探究未知的领域。(……)就学习所谓学习的目的性吧。(……)在这个过程当中,那自然也找到一些心理学之类的东西,以及去了解他人的心智,了解他人的历史,这样的一些窗口,对我来说,它会引起我的好奇心。……然后第三部分就是我的职业惯性的一种体现:发现问题,分析问题,解决问题,设置计划。我个人的生活跟职业不太分的。这就是一种惯性吧,当我看到问题的时候,我就会想我手里的工具箱,我就想试试看。}
\dialogue{August}{往里面怼。}
\dialogue{Neruthes}{有一个经典的修辞是,当你手上有锤子的时候,你看什么东西都是钉子。所以在你的文本当中,有一些具体问题会引起我的一些共鸣,所以我就会拿着锤子上去敲敲看。\\
然后接下来就要深入的讲这个第四点。在刚才这三点过程中,其实都是我的事情。这个过程当中,我只是把你的文本作为一种材料,来养育着我的什么东西。对于深入的去进入到你的语境里面去,站在你的角度上去思考,去体验这件事情,一方面我是略微有点恐惧感,因为那是一种凝视深渊的过程,(……)另外一部分就是这是一个社会礼仪问题。这似乎是会有点失礼的。(……)你在文章中提到过,你不太喜欢别人去揣测你的心态,去把你的心态、心情状况往某个成文的心理学著作里面的某个模板里面去套。我觉得这种行为它没有去重视你的个体状况。这种行为它是在以你的状况为材料在尝试论证某种理论的有效性。\\
你也表达过我的一些猜测分析是让你觉得,一方面可能有刚才描述的这种原因,不太舒服,另外一方面就是,确实是很难站到你的角度上去做有效的分析。我们的这个分析中的不符合实际情况的假设有点过多了。当然从科学研究的角度上讲,这是一种贴近真相的过程。但是从社交的角度上讲,如果排除错误答案的这个人感到不舒服了,那么这个过程其实也没有什么继续的必要。\\
因为这样的原因,所以我会觉得在礼仪这种层面上,让我不太对、不太能够去站在你的角度上去分析、去思考这样的事情。一方面是这种所谓的分析错了,你会不开心。另外一个就是,就算分析对了呢?那么这就是会是你欢迎的吗?如果你欢迎别人来做合理的分析,那么我会是这个列表中的其中一员吗?我要去过早的建立起这种信心吗?我觉得还是放松一点好。至少在有更明确的欢迎信号之后再做。在那之前就前三个个人享受的那个范围(借题发挥、好奇心和职业惯性)内对所有人来说都是开心和安全。}
\dialogue{August}{我觉得没有人会喜欢被分析吧。因为即使分析对了,自己也会有一种感觉,就是自己变成了一道题目,或者是一本作业本,就是只是一个供他人分析的对象,就是变成了一个死物。可以说是被异化了。}
\dialogue{Neruthes}{就是这个和我的个人想法会有些区别。我觉得理解我自己,这个是一件浩大的工程。这个过程当中,我不排斥其他人的参与。就是说别人的参与的形式可能会是我不喜欢的,或者参与的过程当中,在隐私之类的这些层面上,会有一些我不欢迎的点,那么那种时候,我当然会告诉他。在某个具体的问题我不怎么谈,会以合理的形式去规定他参与的边界。但是在这个边界之外,我仍然是欢迎参与。理解我自己这种浩大的工程,他凭我自己一个人的智力,可能是不够用的。这种工程当中,如果有他人参与进来,提供一些有效的帮助,是会相当欢迎、相当支持的。\\
比如刚刚说的那个喂饭问题。这个在我的记忆中是相当明确的,但是如果没有今天的这个讨论,把它重新给拎出来,我可能不会用这种新的视角再去重新解读这件事情、这种经历,对于人的长期的后续成年后的互动角色偏好可能会有影响。今天的这种讨论就是给我增添了一个新的视角。(……)毕竟他人是自己的镜子,不以他人为镜子是看不清自己的。那么这种镜子里怎样的形式发挥作用?比如说今晚这种形式是一种很好的形式,当然也可以有其他的形式。\\
(……)\\
还有一个事情是,在我的心态中,我会有比较多的,给未来的史学家留研究材料的这种心态。其实我知道未来的史学家们很大概率不会来研究我,我在史册上未必会留下什么大的名字,值得一堆史学家的研究。就算是一种自我督促,还是一种历史责任感?反正就可以留的史料就多留点。(……)我觉得这也是一种很好的给后人做贡献的形式。就从更高的抽象层次上来说,我觉得这可以说是我的人生经历,它并不只属于我。我的人生经历是指我这具身体留在这个时代上的一条痕迹。我们把这条痕迹被更多的人、被将来的人观测到。它不只属于我,它还属于这个时代,它还属于全人类。}
}

\section*{聊天,解释权}

\dialoguelist{Neruthes}{
\dialogue{Neruthes}{有两个问题想展开聊一下。第一个是关于自己的历史作为被研究的材料这种事情,这个心态的差异是怎样产生的?为什么会有这两种不同的心态?(……)另外就是这种视频通话的形式的转变,对你来说是否有着属于你的某种心路历程?来解释这种今天互动形式的这种变化。事实是,这是怎样的因素推动你来选择用一种与以往不同的互动形式来进行一些互动。这是我比较感兴趣的地方。这两个问题,我按照怎样的先后顺序来研究一下?}
\dialogue{August}{你说第二点的时候我可能已经没有这么记得第一点了,我先回答第二点吧。然后你可以再回到第一点。(关于)第二点,其实是你看见我公众号里面也有说可以聊天,然后记录文字稿,其实这种方式是我一直蛮喜欢与另一个人就这种即时面对面的互动,而不是文字。因为文字的话我会考虑到,因为我一开始提的时候是因为我意识到我那时候在工作,然后我会没有那么多的精力分出来去全神贯注地和另一个人聊天,而且要理解你的文字里面那种逻辑,我觉得有的时候我是抽不出精力去理解、去看透背后你可能真正想要说的是什么,所以我觉得可以在一段有限的时间里面集中地去投入到这件事情里面,而且文字稿也可以帮助我们记录下这个过程。这个形式是一直很喜欢,而且和你聊天的时候发现这个话题好像你想展开地说。一个过程,就是你想延伸出来聊更多的时候,我觉得好像可以提供这种形式,看你会不会想参与进来。然后你说前一个点是什么?}
\dialogue{Neruthes}{我们这两种对于各自的个人经历的处理态度的差异,是什么让我们走向了这两种不同的态度?}
\dialogue{August}{我想每个人都有应对自己周围环境的办法。可能你的环境和我的环境也不一样\pozhehao{}就是环境上会有差异的地方,另一方面是每个人的创造性适应\pozhehao{}就是每个人如何选择去适应这个环境,也有很大的不同。就算是同一道题,每个人都会有不同的解法。}
\dialogue{Neruthes}{好抽象,可是如果能具体一点的话……}
\dialogue{August}{心的心理本来就很难去,比如说归因、假设、归类,因为人的心里总能想出各种各样的办法、各种各样的差异。}

\dialoguesepline{Neruthes}{\threestars}
\dialogue{Neruthes}{你在对待自己的个人经历的事情上,关于你不行为什么你会不喜欢他人来把你当一道题做,那么这种心态我们把它扩大化,把它去升华一下,那是不是可以这么来思考这个问题:可能是你会觉得你的个人经历这种历史它是某种值得需要被你独占的东西。你不希望这个经历的解释权由他人来染指。}
\dialogue{August}{更像是一种占有感吧。就是一些很个人的意识想法,思想也好,不会想被他人所改变。}
\dialogue{Neruthes}{这个事情跟我的理解好像是稍微体现这些差异,我所关注的点是分析、解释、探究的这样的一种去知晓、去思考的过程,但是你刚刚说到的改变。这个好像和我描述的那种对你经历的研究的那种行为,好像是不太吻合。}
\dialogue{August}{但你在推论,你在组织的过程当中,同样也是对你手中的素材进行了改变。}
\dialogue{Neruthes}{所以你说的是在回忆中,塑造历史,塑造正确的集体记忆,这种意义上的这种改变什么?在每次回忆的时候,作为记忆的历史都会被改变。}
\dialogue{August}{会被改变,但会是一些自己想要接受的改变,而不是对方胡乱弄成一杂,但是自己并不那么认为。而是自己也能够接受或是能够认同的。决定权是不是在自己手里。}
\dialogue{Neruthes}{我觉得这里还有一个微妙的差异问题是(……)那么别人对你的历史的这种探究、分析、解释,这种行为在你看来这是一个值得被劝阻的行为,而不仅仅是你不喜欢的结论需要被排除掉的情况?}
\dialogue{August}{我觉得对方可以做他想做的事情,但是我不一定会认为这就是我会认同的。就是对方可以说把我的素材也好,把我的文字也好,各种东西也好去分析、去推论、去研究,但是我觉得最终的解释权依然是在自己手里。就不会说,我可能写了一些东西、说一些东西,然后有人看完之后:“噢,你就是有抑郁症”“噢,你就是精神病”。最终的决定权还是我,就是说我觉得我自己不是这样的,不是对方所认为的那样,不是对方是推论的那样,那我就不是这样子,而不是会因为对方的一个推论或者是对方自己创造出来的东西盖住了我本身拥有的东西。}
\dialogue{Neruthes}{所以这个表现的形式就是关于对解释权的垄断、的独占这样的一种想法。再往下深挖一点,是不是就可以理解成不喜欢自己的独特性被抹杀。}
\dialogue{August}{可以这么说,不喜欢变成了那一种他人可以任意归类、任意分类的那种抹杀、抹去。}
\dialogue{Neruthes}{你有没有其它的关于独特性被抹杀这种,能靠近这个模板里的经历?}
\dialogue{August}{比如说学生的时候,好像就只是分数而已,每个人都可以被归类进多少分。然后最近其实也有一些课程里面的作业,就是助教会以评优来作为一种分类,就是(作业是)优秀的还是不优秀的。就好像没有被评为优秀,就是无论自己写的东西有多么好或多么坏,但是一个优秀和一个不优秀,就好像将这一切都抹去了。}
\dialogue{Neruthes}{(这是)一些比较正常的程度的,是正常生活中往往会有的程度。它没有显得很突出的。所以我这个推测的思路可能也就在这里就告一段落了。这个东西还是很难往下再扩大化。}
}

\section*{梦境,没有多少喜悦感}

\dialoguelist{Neruthes}{
\dialogue{Neruthes}{还有一个点,就是关于这个成为自己的超人。这篇文章中提到了:“总是一个人面对任何事情、所有事情”,这是一个非常值得庆祝的经历。战胜了困难。打倒了楼梯间,获得力量,去成为更好的人,去成为不再需要被拯救的人。这种喜悦感,我在你的文本里面并没有捕捉到。会不会你对战胜那份漆黑、去打倒这个楼梯间的在这个过程,你有着不太一样的解读。}
\dialogue{August}{因为好像我并没有打倒什么。?因为那些梦境里面的意象其实都是一些内心世界的一种现实,其实都是自己的一部分,只不过是自己从一个现实到另一个现实。\\
然后那种喜悦感,但是后来其实这种情况越来越常见,就是好像我会经常经历了一件事情又经历了一件事情,然后从中收获了什么。就是那种喜悦感,开始在我的自我探索、自我历程里面已经习惯了,所以我没有表露出太多的喜悦感。}
\dialogue{Neruthes}{哎呀,听起来感觉有点困难,不是很懂。}
\dialogue{August}{就是这种经历,其实对我来说已经不是第一次。这种克服了什么,或者是走出了什么,这种经历对我来说真的不是第一次,已经很常见,常见得我好像喜悦感已经很少很少很少。只是知道,自己又做到了。}
\dialogue{Neruthes}{都做到了。已经不是的是什么值得特别庆贺开party的事情。}
\dialogue{August}{以前会,很久以前。}
\dialogue{Neruthes}{然后,这一段的最后一句话:“也许我不想再一个人了”。某种意义上,这是在尝试收回一开始的关于亲密关系的那个讨论的伏笔。但是,好像不完全是这么一回事。我在这段文本中所体会到的那种情感是:打倒了一个怪兽,唉,怎么这次还是我独自打倒的,下次换别的怪物的时候,能不能多一个人来陪我呀?我所读到的是这样的一种感觉。这是你想表达的吗?}
\dialogue{August}{我想表达的是不是什么都需要自己一个人去做,或是独自去面对,但是不完全是打倒什么东西,而是不完全需要自己去完成某一个过程。比如说可能看见什么好看的电影,可以不用一个人去看。有点类似于一种经历,不是每次都需要独自一人去经历,也许可以有另一个人在。}

\dialoguesepline{Neruthes}{\threestars}
\dialogue{Neruthes}{这(看电影)是一种欣赏型的。在预期中就会有喜悦的活动,而从黑暗楼梯间中突破,或者说去面对是否能够突破它的这种挑战,这是一种危险,或许还会带着恐惧感。完成之后才能够获得那种肾上腺素刺激感的不确定性。在这种活动当中能够收获到怎样的心情,是有很大不确定性的。(……)所以听你的描述,就感觉从黑暗楼梯间当中走出来的这个过程,它像是某种命中注定的、最终必然要发生的。所以,它不太值得喜悦,或者赋予太多的感情色彩。}
\dialogue{August}{可能因为这个事情我已经回顾了很久。可能小时候最终梦到走出来的那一刻会很开心,但是那个时刻距离现在已经很多很多年了。}
\dialogue{Neruthes}{也会被冲淡很多情感。}
\dialogue{August}{嗯,有可能会。}
}
