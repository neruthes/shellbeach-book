\chapter{自我探索 | 18}

\ardate{2022-01-19}{SP0KCnFrNZtJMg5\_1nSA6Q}


\blockquote{
昨天,在和课程同学进行倾诉练习时,我说到了在上周末去另一个城市的社区街道里我所感受到的宁静感,我想知道那份宁静感是什么。在倾诉练习结束后的互相反馈环节,对方说TA在一开始不敢打扰我,因为那个场景对我来说好像很private.   同时TA说TA在一开始感受到了我的vulnerability,这会让TA不敢靠得太近,但TA在整个倾听过程中都感觉到我好像不想TA靠得太近,但又不要离我太远。

\blockquotesource{自我探索 | 17}{白色灯塔先生}{2022}
}

今天又一个人告诉我,她在我身上也会感受到这种感觉\pozhehao{}“我好像不想TA靠得太近,但又不要离我太远”。这让我更加好奇:为什么。

\tristarsepline

我想起之前和一个男生视频聊天时,他一直在说自己的事情。每当我试图回应或产生互动时,我都无法打断他\pozhehao{}不是说他会一直说下去,而是在每次的短暂沉默后,当我打算回应时,他都会继续说下去,就像一个卡顿的视频。后来我意识到这并不只是出现在视频聊天,他在面对面交流和直播时都是这样的。我知道他在沉默的时候是在继续思考,但我有一种感觉:即使他人在他身边或他正对面,他都只是在独自一人地思考,干着自己一个人的事情。

我想到我在咨询室里的大多数时候也是这样:沉浸于内心世界的自我探索当中,但我会时不时地留意咨询师是否有想要表达的内容,时不时地从内心世界回到咨询室。我在想,当我沉浸于内心世界的自我探索当中(无论是在咨询室、倾诉练习还是文字),是否正是这种沉浸让对方感到不敢打扰,感到“我好像不想TA靠得太近,但又不要离我太远”。

\blockquote{
“不想TA靠得太近,但又不要离我太远”这句话也会让我脑海里有一个画面:一个很怕他人的靠近,但也很怕他人的离开的小孩。如果从依恋类型维度的角度来看,那可能会是高回避亲密、高忧虑被弃。

但我知道在现实生活里的我并不是高回避亲密、高忧虑被弃。我和我的咨询师以及身边某些朋友正在建立或保持着深入的人际关系;我也不担心被弃,因为我并不拥有任何人,也并不属于任何人\pozhehao{}如果从未拥有,又怎能称之为遗弃。所以我在想,那个高回避亲密、高忧虑被弃的部分,会不会是我内心深处的自我,那个更偏向于小孩子的自我,那个既害怕被打又害怕被赶出家门的那个小时候的我。也许当在自我探索、在进入内心深处的过程中,我在一定程度上成为那个更弱小的自我,那个能够感知到各种各样的丰富情感的自我。我记得那个自我正是我在大学时期一直试图找回、试图挖出的。而现在我好像不只是找回了充满丰富情感的那部自我,而更是能够在特定的范围里(比如说在写作或倾诉或咨询时)主动地成为那部分自我。

即使你找到了你所爱的人,找到了爱与幸福,你仍然是你\pozhehao{}那个独自站在黑暗里受惊的小男孩,因为对每一段亲密的关系都感到畏惧,从而把所有真正了解他的人都排斥到了尽可能远的地方。那个小男孩该怎样才能变得幸福?

你讨厌他,因为他为了自己的安全感而放弃了所有可能会得到幸福的机会。但你却无法摆脱他,因为在你内心深处,他就是你,他是那最深处的、最纯洁的你。他是你偶尔仍然能够感受到幸福的唯一原因。

大多数时候,你感受不到任何感觉。你无法感受到幸福、悲伤、渴望、羞耻、孤独。但时不时那个小男孩会走出来,而你就感觉到了一切,虽然说大多的是孤独与悲伤。你试图把他推回去,但你又不能这样做。因为你已经谋杀了那个小男孩的大部分存在,你谋杀了他的单纯、他的欢乐、他那能够感受感觉的能力。在你谋杀他的同时,你也谋杀了自己。这也就是为什么现在的你无法感受到任何感觉。

\blockquotesource{随笔 | Crawling in Vain(徒然地匍匐前进)}{白色灯塔先生}{2018}
}

\useimg{aimg/2022-0119-1.jpg}

\tristarsepline

我会感到很开心,因为那个自己一直想要拯救的那部分自我、那个我想要亲近的自我,原来我在某种程度上早已经成为了“他”。我不再是通过写作或咨询的文字或言语表达将自己内心最脆弱的部分、将那个“他”带出来,而是成为了“他”。

虽然我不知道那个“他”在将来会不会变得没那么回避亲密或忧虑被弃,因为毕竟那是小时候更本能的、更“根深蒂固”的自我,但我不会想刻意改变“他”。就这样就好。

\tristarsepline

在昨晚冥想时,当试图想象那片蓝天下的平原,我发现自己所在的地面开始下沉、下落。我落入了黑暗,周围有着很多繁星,而我的身体也在发光。我一直下落,落入黑暗的更深处,但看不见更深处有些什么,也不知道自己为什么要下落。我在问自己,那些繁星是什么,那片黑暗又是什么?那些繁星像是生活里人来人往的他人,而那片黑暗像是存在虚无。然后我继续问自己,不断下落的我在寻找些什么吗?对此,我找不到答案。

“不想TA靠得太近,但又不要离我太远”这个描述像是在昨晚冥想时那个不断落入黑暗深处的过程:周围有很多繁星,有的星离我很近,有的离我很远,但它们都匆匆而过,或者说是我匆匆而过,在繁星之间穿梭着、下落着。

我会在想以前的我的处境是怎样的?和前任在一起时,我的世界几乎只有他的存在,那颗最耀眼的“星”。和前任关系破灭后,我陷入了抑郁、无意义和虚无的“黑暗”里,身边没有任何一个人能帮助我、为我而停驻、陪我呆在当下、拯救我。那更像是一片孤独一人、没有任何光亮的黑暗。现在,那片存在之虚无依然存在着,但在这片黑暗里,我好像更能看清其中的繁星\pozhehao{}那些和我一样身处于这片存在之虚无的黑暗里的他人。

我没有让任何一颗星接近太久是因为,无论那颗星有多近、有多耀眼,它身后的那片存在之虚无的黑暗依然在那,我终究还是要“面对”那片黑暗, 而 且 我 还 在 不 断 下 落。那片繁星没有离自己太远是因为,他们本来就存在于自己身边\pozhehao{}他们就和我一样,都是这片黑暗里的一颗颗星。与其说我没有让自己离他们太远,也可以说他们也没有让我离他们太远。

不过,在一直往下落时,我感觉周围的繁星似乎只是停留于原地,而我则是在不断下落。至于会下落到哪里,我还不知道。
