\chapter{白色灯塔先生的情感笔记 | 18}

\ardate{2022-07-29}{tbdqxGvqJTt9yNSCfKu-EQ}




% \subsection*{对作业可能被攻击的恐惧感}

% 然后我试图去觉察自己的感受,并对冬裔说:“在你问‘看看你的观点’时,我会很抵触,因为我不想去触碰那个‘伤口’。我会怕被攻击,因为我的成长经历里很多这样的被攻击的经历\pozhehao{}自己的爱好、感受和想法被否定、被打压和被抹去。我不希望又多一个人(你)来判断我的‘对错’。我可能刚刚有点过度防御了,就好像助教没有评优一样,会很容易将这些信息和以前的经历带来的感受联系在一起。我可能也很不相信一些没有看见我的情感的人,会极度地不信任对方,因为我不知道对方会不会突然来攻击我。”

% \blockquotesource{随笔 | 街区宁静感,停驻,畏惧}{白色灯塔先生}{2022-01-17}















\subsection*{来自他人的部分的硬塞感}

但另一方面,那种自我攻击背后又会是些什么?我回想起之前在咨询室里描述的那个画面:把自己按下水面地去直视自己所恐惧的事物。那种自我攻击就像是把自己按下水面的力量,只不过这种自我攻击是想让我符合自己过往的那一部分自我认知。这种想要让自己符合自我认知的背后又会是什么?我想这可能是想要保护自己\pozhehao{}如果我确实是自己所一如既往地糟糕,那么how worse could it be? 但如果我并不是自己以前所一直设想的那么糟糕,而是周围各种各样的人所硬塞给我的认知、评价、反馈、批判、否定,那么我需要面对的则是周围的那些他人,那些比自己糟糕得多的他人,需要承担反抗他人对自己内心世界的入侵的责任。这似乎也是为什么在那个当下我会感觉到那种被挤感\pozhehao{}反馈里有属于他人的部分试图挤入我的内心世界,试图让我内化本属于他们的部分。

\blockquotesource{自我探索 | 17}{白色灯塔先生}{2022-01-18}






\subsection*{对所谓的“照顾”的回避感}

最后,我留意到被照顾感会让我想要回避对方。这种抵触感来源于我妈和外婆在我的成长历程里的“照顾”,但那种“照顾”并不是真正的照顾,而更像是一种借照顾为由的操控,一种否定我的能力、认知、感知、想法、情感甚至是存在本身的操控。这样的“照顾”并不是我想要的,而是我被施加的。我想要的照顾、我需要的照顾仅仅是一个拥抱或是一场倾听便足够了。

而且,我会认为这种“照顾”背后还有着一份不信任:不信任我有能力解决我自己的问题,不信任我能处理我自己的情感,不信任我能照顾好自己。这种不信任也在一定程度上没有看见我有能力的部分在,甚至一定程度抹去了我有能力的部分。对方眼中的那个需要“照顾”的、没有能力的“我”,并不是我。

\blockquotesource{自我探索 | 17}{白色灯塔先生}{2022-01-18}






\subsection*{对马后炮的憎恨}

当我将想换平板的想法告诉一个朋友后, 他一如既往地给我想好了接下来的流程:去回收旧平板—拿回收的钱加多一点点钱就能换一部新的—哪里有回收店\pozhehao{}新平板买哪个型号比较好。一开始我是蛮兴奋和充满期待的,直到我发现旧平板的回收价只有三百块时,他说:“所以我都是用了半年至一年就马上换新的,就是为了防止你这种情况”时,我的第一反应不是想砸东西,而是想用东西砸他。

\blockquotesource{随笔 | 换平板,老中医,“替”}{白色灯塔先生}{2022-01-20}






\subsection*{“那并不是我”的非我感觉}

同时我也在作业区里看见有同学的5年、10年、30年之后的计划(早已远超于描述理想自我的范畴)很宏伟很宏伟。当我下意识地把自己和对方进行比较,并开始感到自卑感时,同样有另一种感觉冒了出来,那种感觉好像在说:那并不是我,那不完全是我想要的,我想要的是在这个当下成为我自己。

……我不想去投射一个理想自我的自我意象,我只想在现在这个此时此刻、这个当下成为我自己,这个我一直以来、每时每刻都在用心培养的自我。

\blockquotesource{“理想中的自己”就是现在的样子}{白色灯塔先生}{2022-02-16}






\subsection*{对自我表达的压抑感}

小时候的我(大概是上幼儿园和小学时期)很健谈,基本上是想到什么就说什么,所以会不经思考地说出一些有意或无意伤人的话,不过那些话的具体内容我已经想不起来了。家里人和亲戚当时对我的教养方式是辱骂和家暴,所以后来的我(现在也是)在他们面前越来越寡言,不会在他们面前说什么话,也不会透露多少关于自己生活的什么信息。

在踏上自我探索历程的路上,我开始发现压抑着自我表达的那份力量,是在保护着我自己,保护自己不要因为说错话而遭到言语和肉体上的攻击(虽然童年的经历早已过去了十几年)。这种保护的力量一直都在,而这种力量在意识层面上的感觉则是一种自我表达的自我压抑感。这种自我压抑的感觉并不好受,但无疑是十分适用的,适用于我的生存。但当我可以意识到那份自我表达的自我压抑感背后是一种试图自我保护的意图后,我便可以从以前的无意识自我保护转为现在的有意识自我保护\pozhehao{}有意识地保护起自己,不和家里人和亲戚建立起人际联系,不透露不必要的信息。

\blockquotesource{羞愧,“内向”}{白色灯塔先生}{2022-02-17}






\subsection*{对逃离的渴望}

我会想起大四在书店实习时,有上司在的时候我极度不喜欢待在店里,所以就会经常去洗手间,找个隔间刷手机。但那里的洗手间很破旧,灯光也很暗,还有一股挥之不去的臭味(差不多是恐怖电影的氛围)。因此,后来我就去了洗手间旁边的消防楼梯通道里,坐在台阶上刷手机。我记得那时候我还在玩一个农场养成类的游戏,通常玩个15分钟才回去店里。

所以现在有时候我去商场洗手间后尝试去探新路线时,我会不时在几乎不会有人经过的楼梯间里看见不同的餐饮业店员坐在台阶上刷手机、吸烟,甚至是睡觉。我想我可能在某种程度上能理解为什么他们想在一个off-the-gird的角落呆着里,因为在剩余的时间里他们被迫要呆在一个他们只想逃离的地方。

\blockquotesource{零碎的想法 | 33}{白色灯塔先生}{2022-02-18}






\subsection*{对没有付出更多的陪伴的后悔感}

我会想起我的朋友琥珀,他离开这个城市回家生活也快一年了吧。虽然他说他被诊断为双相障碍,但和他相处时,我看到的几乎只有他抑郁的一面,而没有狂躁的一面。在每次见面时,他会想我多陪陪他,比如说在他的住处过个夜或待久一点,不过那时候我因为过往的一些不好的经历而本能性地排斥任何试图“粘着我”的人,所以那时候我并没有陪他久一点。

当他离开了这个城市后,我一直感到蛮后悔的,后悔于自己没有在他需要我的陪伴时多陪一陪他,也后悔自己因对过往经历的抗拒而没有从他那获得更多的陪伴,而现在他也早已不在这座城市生活了。

\blockquotesource{零碎的想法 | 33}{白色灯塔先生}{2022-02-18}






\subsection*{对不知道下一次见面是何时的难过感}

见面前,我的想法是想知道他无故消失的原因。见面时,他说他是因为太忙了才在这一年多里没有回我信息。(我信你个鬼!)我还问了他他觉得我们分手的原因是什么。我记得他当时的回答是我只是想找个人依赖,“找另一个你能依赖的人不就好了”。

见完之后,我感到更加难过,对又一次的分离更加不舍,因为不知道下一次再见会是什么时候,甚至不知道是否还有下一次的见面。只好继续一人独自面对未来的未知,面对那个曾经唯一的依靠、现在唯一的依靠消失在现在的茫茫人海里,消失在过去的回忆里。

\blockquotesource{零碎的想法 | 33}{白色灯塔先生}{2022-02-18}






\subsection*{“一切都是错的”的违和感}

我想起一开始用社交软件时(大概是在16\~17岁的时候),有两个男生同时喜欢我。我后来跟他们说我会选其中一个在一起,后来也确实在一起了,但继续相处后觉得那个男生很少回我信息,并给我一种感觉:我并不是他生活里重要的人。所以后来我提出了分开。

我想那时候的我甚至连喜欢都没有喜欢上对方,而只是因为有这样的选项、这样的人就去试错了,更像是以约炮的态度进入一段情侣关系。

在选择和谁在一起时,那时候的我会比较两人的各种特点和优势,比如说能腾出来的时间、金钱、身高、身材、性格等方面,一些最为表面、标签化的东西。即使在这两个被我所量化的人里选出了更优的那一个,但这依然不是自己所想要的人。和自己不喜欢的人做一些自己设想会感到很亲密但实际上并没有亲密感的事情(例如牵手)时,那种违和感,那种“一切都是错的”的感觉。

\blockquotesource{零碎的想法 | 33}{白色灯塔先生}{2022-02-18}






\subsection*{对“足够好的父母”的幻灭和渴望}

所以当读到“愿意放开你的父母”时,我会感到蛮伤心的,因为这意味着detachment,与自己的现实父母进行分离,同时接受自己的现实父母永远都不足够好的这一一直以来的现实,以及敞开自己地去感受自己对此感到的悲伤,哀悼对足够好的父母的幻想的丧失,放弃对足够好的父母、对一个家的期望和渴望。不过这份渴望依然还会在,只不过它不再指向现实生活中的任何一个人、任何一处地方\pozhehao{}no place as home. 相反,它开始指向内在小孩、内在父母、梦境里的地方等内在(资源)的部分,一些能够永存于自己的内在世界的部分,而不是处于人来人往、世事变迁的外在世界的地方。

\blockquotesource{零碎的想法 | 33}{白色灯塔先生}{2022-02-18}






\subsection*{对内容匮乏的活动和人的厌倦感}

我们在公交车总站的一间书店里约看书。我在之前的见面里有提到说我在大四实习时一直想象哪天能到这家书店坐在窗边看书。然后我问他:“你之所以会选择来这家书店,是因为我之前提过我有想象在这里看书的样子吗?”他说是的,然后问我现在什么感受。我看着外面的大雨天和店里的摆设,我说:“在我之前来的时候,这里是下午,阳光很好,这里(我用手指着店内的一片空地)摆着一个堆满书甚至看似随时要倒下来但还没倒的书桌,书柜里的书都挤得难以拿出任何一本。但现在,今天刚好是阴雨天,也许之后(再来)会有阳光。不过这里更像是一个卖书的超市,而不像是个书店。这里的书都摆放得很整齐、很工整,就像是超市里的产品。”他说:“所以这和你的想象有差距?”我说:“何止是差距……”他说:“刚刚在坐下来的时候我感觉到了你的情感好像有点微妙”。我沉默了一会儿,然后说:“其实我觉得现在并不足够,看书并不足够。我之前也有和朋友约看书,结果是大家除了看书就没了,没有多少聊天。”他说:“所以你想要的是聊天?”我回应了声嗯。他说:“那好像约看书和你想要聊天两者是不一致的。”我说:“是啊。”

不过在回顾的时候,我会想到:喝茶本身也不是单纯的喝茶,约过夜本身也不是单纯的过夜,去心理咨询也不是单纯的心理咨询本身,那为什么约看书会变成了单纯的看书、约喝东西本身变成了单纯的喝东西、约看电影本身变成了单纯的看电影、约sex变成了单纯的have sex?本是为了让更多事情的发生和展开而提供框架的活动,却变成了框架本身,而内容则空缺了出来。

但我想,匮乏的不是活动框架本身,而是参与活动的人。

\blockquotesource{随笔 | 周六的“雨水”}{白色灯塔先生}{2022-02-20}






\subsection*{被重要他人所遗弃的遗弃感}

在通勤路上,我依然感到很难受。在巴士还在等交通灯时,我想到可以利用这一短暂但能够独处的时间。我闭上双眼,感受那种想哭的感觉,试图回溯这种感觉的道路,试图回忆过去有关这种感觉的场景\pozhehao{}我能回想起去年和落叶相处的画面;回想起前任的无故消失、初恋的无故消失,一次又一次重回故地想要追寻曾经的身影、过去的回忆;回想到小时候被打时坐在地上哭;回想到小时候被赶出家门时边拍打着家门铁门,边坐在台阶上哭。

被遗弃意味着死亡、孤独、无意义、虚无、不值得活下去、一个人在黑暗里,一个人在无尽的漆黑楼梯间里,无论往上走还是往下走都走不到尽头,永远在黑暗里徘徊,there's nothing left and no one left, there's no way out.

然后不知道为什么,我突然睁开了双眼,好像自己从过去很遥远的地方突然被拉回了现实,被拉回了现在的此时此刻。我内心在想:噢,原来是这样的,那种被遗弃感\pozhehao{}被家里人“遗弃”、被初恋“遗弃”、被前任“遗弃”、被落叶“遗弃”,然后现在和最近认识的那个男生的相处好像也激发了我的那份被遗弃感。这种“遗弃”也并不是客观世界里的遗弃,而是主观世界里感受到的“遗弃”。当能在理智上看见那种感觉是一种被遗弃感后,我反而不再感觉到那种感觉了,反倒是,我感觉自己的心脏有一种被勒住的感觉\pozhehao{}躯体上的感觉,而不是心理上的。可能是心理上的情感太难以承受,转为了由躯体来承受吧。

\blockquotesource{被遗弃感,无尽楼梯间,脆弱与全能}{白色灯塔先生}{2022-02-22}






\subsection*{总需要一个人去面对糟糕的事情的孤独感、无助感、无力感、被遗弃感、绝望感}

在这个重复了无数次的梦境的最后,并没有另一个人将自己带出那个漆黑的无尽楼梯间,也没有因为我找到了另一个人而离开了那个楼梯间,最终还是要靠自己一个人走出去,还是要靠自己一个人走过草丛里可能存在的未知怪物。最终还是要靠自己一个人去“救赎”自己,去“充分摆脱”那个漆黑的无尽楼梯间,一个人去面对小时候的糟糕的事情,面对和初恋的分手,和前任的分手,面对生活的变迁。总是一个人面对任何事情、所有事情。也许我不想再一个人了。

\blockquotesource{被遗弃感,无尽楼梯间,脆弱与全能}{白色灯塔先生}{2022-02-22}






\subsection*{对亲密感的抵触感}

听完咨询师对此的猜疑后,我感受到了一种亲密感,以及在此之后感受到的抵触感。我向咨询师表露了自己的感受变化后,咨询师问:“你能多说一说那种抵触感吗?”我继续说:“这种抵触亲密的感觉会让我想起我和我妈的互动。因为小时候经常被我妈打,所以现在当她试图触碰我(即使是隔着衣服地触碰我的肢体),我也会很本能地弹开。而且我也很反感有陌生人去触碰我的身体。就好像那种抵触感存入了自己的肌肤里。咨询师说:“确实,当你被打时,是你的皮肤承受了疼痛。”我点了点头。

\blockquotesource{Life still goes on, I still go on.}{白色灯塔先生}{2022-02-28}






\subsection*{对自己的无趣感}

咨询师继续问我关于这种抵触感是否还能想到更多吗?我说我好像不能想到更多的事情了。在沉默了一下后,我说:“我在想其他人会不会不那么无趣,其他人说不定能联想到一些其他的回忆、一些故事的展开。但我好像只是把它(抵触感)看成是情感本身。情感出现了,就看着它出现,然后它说不定什么时候就会消失。”

咨询师回应说:“好像你会想知道其他来访者会不会更有趣。其实你是不是在担心我会感到无趣?”我马上回答说:“不是的。我不是在担心你会不会觉得我无趣,而是我好像已经有一个内心形象在认定自己是无趣的。这可能更像是一种自责的声音吧,在说‘为什么自己那么无趣’。我想很多人都有自责的这一部分。那个自责的声音还会说:‘如果自己在读大学时就知道自己想学的东西是什么、想走的方向是什么,那该有多好。’但又会出现另一个声音:‘自己不可能突然就找到自己想做的事情。这是需要过程的,我走到现在这一步是需要过程。’”

\blockquotesource{Life still goes on, I still go on.}{白色灯塔先生}{2022-02-28}






\subsection*{对协调者的自我认同感以及与之有关的更深层的迷茫感}

咨询师联想到:“好像之前你在我们的咨询里也会承担协调者的身份,比如说当你发现我们在走的方向不一致时,你会去协调我们两个人走的方向。”我回顾了一下那次的咨询,回答说:“嗯,确实是。那时候的协调者也像是站在第三者的角度去看彼此冲突或者说分歧。”在沉默地思考了一会儿后,我继续说:“我也会想到,如果我去不承担那个协调者的角色的话,那我又能成为谁,我又能站在哪里?如果是以前的我的话,我好像就只会被卷入冲突里。这一协调者的角色好像更像是自我认知、自我认同的一部分。如果没有了它,我好像会更加迷茫。……嗯。我想,比起看着自己面前的内心世界里有着不同欲求的自我的相互冲突,以及面对人际关系里的冲突,会让我更加感到更加迷茫的是,如果我不成为一个站在第三者角度的协调者的话,那我又能是谁?这是一种更深层、更弥漫的迷茫。”

\blockquotesource{Life still goes on, I still go on.}{白色灯塔先生}{2022-02-28}






\subsection*{身处困境的窒息感以及被包裹感}

咨询师回应说:“好像在接触到那种感觉后,你又会理性地去分析这种情感。”我笑了笑说:“嗯!好像当自己开始陷入那种情感时,我的理智化就会开始无意识地工作,试图将自己从那种情感里捞起来。我并不是有意识地去理智化这整个过程\pozhehao{}去将过去的感觉和现在的感觉连接起来,来减缓现在的情感的强度。但自己确实无意识地这么做了。我想可能是因为这个过程自己已经做过很多次了:去感受情感,去追溯情感的起源,去理解情感。”

咨询师回应说:“刚刚你会用‘捞起来’这个词,我在想,如果真的沉了下去呢?那会是一种怎样的状态?溺水?”我感受了下,回答说:“会是一种窒息感。我会想起读大学时自己很想去自杀,其中一部分原因是因为自己每一口呼吸都感觉自己没有在吸入氧气。我一直在呼吸,但吸入的每一口气里的氧气都不足够,一直处于缺氧的状态。怎么呼吸都吸不够足够的氧气,每次呼吸都很痛苦,很难受,很想结束这一切。”咨询师回应道:“这听起来真的很痛苦。”(但我怀疑,咨询师很可能并没有类似的体验,可能并不能切身感受到我的感受,而只是说说而已。)我继续说:“我想这种窒息感是因为自己在心理上困境感在身理上体现为了窒息感。”咨询师回应道:“那好像真的是一种很不舒服的感觉,才会想把自己‘捞起来’。”(我留意到咨询师在用词上把“痛苦”转用了一个程度更低的词“很不舒服”,可能是想让我减轻陷入情感的程度吧。因为我记得之前学的一个边缘性人格障碍课程的关于心智化的内容里有提到这一技巧)。我继续说:“虽然一开始进入到那种情感会不舒服,但之后就会完全沉浸于其中,会觉得舒服,因为那种困境感、窒息感是自己所熟悉的事物,会有一份安全感,就像是一份拥抱,像是一层膜将自己包裹起来,可能是用于保护自己当时所处的困境吧。”咨询师简短地回答了:“嗯。”

\blockquotesource{Life still goes on, I still go on.}{白色灯塔先生}{2022-02-28}






\subsection*{对关系不能继续深入下去的无力感、悲伤、窒息感、失败感、受挫感}

晚上,我们找了另一处江边走。在坐在江边的石凳上聊天时,他说出于一些他的个人原因,他不能和我有更深入的关系了,只能止步于此,最多只能停留在现在的亲密程度。我尊重他的选择,并为彼此的关系不能继续深入下去,不能深入到我想要的程度,不能达到我想要的事物而感到悲伤。这应该是我第三次有这样的经历了,或者说是第三次被激起和强化这种感觉,这种无力感、悲伤、窒息感、失败感、受挫感。

\blockquotesource{Life still goes on, I still go on.}{白色灯塔先生}{2022-02-28}






\subsection*{感觉自己仅仅是在输入输出的机器感}

但在大学后,特别是在学习心理学后,我的写作似乎一直停留于外显心智化的层面,停留在文字的表达里,而没有像大学时试图将一些未被意识到的事物通过文字以外显出来。

这也会给我一种感觉:感觉我自己像是一个机器。我不断地接收各种输入,并在经过自己的处理后进行输出,但好像从来没有一些自发的输出。就好像,如果我没有了输入,那我就不会输出了。如果没有了外显心智化的事物的刺激,那我就不会外显化自己的内隐心智化。

\blockquotesource{活着的意义,心智化}{白色灯塔先生}{2022-03-02}






\subsection*{对与女性变得亲密的抵触、愤怒和憎恨感}

在昨天接电话热线前,我感觉到一种很强烈的焦虑,感觉自己控制不住呼吸了,呼吸紧促地快要过度换气了。这种焦虑感来源于之前接热线的经历触发到了我对自己和女性逐渐变得亲密的抗拒。这种抗拒感让我联想到我和我妈的相处,触碰到了我对我妈的那种本能性的排斥和回避。

……我想我之所以会把女性的naked body视为一团肉,是因为小时候和我妈相处时,我也会把我妈的身体看成是一团肉,一团包裹着无尽的恨意的肉。如果没有了那一团肉,里面的那些憎恨和暴力就会统统释放出来,就像是电影《寂静岭》里,那个母亲在教堂里被教主刺伤后,随着体内的黑色血液滴在教堂的地板上,血液里的黑暗和火焰开始吞噬着周围的一切。

在性取向方面,与其说我喜欢男性,倒不如说:比如男性,我更憎恨和排斥女性。这很可能是还在读小学的我无意识地将自己对母亲的憎恨和排斥转移(移情)到了其他女性上。

不过,在之前接热线的那个当下,我并没有感受到以前和我妈相处时的那种对那一团肉所包裹的憎恨和暴力而感到的恐惧、排斥和憎恨。我在那个当下仅仅感受到逐渐增加的亲密感,而焦虑感则是之后才出现的情感。那份焦虑,并不只是焦虑于自己和女性的亲密,不仅仅因为这可能会触及到我对母亲的憎恨和暴怒,还因为一些更深层的恐惧。

\blockquotesource{情感,性取向}{白色灯塔先生}{2022-03-05}






\subsection*{身处内心空间的闲暇和稳定感}

这个鸟鸣会让我有一个画面。我在冥想的时候也会通过白噪音来想象自己身处于一个场景里,将自己锚在其中,看着周围的想法和情感的飘过。我现在的脑海里的画面就是:森林里有片草地,阳光斜着打在草地上。如果我在冥想的话,我可能就会坐在草地上。我觉得这种状态就像是在接热线时,我一边听着对方的倾诉,一边又在自己的内心世界里。就好像我有一部分的自己是在更内在的空间里,另一部分的自己在和外界做各种回应、干着各种事情。当然,在接热线的时候,我有自己的想法和思考,但我依然感觉有一部分的自己是安坐在更内在的空间里,很闲暇、很稳定。

\blockquotesource{感觉躯体不是自己的}{白色灯塔先生}{2022-03-06}






\subsection*{对急于开始自杀危机干预评估的荒谬感和恼怒感}

其实,在回顾整个过程时,我会觉得自己蛮荒谬的,毕竟自己曾经也是曾经有过大概六年的自杀意图甚至是自杀计划的人。如果那时候的我“深处于”自己以前的状态去拨打电话热线,而电话里另一个头的人反而忙于缓解TA自己的焦虑并且忙于问各种问题、忙于评估,我估计会很恼怒。而那种恼怒感也在那次和我的咨询师的互动里存在着\pozhehao{}恼怒于“我感觉你只是在忙于阻止我,而不是理解我”。但这种恼怒感不单单指向着我的咨询师,也指向我自己\pozhehao{}接热线时的自己。

这种恼怒感\pozhehao{}无论是责怪他人也好,还是责怪自己也好\pozhehao{}的背后,都是在说同一句话:“你只是在忙于阻止我,而不是理解我”。

\blockquotesource{“憎恨这个世界和这个世界上的所有人”}{白色灯塔先生}{2022-03-08}






\subsection*{难以承认自己的无能的无力感}

所以有时候,我也会在想,我是否不能接受自己的有限性\pozhehao{}对理解他人的有限性,对理解自己的有限性,而不仅仅是接受热线的有限性。承认自己帮不到对方对我来说会有点难,因为很大程度上,我更难以承认我帮不到我自己,就像之前的那六年里的我并没有向他人求助,也没有找心理咨询求助。

是啊,我很难承认自己是无能的,难以承认连我自己都帮不到自己。因为这对我来说意味着,其他人也不可能帮到我,只有死路一条了,真正的“死”路。所以,比起我没有能力去救电话另一头的来电者,我更害怕的是,我没有能力救我自己。

就像是,如果我不去心智化他人、不去心智化我自己的话,就没有人能做到了。

\blockquotesource{“憎恨这个世界和这个世界上的所有人”}{白色灯塔先生}{2022-03-08}






\subsection*{对温暖的抵触感}

当他把手放在我头发上的时候,我的脖子和他的手腕触碰着,我感觉到温暖,同时也感觉到自己对温暖的抵触感,抵触于他人可能会伤害自己。这种出于自我保护的抵触感也让我更难以向他人求助,只能靠自己去去解决问题、去面对情感。而我之所以会变成现在这个样子,很大程度上也是因为,我曾经很奋力地试图拯救我自己,而且也成功做到了。虽然如此,我的内心深处依然很害怕他人的触碰,特别是当自己很疲惫或脆弱的时候,因为,毕竟是自己做到了,而不是他人做到了。

\blockquotesource{“憎恨这个世界和这个世界上的所有人”}{白色灯塔先生}{2022-03-08}






\subsection*{对自己身处困境时的周围的人的憎恨感}

“根本没有人去帮助那时候的我,那种憎恨感,憎恨这个世界和这个世界上的所有人。如果我不去心智化他人、不去心智化我自己的话,就没有人能做到了;如果我不靠自己的力量去逃脱困境,去拯救我自己,就没有人会帮助我这么做了。这真是个令人感到孤独和绝望的世界。烂人到处都是,无能的人也到处都是。”

\blockquotesource{“憎恨这个世界和这个世界上的所有人”}{白色灯塔先生}{2022-03-08}






\subsection*{不再投注精力的毫无意义感}

好像每到一定时期,自己的大脑就会自动地暂停工作,或者说不再往任何人和任何事物身上投注精力。而当不再投注精力于这个世界时,一切对我而言都变得毫无意义了。

\blockquotesource{“活着没有什么意义,也没有什么能干的”}{白色灯塔先生}{2022-03-11}






\subsection*{投注精力的乏力感}

“一直在用意义来驱赶无意义”\pozhehao{}在用各种各样的课程、书籍、和朋友相聚、和新的人面基等新鲜的事物来创造意义,以此驱赶其他无意义的部分\pozhehao{}无意义的工作、家庭、同事、生活环境。这就好像是,我一直处于一片匮乏的土壤上,而我试图用自己的意义来构建出一个自己能安驻于其中的乐园。当然,我并不是全能的,这个“乐园”也会时不时not working,而我则因此需要去寻找更多新鲜的事物,向这个“乐园”灌注新活力。

这就像是,我不仅仅要照顾自己的身体需求(比如说满足饥饿感和性欲),还要照顾自己的心理需求(比如说对人际关系和新鲜事物的渴望)。而当心理渴望没有得到满足时,此时此刻的我的感受是:我感觉自己快呼吸不过来,有一种窒息感。比起写作的一开始,现在的我更加确信自己想要什么,而那份想要买网课、想要获得新鲜事物的渴望就像是在掐着我,越掐越紧,越来越难呼吸。

\blockquotesource{“活着没有什么意义,也没有什么能干的”}{白色灯塔先生}{2022-03-11}






\subsection*{对未来憧憬的辜负感}

然后我感受到一种伤心,好像那个脆弱的、容易受挫的那部分自我在哭泣着在大学时候的那些渴望都没有被满足、都辜负了,辜负了那时候的我对未来的憧憬……好像现在的我辜负了大学时的我对未来的憧憬。大学时的我总是在教室里看着窗外,或是下课后走在放学路上,看着校园的风景,憧憬着自己的未来能够像这个大学校园一样,充满着新的风景、新的未来、新的事物。结果毕业后,我却选择了回原来的城市生活,选择回到大学前的生活。那时候之所以会回来这座城市,是因为那时候认识了前任,而那时候的我以为自己能会在前任公寓一直生活下去,结果后来还是回了父母家,这个让我厌恶和憎恨的充满着无数痛苦的童年经历的地方。

好像现在的我在那么长的时间以来,一直做着大学时的我所厌恶和憎恨的事情\pozhehao{}回到原来的地方、回到那些糟糕的过去。

想换一个城市、换一个地方生活了,想离开这里的一切。

\blockquotesource{“活着没有什么意义,也没有什么能干的”}{白色灯塔先生}{2022-03-11}






\subsection*{对他人合理化我自己的想法和感受而感到的孤独感与无力感}

而且在留言里,我也看到了一个朋友试图合理化我的想法和感受,比如说:「当一个东西和任何其他东西都不会产生互动,那它存在不存在似乎都没有区别,所以“我不去推动事情发生,一切就都不会发生”,在我看来倒是很自然而然的事情。」当然我知道他的本意可能并不是想合理化我的想法和感受,而可能只是在表达关于他自己的事情。但这依然让我感到很孤独和无力。就像是:“嗯,没有人能懂你的痛苦的。”

\blockquotesource{“你还不如死了算了,生块叉烧都好过生你。”}{白色灯塔先生}{2022-03-12}






\subsection*{对家的温馨感和连接感的向往}

在上周周末,我和朋友Rue和他对象Yang一起约饭。约完饭后,我们回到朋友Rue的住处,他对象Yang给他订了一个生日蛋糕。在我们一起吃蛋糕的时候,Yang坐在地上的垫子上,Rue坐在沙发上,Rue把双腿放在垫子上的Yang的大腿上,然后旁边还有其中一只他们养的猫,在Rue坐上沙发前,那只猫还在沙发上躺着。这一幕激发起了我对家的温馨感的向往,同时也回想起,以前还住在前任公寓时,我在客厅的地板上铺了被子,然后我和前任有一晚就坐在被子上,背靠着沙发,通宵打游戏,不过基本上是我看着前任在打游戏。那时候前任养的那只猫也躺在被子上犯困。这两个场景好像都让我感觉到了一种“相遇”、一种连接。

\blockquotesource{相遇,想法和感受,纯理智,机械体}{白色灯塔先生}{2022-03-15}






\subsection*{对大量文字信息感到的被淹没感}

直到现在,我都很反感回看那时候我和TA的聊天记录,因为我很不喜欢对方发一大段又一大段的文字,对方最好能像我一样把所有内容整合成一篇文章。因为那种一大段又一大段的文字会给我一种被淹没感,好像我在被对方的话语以及话语里充满着情感的文字所淹没着,我就像是站在沙滩边,而浪潮开始掀得很高很高,而且还一浪又一浪地涌了过来。

\blockquotesource{心智化,防御,脆弱,残暴,车流}{白色灯塔先生}{2022-03-16}

