\chapter{随笔 | 元旦,生气,冲突,意义}

\ardate{2022-01-03}{xgeE9XaLnaKGN3xmmsRDPA}


元旦假期,我去了另一个城市找朋友L玩。在假期的第二天,我和朋友和朋友L所认识的另一位男生M一起约去艺术展观展。

我们先是吃了顿午饭,午饭的大多数时候都是我听着朋友L和男生M在聊天,然后我们去了那个艺术展。在逛艺术展时,朋友L在男生M不在附近的情况下跟我说:“看完展我们就回去睡个午觉,然后晚饭叫个外卖。”后来,逛展越逛越晚,在快到下午五点时,男生M提议我们去吃晚饭。我知道朋友L只想回家休息,然后我看朋友L对此是否会有什么反应,但朋友L只是把手机摆在脸前玩,没有什么回应。然后男生M便开始在手机上搜附近的餐厅。我想到朋友L想早点回家,然而现在已经离朋友L的设想(看完展就回去睡个午觉)已经相差甚远,所以我在手机上找了几家艺术展附近的餐厅向男生M提议。但男生M并不想去那些餐厅,然后找了一家需要坐地铁去的有一定距离的餐厅。在去餐厅的路上,朋友L依然只是在玩着手机。

当去到那家餐厅吃饭时,朋友L依然全程用手机遮挡着脸,全神贯注地玩着手机。在我和男生M吃得尽兴,同时开始熟起来地聊天时,男生M开始注意到朋友L一直玩着手机有点奇怪,便向朋友L招手,想要引起他的注意,但朋友L一直没有回应。在和男生M聊天的过程中,我开始觉察到他从午饭一直到晚饭的过程中,似乎都很主动地承担活动组织者的角色,会去记住谁是哪里人、喜欢吃什么、不喜欢吃什么、有什么忌口,会去找性价比高又好吃的餐厅、控制成本(人均消费)。我在午饭的时候便向他提出这一点,他发现这更像是他在工作方面的习惯。

吃晚饭时,男生M聊到他生活在公司的宿舍里,没有太多自己的空间。这让我想起刚刚在逛礼品店的时候,男生M看见了自己喜欢的东西,我问他他打算买吗,他说他还不打算,因为他住的宿舍没有这样的空间能放那件物品。吃着吃着,店员上了一杯柠檬茶,另外还送了另一瓶瓶装的柠檬茶。他想让我把瓶里的柠檬茶喝了,我问他:“你是想要这个瓶子?” 他回答说:“是的。”我接着问:“你是想用这个瓶子干些什么吗?”他说:“打算用来种绿植,装土壤。”我回应道:“虽然你说你没有太多自己的空间,但你好像依然会让一些属于你自己的东西‘生长’出来。”他回答说:“是啊,毕竟还是要过生活的嘛,就像自己一个人吃饭也要找一些性价比高又好吃的餐厅。”然后他聊到他并不喜欢呆在宿舍的生活,所以到了周末就会尽可能出来玩,多在外面逛,而不是呆在宿舍里,因为一呆很可能周末就过去了。这让我回想起我以前在大学时的宿舍生活,我在那个当下能理解他想要多在外面度过时间的念头以及这个念头的背景。

聊着聊着,男生M说他想去看一下附近的一家有猫的饮品店,我问朋友L想去吗,他说他不想,但声音小得只有我能听到。吃完饭后,男生M在找那家饮品店在哪,朋友L在跟着他走的时候突然大声地说了句:“我不要跟你玩了!”然后一个人转身走向离开商场的方向。我和男生M跟了过去,男生M问我他怎么了,我说:“他可能早就想回家了。”

在地铁里,和男生M匆匆道别后,我和朋友L坐在地铁里,我问朋友L:“你会想他(男生 M)在定行程的时候多关心一下你吗?”朋友L回答说:“他总是会在计划好的行程后加更多东西,我一开始就说好只是约艺术展。(他)不像你,你问了我是不是想去饮品店。”我说:“如果你可以早一点跟他表达这一点,说不定他就能更早地知道(这一点)。不过,他下一次估计也就知道了吧。”朋友L回答说:“他总是这样,又不是一次两次了。”我回答道:“噢,那这可能是他的一种模式吧。”

我猜想到男生M为什么会有这一模式:因为不想呆在宿舍而想多在外面逛。但出于这样的背景和意图,男生M可能在定行程的过程中忽视了同行人的想法和意见。另一方面,朋友L当时的回答“我不要跟你玩了!”听起来更像是一个小朋友会说的话,或者说这是他内心更偏向小朋友的部分、小朋友的自我所说的话,我猜想这可能是因为朋友L在那个当下感受到了小时候所曾经感受到的感觉,猜想到小时候的他说不定经历了某些创伤,一些不被他人关心和重视甚至是不被他人“看到”的创伤。同时,把手机摆在脸前来回避与外界的互动的行为也很像小朋友生闷气的表现。

当朋友L生气时,我的第一反应是回避,因为在我的原生家庭里,回避生气的人意味着少挨打。但我很快意识到自己开始表现出回避甚至是有点讨好攻击者的行为,并将焦点转向朋友L,试图去设想为什么朋友L会生气。我想起近两个月前开始学的人本课程内容里,当事人的痛苦的背后往往意味着TA所珍重的意义的受创。所以我想知道,朋友L的生气背后,那个受伤的部分背后所代表的他所珍重的会是怎样的意义。当我得知他重视的是他人对他的关心后,在之后和他的相处里,我也逐渐意识到,朋友L很少会主动与人(包括我)产生连接,很少主动关心他人的内心事物,而只有在他人开始关心他内心的事物时,他才会去反之关心他人。

\tristarsepline

以前的我(特别是小时候的我)总是回避攻击者,甚至讨好攻击者,认为任何的不和谐和冲突甚至是攻击都是自己所不能忍受的、“不好”的事情。但这次我意识到,这些冲突背后有着它们的意义、暗藏着那些身边的他人所珍重的意义、他们更为内在的事物。冲突并不是一件“不好”的事情,因为在冲突背后有着许多真实的自我,那些当事人可能早已遗忘或隐藏起来的自我。冲突反而能让我更好地理解对方、看见对方、看到对方所珍重的意义。


\useimg{aimg/2022-0103-1.jpg}

\useimg{aimg/2022-0103-2.jpg}

\useimg{aimg/2022-0103-3.jpg}

\useimg{aimg/2022-0103-4.jpg}

