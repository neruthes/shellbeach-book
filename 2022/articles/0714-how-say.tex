\chapter{“真正重要的往往不是双方说了些什么,而是双方是怎么说的”}

\ardate{2022-07-14}{slPuVIMzP0Ebrhgvm3nmpQ}



最近在工作上,上司想安排一个更高级的同事加入我们小组,然后说到忙季快要来了,而我们小组的同事都不在状态。当上司这么说时,首先我是对自己的工作质量感到不安和自卑,但想到“不在状态”、“质量不行”这样的话从入职开始就陆陆续续说到现在,我便开始转移思考的焦点:为什么上司想要说这样的话?Ta的言语给我的第一感觉是:不安\pozhehao{}对忙季感到不安、对我们小组组员的工作质量感到不安,所以要安排一个更高级的同事到我们组里,但这对ta来说似乎依然不够,因为ta还继续说我已经对流程、规范都很懂,对待工作一定要谨慎、认真。在我看来,这更像是典型的边夸边提要求的句式。

在那份不安的背后,我还感受到了操纵感\pozhehao{}将新的更高级的同事安排进小组以及边夸边提要求地想要组员(我)保证工作质量。这也让我想起,我在接热线的过程甚至是在和朋友的相处里也不缺乏这样的让我感受到操纵感的时刻。

在接热线里,我会遇到某些来电者会说类似于“你认为我不危机吗”的话来试图延长热线时间。我的一贯态度是不去满足对方的这些让我感到被操纵感的要求。例如当对方表露ta担心我无法理解ta时,我会沉默。督导师指出了这一点,并提出说我们接线员要呈现出主动的态度,鼓励来电者多说一点,而不是沉默\pozhehao{}“就像是在说:你爱说就说,不说拉倒”。同时其他学员对此的反应也不会是沉默,而是会回应一些类似于“我会尽可能理解你”的话。

当然,不去满足对方的这些让我感到被操纵感的要求的态度也和我的成长环境有关。在我的主观现实里,我妈一直以来都是个manipulative bitch,在过去试图操控我生活的各方各面(特别是客体),不只是操纵物理现实,还是操纵精神现实,比如说天气的冷热、考试的难易、挫折的大小\pozhehao{}“现在的天气又不热”、“这次的考试又不难”、“就只是这么点事”。

因此,现在我的下意识做法是拒绝这样的诉求。我能看见那诉求是什么,仅仅是看见本身就足够了,至于我做什么或者是做与不做,那是事后的后事,而不是在那个当下我就要付诸行动地通过满足对方来消除那份对方的不满给自己带来的被操纵感和压迫感。

而如果我真的要对此回应的话,我也不会是直接满足、迎合对方的诉求,而是指出对方这一诉求本身背后可能存在着的事物。例如,如果对方说:“我担心你不能理解我”,我可以回应说:“听上去,好像你蛮看重被理解的。”

我越来越发现在人与人之间的互动里,真正重要的往往不是双方说了些什么(what),而是双方是怎么说的(how)。例如在一次朋友聚会里,有两个朋友交谈甚欢,其中一个朋友会特意提一点来惹怒对方,但又不是以一种敌意的方式去惹怒对方。例如说开玩笑的朋友会说:看XXX会喷那么多种类的香水,就知道他面基过不少男生。而那个被开玩笑的朋友就会极力想去否定这一点。

我会留意到这些惹怒的点都是基于其中一个人对另一个人有一定程度的理解,甚至是知道对方的脆弱之处是什么,并且拿这一脆弱之处以幽默的方式来开玩笑,却不是攻击(明明可以攻击)。当这样的方式在聚会上发生了两三次后,我开始留意到主动开玩笑的那个朋友好像想要通过这种开玩笑的方式来让另一个朋友更加投入到彼此的互动当中,好像只是普通的聊天对他来说並不足够。

所以后来我说:“你们看上去像是在打情骂俏。”然后他们马上就沉默了。后来其中一个朋友说道他们刚才确实是在打情骂俏,但彼此还是保持着各自的边界的,毕竟他们各自都有自己的对象。

在言语层面,很多人都自我社会化得能熟练运用各种防御,甚至能以书面语来口头聊天。但一些更内隐的东西(例如语音语调、用词、肢体语言、面部表情等)总是在传达一些与言语内容不一致的东西。例如说在外人看来很无聊的东西,两人却能聊得起劲;一个人在谈论着下属工作安排的变动,但好像无论怎么安排都并不足够;一个人在认为热线设置时间不够时,却以一种缺乏对自我(自体)边界的自信和果断而是想要跨过边界去摧毁对方的态度来表达愤怒。


