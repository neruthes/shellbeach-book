\chapter{Nothing matters}

\ardate{2022-06-05}{6i6ilPr4sA8bveuLF5K5Wg}



\dialoguelist{咨询师}{
	\dialogue{我}{我最近的状态有点崩。这周我都没有写什么东西,因为上次咨询发现写作好像只是为了确定自己那转瞬即逝的存在感而已,所以就不想写了,没有什么意义。这周就是上班,吃饭,还有打游戏。这周我有三门网课要上,但我都没去上。
		这周我脑海里一直有一个画面,是之前看的电影《瞬息全宇宙》里面的一个反派说的一句话:“Nothing matters.” 什么都不重要。那个反派就像是欧文亚隆写的那本《存在主义心理治疗》里的上帝视角,因为她的自我意识被分裂到了各个平行世界里,所以她看透了一切。就是nothing matters.}
	\dialogue{咨询师}{当你这么说的时候呢?你会有什么感受?}
	\dialogue{我}{我会感到很伤心和难过。因为现在这样的状态是因为,如果是短期的话,是之前喜欢的那个男生拒绝了我,如果是长期的话,就是前任。当那个男生拒绝了我后,我就没有再去设想彼此能有更好的未来了。当我发现前任不是自己所设想、所投射的那个样子时,他在我内心的形象就消失了、死掉了。同时,好像我自己的一部分,那个和他们相处时的我,或者说他们眼中的我也死掉了。}
	\dialogue{咨询师}{这种状态还会让你想到些什么吗?}
	\dialogue{我}{我会想起之前有一个男生无故消失后,我有很强烈的自杀意图,甚至打算去实施自杀计划,但后来我放弃了。现在的我就像是那时候放弃了自杀的我,只是一副行尸走肉,仅仅是延续着、存在着,并不是真正意义上地活着。}
	\dialogue{咨询师}{那你现在呢?会和那时候的感觉那么强烈吗?}
	\dialogue{我}{嗯,会。……但同时我也知道自己能够在这种状态里活下去,而不会去自杀,不会去选择死亡,因为连死亡本身也没有了意义。有时候接热线的时候,会接到一些高风险来电,他们努力寻求出路、想要摆脱困境、受尽痛苦,甚至是追求死亡。他们会有那份动力,但我并没有这样的部分。当我没有往这个世界上的任何人投射任何东西,没有向任何事物和人投注精力时,这个世界也就变得毫无意义了,充满着虚无。甚至连咨询也没有了意义。}
	\dialogue{咨询师}{好像你真的没有和任何事物产生连接。那在咨询,你会不会也感觉我们之间没有多少连接?}
	\dialogue{我}{嗯,会的。我会感觉没有连接,和任何事物和人都没有连接。}
	\dialogue{咨询师}{那你能回想起以前的感觉吗?}
	\dialogue{我}{我好像回想不起来,现在的我的状态回想不起来。我能记起我们谈过的内容,但情感……我记得有连接感、开心和兴奋。但现在的我在说这些话的时候,那些好像就只是文字、词语而已,那些词语都是空洞的,它们背后没有情绪。因为如果有情绪的话,至少我还能利用它们来感受到这些情绪。}
	\dialoguesepline{咨询师}{(沉默)}
	\dialogue{我}{我会留意到你今天穿的衣服上有一个词“childlike heart”,但我看不清下面那两行的字,所以会好奇那两行字写的是什么?}
	\dialogue{咨询师}{你好像把焦点转移到了我身上。}
	\dialogue{我}{嗯,是的。因为我想去看看适用于你的部分是否也能适用于我。想去照搬一些对方的东西,像是套模具一样将自己套在别人身上,看看能不能摆脱那种无意义感和虚无感。比如说刚毕业的时候我会想去成为前任的样子,像他那么工作狂。但后来当我找工作的时候,我发现我做不到,我很无力。我想可能我内心有很大一部分、绝大多数的部分在抗拒着自己去做这件事,抗拒着自己去逼自己做自己并不想做的事情。当我做不到的时候,我就感到很无力,同时也摆脱不了无意义感和虚无感。}
	\dialogue{咨询师}{那我大概能明白为什么你会总是想占据他人的空间,就像是之前你说的萨特的《他人即地狱》一样,你真的很需要从另一个人的瞳孔里看到自己。}
	\dialogue{我}{嗯。在我眼中的对方总是很固定,比如说有固定的事业爱好,虽然我也知道每个人都会变,但在我看来好像是阶梯式的变化,不如说不断变好或变坏。但我不是,我好像没有什么形状。}
	\dialogue{咨询师}{为什么那些部分会给你固定的感觉?}
	\dialogue{我}{我也不知道,但就是会有这种感觉。}
	\dialogue{咨询师}{那你是怎么看到对方的那个部分的?}
	\dialogue{我}{比如说你今天穿的衣服上的这个“childlike heart”就是你的一部分,因为刚刚我提的时候你笑了一下。比如说你的袋子将近一年了都没有换,即使上面已经有很多褶皱。当我看见他人向这个世界上的人或事物投注那些部分的时候,那会给我一种固定、稳定的感觉,因为我自己并没有这样的部分,我并没有向这个世界的任何事物或人投注精力。\\
		而且这个部分也不是理所当然的,向这个世界上的事物和人投注精力本身不是理所当然的,因为我就不会。当内心的那些重要他人的形象在自己内心死掉后,他们对应的那些我的自我也死掉了,而剩下的部分我拼凑不出来些什么,拼凑不出来自己究竟是谁、究竟是什么。以前的我好像一直都在用他们的形象来拼凑出自己是个怎样的模样,比如说和他们相处的时候或者是当我幻想和他们相处的时候我是怎样的,用那个形象来构成着自我。但现在没有了,什么自我都没有了。只是一团无形的东西,说不上来是什么,只是在无形地飘着。\\
		当我回想起过去,小时候的时候,虽然现在说起来会有点傻,但那时候的我会希望有人能告诉自己的人生道路、活着的意义、自己擅长些什么、自己应该往哪个方向走?}
	\dialogue{咨询师}{你会说好像现在想起来又会觉得有点傻?}
	\dialogue{我}{嗯。因为这些事情都并不会有一个固定的、唯一的、稳定的答案。未来的道路是虚无的,过去也是虚无的,此时此刻也是虚无的。}
	\dialogue{咨询师}{那当我穿着这件衣服、带着这个袋子的时候,你会怎么理解这件事呢?}
	\dialogue{我}{我会设想你可能把自己的一部分投射到了这些东西上,比如说那个“childlike heart”的那部分你的自我。然后当你穿上这件衣服、带着这个袋子的时候,你也和那个你投注于它们的那部分自我接触了。而且绝大多数人几乎不会随意就换东西,比如说你的衣服的领子已经有点皱但你还穿着。如果扔了的话,那可能也意味着扔掉了投注到这些物体身上的那部分自我,或者是那部分自我转移到了其他事物上。好像外界的事物和他人都是客体,总是要把自己的精力投注到外界的客体上。}
	\dialogue{咨询师}{我留意到了你在皱眉。}
	\dialogue{我}{嗯,我会感到很悲伤、很难过,一切都没有意义了。当对方接触到那些对方所投注的事物或深爱的他人的时候,他们同时也接触到了那部分的自我。但我没有,我没有重要他人。虽然我不确定未来什么时候才会遇上另一个喜欢的男生,但此时此刻的我没有,所以我也连接不上和重要他人相处时的那部分我的自我。\\
		是啊,正是和重要他人相处时的那个自我,才是最能感受到各种丰富的情感的自我。我需要那部分自我才能感受到情感,无论是开心也好,伤心也好。就像是有的人一进入亲密关系就像是变了一个人,那个处于亲密关系里的那部分自我。但我好像就只有亲密关系里的自我了,而除此之外,就什么都没有了。如果我不在别人的家里、不在他人的内心空间里,那在那外面就只有一片虚无,什么都没有。当一个人的时候,我就什么事情都不想做,没有什么想干的,就只是打游戏,就像小时候一样。但其他人会有恋爱关系之外的那部分自我,但我好像就是没有。需要有另一人去确定自己的存在,需要对方在内心留有关于我的形象,好让我将这些形象拼凑起来。}
	\dialogue{咨询师}{你是怎么看到对方眼中那个你的部分的?}
	\dialogue{我}{比如说你会在咨询里说我好像和之前的几次咨询有所不同,那时候我就知道你在内心里留有着一部分的我的自我。比如说之前喜欢的那个男生总是在见面的时候说我很低沉,他需要去说些什么或做些什么才能让我活跃起来。那时候我就知道他内心有一个我的形象,他们眼中的我。正是因为他们的言语或所做的事情。但现在不太能了,因为现在那些部分都太小太零碎了,它们拼凑不出来些什么。}
	\dialogue{咨询师}{那会有人主动连接你吗?}
	\dialogue{我}{之前那个喜欢的男生会偶尔找我约饭,但我去约他的话他总是没空。加上我也没有把他视作重要他人了,因为我没有再向他投射、投注希望了,不够重要。}
	\dialoguesepline{}{(沉默)}
	\dialogue{我}{这时候我的一部分又会想自我指责,想指责自己为什么总是需要他人来确定自己的存在,自己就不能独立吗,会要求自己拥有着一切的答案。这个部分在之前是投射给了前任,认为他总是有着我提出的任何提问的答案,但后来我发现他并没有答案,所以我就把这个部分收了回来,但现在好像我又在要求自己要做到拥有一切答案。我也会在想刚刚我提到的那些特质(独立、拥有一切答案)是从哪里来的。}
	\dialogue{咨询师}{我想这次的讨论能留到下次再继续。}
}
