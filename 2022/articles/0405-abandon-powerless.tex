\chapter{被遗弃的无力感}

\ardate{2022-04-05}{HeiZZUvKuB\_SXAC8wtEK6w}




上面那篇文章里有一个评论:“那个年纪,只能那样了。”看到评论时,我会有点感动。嗯,是啊,那个年纪的自己也许也尽力了。

但那种无力感一直都在,一直延续着,而即使是现在的我,我也不知道我能怎么样。我又能做些什么呢?

最近喜欢上了一个有夫之夫的男生,而我特别不喜欢他摸我的头。我说“我不敢完全依靠你,毕竟你还有你对象,而我没有。”我只有他。当和他见完面后,在各自离开的路上,我感到很脆弱,同时也意识到自己在感受着脆弱感,并对自己说:让自己在另一个人面前变得脆弱,勇于感受这种脆弱感也蛮重要的。

我不敢依靠他,会有一种恐惧感,但同时也有着渴望,渴望能和他有更多的连接。恐惧和渴望同在。而我也没有因为恐惧就停滞不前了,因为那份渴望足够强大。我会想起自己关于去考研的念头,在那方面,是恐惧胜过了渴望。

大学时认识的一个朋友/学长读了上面那篇文章后,说了一些话。他没有直说,但通过另一个他身边的人转述到了我这的话是:“有点太钻牛角尖,就该干嘛干嘛就好”。我的第一反应是很气愤,然后想到“思维反刍”。

我会想到,有的人可能就那么“幸运”,想做什么就去做,因为关于想做的某件事情上并没有经历过什么负性事件/创伤。比如说想坐车就坐车,而没有经历被车撞过后出现的PTSD经历;比如说参加什么聚会就参加什么聚会,而没有经历被霸凌、被社交隔离的经历;比如说想做爱就做爱,而没有经历被性侵的经历。然后在那些“正常人”的眼里,一切看似都那么的正常、那么的“该干嘛就干嘛”,那么的自然而然地发生、进行。对于那些“正常人”,我会既感到羡慕,同时也感到嫉妒和憎恨。

不过有时候,我也会觉得自己蛮幸运的,毕竟还有许多人经历过比自己更加糟糕、更为残忍的事情。

清明假期的最后一天,想见那个有夫之夫的男生,但他因为有事而没见。最后一天,我用一些需要出门折腾的面基/聚聚来填充最后一天。临近傍晚,自己真的很困,就回家去了。但我睡不着,想写一点东西或上一节网课又发现自己无法集中注意力。后来躺在床上也一直睡不着。我觉得自己很没用,然后发现家里人做好了饭没有叫我就他们自己吃完了。躺在床上的我感觉自己被遗弃了,然后马上起床出门,踩车去附近的一个我熟悉的社区里找吃的。在夜晚的马路上,踩着单车的我吹着冷风,发热的头脑、身体和皮肤都有所冷静。我想到家里人可能是以为我睡着了才没有叫醒我吃饭,但我依然不想去思考这个可能性,并认定他们就是遗弃了我。当我意识到自己拒绝去思考时,我用仅有的心智化精力意识到自己在当下有着认知扭曲和拒绝心智化,并聚焦在被遗弃感上。我意识到这种被遗弃感在这一整天一直存在着,只是在接近这一天的结束时抵达了情感峰值。而我要去的那片熟悉的社区,是我唯一能获得安全感的地方了,而不是他人。那片社区曾经是我和初恋有过共同经历的地方。

被遗弃感,不知道是因为上一次心理咨询的所激发的,还是假期最后一天的那个男生因为有事而没能见面所激发的,但这种感觉一直存在着,一直存在于自己内心深处,一直存在于自己的经历里。

即使是现在的我,面对这种被遗弃感、面对被遗弃,又能做些什么呢……被遗弃的无力感。

