\chapter{“活着没有什么意义,也没有什么能干的”}

\ardate{2022-03-11}{9xikidne6dGCsvW6sHQjLw}


我觉得活着没有什么意义,也没有什么能干的,等到了死了之后就更加什么也干不了了。

这种“没有什么能干”的感觉一直伴随着我以前的自杀意图,直到现在也很多年了。这种感觉可能就是抑郁症症状之一的丧失兴趣或乐趣。虽然这种感觉已经很熟悉,但每次“来袭”的时候依然感觉很难受。

在现实生活的时间轴上看,这种感觉的出现很可能是因为我停止了接电话热线,因为这个月的接线时间Quota已经达成了,所以我想将腾出来的时间用在上网课上。但好像我的大脑并不想我这么做……这种感觉之前也有出现过,比如说周末有一个很开心和难忘的活动,而之后的工作日就会跌入低谷。有的人可能会称之为节后综合征,这暗示着日常的工作生活并不利于心理健康,而从节假日的放松或兴奋回到令人压抑的生活后,对我来说就会觉得:活着没有什么意义,也没有什么想干的。人为什么要活着?

\blockquote{
后来继续走在江边时,我恢复了点精力,然后开始心智化我自己地对他说:“有时候我感觉活着也没有什么事情可做。当我回想起之前那些自己觉得很有意义的事情、觉得这个世界很丰富的时候,我发现其实丰富的、充满意义的更多是来源于自己。而一旦我不再将精力投注于这个世界、不再投注于身边的人的时候,一切都会变得毫无意义。而且也没有人会心智化我,只要我不去推动一些事情的发生,那么事情就不会发生了。就像是现在在走路一样,如果自己很累,不想往前走了,那么这条路就会停在这里。”他问我我想指的人是否就是他自己。我说:“是的,以及生活里的其他人都是这样。”

\blockquotesource{“憎恨这个世界和这个世界上的所有人”}{白色灯塔先生}{2022}
}

不过这种感觉早就在上周周末的时候出现了,并不完全是因为自己在这周暂停了接电话热线。好像每到一定时期,自己的大脑就会自动地暂停工作,或者说不再往任何人和任何事物身上投注精力。而当不再投注精力于这个世界时,一切对我而言都变得毫无意义了。

我有想过怎么“自救”,比如说月中发工资时去买一个新网课。我记得在之前的咨询里,咨询师有留意到我总是在上各种课程或看各种书,然后问我这对我来说意味着什么。那时候我想不到什么回答,因为对我来说,在日常生活里进行各种学习早就是我多年来的日常。不过如果要让现在的我去回答的话,我会回答:“一直在用意义来驱赶无意义”\pozhehao{}在用各种各样的课程、书籍、和朋友相聚、和新的人面基等新鲜的事物来创造意义,以此驱赶其他无意义的部分\pozhehao{}无意义的工作、家庭、同事、生活环境。这就好像是,我一直处于一片匮乏的土壤上,而我试图用自己的意义来构建出一个自己能安驻于其中的乐园。当然,我并不是全能的,这个“乐园”也会时不时 not working,而我则因此需要去寻找更多新鲜的事物,向这个“乐园”灌注新活力。

这就像是,我不仅仅要照顾自己的身体需求(比如说满足饥饿感和性欲),还要照顾自己的心理需求(比如说对人际关系和新鲜事物的渴望)。而当心理渴望没有得到满足时,此时此刻的我的感受是:我感觉自己快呼吸不过来,有一种窒息感。比起写作的一开始,现在的我更加确信自己想要什么,而那份想要买网课、想要获得新鲜事物的渴望就像是在掐着我,越掐越紧,越来越难呼吸。

现在往回看,“活着没有什么意义,也没有什么能干的”并不是真实的,那种“活着没有什么意义,也没有什么能干的”的感觉的下一层暗藏着一个更汹涌的渴望,而我在此之前可能都一直无意识地压抑着这份渴望。此时此刻的我很想找朋友借钱去买网课,如果我不尽快获得这一新鲜事物的话,这种窒息感似乎只会越来越难受,这种渴望快要失控的感觉,越来越难 keep it at bay。

我会在想,为什么之前的我要无意识地压抑着这份渴望?我记得大学时候的我会因为之前的受挫经历而无意识地压抑着自己想去探索校园外的世界的渴望,还会加上一定的自我攻击。但现在的我已经不需要像以前那样通过压抑自己的渴望来自我保护,或者说那时候的我本不需要通过这种方式来获得自我保护,因为有其他的方式能更好地自我保护。那现在的我会不会也在无意识地保护着些什么?可能在保护着自己内心那个脆弱的、容易受挫的那部分自我?Maybe. 那部分的自我可能会说:“我真的很害怕受挫,所以我宁愿相信一切都是没有意义的,没有什么事情好做的。”

感觉自己此时此刻的心情和呼吸都好点了。然后我感受到一种伤心,好像那个脆弱的、容易受挫的那部分自我在哭泣着在大学时候的那些渴望都没有被满足、都辜负了,辜负了那时候的我对未来的憧憬……好像现在的我辜负了大学时的我对未来的憧憬。大学时的我总是在教室里看着窗外,或是下课后走在放学路上,看着校园的风景,憧憬着自己的未来能够像这个大学校园一样,充满着新的风景、新的未来、新的事物。结果毕业后,我却选择了回原来的城市生活,选择回到大学前的生活。那时候之所以会回来这座城市,是因为那时候认识了前任,而那时候的我以为自己能会在前任公寓一直生活下去,结果后来还是回了父母家,这个让我厌恶和憎恨的充满着无数痛苦的童年经历的地方。

好像现在的我在那么长的时间以来,一直做着大学时的我所厌恶和憎恨的事情\pozhehao{}回到原来的地方、回到那些糟糕的过去。

想换一个城市、换一个地方生活了,想离开这里的一切。
