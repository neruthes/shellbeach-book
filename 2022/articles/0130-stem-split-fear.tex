\chapter{随笔 | 支柱,撕裂感,恐惧}

\ardate{2022-01-30}{Z1xix945oVKtO6A2yBlLzA}




\blockquote{
    我开始谈到最近一周里发生的一些事情让我感觉这些事情好像都和家这个主题有关。在这一周,我会想去联系之前的两个无故消失的男生。在理智上,我不想再重新经历过去那些被伤害的经历,但我在感觉上依然很想去联系他们,因为他们是离我所可能拥有的家最近的人,我的生活需要支柱。而春节的临近,以及身边一些人开始打算回家的这种节日气氛的逼近也让我的内心有所波动,一些隐隐约约的东西开始涌现出来\pozhehao{}孤独、悲伤、难过、被落下的感觉。

    \blockquotesource{呆在那里,不想说话,沉默}{白色灯塔先生}{2022}
}

\tristarsepline

在那之后没多久,我就冲动消费地买了三门自学的网课,打算春节宅在家里上课。后来我发现我更像是在利用新买的网课来回避过年焦虑。

昨晚洗漱完躺在床上,打算睡前翻一翻手机,在翻到前任朋友圈时发现他春节有空。我在考虑要不要约他见面。上次约他见面时,我突然取消了见面,而那时候他说:“见不见的都行,随意”。这句话像是一堵墙,挡在了我想要主动联系他的路上,那种自己在对方眼里根本不重要的感觉,而我不想再触碰到这种感觉,更别说是撞穿这堵“墙”。

我在床上辗转反侧,想到:一方面,我感受到一种对他的爱意,那种爱意一直试图push着我去向他靠近;另一方面,我感受到一种对他的恨意,那种恨意在告诉自己我会因为靠近他而受伤的。这两种感觉开始“撕裂”我,像是两股气流螺旋状地包围着我,试图将我往两个方向拉。那种撕裂感让我呼吸不过来,喘不过气。

我决定发消息给他,提出约见个面。当发送信息后,我发现那种撕裂感消失了,随着而来的是一种确信的感觉,那种在冲动消费买完网课后更加确信那就是自己想要的东西的感觉\pozhehao{}我很确信自己想要朝着某个方向走,很确信我需要利用自己所渴望的知识来回避过年焦虑,也很确信我想要拉近和他的关系。

我也会想到,想要约他见面是否也是因为我想要在生活里找个支柱,就像是在上次的咨询室里所说的\pozhehao{}“我会想去联系之前的两个无故消失的男生。在理智上,我不想再重新经历过去那些被伤害的经历,但我在感觉上依然很想去联系他们,因为他们是离我所可能拥有的家最近的人,我的生活需要支柱”。但我似乎并不是因为想要为生活找一个“支柱”,因为我已经有网课这一支柱了。

那么,是什么使我愿意冲破那堵“见不见的都行,随意”的墙呢?我发现我之所以这么做了,是因为我想要摆脱那种撕裂感。在上一次约见面时,我选择了“恐惧”的一方,并退了回去,取消了见面。在这次约见面时,我选择了“爱意”的一方,向他提出了约见面。那现在的我(1月底)和那时候的我(11月初)之间又有怎样的不同?我在这三个月里又发生了怎样的变化?

我开始回顾在约见面之前的那个呼吸不过来的感觉,那种感觉背后有一份恐惧。我自问:我是恐惧于什么吗?会是恐惧于约他见面这件事肯定会再次触碰到过去的伤口?我自我感觉了一下,是这样的,但又不完全是。再次触碰过去的伤口并没有让我那么的恐惧,虽然说过去的伤口我已经放下好几个月了\pozhehao{}这可能是我没有再次选择“恐惧”的一方并退回去的原因。我感到更为恐惧的是,万一我和他的关系走近了呢?万一和他复合了呢?万一和他过往的事情又再一次重演,我又再次经历幻灭呢?

但想到这一点时,我的恐惧感便消失了,因为我知道现在的自己有能力阻止/跳出事情的重演。

睡醒后,我的心情是开心与悲伤的交杂\pozhehao{}开心于能够再见到他,悲伤于我和他的关系被疏离了那么远、悲伤于对他的那份爱意在一路上的受挫。

