\chapter{写作,言语,人际关系,不安感}

\ardate{2022-04-10}{-3uw54vYo9RX7PbOGJAZeA}


\dialoguelist{咨询师}{
	\dialogue{我}{你会想知道上一次咨询后的后续吗?因为之前好像也有不少次咨询也是会有后续的,但我都没有在之后的咨询里告诉你后续是怎样的。}
	\dialoguesepline{咨询师}{咨询师用肢体语言示意可以继续}
	\dialoguesepline{咨询师}{我从口袋拿出手机,开始读文章“无力感,僵住”里的最后一部分}
	\dialogue{我}{我试图将焦点不放在现实世界里,不放在现实世界有多么危险、多么无力,因为好像每次想到现实情况的时候,我都是在无意识地安慰自己:这并不是现实,现实并没有自己所设想的那么可怕、那么令我恐惧、那么令我感到无力。就像是一个父母安慰自己的孩子,黑暗里并没有怪物,这不是真的。\\
		我试图把焦点放在主观世界里、情感世界里,试图去挖掘这种无力感,去体验这种无力感。这种无力感会让我再次“回到”那个无尽黑暗楼梯间的梦境\pozhehao{}永远没有出路,还有小时候被我妈赶出家门的画面\pozhehao{}坐在台阶上在门外敲打着铁门哭泣着。我突然将这两者联系在了一起,梦里在无尽黑暗楼梯间里的那种无力感和被我妈赶出家门的感觉是那么的相似,而父母家门外、那个我被赶出去的家门外就是一个楼梯间。被遗弃的无力感。这就是从前任公寓搬出来时,走在公寓小区路上的感觉。这种感觉一直贯穿着我过往的许多(无数)经历。\\
		被赶出家门的时候,我有想过,要是我走出去呢?我会活不下去的,我会生存不下去的,我会死掉,我很(好)害怕。所以我没有走出去,没有像梦境里的那样不断寻找出路并最终走出去,而是僵硬地凝固在了楼梯间的台阶上,我什么也做不了了,什么也改变不了,没有人会来开门,没有人会发现我,没有人会要我。我想走出去外面,但我又害怕外面的黑夜。我想死,但我又害怕死。我只能这样了,只能在这里了。}
	\dialogue{咨询师}{那这会给你带来一种怎样的思考吗?}
	\dialogue{我}{我会想到:噢,原来那种无力感和梦境里的楼梯间是有关联的,怪不得自己会在被我妈赶出家门后一次又一次地梦到黑暗的无尽楼梯间。这一切都make sense了。}
	\dialogue{咨询师}{除了这都make sense了,还会有其他的吗?}
	\dialogue{我}{好像没有了。}
	\dialogue{咨询师}{其实听你写的文字,我会觉得里面的情感程度还蛮深的,但当你用语言说出来的时候,反而没有太多的情感,好像你只是把这个东西带进来咨询,只是让我知道有这件事的发生,但没有继续地深入和探索。好像你只是想告知我这件事而已,并不想多说些什么。}
	\dialogue{我}{嗯,是的。好像当我写作的时候,在那个环境里我会感到更安全、更自在地表达自己的情绪,但在现实生活里,我需要去考虑到另一个人的反应。}
	\dialogue{咨询师}{你会设想我会有怎样的反应吗?}
	\dialogue{我}{我好像没有这样的设想。但是感觉现实世界的我好像在阻挡着写作时的我想要表达的一些事情,比如说刚刚我读那段话的时候其实我是有改一些用词和表达的。}
	\dialogue{咨询师}{噢?比如说是一些怎样的用词呢?}
	\dialogue{我}{比如说把“无数经历”改成“许多经历”、把“我好害怕”改成“我很害怕”。}
	\dialogue{咨询师}{为什么你会想这样做呢?}
	\dialogue{我}{其实我在读的时候也有意识到自己在改词,我想我是在试图抵抗那种无力感、那个弱小的自己。当我在写作的时候,我感觉自己就是那个场景、那段记忆里的小时候的自己,但当我把它读出来的时候,我觉得我是现实世界里的我,而不是小时候的那个自己。}
	\dialogue{咨询师}{听起来,好像当你写作的时候和你在说话的时候是两个不同的你,一个是能够更自在地表达你想要表达的情感的你,另一个是你现在是什么样子的你。}
	\dialogue{我}{嗯。是这样的。}
	\dialogue{咨询师}{为什么会是这样的呢?你会想到些什么吗?}
	\dialogue{我}{我会想到小时候的自己的毒舌。那时候的自己好像更能自然而然地表达出更加流动的情感,但现在我好像做不到了,至少不能通过言语这么做,只能通过写作。所以有时候当我在用言语表达一些事情的时候,我感觉自己更像是在写作,而不是在无意识地表达,就像是在一边写稿一边改稿。可能是因为小时候毒舌经常被我妈打,所以后来我就压抑着自己不去说一些伤人的话,直到后来自己什么都不想说了。后来当自己开始写作写到一定程度后,自己才开始觉得我能通过言语的方式进行写作,然后才开始习惯用言语表达一些事情。}
	\dialogue{咨询师}{好像你在言语表达的时候会压抑着自己去表达些什么?}
	\dialogue{我}{会有一点。会去看自己表达的内容是否会伤到人。}
	\dialogue{咨询师}{那现在呢,你会感觉自己像是在写作吗?}
	\dialogue{我}{现在不会,在咨询里的体验里都不会,都大多是在无意识地说些什么、不经思考地说些什么。}
	\dialogue{咨询师}{你还会记得小时候的一些怎样的事情或经历吗?}
	\dialogue{我}{我会记得小时候会在发廊阿姨那说她坏话,说“我以后不光顾你了”。还会在亲戚面前说父母的坏话。还有读高中时在校外认识了一个小群体,其中一个女生跟我说了一些她的秘密,但我把这件事告诉了另一个男生听,但在讲的时候自己越来越亢奋,越来越丧失理智,后来我跟那个男生说不如你去告诉那个女生,看她会有怎样的反应。那个女生知道了之后就很生气,然后我也和那个小群体分离了出来,因为他们都很憎恨我,而我也责备自己为什么会犯了一个那么小的错误,而导致了那么大的后果。}
	\dialogue{咨询师}{Em……这些经历好像都是你在挑战规则?}
	\dialogue{我}{挑战规则?}
	\dialogue{咨询师}{好像你会去挑战一些现有的人际规则,比如说在亲戚面前说父母坏话,比如跟另一个人泄密那个女生的事情。}
	\dialogue{我}{好像是。但为什么自己当时要那么做?}
	\dialogue{咨询师}{可能是想通过挑战规则来获得些什么?}
	\dialogue{我}{获得些什么?}
	\dialogue{咨询师}{好像你在挑战这段人际关系的同时,你从这种挑战里获得些什么或想要获得些什么?}
	\dialogue{我}{Em……可能会想获得一种安全感。就是如果对方能够承受得住这种挑战的话,我会得到我所想要的那种安全的关系。}
	\dialogue{咨询师}{那在那种安全的关系里,你会做些什么吗?}
	\dialogue{我}{我想我可能会更自然而然地表达自己想表达的内容,而不用担心被对方攻击或回避。即使是表达一些关系里的不愉快的部分也可以。\\
		我会想到,好像我就是没有这样的基础,在过去的经历里都没有能够让我表达关系里不愉快的东西。如果我在父母面前说他们对我怎么不好、怎么糟糕的话,他们会打我。所以我才会在亲戚面前说,因为我想说出来,但又担心被攻击。高中时和校外的小群体也是,那个女生告诉了我一些秘密,但我想将这些东西表达出来,所以就跟另一个男生说了。}
	\dialogue{咨询师}{那你现在呢,也会不敢表达关系里的不愉快吗?}
	\dialogue{我}{会有一点,会担心自己如果表达了的话,关系就受损了。}
	\dialogue{咨询师}{啊~ 好像在过去的经历里你都不能直接表达你对关系的不满甚至是一些带有攻击性的毒舌。我会想起,之前你有邀请过我去读你公众号里写的文章,而你也说你会把一些咨询里的故事写在了上面。那时候你之所以会邀请我去读你写的东西,是不是也因为有一些你对我们的关系的不满也是没有直接提出来,而是通过写作写了出来。}
	\dialogue{我}{嗯,会有这个部分在。毕竟关系里总是有不愉快的成分在。如果那种不愉快的感觉很强烈的话,我就会通过提建议直接提出来。但如果只是路上的一些很小的石子的话,我会下意识地把它磨平。或者是看到了这些不愉快的部分就放在一边,过去就过去了,然后继续看整条路是怎样的。}
	\dialogue{咨询师}{那些不愉快的部分,比如说会是些什么吗?}
	\dialogue{我}{比如说……比如说刚刚你会有一个语气词“啊~”。那时候我就感到不舒服,因为我会感觉你更像是看到了你想看到的东西,可能是某些理论或知识体系。更像是看到了你想看到的物体,而更少地看到我,我的成分更少了。所以你刚刚在“啊~”的时候,其实我没有在听你之后在讲的是什么,我走神了,我在思考你究竟想看到的是什么,后来我才把注意力拉回到你说的话上。}
	\dialogue{咨询师}{好像你走神的时候也在担心我是不是更少地看见你。}
	\dialogue{我}{嗯,会有点不安吧。}
	\dialogue{咨询师}{那现在说出来之后,你会有什么样的感觉吗?}
	\dialogue{我}{我会感觉这里是足够安全的,能让我安全地表达我想要表达的内容,而不会被攻击或回避。}
	\dialogue{咨询师}{但好像如果我不提的话,你还会主动说出来吗?}
	\dialogue{我}{应该不会。可能会在我的写作里写出来,但不会主动拿出来。因为和朋友的相处里也蛮多这样的情况,我能在言语之下看见很多很多东西,但我不会把什么事情都拿出来讲,自己看到就行了。}
	\dialogue{咨询师}{好像当我说“啊~”的时候,你马上产生了一种感觉,而你也很确信这种感觉是真的。好像这种感觉在你的人际关系里也经常出现。}
	\dialogue{我}{嗯。就是那种对方只是想看见对方想看见的事物的那种感觉。和朋友会有那种感觉,比如说我会和一个朋友感到很不同频,就是他说他的,我说我的。如果是和父母的话,我发现不同频的时候我就直接不想说话了,因为说了他们也听不懂。我记得之前面基的时候我有跟一个男生说起我和前任的经历,然后对方马上说“那就是渣男”,然后开始说他是怎么识破渣男的,他和渣男的经历。我感觉他只是在我的话里找到了一个钉子,就好像手里拿着个锤子就看什么都是钉子,然后就在我的话里找到了那个钉子,拿了出来,还放大地看那个钉子,再往里敲。但我想表达的内容并不只有那个钉子。}
	\dialogue{咨询师}{好像对方只看到了他想看到的那个钉子,而没有看到你想让他看到的东西。}
	\dialogue{我}{嗯。}
	\dialogue{咨询师}{那你想让对方看到的是什么东西呢?}
	\dialogue{我}{会想让对方看到更加丰富的自己,想让对方看到自己的各方各面、不同的自我。而不只是对方想看的那个部分而已。}
	\dialogue{咨询师}{而且我在说“啊~”的时候,你也不会想打断我。}
	\dialogue{我}{嗯。因为我会想继续听你说你想表达的内容,想听整体是怎样的,而不是只抓着一个细节不放。我总不可能打断你,然后说:“我觉得你只是看到了你期望看到的东西,你就是想拿理论啊体系啊来套我,看套不套得上”。我会更想去看完整体再看这些细节是否真的那么重要。而且我也不可能让你以后也不“啊~”,因为这些自然而然地表露的细节会让我有一种安全感,就是我能知道你在自然地表露着。而没有了这些自然而然的表露的话,会让我感到不安。}
	\dialogue{咨询师}{好像现在的你更能容纳关系里不舒服的部分了。}
	\dialogue{我}{嗯。现在的我如果有不舒服的部分的话会写出来,在写的过程中我很可能就已经把一些东西给加工了,给心智化了。这样的方式比起出于情绪化地说一些事情更能保护彼此之间的关系。}
	\dialogue{咨询师}{那你会敢直接地说一些可能会让关系受到破坏的话吗?}
	\dialogue{我}{嗯,我会敢。比如说一个我喜欢的男生很少回我微信的时候,我就跟他说:“我感觉你在利用微信来疏远彼此的距离”。我会有意识地出于情绪而说一些话,一些不经心智化的话,但有时候我也意识到自己是不是会说太多这样的话而让关系受损,我是不是应该经过大脑的思考后才说出来。但我想那种不安感还是会在,担心自己说的话会让关系受损,会想我这样做会不会破坏到关系,特别是自己在意的关系、不想关系受到破坏的那些关系,可能就是亲密关系吧。不过通常都是通过写作自己处理了,如果真的想跟对方提出来的话,就会把写完发东西发出来,或者是(把文字)转给对方。}
	\dialogue{咨询师}{好像现在的你找到了适合你自己的方式。但这种对关系的不安全感,特别是对最亲近的人,这会影响到关系的相处。我想我们可以在之后的咨询里多讨论这部分的内容。}
	\dialogue{我}{嗯。}
	\dialoguesepline{咨询师}{“之后”,噢,好的,赶客了}
}

我会想起我和另一个朋友也因此产生了冲突。对方希望我能更直接地向她表达我内心的情感和想法,但我只是把自己内心的事物写了下来就没有直接告诉她了。对方可能会因此而感到很失落。她会期待我直接向她表达那些反感和不适,但我没有,反而是自己消化掉了。可能直接表达内心不愉快的感受和想法对于对方而言是一件再正常不过的事情,但对我来说,这并不是一件容易的事,这需要勇气和冒险、需要一定的安全感、需要面对不确定性。

在咨询结束后,我想记录下回忆稿的时候,其实我是什么都想不起来,脑子是崩住的。那时候的我感受到了一种恐惧和悲伤,恐惧于关系的受损,恐惧于如果直接地表达自己的感受,周围的人都会离自己而去,自己又会孤独一人。那种害怕,会害怕对方会不会无故消失,会不会是因为自己做了什么、说了什么而导致对方无故消失。那种对无故消失的恐惧感好像一直扎根在我和任何人的亲密关系当中。好像越亲密的关系,越会不堪一击,越会无故消失。当然我知道这种想法听起来就很不符合逻辑和常理,但当处于那种无意识的恐惧和悲伤的时候,我确实会有这样的想法。



