\chapter{“是个安于现状的人”}

\ardate{2022-07-25}{L5ETqVw7DqxlQ3QgfVrJFA}





“感觉你比较安于现状?”

“嗯,是个安于现状的人。”

我会感到困惑和愤怒的是,为什么要用“安于现状”来指责他人。当然也有可能是指责者自己本来就不安于现状,但又无法改变,便将“安于现状”套在他人身上。因此,“安于现状”的潜台词可能会是:不思进取或不想改变。同理,“安于现状”似乎也能套在很多固化/稳定的心理/心境状态甚至是心理疾病之上,例如抑郁症、孤独症、焦虑症、恐惧症、强迫症、PTSD、贪食症、厌食症,甚至是人格障碍。

另一方面,指责另一个人安于现状真的对对方有帮助吗?在我看来,这无异于对一个有自杀风险危机的人说:如果你真的那么想自杀,你早就死了,还站在这里干嘛??

在我看来,安于现状是一种能力。在格式塔疗法里有一个体验循环圈,循环圈的最下面一部分便是休憩状态,指的是一个人的欲求得到满足后,这种欲求开始逐渐消退,直至消失不见,而下一个欲求又还没有出现。有的人会难以呆在休憩状态,比如说成瘾\pozhehao{}成瘾于物质、成瘾于人际关系等,有时候甚至只是无法呆在什么事情也不做的状态当中,而一定要找些事情干,即使干的是什么事情本身并不重要,例如没事就刷手机。而当他真的要进入什么都不做的状态里(例如睡前),他便开始焦虑、烦躁地失眠,逼着自己一定要去入睡,因为入睡本身就是一件要去做的事情。

从自体心理学的角度来看,如果一个自体感弱的人停不下来地不断忙于某些事情(例如成为工作狂),那么他有可能是在利用这个过程来防御无事可做时会感受到的虚无感和无意义感。

另一方面,当有的人指责别人安于现状时,他真正想说的并不只是对方真的只是安驻于当下,而是想说对方不思进取、不想改变。我会认为,这样的指责与当代的社会文化价值观有关,例如追求高成就、高收入、高效的社会文化。除此之外,作为一个个体,他想说的似乎是:对方既想要改变,但又改变不了。这让我想到了改变的悖论\pozhehao{}一个人越努力去改变,反而越被卡在原地。

不同的咨询流派有对此的不同解释,例如说格式塔的极性、精神分析的强迫性重复。但我还会想到的是,处于这样一个既想要改变,但又改变不了的人,内心一定充满着困惑甚至是痛苦。例如一个想要自杀但又还活着的人。但这份痛苦往往不被人所看见,比如说当我有伤心的事情想跟身边的某个朋友倾诉时,那个朋友问我最多的提问是:为什么XXX?那为什么XXX??那又是为什么XXX???好像在一部分人的眼里,他们只看见了一个有待解决的问题,而不是看见我这个人本身。

在人本主义看待痛苦的视角里,有一点是:痛苦来源于生活意义的追寻。也就是说,那个被卡在原地并对此感到痛苦的人,内心依然留存着一份对生活意义的追寻,而正是那份意义的破损带来了痛苦。那个被卡在原地的人真正想要追寻的是什么?那份他所看重的意义是什么?是什么让这份意义破损了、让这个人破损了?

我之所以会介意“安于现状”,是因为小时候自己的父母一直在说很多贬低我的话,例如:“你就是那么的没用”、“你就是个废物”、“你就只能这样了”。听多了,那时候的我便认同了,逐渐地不对学习成绩、不对自己的能力、不对身边的他人、不对这个世界抱有什么期望。但现在回想起来,我会认为那时候的他们更像是在将他们“安于现状”的部分投射给我,通过指责另一个人的“安于现状”来缓解他们那难以改变的人生道路的无力感以及从中获得一定程度的力量感\pozhehao{}通过让别人感到无力来让自己获得有力。


