\chapter{自体客体,镜映物}

\ardate{2022-06-18}{9jboVMC-lQJ\_PlnPA6YWQg}


\midnote{以下来源于课程作业(经修订)}

\textbf{试著分析什么是“自体客体”?}

“自体客体”指的是一种共鸣性情感体验,一种个体从诞生开始便终其一生去寻找的体验。当个体获得自体客体后,个体的自体感会得到增强,日积月累下会逐渐形成这个人的自体的内聚性。

在我的个人理解里,自体客体更像是一种镜映物。一个人从诞生开始便需要去找一面镜子\pozhehao{}无论是真实的镜子还是他人的瞳孔\pozhehao{}需要从世界上的另一个客体(事物及他人)身上看见自己的模样,从中确定自己的存在,即自己是个怎样的人。毕竟,如果每个人都是一样的,或者个体除了他自己之外就没有其他人在了的话,那个体也就不再是“个体”了,个体需要找到另一个人和自己不一样的人,但那个与自己不同的人(最初是母亲或养育者)眼中又拥有着属于自己的镜映物(自体客体),那个个体才能知道:这个人眼中的“我”就是我。

但如果个体缺少这样的(自体客体)体验\pozhehao{}从另一人身上确定自己的存在,他的自体感便无法建立起来,因为他从一开始就不知道自己是怎样的,没有一个对自己的自体表象,更没有足够的自体感(对自我的感觉)。

同时我也会想起前段时间我和我的咨询师的一段对话:

\dialoguelistthin{咨询师}{
	\dialogue{我}{……当我跟你这么讲的时候,我是能感觉到我自己的存在感的,我能感觉到自己是存在的。但如果我不说、我不写的话,时间久了,我会感觉自我在消融,特别是孤独一人的时候,消融到不复存在,不知道自己是谁。这也是我想脱单的原因之一,因为好像我一直需要另一个人来确定自己的存在,就像是一个个体需要他人才能确信自己是个‘个体’。就像是萨特写的那个剧本《他人即地狱》,在里面的其中一个场景是地狱里的人没有镜子,所以他们要用另一个人来作为自己的镜子。比如说,你来看看我,我的脸有没有脏,我要涂口红,你来看看我有没有涂歪。但另一个人会说,那如果我不看你呢?如果我说谎呢?你没有了我这个镜子怎么办?所以他人好像充当着一面镜子的角色,而我又不敢相信他人眼中的镜子,因为每个人都有他们自己的扭曲。所以我不断地说、不断地写也写了那么多年了,但我拥有的也只是一些转瞬即逝的存在感,这种存在感并不稳固,而我需要继续去说、继续去写才能继续保持这种短暂的存在感。而当身边没有他人的时候,我就会把这些东西投射到纸张上,不断地去写,从文字里确定自己的存在。}
	\dialogue{咨询师}{从文字里确定自己的存在,这是一种怎样的感觉?}
	\dialogue{我}{可能是一种真实感吧。因为我能往回看自己写的文字,觉得那就是自己。但时间久了再回头看自己写的东西,又会觉得并不是自己,那种存在感又消失了。我也会在想,自我意识究竟是什么,难道只是情绪和想法和念头飘过的地方而已吗?}
	\dialogue{咨询师}{当你说了那么多你的这些感受和想法,我好像比以前更能理解你一点了。我不知道当你说了那么多,你会感觉我有更理解你、更看到你吗?}
	\dialogue{我}{我好像暂时没有这样的感觉,因为我好像一直以来都不期待他人能理解我、能看见我。跟你说了那么多,就好像是写作,我会把属于我自己的部分呈现出来,我做了我能够做的了,至于对方怎么做是对方自己的事情。我也不知道这种一直在写、一直在说的方式是否健康。}
	\dialogue{咨询师}{你能多说一说在你看来健康的方式是怎样的吗?}
	\dialogue{我}{可能更应该说是普通,就是身边的普通人是怎样做到的,怎么做到确定自己的存在,怎么自然而然就知道自己就是自己的。}
	\dialogue{咨询师}{我想,我不确定这是否是你的感受,这只是我的猜想:可能是在你的生活里一直都没有一个能够确定你的存在的人在。}
	\dialogue{我}{Em……很可能是。好像母婴关系就是这样,婴儿就是需要通过母亲这面镜子来确定自己的存在是怎样的。而我好像一直以来都没有这样的一个部分在。}
}

\blockquote{\blockquotesource{白色灯塔先生}{“自我意识究竟是什么,难道只是情绪和想法和念头飘过的地方而已吗”}{2022}}
