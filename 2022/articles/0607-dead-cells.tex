\chapter{魂系游戏,Dead Cells,无数次的死亡与失败}

\ardate{2022-06-07}{Zu3UKj4RSrtWNtvn2\_xt3g}




最近在玩一个游戏"Dead Cells",然后当看到“虎嗅”的短视频「发售一个月销量超 1200 万,《艾尔登法环》让“魂系游戏”的说法更加成立了。什么是魂系游戏?它到底有多难?是什么样的人在迷恋它?》」时,我发现Dead Cells更像是魂系游戏\pozhehao{}弱引导(加深世界的荒漠感)、弱叙事(剧情不是平铺直述的传统故事,而是被打碎的、藏在各种NPC的只言片语中,没有官方解释,每个玩家都是作者)和不可抗的失败(无数次死亡)。这三者导致了玩家会体验到一种匪夷所思的失败感。而确定的失败感和放弃努力的心理状态就是习得性无助。还有一点是,其他游戏里的玩家是世界中心、天选之子,而魂系游戏里的玩家只是个无名小卒,即使从菜鸟成长到灭神者也改变不了历史和现状。

但这样的游戏没有降低玩家自我效能感,反而大幅提升了自我效能感。有人认为魂系游戏是在筛选阿斯伯格综合征和斯德哥尔摩综合征的潜在患者\pozhehao{}低层次的重复,不需要也用不上技术,只需要不断赴死,凭借无意识本能的进步;通过确认偏误,玩家合理化自己的处境,给坏事找一个好的解释,甚至对自己的遭遇和自己的施暴者产生积极情绪。

以上只是短视频里的内容,因为我并不完全是这么看待和认为的。我觉得Dead Cells更像是一个锻练场:在里面,我死了无数次,但每次的死亡都不是真正意义的死亡,因为每次回到重生点的我总会再次踏上赴死的历程,并希望能在每一次的死亡中走更远的路。Dead Cells里有很多路线,但最终的路线都是通往最终的那个大Boss(当然中途会遇到很多小怪和小Boss),而每一次踏上历程的时候都不会知道自己会死在哪个地图。不过,在无数次死亡里,我大概摸索出了一条我最不容易死亡的路线,也逐渐摸索出每个小怪和小Boss的招数和应对办法,以及当一大堆小怪聚集在一切时,如何“破阵”。

我想我之所以会一次又一次地玩Dead Cells,是因为我的生活里本来就充斥着习得性无助,而我也在生活里死了很多次\pozhehao{}不同的自我在不同的人生经历里不断死亡,死了一个又一个,直到没有任何自我的存在。在无数次的死亡后,我会在想,剩余的自己又是谁、又是什么?难道玩家的存在本身只是一个操作着游戏角色的生命体吗?难道玩家只存在于这段历程里吗?如果没有了这段历程,如果哪一次我不再选择踏上这段历程,而是离开了呢?如果我不只是离开了那个游戏,而是离开了这个世界呢?那我的存在又存在于何处、存在于何者?

在一次又一次的死亡里,我并不认为这是习得性无助,而是一种对习得性无助的超越\pozhehao{}玩家在一次又一次的死亡里前往了、抵达了之前从来无法抵达的地方。虽然死亡几乎是每一次的命定结果,但每次死亡都是独一无二的,因为在每次死亡中的极其微小的一部分都在不断构成了我下一次、下下一次、下下下一次所走的道路。看似重复的死亡,但并不重复,因为有一些独一无二的、富有生命的东西在一次又一次的重复地诞生了\pozhehao{}玩家的选择、技术、无意识操作、靠一次又一次地死亡换来的新景色和新剧情。

而且很重要的一点是,Dead Cells是单人游戏\pozhehao{}我不可能通过依靠他人来摆脱一次又一次的失败和死亡。但如果自己真的通过了一个自己之前无法通过的地图,打败了一个之前无法打败的Boss,那不是因为他人的努力,也不是完全是因为自己的运气(天知道我在此处死了多少次……),而是因为自己,仅仅因为自己\pozhehao{}自己的无意识深处是有能力战胜一次又一次的失败和死亡的,至少是当下的这一次。

短视频里还提到了一点:“如果死亡不是失败的标志,那应该如何向死亡赋予意义?”如果游戏里的每一次死亡都不是失败,如果生活里的每一次“失败”都不是失败,那它们又是什么?作为游戏玩家、人生玩家的我们,应该给这样的游戏经历、人生经历赋予怎样的意义和解释?

我对此的解释和意义是:在一次又一次的死亡和失败里,我并没有真正意义上地死亡,而是在无数次的重复、试图完型中,一些连死亡和失败都无法磨灭的东西生长了出来,而我可以尝试将这个东西从锻炼场般的游戏(象征系统)带入到(现实)生活里。

