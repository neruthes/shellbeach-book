\chapter{“如果我向对方投入了精力的话,我就有可能会受伤”}

\ardate{2022-07-24}{\_y4yj2Iz8AqoeCWPyZqvWQ}





进咨询室后,我跟咨询师讲了下那个前几天去了另一个城市的朋友的事情,以及我对他带有的情感\pozhehao{}羡慕、愤怒和被遗弃的憎恨感。

\dialoguelist{咨询师}{
	\dialogue{咨询师}{好像无论是那个朋友,还是最近喜欢的男生,在你看来,好像他们都没有选择你。}
	\dialogue{我}{是啊,他们都选择了去和另一个人发展亲密关系。就连前任也有对象了。}
	\dialogue{咨询师}{这会给你怎样的感受吗?}
	\dialogue{我}{会感到蛮孤独的,就好像自己在被不断地遗弃。同时也感受到那种被遗弃的憎恨感。}
	\dialogue{咨询师}{我好像还蛮少听你表达憎恨的。}
	\dialogue{我}{我对前任也会有憎恨。可能是因为这个(憎恨的)部分不多,所以可能在咨询里我提起来的次数也不多。不过我在日常生活也蛮少表达愤怒的。表达愤怒会让我感觉到……自己更加处于当下。前几天,那个自己有好感的男生没有回复我消息了,所以那几天都会很难过,甚至意识经常走神,但好像最近几天又会感觉到特别的愤怒。这种愤怒也有一部分会指向那个有好感的男生。}
	\dialogue{咨询师}{那这种愤怒会怎么在你的生活里体现吗?}
	\dialogue{我}{当我表达愤怒的时候,我会感觉自己更加处于当下了,就不是像前几天那么的难过。那时候我总是无时无刻在走神,甚至只想躺在床上,什么也不做,没有任何活力。但表达愤怒的时候,我会感觉自己更有活力了。}
	\dialogue{咨询师}{多说一点这种愤怒。}
	\dialogue{我}{我会想起上周和亲戚的饭局。在饭桌上我少有地去表达我自己的观点,然后马上就遭到攻击了。那时候我是说叔叔阿姨他们的孩子(也就是我的表弟),他倒茶的时候分不清谁的杯子里面的茶是什么颜色,就是会将不同的茶混在一起倒茶。他拿茶壶的样子也不稳,而他母亲(就是那个阿姨)会跟在他后面,害怕他烫伤。然后我就说:“他迟早要学会这些东西的”。然后那个阿姨就说:你是在说风凉话,本来就是应该小的去倒茶,为什么你没有去倒茶?在那一刻,我会感觉到蛮受伤的,然后我马上去心智化:为什么她会认为我是在说风凉话?为什么我会感到受伤?后来我发现那种受伤的感觉背后有一种幻灭感\pozhehao{}好像我依然希望自己能够从他们的家庭里获得一个好的父母,但当被她攻击的时候,我就意识到这个幻想破灭了,而且他们也有他们自己的孩子了。当我意识到自己为什么难过、悲伤的时候,我就撤回了\pozhehao{}甚至在那一刻我是整个人都呆住了、待机了,然后我的意识开始不断心智化。当看清楚自己的部分之后,我就不想去回应他们了,也没有想向他们表达愤怒或攻击,也没有去骂他们。
		我知道有的人是会反击的,比如说小学和初中的那些霸凌者,我知道他们的父母会打他们,而他们会和他们的父母开打。他们是那群反击的人,而我是那群不反击的人。如果反抗的话,自己更可能会受伤。而且如果我在父母面前、在亲戚面前表现出自己的脆弱,我只会被他们打得更惨、骂得更惨。所以小时候的我没有去反击,而是先保护好自己,可能因为自己本来就已经够脆弱的吧。
		所以我也不期望在这个家庭里去得到任何东西了,整个人就撤回了。也许我能从其他人身上获得我想要的东西吧,至少是在这个家庭之外的人。}
	\dialogue{咨询师}{当你说撤回的时候,会有怎样的感觉吗?}
	\dialogue{我}{会感到蛮孤独的,还有一点点悲伤。饭局的那一刻会感到蛮悲伤,但现在只有很少的一点点了。}
	\dialogue{咨询师}{那和父母的相处呢?也会不敢表达吗?}
	\dialogue{我}{倒是没有,在父母家我会敢于表达自己的想法和看法,但那只是一种有话就说、下意识随便说说的状态,而且我也不在乎我说是一些什么,也不怕他们会攻击我。如果他们攻击我的话,我能用各种方式将这种攻击反击回去,让他们说不出话来。}
	\dialogue{咨询师}{听起来好像你不对他们抱有什么期望。}
	\dialogue{我}{是的。因为这个环境让我感到我是得不到我想要的东西的\pozhehao{}得不到镜映,得不到共情。这也会让我想起大概是小学一、二年级的时候,有一次我也是在亲戚饭桌面前,我表达了自己的观点,然后就被整桌人轮流骂。先是外婆骂我,然后是叔叔阿姨家那边骂我,最后是父母家那边骂我。当时就会有一种感觉:没有人会站在我这一边,只要我表达了自己的观点,我就会被攻击。那一次我被骂得蛮想哭的,然后我就去了洗手间,洗手照镜子的时候,我看见镜子里面自己在笑,而且整个脸部是僵硬的,脸部的肌肉已经绷紧得拉不下来,而我开始认不出镜子里面的那个人是谁。}
	\dialogue{咨询师}{就像是你撤回了。}
	\dialogue{我}{嗯。当我的身体没有办法撤回的时候,我的意识就撤回了。}
	\dialogue{咨询师}{我留意到你皱眉了。}
	\dialogue{我}{因为我在思考。好像我在思考的时候,我就会皱眉。但好像在你看来,你能留意到的也就只有皱眉了。}
	\dialogue{咨询师}{当你看到我只能留意到你皱眉的时候,你会想到些什么呢?}
	\dialogue{我}{我会想到身边也有几个人说我不怎么表达情感、不怎么笑。其实现在我也留意到我的语音语调很平淡,很难用一些非言语的方式去表达情感。有点像是小时候的我将自己的情绪封闭起来后,当自己再尝试去打开时,我好像就只能通过言语或文字去表达,而非言语的方面依然是封闭的,比如说语音语调、面部表情。我会想到有的人会在非言语信息里透露出一些信息,比如说当那个人还在激烈地讨论着理论、逻辑的时候,我会跟他说:“听起来你好像有点悲伤”、“听起来你好像有点生气”,就像是降维打击。但我好像就连一些非言语信息也表达不出来,就像是以前封闭情绪的一些后遗症,而我能够做的好像就只有用言语或写作的方式表达出来了。}
	\dialogue{咨询师}{当你这么说的时候,你会感觉到一些什么吗?}
	\dialogue{我}{我会感觉到有点悲伤,还有一种隔阂的感觉,就好像我的内在有很多丰富的情感,但那些情感没有办法跨过这个身体的隔阂表达出来。不过我也不想对此做些什么,因为小时候会很刻意地封闭自己的情感。那时候用的方法是每当自己感受到悲伤、难过、痛苦的时候,就深呼吸\pozhehao{}吸气,然后屏气数5秒,然后再呼气,而屏气的时候不只是屏住呼吸,同时也将自己的情感压抑下去。每当感受到难过的时候,就一直这么做,直到情绪被消失得差不多。后来只要一屏气,我就感受不到情感了。而现在我不想对此做些什么,因为我不想再控制自己的身体和情绪了,就让它顺其自然。不过,当然遇到一些自己有好感的人,我还是会刻意把自己开心的那一面带出来。但是时间长了,如果我觉得累了的话,我就不会太刻意那么做了。}
	\dialogue{咨询师}{那你现在是怎么处理悲伤的?}
	\dialogue{我}{现在我可能会去写作,可能会去找朋友聊天,也有可能去打热线。不过主要还是写作吧。}
	\dialogue{咨询师}{好像当你在写作的时候,你也在梳理这些情绪。}
	\dialogue{我}{嗯。}
	\dialogue{咨询师}{我会留意到有好几次咨询,我都感觉你像是在脑海里写作。}
	\dialogue{我}{可能因为这是我最熟悉的方式吧。而且我也没有办法改变他人,比如说前几天离开了这座城市的那个朋友,我也没有办法去改变别人的决定、改变别人的命运,这条路终究是要对方自己走的。我能做的也就只有处理我自己的部分,而且处理自己的部分是最快、最有效地让自己撤回、减轻痛苦的方式。}
    \dialoguesepline{咨询师}{(我看了看窗外)}
	\dialogue{咨询师}{你留意到些什么吗?}
	\dialogue{我}{我会留意到外面天黑了。有时候我去逛江边或海边的时候,会觉得天好蓝,风景蛮好的。但天黑之后,就会感觉外面的一切都消失了,就算是现在我也会觉得外面的一切都消失了。}
	\dialogue{咨询师}{我记得你在大学时写的那个小说也是,只要灯光还在,那些人就还在,但一旦灯光消失了,那些人也就消失了。}
	\dialogue{我}{是啊,如果我不向他人投注精力,那么他人在我眼中也消失。前一段时间还撤回过头了,以至于自己不在乎世界上的任何事物和人,觉得任何事物和人都毫无意义。}
	\dialogue{咨询师}{你说前一段时间,那你最近是怎么样的?}
	\dialogue{我}{最近我开始捡起来一些在我看来很微不足道的东西,比如说和朋友见个面或者是看一本书或者是看风景。对我来说,这些事情的意义很小很小,但还是有的。现在那个在我看来一切都毫无意义的自我好像被整合了,被整合进之前那个觉得一切都很有意义、很丰富的那部分自我。}
	\dialogue{咨询师}{当你这么说的时候,你会有什么感觉?}
	\dialogue{我}{我会感觉现在的状态更加的平衡,就是我不需要过多地去考虑一件事情是否有意义。如果一件事情没有意义,我可能就不会再做了;见一个人如果没有什么意义,我可能之后就不会再见了;或者是将无聊的感觉告诉对方,看会发生些什么;如果是工作上觉得无聊的话,我就会考虑是不是应该换一个职业方向了。}
	\dialogue{咨询师}{我会留意到你说“整合”、“平衡”,好像之前蛮少听见的。}
	\dialogue{我}{(沉默)就像是上一次咨询我发现我还是想在结束时跟你说“下周见”,但之前我不想那么做是因为我离开了咨询室的空间才跟你说“下周见”。有时候走廊上还会有人在,就会给我一种被暴露、不安全的感觉。所以我想之后我还是说“下周见”,但是是在咨询室里说。
		我会留意到你的嘴部做了一个这样的表情(露出牙齿地一字往后缩),而且留意到你的眼角有一点泪水。但我想不到为什么你会这么做,好像我无法将这两者拼凑起来。}
	\dialogue{咨询师}{当你看到我的眼角有点泪水的时候,你会有什么设想吗?}
	\dialogue{我}{没有,也就是因为我想不到,所以才会觉得奇怪。这可能也是一种安全感的来源吧,需要知道对方为什么会这么做、对方在想什么。}
	\dialoguesepline{咨询师}{(我看了看窗外)}
	\dialogue{我}{然后我留意到你看了看时间,我会猜你在想剩余的时间是否应该展开一个新的话题。}
	\dialogue{咨询师}{好像一部分是你会去猜我想到些什么,另一部分是你不想去猜。}
	\dialogue{我}{其实我也会有猜想,只是我不太想去承认。我猜想你流泪是因为刚刚我说结束咨询的时候我还是说“下周见”吧。可能是这种人际距离的拉远又拉近让你流泪,但另一方面我不想去承认这一点是因为我不太相信我的存在会令另一个人在乎、会令一个人为我而流泪。因为如果去承认这一点的话,就意味着一种精力的投注。如果我向对方投入了精力的话,我就有可能会受伤。}
	\dialogue{咨询师}{是啊。}
}

