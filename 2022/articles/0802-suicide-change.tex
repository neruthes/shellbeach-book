\chapter{“至少自杀本身也是一种改变”}

\ardate{2022-08-02}{vzkTSqpJRGGMeoYLG3iPVQ}




最近在社交软件上看见了其中一个前任男生的动态。

那个男生是我在读大二时谈的一个对象,他一直在我现在所处的城市里生活,而那时候的我则是因为要读大学而离开了这座城市而去了另一个城市读大学。而且那时候的我还在离开这座城市读大学时和初恋分手了,所以每当回到这座城市时,我都会感到很伤心和难过,每次都会想到初恋。那时候我想找一个人在自己回来这座城市时陪伴着自己,让自己不需要每次回来这座城市都会回想起初恋,所以就和那个男生试着相处了一段时间。

但在相处的过程中,我一直有一种不被看重、不被在乎的感觉。其中一个让我印象深刻的经历是,有一天我去他家和他一起打游戏卡牌。那天我们打了一个下午的卡牌,每一局都是他赢,最后他赢了二十多局。在打牌的时候,我一直在想:他会不会有哪一次会让我赢一次,但并没有。他反倒是一直很沉浸于那种胜利感,丝毫不会察觉到和他打牌的那个人是他对象,甚至不会察觉到我是一个有情绪的人。

在那之后不久,我就和他分开了。

最近我看见他的动态里写,他男友失踪了一个多月,他以为他男友是在搞冷战,便默认和他男友分手了。但那天他男友的家属在微信上找到了他,因为他男友将他设置为“亲人”。通过他男友的家属,他才知道他男友自缢走了,因为他男友和同学玩真心话时将秘密说了出去,被网暴得抑郁。他说他很自责为什么他没有重视他男友的抑郁症状,甚至是失踪了一个多月都没有去他男友家找他男友。

我会想到,如果那时候的我没有选择和他分开,甚至一直都还是他对象的话。当我抑郁了、想自杀了,而如果我真的那么做了,他可能也会默认为是冷战并分手了。在以前那段自杀冲动很强烈的期间,我会有想过:如果我消失了,这个世界会变得更好,身边的人也会过得更好。但当现在我看到他的动态里所写的话,以及他在社交软件主页简介写的“有没有约会的呀”,我会在想:但有的人未免也过得太TM好了吧……

今天在一个群聊里看见有人谈到改变,大意是:“如果一个人真的受不了的话,就肯定会改变,不然其实难受了一阵子还是会继续回到原状。不改变是因为不管有多难受,那份难受还是能够承受下来的,而如果选择改变的话,改变本身要付出的代价反而会更加沉重、更加难以承受。”我还蛮认同这段话的,因为我之前上的格式塔课程也有相似的观点:一个人之所以没有改变,是因为在极性上还没有走到尽头,还没有彻底地痛苦和绝望。然后我在那个群聊里回应道:“(一个人)真的受不了也许会自杀,不过至少自杀本身也是一种改变。”

在我看来,那个男生的男友的自杀本身也是一种改变,而那个男生倒是没有多大改变。


