\chapter{自我探索 | 15}

\ardate{2022-01-09}{NiD5s1cZtoAeudpHelPJTQ}


回到咨询室里的我又处于无话可说的状态,咨询师再次提及之前我会准备一些材料带入咨询,我说:“现在的我觉得,让一些东西自然而然地发生在当下,这些自然而然的东西好像比我之前所刻意准备的东西更真实、更自然、更处于当下、更重要、更有力。

这也会让我想到,在人际关系里,有的人会说很多关于TA自己的事情,有时候我也会说很多关于我自己的事情,这样的情况就像是自己准备了些材料带入聊天,只是各讲各的。但如果是发生在当下的真实互动的话,彼此都会参与进来,而不只是一个人的事。当我一开始学咨询时,我会用各种咨询理论框架或视角来看身边的人,但现在我更多会用自己的直觉\pozhehao{}自己和对方相处时自己的感觉和自己所觉察到的事物\pozhehao{}来看对方的肢体语言,来判断对方是否对自己所说的话感兴趣。”

咨询师回应说:“好像你现在更多用自己的直觉而不是咨询的理论来看身边的人,你会看到些什么吗?”我回答道:“比如说我在微信群里看见有群友评论之前另一个微信群有一个很“火爆”的前群友。那个前群友将旧群的所有人都赶跑了,都到了没有他在的新群。他们评价他(那个前群友)是“战争之王”,而有的群友说:‘好像能在他的话语里感觉到孤独’,另一个群友则回应道:‘可他偏偏成为了战争之王。’这会让我想到自体心理学里的水平分裂和垂直分裂,但如果不放在咨询的理论框架视角来看,我好像能看见一个害怕受伤、害怕被攻击的小孩,他害怕他人靠得太近,所以会把他人推开,但又在把他人推得太远后感到孤独。”

咨询师问:“既然你选择将这个故事带入咨询,我在想,你是否也在说有关你自己的经历。”我回答道:“我相信共情总需要基于一定的个人经历之上。当然,我并没有成为‘战争之王’,但我曾经在学生时期会因为他人靠得太近而将他人推开,并将周围的人保持在一定的安全距离里\pozhehao{}既不会太远、也不会太近。但当最近在冥想体验营里我需要写下25个感恩的人(或者更应该说是生命里的重要他人)时,我的第一反应是:好多啊!我发现自己在过去的学生时期里并没有真正和他人产生过连接。”

\tristarsepline

\blockquote{
元旦假期,我去了另一个城市找朋友L玩。在假期的第二天,我和朋友和朋友L所认识的另一位男生M一起约去艺术展观展。

我们先是吃了顿午饭,午饭的大多数时候都是我听着朋友L和男生M在聊天,然后我们去了那个艺术展。在逛艺术展时,朋友L在男生M不在附近的情况下跟我说:“看完展我们就回去睡个午觉,然后晚饭叫个外卖。”后来,逛展越逛越晚,在快到下午五点时,男生M提议我们去吃晚饭。我知道朋友L只想回家休息,然后我看朋友L对此是否会有什么反应,但朋友L只是把手机摆在脸前玩,没有什么回应。然后男生M便开始在手机上搜附近的餐厅。我想到朋友L想早点回家,然而现在已经离朋友L的设想(看完展就回去睡个午觉)已经相差甚远,所以我在手机上找了几家艺术展附近的餐厅向男生M提议。但男生M并不想去那些餐厅,然后找了一家需要坐地铁去的有一定距离的餐厅。在去餐厅的路上,朋友L依然只是在玩着手机。

当去到那家餐厅吃饭时,朋友L依然全程用手机遮挡着脸,全神贯注地玩着手机……

聊着聊着,男生M说他想去看一下附近的一家有猫的饮品店,我问朋友L想去吗,他说他不想,但声音小得只有我能听到。吃完饭后,男生M在找那家饮品店在哪,朋友L在跟着他走的时候突然大声地说了句:“我不要跟你玩了!”然后一个人转身走向离开商场的方向。我和男生M跟了过去,男生M问我他怎么了,我说:“他可能早就想回家了。”

在地铁里,和男生M匆匆道别后,我和朋友L坐在地铁里,我问朋友L:“你会想他(男生M)在定行程的时候多关心一下你吗?”朋友L回答说:“他总是会在计划好的行程后加更多东西,我一开始就说好只是约艺术展。(他)不像你,你问了我是不是想去饮品店。”

……朋友L当时的回答“我不要跟你玩了!”听起来更像是一个小朋友会说的话,或者说这是他内心更偏向小朋友的部分、小朋友的自我所说的话,我猜想这可能是因为朋友L在那个当下感受到了小时候所曾经感受到的感觉,猜想到小时候的他说不定经历了某些创伤,一些不被他人关心和重视甚至是不被他人“看到”的创伤。同时,把手机摆在脸前来回避与外界的互动的行为也很像小朋友生闷气的表现。

\blockquotesource{随笔 | 元旦,生气,冲突,意义}{白色灯塔先生}{2022}
}

后来,我又说道:“又比如说上周周末我去了另一个城市找朋友L玩。从那个城市回来后,我发现自己感到悲伤,当时我在想这份悲伤背后是不是因为和那个城市的男生L的分离,但我发现自己无法用理智去看清楚这份悲伤背后有些什么。所以我在冥想的时候,试图将这份情感带入到那片蓝天之下的平原(一个我在冥想时经常用于意象化情感的场景)里,我好像在平原上看见了一个小孩,我会想要去拥抱那个小孩,或者想牵着他的手向前走。当冥想结束后,我意识到自己可能在朋友L身上看见了过去的那个我\pozhehao{}那个内心小孩。这会让我感到警惕。如果用咨询的话语来说,我会担心这是否是一种认同或投射。但如果用非咨询的话语来说,这种情感似乎更像是一个普通人对一个小朋友的关爱,甚至是一种拯救之情。但我并不想陷入其中。”

咨询师回应说:“好像你发现你有能力去拯救他,但又能不陷进去,和他保持一定的距离。”我回答说:“嗯。因为这毕竟是他的议题、他的创伤,我能做的只有陪伴在他身边,但要跨出改变的那一步的终究只能是他自己。他终究要去面对这一步。

我也会想到,两年多前快和前任分手时,我逼问他为什么当时要选择和我在一起,他说是因为对我产生了保护欲。现在的我想到,那时候的他是否也像是现在的我,在对方身上看见了曾经的自己,想要守护对方的那个部分。但这种处于保护欲所做出的行为,对我而言反而造成了更大的创伤,甚至让我的自杀意图又上了一个台阶\pozhehao{}一个更加绝望的台阶。其他人的自杀意图可能只是:‘我很想死’,但我的自杀意图会加入很多有关虚无和无意义的元素,会更加绝望。而且每次亲密关系破裂都会让我的自杀意图上一个台阶,一个我无法回到开始亲密关系前的状态的台阶。如果再经历一次,我会害怕自己无法阻挡自己的自杀意图。

而且,我并不想成为又一个前任、又一个初恋、又一个母亲。”

\tristarsepline

在短暂的沉默后,我继续说道:“这也会让我想起以前和前任、和初恋的经历。和他们的经历就像是坐过山车,有点像是双相障碍患者会跟随着自己的情感波动而波动,而我就像是在坐过山车:先经历高峰,再落入低谷。”

咨询师回应说:“那种坐过山车的感觉会是一种怎样的感觉?当时准备坐过山车时,你会有怎样的感觉?”我回答道:“一种get high的感觉,就像是买醉。这种感觉背后也有一种熟悉感,好像我又能赶上一趟过山车了,又能开始一段难忘的历程了。这好像能为我的平淡生活带来something new,但当我这么说时,我也意识到,这并不是something new,而是something old,一种旧的模式。当然我也会想到,这种重现是否也是出于我的潜意识想要通过不断地重复来冲破那个原有的困境,毕竟只需要一次的成功便足够了。这似乎也是人类的本能之一。我也会感到一种悲伤,一种要放弃自己所熟悉的事物(模式)的悲伤,因为无论好的部分也好、糟糕的部分也好,这样的经历依旧是我会深深铭记的经历。我会觉得很遗憾,这一切本可以再次发生。不过,当我看到喜欢之情背后的拯救欲后,我开始怀疑这是否真的是自己想要的。”

我沉默地思考了一下,然后继续说:“我也会想到,之前之所以会进入亲密关系,背景是那时候的我正处于生活的变迁:高中毕业和大学毕业,而现在的我并没有在经历生活的变迁,所以没有了这个因素来push我去进入一段亲密关系。或者更应该说,现在的我不需要利用一段亲密关系来逃避以前的我所不敢去面对的事物,比如说自己对生活的变迁所需要承担的责任、自己想要成为一个怎样的人、自己想朝着哪个方向走。

\tristarsepline

我觉得去看见自己所害怕面对的事物蛮重要的。在和前任关系破灭的这两年里,我通过自我探索来逼自己沉下水面去看自己所害怕面对的事物。我想起我现在在上的人本课程里,有个来访者形容咨询的过程就像是咨询师把来访者按下水面,在受不了的时候又把来访者拉起水面透透气,然后又按下去。这让我想起游泳教练:他会把学员踢下水,然后在快溺水的时候提起来,然后又压下去。这种方式还蛮残忍的。但另一方面,在这两年里,我开始知道自己有这份力量,这份将自己按下水面去面对自己所害怕面对的事物的力量,而且我好像也看到了越来越多的自己所畏惧的事物。”

咨询师问:“这会给你一种怎样的感觉?会是轻松感?”我回答说:“嗯,其实还有掌控感,一种能掌控彼此的关系的掌控感。和前任在一起时的我通过亲密关系来回避那些自己不想去面对的事物,然后当关系破灭后,我发现自己所恐惧的事物早就把自己包围了起来,就像一团迷雾,我逃不出去,也没有人把拉我出来。而当我真正去面对自己所害怕的事物、去看清它们原本的样子后,它们回到了原来的样子\pozhehao{}只是一个有着固定的形态和形状的死物,而不再像是一团活物一样将自己给包围。”

咨询师再次问:“那这种活物变回死物的过程,会让你感受到些什么吗?会是安全感?安心感?”我回答道:“会是掌控感。因为我不再像以前和前任的关系那样,彼此都被卷了进来,彼此都逃不出来了。如果用精神分析的术语,那会是一种重现/重演,就像是彼此一次又一次地重演着同一份剧本,陷入一次又一次同样的争吵。但现在我可以在半只脚踏入重演的剧场时把脚收回来,和那个剧场保持距离,和那个男生L保持距离。”

咨询师微笑地看着我,然后我回顾了下咨询的过程,继续说:“好像在咨询的交谈过程中,我越来越确定我是个怎样的人、我想要的是什么。”咨询师的笑容越来越大。

\tristarsepline

\useimg{aimg/2022-0109-1.jpg}

\citebook{The Interpreted World}

但之后咨询师的笑容开始褪去,因为我继续说:“但这也会让我感到警惕。我想起前一两周我看的一本关于现象心理学的书,里面有写到现象心理学与人本心理学的区别:人本心理学更强调自我实现等一些看似很美好的东西,但现象心理学更强调一些无法改变的东西、一些更为残酷的现实、一些身为一个存在所必须去面对的现实,比如说the inevitable incompleteness(不可避免的不完整性)。  

越来越确信自己想要成为怎样的人也好,自我实现也好,但我会警惕,自我实现这一看似美好的事物的背后、的另一面会是什么?”我沉默地思考了一会儿,然后继续说:“我好像暂时还看不见这个部分,看不见另一面会是什么。”听到这里的咨询师打算说点什么,但被我打断了:“我好像隐约‘看见’了些什么。”我闭上双眼,双手向前“触摸”,然后我“看见”了。我睁开双眼,看着眼前的咨询师,说道:“我‘看见’了,我好像要放弃通过他人来与自己的内心小孩产生连接,甚至是纠缠在一起。我感到悲伤和孤独,因为我好像无法在这个世界上再看到另一个像是自己的内心小孩的人,因为每个人都是独一无二的个体,我也是。这可能也像是一种存在隔阂,我永远无法触碰到对方真正的存在,我也永远无法通过他人来与自己内心的小孩产生连接。”

咨询师联想起之前我带入咨询的短篇故事《短篇故事 | Aboard, Tram, Shell Beach》,然后说:“我会想起之前你带入咨询的那个短篇故事里的欧文之死,我会想,这是否和结局里欧文死掉的那个曾经的自我有关联?”我知道我要否定咨询师的联想,但我思索了一下要怎么回答才既不会直接否定对方,又能够继续推动咨询。我说:“Em……欧文死掉的那个自我更像是他想要放弃的那部分自我。但我并不想放弃自己的内心小孩,因为我可以通过他来感知到很多丰富的情感。我知道我可以通过很多自己的方式来与自己的内心小孩产生连接,比如说冥想和写作。但我不能通过他人来与自己的内心小孩进行连接。我不应该这么做。”

咨询师回应说:“好像你不会再带一些东西进去关系里,比如说投射和认同。关系回归到了关系本身,你也不再把他人当作连接自我的桥梁或管道。”我有点诧异于咨询师的洞察,并回答:“嗯,是这样的。当你这么说,我也会想到,之前的咨询里,我刻意准备东西(材料)带进来,那时候的我是否也把你当作了连接我的自我的桥梁或管道。”我感受了一下,然后继续说:“嗯,我想是这样的。而现在(咨询的进程)自然而然地发生,好像也和以前的(咨询)过程没有多大区别。但我觉得我们(现在)彼此的关系更像是普通的人际关系,而不是(曾经)刻意的咨访关系。”

咨询师回应说:“嗯,当你不带准备地进入咨询,我能感觉到一种轻松感。”

\tristarsepline

在咨询结束后,我意识到咨询师的笑容之所以褪去,可能是因为我对咨询进程的转向\pozhehao{}从越来越确定“我是个怎样的人、我想要的是什么”转向存在既定(the inevitable incompleteness)\pozhehao{}就像是之前的那个比喻:把自己按下水面,去看自己所恐惧的事物。

另一方面,我开始有点怀疑咨询师是否知道了我的公众号,因为TA的引导“既然你选择将这个故事带入咨询,我在想,你是否也在说有关你自己的经历”很像是我不久前所写过的一个技巧:

% 3

\blockquote{
最近一周分别和两个互不相识的男生聊天时,我都发现他们都会在讨论关于他们自己的价值观时,会说:你怎么样怎么样。或者是在表达观点时省略主语,听起来像是在囊括彼此的共同看法(把我卷入他们自己的观点),比如说:“很多时候,不是……,而是……。而……,真的就是……,而……。我相信你……,而这不是……。”

但问题是,我和对方的观点并不一致,但我并不想去否定对方的观点,我只是不喜欢对方将他自己的预设和价值观/恋爱观/世界观等投射到我身上。

后来,我逐渐摸索出了一个屡试不爽的回应:“我觉得你好像是在诉说着关于你自己的故事。”这时候的对方便会开始回顾他自己刚刚所说的话里,有多大程度是受限于TA自己的主观性、过往经历和认知的。同时对话的焦点也从我身上转移到了对方身上。

\blockquotesource{零碎的想法 | 30}{白色灯塔先生}{2021}
}
