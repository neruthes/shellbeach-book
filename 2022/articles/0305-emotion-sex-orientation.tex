\chapter{情感,性取向}

\ardate{2022-03-05}{esnmg7XwPuu3KWUArsrscA}


在昨天接电话热线前,我感觉到一种很强烈的焦虑,感觉自己控制不住呼吸了,呼吸紧促地快要过度换气了。这种焦虑感来源于之前接热线的经历触发到了我对自己和女性逐渐变得亲密的抗拒。这种抗拒感让我联想到我和我妈的相处,触碰到了我对我妈的那种本能性的排斥和回避。这种本能性的排斥和回避应该也和我的性取向有关。还在读小学时,我便开始知道自己的性取向。

我记得大概是小学五、六年级时,那时候电视上在播着一部恐怖电影《恐怖蜡像馆》,其中一个片段是一个男生被药物麻痹了,任由另一个男的将他的衣服剪开,缝合伤口,脱毛,然后将他固定在座椅上,将喷蜡洒在他的 naked body 上,将他做成活人蜡像。小时候经常看恐怖电影的我在看到这一幕时,除了觉得恐怖外,那时候的我还有了生理反应,甚至想向电影里的那个无法动弹的男生施虐。

但对于女性的 naked body,我并没有想要对此施虐的欲望,而只是将其视为一团肉罢了。我想我之所以会把女性的 naked body 视为一团肉,是因为小时候和我妈相处时,我也会把我妈的身体看成是一团肉,一团包裹着无尽的恨意的肉。如果没有了那一团肉,里面的那些憎恨和暴力就会统统释放出来,就像是电影《寂静岭》里,那个母亲在教堂里被教主刺伤后,随着体内的黑色血液滴在教堂的地板上,血液里的黑暗和火焰开始吞噬着周围的一切。

在性取向方面,与其说我喜欢男性,倒不如说:比如男性,我更憎恨和排斥女性。这很可能是还在读小学的我无意识地将自己对母亲的憎恨和排斥转移(移情)到了其他女性上。

不过,在之前接热线的那个当下,我并没有感受到以前和我妈相处时的那种对那一团肉所包裹的憎恨和暴力而感到的恐惧、排斥和憎恨。我在那个当下仅仅感受到逐渐增加的亲密感,而焦虑感则是之后才出现的情感。那份焦虑,并不只是焦虑于自己和女性的亲密,不仅仅因为这可能会触及到我对母亲的憎恨和暴怒,还因为一些更深层的恐惧。

其实我并不对除了我妈之外的女性感到憎恨和暴怒,not anymore。现在的我比小学的我更能看清他人(无论对方是男性还是女性还是其他生理性别)背后的那个真实的人。而我会真正喜欢上的,是生理性别背后的那个真实的人,无论对方的生理性别如何。如果是这样的话,这会给我带来更深一层的恐惧,那就是:如果我会在情感上爱上一个女性,那么我的性欲是否也可能会导向一个女性,而我的身体又是否能接受异性性行为?如果是的话,那我是否不完全是一个同性恋者?

这让我感到很恐惧,因为过往的自己的情感和性方面一直很确信自己是个怎样的人,而现在,I'm not sure anymore。I'm not sure of anything anymore. 我觉得我自己好像在被一点点地消融。随着对自己的情感越来越敞开,我似乎也不再像以前那么确信自己一定就是一个怎样的人了。
