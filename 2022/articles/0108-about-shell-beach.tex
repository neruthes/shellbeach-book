\chapter{About Shell Beach}

\ardate{2022-01-08}{lYugyqzQRWtJ6MpX4W4zCA}




% Original 白色灯塔先生 Shell Beach 2022-01-07 16:29

这个月的月初,我就马上将公众号改名了(公众号改名只能在一个自然年里改一次)。

Shell Beach来源于1998年的一部电影《移魂都市》,我在去年四月也写过一篇有关《移魂都市》的writing。

\blockquote{

《移魂都市》的剧情大致是(以下涉及剧透)外星人绑架了一群地球人,将他们放置到另一个地方去开展实验。外星人在每天准时12点时让每个人陷入沉睡,交换人类各自的记忆,改变城市的布局,随后观察结果。在到了12点时便再次交换人类的记忆,一直持续下去,通过实验来研究到底是什么使人之为人(what makes human human)\pozhehao{}这样的表达或许能用“人类的本质”来形容。

其中一个外星人注入了人类的记忆(即使这会逐渐导致那个外星人的死亡),因为它想知道人类拥有着自己的记忆的那种感觉是怎样的。

\dialoguelistthin{外星人}{
\dialogue{Alien}{We're very lucky, when you think about it, to be able to revisit those places which have meant so very much to us.}
\dialogue{外星人}{我们非常幸运,能够重游那些对我们而言意义重大的地方。}
\dialogue{Human}{I thought it was more that we were haunted by them.}
\dialogue{人类}{我还以为我们更像是被这些阴魂不散的地方所萦绕于脑际。}
\dialogue{Alien}{Perhaps. But imagine a life alien to yours, in which your memories were not your own but those shared by every other of your kind. Imagine the torment of such an existence \textendash no experiences to call your own.}
\dialogue{外星人}{也许吧。但想象下有这么一种对你们而言如此陌生的生活:你的记忆不再属于自己,而是在每一个同类中共享。想象下这样的存在是如此的折磨\pozhehao{}没有属于自己的经验。}
\dialogue{Human}{If it was all you knew, maybe it would be a comfort.}
\dialogue{人类}{如果这就是你所知道的一切,或许会是个慰藉吧。}
\dialogue{Alien}{But if you were to discover something different, something better\ldots}
\dialogue{外星人}{但假如你体验到了一些不同的东西,一些更好的东西……}
}

外星人正在灭绝,而它们所寄望的出路就在于探索人类的本质,融入人类。从上面的对话来看,人类与外星人在意识上的本质区别在于,人类仅仅拥有个体的意识,而外星人则是无法分离地同时拥有着集体意识和个体意识(“你的记忆不再属于自己,而是在每一个同类中共享”)。外星人所提及的“更好的东西”恐怕就是人类的那属于自己的经验吧。

\dialoguelistthin{外星人}{
\dialogue{Alien}{I'm dying, John\footnote{A human.}. Your imprint is not agreeable with my kind. But I wanted to know what it was like, how you feel.}
\dialogue{外星人}{我快死了,约翰。我们的存在并不允许注入记忆。但我想知道拥有属于自己的记忆是一种怎样的感受,想知道你的感受。}
\dialogue{John}{You know how I was supposed to feel. That person isn't me. Never was. You wanted to know what it was about us that made us human, you're not going to find it in here\footnote{John was pointing his head.}. You went looking in the wrong place.}
\dialogue{约翰}{你所知道的是我本该感觉到的感受。那个人并不是我,从一开始就不是。你想知道是什么使人之为人,你是无法从这里找到答案的\footnote{约翰正指着他自己的脑袋。}。你们找错地方了。}
}

约翰是整座被用于实验的城市居民里唯一一个最先进化出调谐能力(一种能够改变实体的心灵能力)的人类。在没有被注入的记忆下,约翰的记忆是空白的。在这样的状态下的约翰开始探索这座城市,并最终将外星人击败,夺取了外星人创造这座城市的机器。他能利用外星人的机器去创造自己的世界,只要能想象到,便能创造得出来。然而他创造了一个“贝壳海滩”\pozhehao{}那是一个他自从醒来就一直想去的一个地方,以及创造了他所遇到并喜欢上的女生安娜在那片海滩上的再次相遇。即使约翰失去了自己的记忆,他依然拥有着“使人之为人”的那份人类的本质,他所创造的贝壳海滩以及与安娜的相遇便证明了这一点。

\blockquotesource{越过山丘,遥远的前方}{白色灯塔先生}{2020}
}

\tristarsepline

Shell Beach(贝壳沙滩)是约翰自从醒来就一直想去的地方,但最近在重温这部电影时,我会好奇为什么:为什么他想要创造Shell Beach?为什么要创造一个他从来没有真正去过的地方?

对于约翰而言,Shell Beach象征着他童年的很大一部分,象征着失忆的自己所竭力寻找的过去,即使这只是一个虚假的过去。但在寻找一个虚假的过去的同时,约翰创造了一个真实的未来。这让我想到了自我探索和心理咨询:每个人都在不断地重构(包括再阐释)过去的回忆,而在重构回忆的同时,我也在构建着新的现在、新的未来。我无法改变过去那些痛苦的、支离破碎的部分,但我可以在这些创伤、伤痕之上构建出属于我自己的东西。

对于约翰而言,Shell Beach就像是他在这个陌生的世界的一个锚点。约翰和其他人类被外星人绑架到了另一个世界,每个人所曾经拥有的关于过去的回忆都被外星人夺走了,没有人记得自己曾经从哪里来。在这个陌生的世界里,作为第一个也是唯一一个拥有tuning能力(该能力能够创造和改变物质世界)的人类,约翰创造了Shell Beach。

利用tuning来改变物质世界时,影片中强调的一点是:要集中精力。这让我想起我一个星期前开始参加的冥想体验营的经历。初阶冥想的其中一个重点便是找到一个内在或外在的锚点,将自己的意识/注意力锚于其中,保持对外界和自身的清醒的觉察。或者可以说,初阶的冥想像是一个集中精力,但又不要刻意集中精力的过程。如果太集中精力,反而会陷入更多的思绪和情绪,但如果太放松,则可能会睡过去,留意不到周围和自身发生着什么。Trying but not trying.

当然,约翰不只是创造了Shell Beach,他还创造了海洋、阳光和大陆,这些外星人所夺走的事物。后来约翰也意识到了那些过去的记忆只是他本该有的感觉和回忆\pozhehao{}“你所知道的是我本该感觉到的感受。那个人并不是我,从一开始就不是。”但约翰依然创造了Shell Beach。我的理解是,约翰能意识到那些他本该有的记忆只是被制造的记忆、虚假的记忆,那些他人所试图强加予他的部分。他并没有全盘否定这些他人试图强加给他的部分,而是以自己的方式整合了这个既否定又接纳的部分,并在此之上创造出了属于自己的事物。我也和约翰一样,并没有全盘否定自己的那些悲痛的、支离破碎的过去,而是承认那些我所不认同的部分,并在此之上创造着属于我自己的事物、发展出属于我自己的能力。

约翰没有利用他的能力去创造他人,而只是创造和改变物质世界,并将他所喜欢的女生Anna“引导”到了临近Shell Beach的码头,从而与经重制记忆后的她再次相遇。我也和约翰一样,并没有沉迷于只创造属于自己的事物、属于自己的世界,而是保持着与身边的他人的一次又一次的“相遇”。

\useimg{aimg/2022-0108-1.jpg}
