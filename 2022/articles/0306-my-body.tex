\chapter{感觉躯体不是自己的}

\ardate{2022-03-06}{KHqtz2b-vlQQ9Xf2PhSPDA}


\dialoguelist{咨询师}{
\dialogue{我}{我留意到走廊那里有个白噪音音箱。那个音箱是我半年前在淘宝看上的,不过因为太贵,所以那时候克制着自己没有买,结果现在又心动了。}
\dialogue{咨询师}{噢?你说你克制了自己。}
\dialogue{我}{嗯,因为买一件电子产品只是一瞬间的体验,而买其他东西(比如说课程、热线)则是一个过程,一个能延续下去的过程,而不至于只有一瞬间。就像是性高潮只有那一瞬间,但在那一瞬间过去后,彼此的安抚、亲吻和言语的亲密则是能继续延续下去的,不至于在(性高潮)那一瞬间后,就马上从高峰跌入低谷。}
\dialogue{咨询师}{那这样的过程对你来说意味着什么?}
\dialogue{我}{Em……可能意味着,在这个过程中,我能够呆在当下,我能够呆在这个过程里,停驻于此。这也会让我想起,最近我在接热线的时候,我会有一系列的生理反应,比如说脑壳疼,但大脑内部反而觉得很轻盈;腋下出汗;脸部血管膨胀。我在之前的咨询里也会有这样的生理反应,只不过在接热线的时候更加明显。我也会想起,以前我一开始写作的时候也会这样,我想这可能是一种心流的体验。}
\dialogue{咨询师}{那你觉得为什么会有比如说腋下出汗的生理反应?}
\dialogue{我}{我想可能是因为在接热线、之前的咨询、写作时,我好像都在用言语或文字来试图捕捉一些看不见的东西。我最近在读一本关于心智化的话,里面有提到:‘外显化内隐的心智化’。我想我可能是在外显化内隐的心智化的时候,就会有这样的生理反应。不过要进入心流的话,任务需要有一定的难度,如果之后我适应了接热线的难度的话,我可能就不会有这样的生理反应了,就像写作一样。}
\dialogue{咨询师}{当你处于那个状态的时候,你会有怎样的感受?或者说你现在会有怎样的感受吗?因为你说到在之前的咨询里也会处于这样的状态。}
\dialogue{我}{我会感觉蛮舒服的。现在的我的脖子(用手摸着脖子的后半部分、和衣服相接触的部分)有点温热了起来,我的大脑偏外壳的地方会有点开始疼,但大脑内部会觉得很轻盈,而且感觉我的脖子后面好像有两条血柱,在往我的大脑泵血\\
……而且在咨询室里也能听到(走廊的白噪音音箱的)鸟鸣。}
\dialogue{咨询师}{噢!是吗?我好像还没听到。}
\dialogue{我}{嗯,我听到了,好像在咨询室里有了新的东西。这个鸟鸣会让我有一个画面。我在冥想的时候也会通过白噪音来想象自己身处于一个场景里,将自己锚在其中,看着周围的想法和情感的飘过。我现在的脑海里的画面就是:森林里有片草地,阳光斜着打在草地上。如果我在冥想的话,我可能就会坐在草地上。我觉得这种状态就像是在接热线时,我一边听着对方的倾诉,一边又在自己的内心世界里。就好像我有一部分的自己是在更内在的空间里,另一部分的自己在和外界做各种回应、干着各种事情。当然,在接热线的时候,我有自己的想法和思考,但我依然感觉有一部分的自己是安坐在更内在的空间里,很闲暇、很稳定。}
\dialogue{咨询师}{听起来,好像确实是这样的一个画面。一个更内在的你在内在的空间里,而另一个你则在相对外界的空间处理着外界的事情。我记得在之前的咨询里,你是一个协调者的角色,而这次又是……}
\dialogue{我}{嗯,好像确实是两个不同的意象。现在更像是在内在空间里的自己和外界的自己。}
\dialoguesepline{咨询师}{短暂的沉默}
\dialogue{我}{其实我会想提一个问题,这个问题已经困扰了我一段时间(或许算不上是真正的困扰)。就是,我会感觉自己的手不是自己的,手掌不是自己的,有时候看着镜子也不知道镜子里的那个人是谁,有时候摸着自己的脸庞也觉得不像是自己的。我记得我小时候\pozhehao{}更多是读小学的时候\pozhehao{}自己的手臂没有那么修长,只有那么长(我指着小臂的一半),所以现在看起来很奇怪;手也只有那么大(我指着手指和手掌的交界处),而且手上的血管也没有那么多,只有一些主要的主路,但是现在多了很多支路的血管。
我有向一个以前读麻醉科的朋友说起这件事,他问我会因此而想截肢吗,我说当然不会。}
\dialogue{咨询师}{好像你能记得读小学的时候,你的手是什么样子的。}
\dialogue{我}{嗯,我也会在想,为什么我会那么清楚地记得那时候自己身体的样子。那时候会不会经历了些什么特别的事情?}
\dialogue{咨询师}{那时候的你会是一个怎样的状态吗?}
\dialogue{我}{Em……好像没有什么特别,就是几乎每天晚上都会被我妈打,然后在学校里也经常被霸凌,即使到了周末,我妈会拉我去和她的朋友的孩子们玩,但他们也依然会霸凌我。所以那时候的经历好像就是时时刻刻都会被打而已,也蛮普通的。\\
我也会想到一种可能性。其实我在上一次咨询结束后,在等电梯的时候,我试图继续往深处看,我看见了一个大学时的自我依然身处于痛苦当中。我本以为随着大学生活的结束,那部分的自我就会消失,但事实上并没有。那个大学时的自我之所以还处于痛苦,是因为我在现在的生活里一直没有满足大学的那部分自我对热爱生活的欲求,比如说想看看天空、背背单词、看看外文书。所以当我现在在生活里腾出时间去做这些事的时候,那个大学自我的意象就消失了。\\
所以我也会在想,会不会是小学时的自我也有一部分没有被现在的我所看到。}
\dialogue{咨询师}{好像小学的那个你承受着很大的痛苦?}
\dialogue{我}{嗯。我想,那时候的我不只是承受着痛苦,而且……当我看着自己的身体、自己的手臂的时候,我(当时)会在想:‘这副身体究竟能承受多大的痛苦,才会崩溃?正是这副身体让我被禁锢于日复一日的生活里。当时我唯一的解脱是,比如说课本上会有一个插画,然后我就想象自己在那个插画里,我会把2D的插画想象成3D的,而我就置身于那个场景里。通过这样,我就摆脱了现实,不再被禁锢于现实里。’不过(现在的)我依然会在想,为什么我会把小学的自己投射到现在这副躯体上?}
\dialogue{咨询师}{我留意到你会用‘为什么’。好像你又开始以纯理智的方式去思考一些事情。这会不会是一种抽离?比如说通过这种方式将自己从情感里‘捞起来’,而且之前的咨询里你也会准备一些小故事或纸条或iPad带进来。}
\dialogue{我}{Em……抽离……当我沉浸在内心的世界里,我确实抽离了,但又不是完全的抽离,比如说在接热线的时候,一部分的我抽离了出来,但并没有完全抽离。怎么说呢……就好像一部分的我在回应着对方,另一部分的我沉浸在自己的内心世界里,将自己的内心世界打造得像是对方的内心世界,从而更好地与对方连接在一起。所以好像並不完全是抽离。……我想你所说的‘抽离’可能更像是我从外界世界转入到内在世界,这确实是一种抽离,但也並不完全是抽离。不过这种‘抽离’确实是小学时的我得以逃离现实世界,去内心想象的世界的方式。\\
我想,可能是当我身处于内心世界的时候,我更像是小学时的自己,而当从内心世界回到现实世界时,我可能无意识地把小学时的自己投射到了现实世界的躯体里\pozhehao{}看着自己的躯体觉得很陌生,觉得不像是自己的,不像是自己内心的意象。\\
我最近在上的释梦课程里经常会提到‘内在资源’这个词。所以我想,当我运用心智化能力去构建内心的世界的时候,我可能就是在运用小学时的自己的那部分内在资源。}
\dialogue{咨询师}{我一时间不知道怎么回应了。(笑了笑)好像你确实会用那一部分的内在资源去想象内心的世界。听起来,那个自我的资源很丰富。}
\dialogue{我}{嗯。我会在想,我内心的意象并不一定要和现实世界相匹配。我内心的意象更像是小学时的自己,而现实世界则是现在这副躯体。这两者并不冲突。我会想到自己的一个有双相障碍的朋友,他说他连自己胖了一、两斤都能从脸上看得出来。天呐!一、两斤,还不够我吃的苹果的重量,这都看得出来?我会想到,等自己手上的皱纹、脸上的皱纹多起来后,当自己的身体开始衰老,甚至开始发胖的时候,我不需要担心自己外表的衰老,因为我内心依然是小学时的自己。}
\dialogue{咨询师}{我留意到你在皱着半边的眉。你会是在……}
\dialogue{我}{我可能在思考着未来的可能性……(短暂沉默)……我可能也在感受着这种(内心世界和外界世界)分离的感觉,好像自己的内心世界和躯体都各自找到了属于自己的位置,安放在了那里。}
\dialoguesepline{咨询师}{咨询师笑了笑}
\dialogue{我}{我也会在想,为什么我自己一个人做不到、想不到这些。我之前有一个人试过,但想不到、走不到。我想这会是因为……我想可能是因为有鸟鸣声。其实我脑海里一直有一个画面:小学时的自己在森林的草地里,在斜着打下来的阳光下奔跑。这好像一直给了我一个锚,能让我一直朝那个意象(森林里的草地)走。}
\dialogue{咨询师}{好像不仅仅是咨询室外,咨询室内也有了变化。}
\dialogue{我}{(没反应过来)噢?你是指?是指我?}
\dialoguesepline{咨询师}{咨询师点了点头}
\dialogue{我}{噢,我想我今晚在咨询室里呈现的状态也和之前很不一样。}
\dialogue{咨询师}{我终于能听到鸟鸣了。}
\dialogue{我}{可能每个人对外界的感知能力都不同吧。}
}
