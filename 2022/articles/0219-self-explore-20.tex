\chapter{自我探索 | 20\pozhehao{}“内在的丰盛来源于外界的匮乏”}

\ardate{2022-02-19}{Fi1oc\_LTynLhwxgi2X5YGQ}

\dialoguelist{咨询师}{
\dialogue{我}{当我在整理上周的咨询回忆稿时,我发现你会有意地把话题拉到咨询前期那段在我看来你更像是在机械性地回应的时期,而且这个议题(可能是个议题)好像出现得越来越频繁。我会在想,这可能是一种技术,将咨询外的人际关系拉回到彼此的人际关系,这样的做法确实会很有用,毕竟谈及彼此发生的事情会比谈及咨询外的关系更有用。但我也会在想,你频繁地提起这个议题是因为你出于关心我的感受和想法吗?还是关心彼此的咨访关系呢?亦或是出于你自己的顾虑,比如说对自己能力的恐惧或焦虑?而且我也留意到了在上一次咨询里,你会在面质前有所停顿地说“可能有点偏题”。所以我有点警惕,这是否会是你自己的个人议题?}
\dialogue{咨询师}{好像你会想到很多方面,会想到这是你的方面,还是彼此关系的方面,还是我的方面。那你会更偏向于哪个方面吗?}
\dialogue{我}{我好像暂时还没有偏向那个方面,就只是想到了可能会有的这三个方面。}
\dialogue{咨询师}{为什么你会担心这可能是出自我的方面?}
\dialogue{我}{因为我会担心彼此的关系会逐渐固化。}
\dialogue{咨询师}{固化。你能多说一些吗?或者会想到些什么吗?}
\dialogue{我}{我会想到现在和父母、亲戚的关系是固化的,和同事的关系也是固化的。}

\dialoguesepline{咨询师}{(在和咨询师描述了我和父母、亲戚以及和同事的关系后)……}
\dialogue{咨询师}{好像你和父母的关系以及你和同事之间的关系虽然是固定的,但又有所不同的。和同事之间的关系好像是彼此默认的,但和父母的关系里你是主动地去和他们保持距离。}
\dialogue{我}{嗯。我想这种固化的关系是为了keep the danger in check,为了将一些危险的东西包裹起来。}
\dialogue{咨询师}{会是一些怎样危险的东西呢?}
\dialogue{我}{比如说我妈依然和以前一样,在吃饭点菜时她会问我想吃什么,但当我决定好我想吃的东西后她又会说这个不好那个不好,还是点XXX好。所以她根本不想我点我自己想点的菜,而只是想我点她想我点的菜。如果我向我妈敞开更多的自我的话,我就会因为她的这个部分(以及其他更多的部分)而受到伤害。所以我的内心对她依然感到警惕、感到不安全,我也不会将彼此的关系变得不那么固化。而且这样的固化也蛮有用的~}
\dialogue{咨询师}{那如果是面对一些你会感到不那么危险的人呢?}
\dialogue{我}{比如说我的朋友。之前我想换iPad,然后我的一个朋友会鼓励我去把旧的iPad卖了,然后买哪个型号好。但后来我发现收购的价格太低了,换一部新的iPad要花的钱太多。然后他马上说:“所以我才会定期换电子产品,就是为了防止你这种情况。”那时候他就很本能性地防御掉了这个部分,但当他本能性地防御的时候,我反而因此而受伤了。但我能区分除了这个部分对我而言是有危险的之外,其他(关于他)的部分(对我而言)是安全的。所有当我能区分对方不同的部分哪些是危险的、哪些是安全的,我内心的感觉就不会只是感到危险,而是能够区分出不同的部分。}
\dialogue{咨询师}{这种危险,是否也像是在咨询一开始你提到的那种警惕的感觉?}
\dialogue{我}{嗯……是的,就是那种警惕的感觉。我会警惕彼此的关系越来越固化,就像是这个房间被一块又一块的石头挡住,直到空间越来越小,能聊的东西越来越少。但如果我能依靠自己的能力向你指出那里有一块石头,说不定就能把那个石头挪开,看见更多的部分、更多的东西。}
\dialogue{咨询师}{我会想起在上一次咨询里,以及在之前的咨询里,你都会提到“新鲜感”这个词,比如说你会和不同的人见面。好像新鲜感对你而言很重要。}
\dialogue{我}{嗯。因为当我在一段关系里,从同一个人身上看见不同的东西时,就像是一起去一个新的地方。但这个地方不是外界的地方,而是内在的地方\pozhehao{}我的内在、你的内在、彼此关系的内在。而且,我也不想换完一个又一个的人,在看完一个人的内在后又换,看完又换,看完又换。如果是这样的话,我好像也没有真正呆在什么关系里。但如果我能将一些东西、一些石头指出来,而那个石头又能被挪开的话,那每次见面我都看见焕然一新的对方。当然我不是去冲破对方的个人议题,不是去撞开对方的墙,而是向对方指出这一点,看对方、看彼此能做些什么。}
\dialogue{咨询师}{好像你能看见自己不同的部分,也能看见他人不同的部分。}
\dialogue{我}{嗯。对我而言,我是先看见我自己的不同部分,然后才能看见他人不同的部分。不然如果自己(的内心)是揉成一团的话,那么就算能看见他人不同的部分,我也不会确信那就是真的,不会有那种确信感。}
\dialogue{咨询师}{我会想起在很前的咨询里,你会说你想要他人能够看见你,但现在你更多是看见了他人,更多是:“我看见了他人的XXX”。}
\dialogue{我}{嗯~确实是有这样的转变。我在想为什么。}
\dialogue{咨询师}{我也在想。}
\dialogue{我}{Em……(思考中)}
\dialogue{咨询师}{我想,这会不会是因为之前你没有看见你自己?}
\dialogue{我}{不,不是的,我一直都能看见我自己,反而是当我能看见自己的不同部分,而他人却看不见我的不同部分时,我会感到孤独,我会想让他人也看见我。但……em……我想……是因为……嗯。我想现在我没有那么渴望别人看见我是因为我好像代替了他人的位置。在我心智化他人和心智化我自己的过程中,我好像代替了他人的位置看见了我自己,而这好像满足了想要他人看见我的那种渴望。就像是之前的咨询里我会代入你的位置去看周围的事物一样。}
\dialogue{咨询师}{嗯。我也会有这种感觉,好像你在内心创造了一个客体来看你自己。}
\dialogue{我}{我在外界缺乏的东西,比如说陪伴,比如说拥抱,比如说家,我都在内在世界里创造了它们,创造了一个又一个的场景\pozhehao{}家一般的场景,创造了内在小孩\pozhehao{}并和内在小孩拥抱。好像外界缺乏的东西,我都会在自己的内在世界里创造出它们,去弥补它们的空缺。我会对此感到很开心和难过,开心于自己内在的丰富、丰盛,但同时也难过于我内在的丰盛来源于外界的匮乏。}
\dialogue{咨询师}{嗯。我也会有你这样的感觉,既为你感到开心,又感到伤心。}
\dialogue{我}{是啊。我想生活里的很多事情都是这样,不会有全好或全坏,而是相互揉杂在一起。}

\dialoguesepline{咨询师}{(我看了下时间,咨询快到50分钟。)}
\dialogue{我}{其实我想问,你会从我身上感觉到一种脆弱感吗?因为我的前任、我的那个资深咨询师朋友、和我一起进行倾诉练习的同学,以及最近认识的一个男生都反馈说能在我身上感觉到一种脆弱的感觉。}
\dialogue{咨询师}{我好像还没有这种感觉。如果之后有的话会跟你提出来。我这么说你会感觉怎么样?}
\dialogue{我}{嗯~好的。}
\dialogue{咨询师}{我也会在想,刚刚你看了下咨询时间,然后在咨询快结束时提起这件事。好像你会对这件事有所介意?}
\dialogue{我}{嗯,我会介意。我会介意身边不同的人对我有这样的感觉,但我却感觉不到这种感觉。而这种感觉似乎也影响到了他们与我的互动,比如说有的人会(因此)想要照顾我、有的人会想亲近我、有的人会和我保持距离。就有点像是一个不安全依恋的人一直遇到的对象都是不安全依恋的,然后当回顾时才发现原来不安全依恋的是自己。我会担心和他人的人际关系会不会因此而固化。我想也是我在咨询一开始提到的那个警惕感吧。}
\dialogue{咨询师}{嗯。现在我会明白为什么你会打算提起这件事。那你自己会怎么看这件事?}
\dialogue{我}{我会想,我有脆弱的那一部分,那一部分就像是自己内心的小孩子的一部分。但我也有其他的部分,比如说我想去冒险的部分、我想独立的部分、我学生时期的部分、我读大学时的部分。而且那个脆弱的部分对于亲密关系而言蛮重要的。如果没有了那个脆弱的部分,如果自己时时刻刻都那么全能,什么事情都能搞定,那为什么还需要另一个人在?为什么还需要拥抱,还需要陪伴,还需要另一个人去contain(涵容)自己的情感,还需要另一个人去holding(护持)自己去外界探索?}

\dialoguesepline{咨询师}{咨询师笑了笑。}
}
