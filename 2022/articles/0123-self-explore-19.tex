\chapter{自我探索 | 19\pozhehao{}倦怠,“这不是我想要的生活!”}

\ardate{2022-01-23}{-6rl5vfrDAf135scWn4-1A}



最近一周,一种倦怠逐渐涌现出来。在昨天的心理咨询里,我陷入了那种熟悉的什么话也不想说的状态。

\blockquote{
    说着说着,咨询师留意到我的情感强度不像咨询一开始那么强烈,并问我为什么会这样。那时候我才意识到,自己的语速早已开始变慢,自己越来越不想说话。我说:“在刚刚的几次沉默里,其实我早就走神了。这种孤独、悲伤、难过、被落下的感觉好像将自己拉去了自己脑海中的一个又一个的场景,比如说那个街区的场景、那个塔的平台的场景。我好像越来越想从这个现实里抽离出来,离开这个咨询室、离开这个世界。”

    \blockquotesource{呆在那里,不想说话,沉默}{白色灯塔先生}{2022}
}

在一开始,我以为这只是我的身心节奏,就像是之前处于非常抑郁的心境时也会每几个月里就有几周这样的状态:不想干任何事情,不想说任何话,不想打任何文字,对自己喜欢的事物完全丧失兴趣,对未来不抱有任何期望。抑郁的反面不是开心,而是活力。当这种抑郁的状态抵达极端时,甚至连自杀的动力也会丧失。

在这周的课程里,一位讲师说到了她应对职业倦怠的其中一个方法:铭记初心。

\blockquote{May you remember this spark that caught you in the first place, that caught your imagination that touched your heart. And may you keep that spark alive, and remember the immense privilege it is to sit with another human being. And at those moments when you forget it, or you get tired, or you get frustrated, or question yourself, take time and remember again the spark and the privilege of being a healer.}

我开始回顾每当濒临自杀成功边缘的那份选择活下去的初心。在读小学时试图从六楼父母房间的窗外“走出去”时,我看着窗外远处的日落,想到:说不定将来会有更美好的事物、值得我活下去的事物。在一年多前同样试图从高楼的窗外“走出去”时,我想到:如果自己真的这么做了,那就意味着我再也不可能见到那些无故消失的人了。

带着选择活下去的这两个初心,我开始回顾自己现在的生活。我找到了未来那个更美好的事物,找到了值得我活下去的事物。我也重新见到了某个无故消失的人,也逐渐放下了他们的那份无故消失。我回想起了在一年多前,当自己决定活下去时,我做的另一个决定:放弃那些自己所无法控制的事物和他人,以及放弃对自己的控制。

\blockquote{
    所以,既然我已经走了那么远,遇到了、找到了那么多值得我继续走下去的事物和人,为什么我就不能休个息、放个假呢?然后,我试着和自己的身心好好相处,放弃控制自己的身心,照顾好自己身心的需求:想休息就休息,想嗜睡就嗜睡,想变得愤世嫉俗就愤世嫉俗。
    在一起走去吃饭的店的路上时,走在一片安静的街区里,我突然感到很宁静。那种宁静感唤起了我过去的回忆,我回想起还在读大三时,当时的我也去了另一个城市Z,在经过城市Z里的一片街区时,那是我第一次感受到这种宁静感。在这两个城市的这两片街区里,街边都有不少在居民楼一楼开的路边咖啡店,偏雪白色的地面砖,干净的路面,间隔得十分整齐的树木,一列又一列的树荫。
    距离上次感受到这种宁静感,已经是四年前的事情了。当我第一次身处于城市Z的那片街区,我的大脑开始自主地产生各种幻想,幻想着这条充满树荫的街道无尽地延伸,延伸至视野的尽头。我感觉这里更像是一个梦境,一个安静得与世隔绝的梦境、一个比现实世界更为宁静的梦境,而我想在这个梦境里一直停驻于此。
    
    \blockquotesource{随笔 | 街区宁静感,停驻,畏惧}{白色灯塔先生}{2022}
}

因此,这周末我和一个朋友去了上周周末去的那个曾让我感到宁静感的社区。我们找了家路边咖啡店坐着聊天,他从聊天逐渐转为倾诉,我也从聊天转变到了倾听的状态。我发现自己越来越擅长进入并呆在倾听的状态。当听着他讲述着关于他自己的童年往事时,我像是在“看着”流水般一个个闪过的画面和一波波涌动的情感流过。那些画面和情感都从我的“身上”流过了,我记得他们的流经,但我也知道那些画面和情感并没有在我的“身上”停留太久。同时,他也会邀请我进行自我表露。每次当我表露到一定深度时,他才会继续进行更深的自我表露。我开始察觉到这种(可能的)模式的重复\pozhehao{}好像只有我愿意表露出关于我自己的一定深度的内容后,他才有足够的安全感去表露关于他自己更深的内容,或他才知道我是能理解他内心更深处的感受和想法的。但这也会给我一种感觉:好像我的自我表露只是一种对方进行更深的自我表露的“筹码”,我不确定对方是否真的那么在乎我的过去,不确定他是否真的像我在乎他的过去般地在乎我的过去。不过我不会介意一定深度的自我表露,因为我的自我探索和防御方式已经能让我不像以前一样那么容易被他人所“入侵”,而且我也对我和他彼此的关系抱有一定的安全感和信任。


在倾听的间隔中,我看着朋友身后的街道风景\pozhehao{}路过的行人,干净且偏白的马路,上方的绿色树叶\pozhehao{}我对活着的倦怠感开始慢慢减弱。但在那个当下,我还不知道为什么那份倦怠感会有所减弱。在我和朋友各自离开后,我回顾起倾诉的过程\pozhehao{}不在于倾诉的内容,而在于倾诉本身\pozhehao{}我开始觉得活着并没有我之前所感到的那么倦怠,因为生活里还有他人的故事,还有我在倾听他人的倾诉时,我能对他人的生活、生命、存在产生影响,我能用自己的倾听帮助到另一个人。

不过,我也开始警惕于,我是否只是利用他人的生活故事来回避我自己的生活问题。我自问道:如果我是在逃避的话,我到底逃避的是什么?我脑海里有一个画面:我的外婆在亲戚的饭桌上训斥我,说我的工作没有前途,也赚不到什么钱。我很生气,直接离开了饭桌,离开酒楼。这个画面是曾经真实的回忆,但我已经忘了这个回忆具体发生在哪间酒楼了,因为同样的画面似乎在不同的酒楼发生过好几次。在我的脑海里,每当我离开了酒楼,没过多久,我又重新回到那个饭桌,外婆继续训斥我,我再次离开。这个画面不断在我的脑海里循环,就像是一个挥之不去的清醒的噩梦。

由于这个画面的重复性已经让我无法无视,所以我试图不再逃避,试图将画面里的外婆看作自我的一部分,那部分的自我究竟想表达的是什么。那部分的自我表面上在说:“你的工作没有前途,也赚不到什么钱。”但我试图将我外婆曾经说过的话从它身上移开后,它在对我大声哭喊着:“这不是我想要的生活!”

我马上感到很奇怪,我明明已经在我想要走的方向上越走越远,无论是理论知识、个人体验,还是实践经验方面。为什么“这不是我想要的生活”?我向自己内心深处问道:“我真正想要的是怎样的生活?”我回答道:“一个自己的存在对于自己而言以及对于他人而言,都是有意义的、有价值的生活。我的存在是有意义的、有价值的。”

我立即回想起最近的工作:在最近几个月的工作里,每日重复着既忙碌又没有意义的工作内容。当我找到“未来”那个更美好的事物、找到了值得我活下去的事物,并朝着自己想走的方向越走越远、能力越来越强后,我就像是看着远处的光亮在我的眼中变得越来越明亮,也看着自己一直身处的黑暗在我的眼中变得越来越黑暗。

The brighter the light, the darker the darkness.
