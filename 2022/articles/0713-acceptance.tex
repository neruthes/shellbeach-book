\chapter{设想,交融感,被接纳感}

\ardate{2022-07-13}{cmR9XPCPvzHfZPTvRzDR9g}



最近几天,我都在想我要怎么和上周周末见的那个对他有好感但他已经有对象的男生再约见面。但因为还只是周初,要约周末的话时间又还很早,会显得好像才过了几天我就想再约他见面似的。我发现自己一直停不下来地不断想这件事情,设想各种场景、对话、情况。

我设想着和他下次见面时我能和他说点什么,比如说上次见面一起逛的时候,他好像也是在通过一起逛他经常逛的地方这种方式来呈现他内心的事物、他所看重的事物,从中好像透露了他对我的在乎。为什么我会想要他也在乎我?因为如果他也在乎我的话,那在我的设想里,这能够带来一种交融、融合的感觉。

我会设想着下次见面时我能表露:我只是普通本科毕业,而现在的工资在我看来真的很低。为什么我会想要表露这个部分?因为我想要他能接受我对此感到自卑的部分。如果对方有着更好的学历和经济能力并且还能够接受自己在这方面的自卑之处的话,在我的设想里,我会有一种被接纳的幸福感。

当发现自己那么渴望想约对方见面,是因为自己想要得到对方也在乎我、对方愿意接纳我感到自卑的部分后,我会想到那种喜欢的感觉背后,除了这两者(渴望融合、渴望自己感到自卑的部分的被接纳)外,还会有些什么吗?如果我将任何情感,无论是喜欢之情还是吸引感还是爱慕之情都分析得彻彻底底的话,好像在这个过程里,那种未知的感觉也会随之消失\pozhehao{}正是不知道对方对自己为何如此吸引,才更为吸引。

我想到他还是有对象的,我如果表达了自己对他的情感的话,会不会给他们的关系带来影响。但这样的事情(向有对象的男生表达我对对方的喜欢之情)都已经发生过两次了,而这两个男生至今都还是朋友关系。况且互动(例如深入地交流)所能带来的被接纳感和交融感并不需要处于情侣关系才能达成,朋友关系也照样可以。但他是他,那两个朋友是那两个朋友,我并不想影响到他和他对象的感情。但如果我不表露自己的情感,我依然会一直想约他见面,而这样的频繁见面恐怕也会让对方感到奇怪甚至是负担,而当对方察觉到这些频繁见面的奇怪之处时,说不定会撤回、离开。同时我也意识到这一系列的设想的方向开始往熟悉的方向走\pozhehao{}隔阂、撤回、离开。

上周和朋友聊天时,那个朋友发现,无论他在某件事情上选择怎么做、做什么,我对他的不同选择的设想都会朝着同一个方向:我不相信他人眼中的我是独一无二的。而现在的我会想到,如果我不相信他人眼中的我是独一无二的,那么对方又怎么会相信他眼中的我是独一无二的呢?

当察觉到自己的心情依然很心烦意乱、很紧张,不知道应该怎么面对他,不知道应该怎么走下一步时,我还是不敢跟他发任何信息、不敢踏出任何一步。后来,在躺在床上半睡半醒的状态下(总是在半睡半醒的状态下才敢去做一些清醒时不敢去做的事情),我给他发了信息:表达了对他的喜欢的感觉,以及表露对他的情感不是为了去影响他和他对象现在的关系,而是不想在和他之后的互动里继续隐藏我对他的感觉\pozhehao{}因为这会让关系变得不那么真实,以及我也会担心他会察觉到我在隐藏着些什么。

第二天,我发现信息已读但他没有回复。


