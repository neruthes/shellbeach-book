\chapter{没有意义,表象,抓住悲伤}

\ardate{2022-06-11}{FwNSLalm\_kRB6mKCMaBFDA}


\dialoguelist{咨询师}{
\dialogue{我}{这周的状态和之前前两周差不多,都是那种无意义的感觉:觉得咨询没有意义,今天还准备和朋友约自习也没有意义,今晚还有课程小组练习也没有意义,除了打游戏之外一切都没有意义。这周还有朋友找我倾诉,我也觉得没有意义。但我不知道为什么自己还在一直做着这些事情。}
\dialogue{咨询师}{来咨询之前,你会设想到如果你提出来咨询没有意义,我会怎么回应吗?}
\dialogue{我}{我猜想你会沉默。你最擅长的就是沉默了。如果我能设想到你会有怎样的回应的话,那说不定我就能从这样的自我对话里找到个答案,那就不会在咨询里表达觉得咨询没有意义了。我想可能是因为我内心没有那个觉得事情是有意义的的部分。}
\dialogue{咨询师}{那打游戏的那种意义感是怎样的?}
\dialogue{我}{就好像打过了之前过不去的关卡的时候会有一种效能感,同时也能看到更多之前探索不到的世界的新鲜感。或者是去逛江边,吹着江风,晒着太阳的那种直接的体验。}
\dialogue{咨询师}{噢?‘直接的体验’。}
\dialogue{我}{嗯,就是能直接感受到快乐或开心或意义感的体验,而不是和人。和人好像就无法直接体验到情感,好像没有真正地与人互动。}
\dialogue{咨询师}{当你说你觉得咨询没有意义的时候,你是在指?}
\dialogue{我}{Em……就好像咨询结束之后,我没有什么特别的情感,不会像以前那样感到开心或兴奋或连接感,就是没有任何的情感存在。而且不只是咨询,和朋友约自习也是,听朋友倾诉也是。}
\dialogue{咨询师}{那现在还来咨询的你,会不会感到和我也没有真正在互动?}
\dialogue{我}{会的。\\
我发现这种无意义感好像都是与他人有关,好像我在和人的互动里不再感到有意义,感到这一切都毫无意义。}
\dialogue{咨询师}{为什么和人的意义感会消失了?}
\dialogue{我}{可能是因为和前任,还有之前我喜欢但拒绝了我的那个男生的互动,他们在我内心的形象的消失。好像当没有了和重要他人的连接,我也就没有了与人连接的感觉,没有了那份与人互动的开心、快乐、兴奋、连接感、意义感。}
\dialogue{咨询师}{但在我看来,你依然还会做这些事情,会和朋友约见面、会来咨询。好像你并没有把自己封闭起来。}
\dialogue{我}{是啊,所以我也在想为什么自己还在做这些事情。}
\dialogue{咨询师}{你会不会是在寻找一些什么?}
\dialogue{我}{如果是寻找的话,可能是在寻找意义感吧,看生活里还会有什么东西能带来意义感。我前段时间看的一本关于冥想的书里说到,对于那些身处抑郁的人,他们可能会想到到处走走没有意义,就不去做了,然而不去做本身就无法给他们带来改善。但如果他们不是先去感受想不想做,而是先去做,然后去看看这件事情能给自己带来怎样的感觉的话,说不定事情还会改善。所以我好像也在试图维持着生活的表象,维持着以前的生活看似的样子,保持着开放性,去看会发生些什么。}
\dialogue{咨询师}{那以前的你会怎么做?}
\dialogue{我}{以前的我会直接根据感觉行事,比如说想到某件事情不想做,那就不去做了。如果去做某件事情、去见某个人没有什么特别的感觉,那我就不去做了、不去见那个人了。但现在我会先去做了,然后看会不会有什么特别的感觉。}
\dialogue{咨询师}{好像你现在还一直在做这些事情,依然在寻找些什么。}
\dialogue{我}{Em……可能只是在被动地接受各种可能性,被动地敞开自己。如果有人找我,我就会去赴约。但如果没有人找我的话,我也不会主动去约人,也不会主动去上什么课程。}
\dialogue{咨询师}{好像你在说你并不是主动地在寻找,而只是被动地敞开自己。}
\dialogue{我}{嗯。}
\dialogue{咨询师}{在咨询里的你会不会也只是在被动地敞开自己?}
\dialogue{我}{是的。}
\dialoguesepline{咨询师}{(沉默)}
\dialogue{我}{我发现刚刚你提出来这一点、提出来了我只是在咨询里被动地敞开自己之后,我们就沉默了。如果你不问我些什么,我就什么也不想说。}
\dialogue{咨询师}{为什么会这样呢?}
\dialogue{我}{我想可能是因为当你提出来这一点后,我就认同了你说的,我只是在咨询里被动地敞开自己。}
\dialogue{咨询师}{但同时好像你并没有沉默下去,而是提出来并且还作了解释。}
\dialogue{我}{可能因为我的情感和想法都凝固了,而当我意识到这一点时,我就会本能性地心智化我自己,从一个第三者的角度去看自己发生了些什么。我还是有这个能力的。}
\dialoguesepline{咨询师}{……}
\dialogue{咨询师}{刚刚你说到表象,那会是一种怎样的表象?}
\dialogue{我}{嗯,有时候感觉我和别人互动就像是带着一副面具,比如说来咨询、去约朋友见面、和课程同学练习作业、倾听朋友倾诉或接热线,都有一种很虚假的感觉,比如说‘你说,我在听’,有一系列的套话来表现得接线员很在乎来电者,但其实就我个人来说,我并不在乎他们。我也不在乎我的朋友和ta对象的关系问题。我会想说:我并不在乎你和你对象的事情,你并不重要,什么都不重要。但我又不能直接表达这个部分,因为我不知道对方会有怎样的感受和会怎么回应,而且如果我说了的话两人的关系说不定就会受损。}
\dialogue{咨询师}{听起来,好像你依然在乎他人的感受,担心两人的关系会受损。}
\dialogue{我}{嗯,会的。因为如果我完全没有在乎他人的部分,那我压根就不会去听朋友倾诉,也不会去接热线了。会有在乎他人的部分,但另一部分也很庞大,那就是我并不在乎他人,并不在乎他们的痛苦和难受和悲伤。他们并不重要,没有什么是重要的。但我没有办法去表达这个部分,好像我不应该去表达这个部分,因为对方带着难受的心情和心事找我倾诉,我不可能跟对方说其实我比ta还要更丧。所以我会觉得即使在和他们互动着、倾听着,但我依然感觉我没有完全在那里和他们互动,因为有很大部分的自己是抽离出来的,将那个觉得一切都没意义的自己抽离了出来,没有真正地在场。\\
我想可能是一直以来都没有人确认过自己的这个部分,这个觉得一切都没有意义的部分。而且我一直以来都不敢跟他人表达,担心关系受损,所以也一直没有人来确认我这个部分。}
\dialogue{咨询师}{如果你真的表达了这个部分呢?如果你直接对他们说呢?你会有怎样的设想吗?}
\dialogue{我}{不知道呢。我会想起以前还在这座城市的有双向障碍的朋友,如果我和他说的话,他可能也会同意我的说法,会说:一切是没有什么意义,生活没有什么意义,活着没有什么意义。……然后我们可能会找其他一些我们都感到有意义的事情去做。}
\dialogue{咨询师}{我能不能这样理解:好像当你这么说之后,会出现转机。}
\dialogue{我}{嗯。但如果除了他之外的话,我不知道其他人会怎么反应,可能会各种防御,比如说回避、否定、合理化。\\
我想起这周找我倾诉的那个朋友,我最后有跟他说我最近的状态就是觉得一切都没有意义,然后他说:我不知道这对你来说是否适用,但我会觉得意义来源于生活的一些小事,比如说和人见面、看书这些事情。我会想说:嗯,说得很好,但下次别再说了。因为我会感到不被理解,因为那并不适用于我。那并不是一个固定的、看得见的答案。}
\dialogue{咨询师}{你会担心我不理解你吗?}
\dialogue{我}{不会耶。好像我都只是会替对方担心,但不会替自己担心。因为好像如果一切都不重要了,那我就能把自己保护得很好、能完全地敞开自己,而没有什么能伤害到我内心的事物。如果一切都没有意义,那么和前任的经历、和那个我喜欢但拒绝了我的男生的经历,和那些在我看来的重要他人的经历,就没有那么的痛苦了。}
\dialogue{咨询师}{我猜,你和他们、在依恋里的经历里的痛苦真的很强烈,所以才会想要降低这样的痛苦。}
\dialogue{我}{嗯。}
\dialogue{咨询师}{而不是为了虚无而感到悲伤。}
\dialogue{我}{Em……我好像不会为了虚无而感到悲伤,因为虚无好像就是一切本来的样子。和他们的关系、那些丰富的经历,我会为那些有意义的部分的消失、的褪去、的破碎而感到悲伤,会为了那些有意义的事物褪去为无意义而感到悲伤。但我不会为了虚无本身而感到悲伤,可能因为我一直在更深层次地认同着这种无意义\pozhehao{}一切本来就是无意义的。他们消失之后只是让一切回到了遇见他们之前的状态。\\
同时我也可以回头去再感受悲伤,但自己真的想这么做吗?在之前很多年的抑郁里,我一直抓着悲伤不放,在自我欺骗:只要抓着那份悲伤\pozhehao{}主要是和前任分手后的悲伤\pozhehao{}那一切都还是有意义的:我的生活是有意义的、活着是有意义的。但事实上真的是这样吗?\\
所以现在的我就不会想去抓着那份悲伤了,那份悲伤过去就过去了。}
}
