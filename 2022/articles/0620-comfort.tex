\chapter{“我并不怎么会安慰人”}

\ardate{2022-06-20}{JyIQ6trX7ZRU4rCX8jyUrw}

% \dialoguelist{咨询师}{
% 	\dialogue{咨询师}{是啊。}
% 	\dialoguesepline{咨询师}{(沉默)}
% }



今天一个朋友找我聊一些他感到不爽的事情,然后我提供了一个替代性的视角,比如说他那边的人可能是那样的,而不像是我这边的人是这样的。然后他说我是在“客观认真地帮我(他)分析原因”。我说你是想我主观点吗?他问我如果我主观点的话会说些什么,我说“我觉得你可以操他们”。后来他说我连一句安慰的话都没有。我说我并不怎么会安慰人。他说他发现了,之前每次有不爽的事情找我我都是这样。我说我现在依然是这样~

我会回想起很久以前,我问另一个朋友:怎样才能学会安慰人。因为那时候的我发现自己并没有内化过任何被安慰的经验,甚至想不起来任何被安慰的记忆,同时前任曾经说过的“你并不会安慰人”这句话对那时候的我的冲击很大。那个朋友说:凭什么一个人就要懂得如何去安慰另一个人了?即使是现在想起来,我依然觉得蛮好笑的,因为确实,很多人试图获得安慰这一行为本身更像是一种讨要甚至是索取。

而且直到现在,我依然觉得“安慰”这件事情很魔幻,因为我依然很怀疑“安慰”是否真的存在。

往回看在大半年前学的人本主义课程,在人本主义的视角里,它更鼓励一个人与另一个人的痛苦呆在一起,而不是通过各种防御(例如回避、合理化、否定)来防御掉对方内心更为痛苦的部分。只有和对方的痛苦呆在一起,对方才能逐渐恢复动力继续前行,因为对方知道有另一个人愿意和自己呆在一起,无论自己变得多么糟糕、多么深陷痛苦。

人本主义也吸取了一些存在主义方面的视角,比如说认为痛苦是不可避免的、痛苦来源于对生活意义的追寻。所以需要被看到的不仅仅是现实层面的痛苦,还是内心世界的痛苦以及痛苦背后的那份被支离破碎的内心部分。

而在我看来,“安慰”更像是一种防御。不过,我不是想说防御不重要或不必要。防御是至关重要的,否则防御能力弱的个体可能从一开始就难以存活下来,就连婴儿都会采用分裂好客体与坏客体这一防御方式来适应周遭的矛盾经验。

我之所以认为“安慰”更像是防御,是因为我的“被安慰”的经历里,几乎无一例外的是,本应安慰我的人更像是在试图安慰他们自己,而不是在真的安慰我。他们把他们自己安慰好了,但对我来说没什么作用,甚至让我感到更不被理解、不被看见、不被接纳。所以我也不知道何为“安慰”。

在电话热线里,我也不会试图去安慰来电者。有的来电者会直接跟我说:我觉得你并没有舒缓到我的情绪。我的回应就是:“嗯。”因为我并没有用力地想要为了舒缓对方的情绪而刻意地做一些事情,比如说用力地去所谓地“安慰”对方:事情会变好、你这没事、可能睡一觉就好了。情绪的舒缓更像是一种不可强求的副产物,就像是冥想时的无我的状态\pozhehao{}越用力想要抵达无我,越“有我”。

相反,让经验自然而然地流动,比如说将对方的防御挪开,让防御背后被压抑和隐藏的情绪自然而然地流动起来,那些情绪自然而然地就会过去,而不会一直凝固在原地、不会让对方深陷于其中。


如果真的要提一个被安慰到的例子。我会想到这周电话热线团体督导的时候,其中一个学员带来了一个很可能是重度抑郁并且有着自杀风险危机的来电案例。临近团督结束时,督导师说我们(接线员)应该学会去怀疑来电者陈述的事实,因为在督导师看来,来电者所陈述的那个充满困难、充满无力感的事实并不是真的。我问督导师:我们真的应该去冲击来电者的主观现实吗?万一冲破了怎么办?事情可能会带来改变,有可能会变好,但也有可能会变坏。改变的方向是我们无法控制的。督导师问:XXX(我的姓名),你害怕的那个事情变坏的部分是什么?我说:我会害怕ta的情绪可能会恶化,如果ta意识到现实并不是ta所设想的那么无力,而是ta选择让自己变得那么无力来防御掉一些其他东西的话,那ta的情绪可能会更加恶化,而且ta还是有自杀危机的。督导师说:嗯,确实有这个可能性,但这也是ta如果要走入心理咨询要走的过程。然后督导师补充说:我想我和XXX(我的姓名)都感受到了来电者在用无力来防御ta对自杀行为的尝试\pozhehao{}如果来电有力量了,会不会就去自杀了。

最后督导师说的一点带给我的安慰是:对我而言,我曾经同时现在也正处于“在用无力来防御对自杀行为的尝试”的状态,而督导师在理解那个来电者的时候,将我拉了进去,同时无意间突然“共情”到了我,让我感受到:ta能在更深的层次里理解到我,虽然这是ta的无意之为,但正因为是无意之为,在我眼里,这更为宝贵,同时也更为安全。

