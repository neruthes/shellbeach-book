\chapter{石头,边界,自我暴露,努力}

\ardate{2022-04-23}{ExmRu8mrEV2mU\_uS99qUNg}


\dialoguelist{咨询师}{%
\dialoguesepline{咨询师}{彼此沉默,我在到处看}
\dialogue{我}{我刚刚在看周围。我发现今天没有袋子了,座位上有一件衣服,你的手环的颜色好像变了,头发好像变乱了点。}
\dialogue{咨询师}{那留意到这些东西会让你感受到些什么吗?}
\dialogue{我}{会感受到一点新鲜感,好像有一些东西是不同的。}
\dialogue{咨询师}{新鲜感,你会想多说一点吗?}
\dialogue{我}{我可能只是在折腾我自己。我昨天失眠整晚没睡,很想睡但又睡不着,所以现在大脑在各种给自己找事情做,比如说去发现一些新的东西、去想为什么。}
\dialogue{咨询师}{那你有想到为什么今天我会有这样的变化吗?我不同在哪里吗?}
\dialogue{我}{我好像没有想到为什么,只是留意到了。}
\dialogue{咨询师}{那当你没有想到为什么的时候,你会怎么办吗?}
\dialogue{我}{我会想到,那就这样吧,那就躺平、摆烂~因为我知道现在(疲惫)的状态终会过去,而且我也不是每次咨询都是这样的状态。我好像是第一次咨询前一整晚没睡。}
\dialoguesepline{咨询师}{彼此沉默}
\dialogue{我}{在上周末的热线团督时,给我一种很不好的体验。那次团督会激发我被攻击的感觉,导致我在那两个小时里一直压抑着自己的情绪。之后我还哭了一回。在弄明白、心智化自己的那些情绪后,当我发现被攻击的感觉是来源于我学生时期经常被言语攻击后,我再悬置那些部分去看督导师,我觉得ta就像是个石头。ta只是在判断哪里做得好、哪里做得不好,丝毫没有任何人本的关怀,包括对来电者和接热线的志愿者的关怀。}
\dialogue{咨询师}{我记得你在上一次咨询也说过我不怎么表露些什么,你会不会也想说我在咨询里也像是块石头?}
\dialogue{我}{嗯,会有这种感觉。当我们彼此沉默的时候,我觉得我们俩就是咨询室里的两块石头。就好像当你是一块石头的时候,我自己的一部分也变成了块石头。但同时我也知道这是一个互动场。就像在微信群聊里,有一个人说另一个人在社交软件上关注了他但没有说话,所以他自己也不敢说话了。我就说:‘但凡你们其中一个人说点什么,也不至于无话可说。’我会觉得很多人只是看到了对方的贡献,而没有看见自己的贡献。就像是在咨询室里我们彼此沉默着,但凡我们其中一个人说点什么,也不至于无话可说。有时候你在咨询室里的肢体语言甚至比言语透露出更多的信息。其实我在这周有几次设想过如果我沉默的话,那你很可能也会沉默,然后我就会说我们就像是咨询室里的两块石头,但我又会因为我说了些什么而没有了沉默。}
\dialogue{咨询师}{但是刚刚你说我像是个石头,但又说我的肢体语言比我说出来的话透露着更多信息?}
\dialogue{我}{嗯,就是一种石头的感觉。就好像你刻意地hold back了一些东西在你的界限里,但又通过肢体语言传达了出来。}
\dialogue{咨询师}{比如说我刚刚的肢体语言里透露着些什么吗?}
\dialogue{我}{比如说一开始你把手靠在沙发扶手上,但现在你把重心放在了身体上。}
\dialogue{咨询师}{这会让你想到些什么吗?我为什么会这么做?}
\dialogue{我}{我会感觉到好像你在用力,因为一开始你发现我的状态好像并没有想说得很多。}
\dialogue{咨询师}{这会给你一种怎样的感觉吗?}
\dialogue{我}{会让我感觉到自己不是一个人在走,这样蛮好的。\\
    但依然会给我一种石头般的质感的边界感,好像你在holding back something,保留着些什么。}
\dialogue{咨询师}{你会设想到我具体在保留着些什么吗?}
\dialogue{我}{我好像设想不到什么具体的东西,好像就是一些属于你自己的部分,比如说此时此刻你会有怎样的感受或想法,一些很零零碎碎的部分。同时我也知道你不会为自己澄清一些属于你自己的想法和感受,而只是跟着我的想法和感受走。你不会说:‘我在那个当下不是这样的,而是有怎样怎样的情感和感受’,你只是一直跟着我走,即使现在我问你在当下会有怎样的感受和想法,你也不会表露。我会感觉到你的界限一直在那里,你会保留着属于自己的部分以及对于自己在那个当下的想法和感受的澄清。不过你(在)之前(的咨询里)也有表露过这些东西(想法和感受),但现在会更少,好像之前你的表露都是要将它们串联成一个解释或一个推论才会拿出来。}
\dialogue{咨询师}{我会想到一个词:‘吝啬’,好像在你看来我吝啬着自己拿出来的东西。}
\dialogue{我}{其实我会想到另一个词:恐惧。不过这可能是我自己的经历。我在一开始接热线的时候,当来电者问:‘如果是你你会怎么办’、‘给我一点建议’的时候,我会害怕。但后来我逐渐摸索出了自己的方式。如果对方是第一次问我要建议,我会共情对方的处境:‘好像你真的很想要一个建议来摆脱现在的处境’,如果ta还问我第二次,我就会分享我自己的经历,但我不会直接给建议,就只是看当我分享完后会发生些什么。但你不会,你的界限在之前的咨询里一直都在。如果我这次不提出来的话,在之后的咨询里它也会一直在那,就像是这个界限从之前的咨询一直延伸到现在。}
\dialogue{咨询师}{我记得之前你说热线的志愿者、我、这个领域的人都是这样。}
\dialogue{我}{嗯,我觉得这个领域的人都不怎么自我暴露,或者说是专业角色上的志愿者或者是咨询室里的你。但我的课程同学以及那个你的用词是‘资深’咨询师朋友不会。}
\dialogue{咨询师}{好像你知道了一些关于咨询的设置,但你依然会在意我没有表露太多关于自己的部分?}
\dialogue{我}{嗯。}
\dialogue{咨询师}{你会感觉到失落吗?}
\dialogue{我}{好像没有太多失落。可能因为我这周也在区分哪些是属于自己的部分、哪些是属于对方的部分,比如说和督导师、和志愿者、和来电者以及和那个资深咨询师的互动里都在这么做。我越来越明确这是个互动场,开始弄清楚哪些是属于我自己的部分,而我也不需要为对方的部分负责。比如说有一个来电者说我并不能理解到ta、共情到ta,我说我很抱歉,我的能力有限,无法共情到每一位来电者,对此我很抱歉。但同时我内心也在想:我很抱歉,但我并没有那么抱歉。这是一个互动场,我有我的贡献,但你也有你的,这并不只是我一个人的问题,你也有你的问题。而且我也会问对方:‘是因为我的哪个回应会让你感受到不被理解吗’,ta说只是一种感觉。我在想:噢,那就这样咯,我做了我的部分了。}
\dialogue{咨询师}{你好像没有想去跨过那个边界?}
\dialogue{我}{嗯,我不需要为对方的部分负责。}
\dialogue{咨询师}{虽然你言语上是这么说,但从你说话的语气里,我好像听到有斗气?}
\dialogue{我}{其实我能感觉到刚刚我在说的时候我的心跳是加速了。我感觉到愤怒,我对对方感到愤怒,因为对方并没有为自己的部分负责。但起码我负责了。}
\dialogue{咨询师}{你现在的状态好像和一开始很不一样。}
\dialogue{我}{嗯,好像愤怒将自己从疲惫的状态里拉了上来。因为那个疲惫的状态里的另一面就是愤怒,而我也在自我监控着不要让愤怒太过于扩散出去。比如说在地铁里上扶手电梯时,我只想把走在前面的人撞倒,就像是方块一样。而我也确实撞开了他们,只不过没把他们撞倒,而且在撞的时候我也在想:我并不在乎他们。}
\dialogue{咨询师}{但好像你现在会在乎我。}
\dialogue{我}{嗯,现在的我的状态好像变了,从不在乎转变成了在乎。}
\dialogue{咨询师}{我会想,你是否也在乎我没有给一些回应、没有表露一些自己的部分,会期望我有所回应、期望我会为自己澄清?}
\dialogue{我}{嗯,而不是像是个橡皮泥一样任由我揉,完全不反抗,也不为自己解释,就我说什么就是什么,我认为的主观现实是怎样就是怎样,完全不反驳。不是我说你是怎样你就是怎样,而是能表露出一些属于你自己的部分。}
\dialogue{咨询师}{你会设想我会表露出一些怎样具体的东西吗?}
\dialogue{我}{我好像完全设想不到。}
\dialogue{咨询师}{我会想起上一次咨询你说的那个雕塑,好像你很快就能够触碰到对方的边界,然后发现那个雕塑雕不出来了,不知道那个部分背后是些什么。}
\dialogue{我}{嗯,会是这种感觉。每个人都有自己的雷区或者说是议题,而当我触碰到对方的有限性时,我就会感到厌倦,会觉得对方也就这样了,不会变了。}
\dialogue{咨询师}{噢?厌倦。你能再多说一点吗?}
\dialogue{我}{就是,我能设想到你会回避自我暴露、自我澄清,而只是跟随着我的感觉走。你之前的用词会用‘就是’、‘只有’这样的用词,好像在检验我的心智化能力。而当我在那个当下的情绪强度而继续说出我的感受和想法后,你并没有去澄清自己在那个当下的感受和想法,而是顺着我的思路继续走了下去。我好像逐渐摸索出了你的固定的模式。这样的模式让我不再对你、不再对对方感到未知。我好像总是冲着未知而去和无论是人还是物接触,而当对方从未知变成已经时,我就会厌倦了,我就想撤回了。}
\dialogue{咨询师}{为什么未知对你来说那么吸引?}
\dialogue{我}{我好像从小就是这样,一个又一个的玩具,总是在探索未知。但现在的话,我总不可能换完一个又一个的人。但当我知道对方的全部已知后,当对方的未知的部分开始消失后,我就厌倦了。我也不知道为什么自己会那么喜欢未知。}
\dialogue{咨询师}{好像如果触碰到了边界后,你会被对方所束缚?会让你感觉被困住、被困在了关系里?一切都动不了了?自己也动弹不了?}
\dialogue{我}{Em……不是。是我在束缚着自己,是我在自我束缚着。因为我知道这是一个互动的场,我有我自己的贡献,但我能将自己的贡献撤回去。就像是有时候跟父母、跟朋友吵架一样,如果我将自己的贡献撤回去,那两个人根本吵不起来,最多只是对方在骂而已。我是一直以来都有撤回的自由的,而不需要总是在场里和对方纠缠在其中。所以是我选择将自己束缚于场里,选择将自己纠缠于其中,同时我也知道自己是有撤回去的自由的,我随时能撤回去。我可以随时选择不自我束缚。\\
    就像我能感觉到你在咨询的过程里一直努力地共情甚至是做解释,我能感觉到你有在努力,但我依然感觉到那些努力只是在既定的边界内。那个边界依然在那里,你并没有踏出去(那个边界)。}
\dialogue{咨询师}{那现在的你会有什么感受吗?会感受到失落吗?}
\dialogue{我}{嗯,会有失落。但同时也有轻松感,因为好像我已经做了属于我自己的部分了,我能做的都做了,我对自己感到满意。剩下的就是你的部分,我并不需要为你的部分负责。}
\dialoguesepline{咨询师}{咨询师不再翘着腿,而是并着脚放在地面上,同时双手叠在双腿之间。}
}

在离开咨询室的电梯里,我偶遇到了也打算离开咨询室的咨询师,我开始明白为什么ta会反常地在临近咨询结束时不再翘着腿(之前的每一次咨询ta都是从头翘到尾的),同时也明白为什么ta的双手要放在大腿之间。在那个当下,还在咨询室里的我猜ta是尿急了。但当在电梯里看见着ta急忙想要离开的样子,那时候还在咨询室时,ta的双腿就像是按耐不住想要离开咨询室,但双手又似乎在安抚着双腿\pozhehao{}要呆在当下。

