\chapter{羞愧的时刻}

\ardate{2022-04-28}{foBb\_RFofWV7nBhV3w\_MQA}




\midnote{以下来源于我的本周课程作业(经修订)}

\noindent\textsf{你有过羞愧的时刻吗?回想一个你最近或最有印象的羞愧时刻,觉察一下,其中的羞愧感是如何影响你的自我感觉的?那一刻,你觉得自己是怎样的人?}

在周末的咨询室和咨询师聊完后,我离开咨询室进了电梯。进电梯后,我才发现电梯是上行的,我还要等它上几层才能下行。下行时,电梯停在了咨询室所在的楼层,我的咨询师进了电梯。我们互相点头后,便没有再正视过,而咨询师看似很匆忙地想离开这个只有我和ta两个人的电梯,ta一直低着头。

我回想起在刚刚的咨询临近结束的几分钟时,咨询师反常地不再翘着腿(之前的每一次咨询ta都是从头翘到尾的),而是并着脚放在地面上,同时双手叠在双腿之间。那时候便给我一种ta很迫切地想要离开的感觉。而当在电梯里又遇到ta时,ta明明可以选择不坐这一趟电梯,但ta依然选择进来了,这让我感觉到:嗯,ta真的很迫切地想要离开。还在咨询室时,ta的双腿就像是按耐不住想要离开咨询室,但双手又似乎在安抚着双腿\pozhehao{}要呆在当下。

在电梯里的我有一点羞愧感,好像我在咨询室外遇到咨询师是一件很不应该的时候,同时我感觉到更为强烈的被拒绝感,被咨询师所拒绝。就好像咨询师很急着想要离开、有要去忙的事情,甚至在咨询里就已经从肢体语言在表达着ta渴望着、迫切地想要离开,而我却挡在了ta的路上,挡在了ta迫切地想离开和我共处的空间的路上。那种羞愧感和被拒绝感的混合情感让我觉得自己很糟糕,自己是不会被爱的,自己总会被遗弃,对方总会无故消失。

\noindent\textsf{助教点评:}

\blockquote{
	你好呀同学,收到你的作业啦!为同学坦诚分享的勇气点赞!在阅读同学作业的过程中,我体会到同学深深的失望、沮丧,甚至是无力、无助和悲伤,事事都在表明着“我不值得”,而咨询师似乎对此毫无察觉,自己在心里默默地“受伤”了,在同学的内心,这些都“真实地发生了”。但从同学的分享中,我又感受到了同学觉察的敏锐,尤其是你分享的对于咨询师非语言信息的觉察,“还在咨询室时,ta的双腿就像是按耐不住想要离开咨询室,但双手又似乎在安抚着双腿\pozhehao{}要呆在当下”,抛开其他,首先想为同学的敏锐点赞,这并不是轻易可以做到的事情哦!很多时候,我们会过度关注在带给自己负面体验的感受或想法上,而忽视了一些值得称赞的部分,或许同学陷入了“假象的被咨询师拒绝”的羞愧中而完全忽视了自己在这一过程中表现出来的优势,羞愧与自我价值有关,它可能会蒙蔽我们的眼睛,或许这也是我们需要针对这一部分开展工作的原因呀!期待同学对自己有更多的看见呀!加油!
}

\tristarsepline

在看到助教的点评后,我想到,“过度关注在带给自己负面体验的感受或想法上,而忽视了一些值得称赞的部分”就像是只看见了图形而忽略了背景,而确实可能存在这样的可能性。同时,在看到点评前,我一直认为自己对非言语的觉察对我自己而言是理所当然的,但我忘记了这对于其他人而言(甚至对于我的咨询师而言)并不是理所当然的,而在助教眼里这种觉察是敏锐的。

我确实无法在每一件事情上都能够从第三者的视角看待我自己,无法脱离自己的主观性,只看到了图形,看不见背景。

也许当察觉到自己正聚焦于某一事物时,我可以自问一下:这里的图形是什么,背景又是什么。

