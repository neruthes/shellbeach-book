\chapter{物化与物质}

\ardate{2022-06-17}{PUD5fu0-f7rIyTglC19miw}



读者群群聊:

\dialoguelistthin{我}{
	\dialogue{A}{我哪会这样防着他(A的对象)。如果会的话,我就不会花那么多钱在他身上。或许群主会懂我,这是一种投资。}
	\dialogue{B}{投资,直接上升到物化男性。}
	\dialogue{A}{物化是最容易感受到的。你工作不也是为了物质吗?}
	\dialogue{C}{这个就叫转移话题,偷换概念。(A的常用语)}
	\dialogue{A}{没有吧。所以最后都是物质。}
	\dialogue{B}{不是啊,我工作是为了实现自我价值。}
	\dialogue{A}{但是最基本问题还是物质。那你可以不要工资。}
	\dialogue{C}{物化和物质是两个概念。}
	\dialogue{A}{那你说下区别。}
	\dialoguesepline{我}{……}
	\dialogue{A}{以后你(A的对象)就要被我吃定了。}
	\dialogue{我}{要被A物化为家庭资产了。}
	\dialogue{A}{我可没把他当家庭资产。资产是金融危机能解决问题的。关键是我没想过以后靠他来解决金钱问题。So,别说的就跟买身体一样。}
	\dialogue{我}{原来只是一种亏损拨备。说不定未来A的对象会越来越赚钱呢~ 就像你一样赚钱。}
	\dialogue{A}{但起码我没想过靠他。所以你说我把他物化成家庭资产是错误的。或许你没谈过一场好的恋爱,所以什么都要计较得失。或许等你有了能力就不会计较太多了。}
	\dialogue{我}{也许等我有钱了,我就不用担心贫穷了。也许等我谈恋爱了,我就不用担心失去了。也许等我死了,我就不用担心活着了。}
	\dialogue{A}{我觉得你是没遇到那个人。如果有能力为对方花钱,那对方高兴了,你是不是也会感到高兴。前提是有余钱,合理地花钱,而不是无节制挥霍。}
	\dialogue{我}{会,但前提是对方不是把这种花钱看作是一种“投资”。小时候我家里人就是跟我说长大了要好好赚钱给他们买一套电梯房,我就跟他们说:原来我的存在只是一个退休养老投资。}
	\dialogue{A}{说投资也没啥问题啊,并不是所有的投资都是需要回报的。我也可以说是对我们未来的投资,对吧。就像你父母对你的投资,也不一定就是为了你能给他们养老。如果是为了这个,那你现在肯定忙着赚钱。所有我觉得你们对投资有点偏见。总让我觉得我在养儿子一样。}
}

刚刚自己感到很烦躁不安,内心充满着各种想法:想到在群聊里看到有群友聊到对对象的投资本身是一种物化;想到小时候认为父母只是把自己的存在当作一份投资计划,好让我在长大后给他们买一套电梯房;想到在毕业后找工作时前任说他不可能等我一辈子,然后就去忙他自己的事业去了,那种被遗弃感,被留下来一个人呆在原地的感觉;想到我并不认同这样的价值观,因为这样的价值观一直以来只给自己带来了痛苦;想到为什么我就一定要改变、要进步;为什么我不那么做的时候,身边的人,特别是对自己而言重要的人就要离自己而去。

后来自己的内心越来越烦躁,越来越乱,我意识到这可能是焦虑,所以就去做了缓解焦虑的冥想练习。身体扫描的时候,我感觉到整个大脑、后脑勺、脖子和后肩部都有一种酸痛和温热的感觉,然后跟着引导语将自己的意识锚定在放在地面上的双脚上、坐在椅子上的臀部上,感觉着地面和椅子的稳定后,自己的心境也平静了下来。

恢复平静后,我在想,如果现在的自己真的什么事情、什么人也不在乎的话,那自己也不会焦虑了。但我依然在乎身边的人会不会突然消失,在乎自己会不会突然被留在原地,在乎自己是否能够做到符合他人的价值观,做到按照他们的意思去改变、去变化、去进步等等等等。我依然有着在乎他人的部分,而这个部分到目前为止只给我带来了对过去的痛苦和对未来的焦虑。这可能也是我不想去在乎任何事情和人的原因之一,因为这样我就不需要依据或服从任何准则或任何人的价值观去做某些一味讨好他人而并不是我能够做到也不是我想要做的事情。

令这份焦虑更为加剧的是,对方并不会提一个具体的目标,比如说前任并不会提一个目标说月薪至少多少,也不会说改变成什么样子,他跟我说的话:“你自己想(思考)去”。所以那时候的我以及现在的我都会感到既焦虑又迷茫,我不知道我应该朝哪个方向走,走多远对方才会满意,才会不因此而离开我。

最近看完型课程的答疑时,讲师又重新提到了马丁·布伯所说的“我—你”关系和“我—它”关系。而我真的很憎恨有人用“我—它”关系来看待我眼中的“我—你”关系:我把你看作是一个人,但你却只把我看作是一个物体,比如说一个投资计划、一个增值资产、一副没有什么特别、随时都能替代掉的躯体。马丁·布伯认为只有“我—你”关系才能缓解人与人之间的那种存在孤独,但在我看来,比存在孤独更为孤独的是,当自己把对方视为一个人,而对方只把自己视为一个“它”、一个物体的时候的那种反差感、失落感、绝望感。那并不只是一种孤独感,而是一种从不孤独落入孤独的跌落感\pozhehao{}原来一直以来对方只是看上了自己在XXX方面的潜力,而对方的陪伴、支持、关注、投注只是一场通过我而达成的对他自己的间接投资。

没有什么是真实的,没有什么是长久的,没有什么是发自内心的,对方终究会离开,而我也改变不了对方些什么,改变不了对方的价值观,改变不了对方看待自己的角度,改变不了对方所作所为背后的理由。当发现自己无论做些什么都改变不了他人,改变不了身边的环境,改变不了这个世界的时候,这本身就是一种被物化的体验\pozhehao{}“所以最后都是物质”。无论付出多少努力、做过多少尝试,无论自己做什么,最终都会化为物质(matter),与其他物质没有任何区别的物质。

Nothing matters. No one matters.

