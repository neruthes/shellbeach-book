\chapter{“一切都是那么的支离破碎”}

\ardate{2022-07-05}{063H7UjaRU884xNjjjn5\_w}



最近一天约饭时,我看见了自己喜欢的男生和他所喜欢的另一个男生在一起了。当时我的内心就像是在被撕裂。然后下意识地,我想到我才不要在乎他。但又马上想到:只要我不在乎的话,那么再多的伤心、难过、受伤感都能够消失掉。但问题是,自己依然在乎,或者说依然有在乎的部分在,而在乎那个我所喜欢的男生的这个在乎的部分让我感到很痛苦。

吃饭时,看着他们在一起交谈,不时牵着手的样子,我会有一种冲动,我很想去加入他们,成为他们的一部分,成为他们那充满爱意的连接的一部分。

后来,在我上的自体心理学课程讲到了自恋疾患当中的自恋人格疾患和自恋行为疾患:(源于笔记)

\begin{itemize}
	\item (自恋疾患)这类结构往往是在发展过程当中,在自尊的位置或在自恋的自体客体需要的位置受到了和羞耻、误解有关的比较大的打击,往往比神经症结构的打击要大、时间更早的打击,或者一些丧失。当这些过程产生的时候,这个人的自体感出现了某种断裂,但这种断裂还没有导致精神病性的问题,远没有到体验的连续性断裂了。这种轻微的断裂对一个人的人际关系产生问题。
	\item 自恋疾患的基本结构主体是稳定的,但是会感到严重的空虚感、无意义感、乏味的感觉、无聊的感觉。这种空虚感是来访者的自体连续性和自体内聚性当中有缺口的地方。
	\item 自恋人格疾患可以通过幻想满足,而有一些高功能自恋人格障碍会在某些领域有高成就。但自恋行为疾患通常付诸实践才可以满足,而且通常是一些社会无法接受的行为。
	\item 自恋人格疾患抵抗自己的这种抑郁感、虚弱感的方法是通过一些幻想来让自己补充力量。而自恋行为疾患的人不是通过这些幻想,而是通过实际做一些什么、发生一些什么来感受到一种自己的存在感。
	\item 这种自恋追求来逃避内心的空虚的办法往往是追求一种卓越的成就,当获得卓越成就时,往往会有一种很好的感觉。但困难在于,这些来访者往往很难去享受自己的这种成就感。
	\item 例如自恋人格水平的企业家,在获得企业发展之后,在努力的过程中很投入、存在感很清晰、自体内聚性很高,但当快要抵达终点、要追求的目标快要被完成时,或者在完成的瞬间,常人感觉会是成功的充实感、喜悦,但来访者会在那个瞬间或前一段时间会突然感觉到一种没有意义的感觉、莫名的无聊感、无意义感。对自己之前热衷的事业突然有幻灭感。成功、获利对他而言已经没有价值了。逃避这个坑的办法往往是创造一个更高的追求。
	\item 这类来访者往往无法维持常态的亲密关系和婚姻。但这类来访者要来做咨询有难度,因为当代社会的许多高成就会掩盖许多问题。
	\item 自恋性人格可能会带来许多创造性以及很大的成就,但他内心的体验往往会有一种不匹配、断裂的部分。
	\item 反社会人格障碍的病人表现出来的自恋特点是完全集中在自己身上,完全不顾及别人的感受。
\end{itemize}

我会想起前任那完全不顾及他人的感受的态度,以及他的性癖好也有反社会的部分。同时我猜想,或许他也和我一样会感受到那种空洞、虚无、无意义感,而他应对这些感觉的办法可能更偏向于付诸行动,而且也并不在乎在付诸行动的同时他人会有怎样的感受、不在乎他人(包括我)会因此而受伤。而我好像能通过幻想来应对这种无聊、无趣的感觉,特别是通过写作(现在也依然如此),尤其是通过写短篇故事,而我写的短篇故事的主题大多是虚无、无意义和死亡。

我记得在上一次心理咨询,咨询师问我:“我也会很好奇,这种无聊的感觉的来源会是什么?因为好像你会在很多事物和人身上都感觉到很无聊、很无趣。” 咨询结束后,我试图去思考为什么我总会感到很无聊、很无趣,但怎么想都想不出来;我试图去追溯这种无聊、无趣的感觉,但我同样追溯不出来些什么。

但是当课程里说到:“在发展过程当中,在自尊的位置或在自恋的自体客体需要的位置受到了和羞耻、误解有关的比较大的打击,往往比神经症结构的打击要大、时间更早的打击,或者一些丧失”后,我试图聚焦在破碎感上,去追溯这种破碎感\pozhehao{}内心的某些东西破碎掉的感觉的起源。

一开始,我回想起一周前我和一对亲戚(夫妻)的约饭。吃完饭后,我和他们一起走去搭地铁。我在旁边看着他们在路上一直都牵着手,我突然想到我从来没有见过我父母牵过手。在那时候,我也有着同样的冲动:我想成为他们的一部分、我想成为他们的家的一部分,就像是我想要成为聚餐时那个我喜欢的男生和他对象之间的一部分一样。

我回想起小时候的我确实有着同样的期望。小时候的我只要一放寒暑假就会去他们(亲戚)家住。他们对待我的态度比我的父母好太多太多:他们不会打我、不会骂我,他们会很考虑到我的感受和状况,以至于我很想要成为他们的孩子。但后来他们有孩子了,还是他们费了很大的努力才获得的他们的孩子。在他们的孩子出生的时候,我知道自己被遗弃了,被遗弃给了我的生父生母,被遗弃在了那种每天无尽地重复的生活\pozhehao{}每天都被打被骂,看不见尽头的生活。那个对一个美好的家的设想破碎掉了,我不可能成为他们家的一部分了,我没有一个家了。

我想起读大三时梦到的一个关于塔的梦(《一个关于塔的梦》)。梦里的我抵达了一个塔顶的平台,看到远处有一对夫妻,近处有个小男孩问我要不要和他一起去远处的那个海贼船玩。那时候的我很想答应,但还是拒绝了,就像是现在的我依然选择和那对亲戚以及他们的孩子(也是个男孩)保持距离。因为我知道自己不可能成为他们家的一部分,那份对有着一对好的父母的家的期望在很多年前就已经破碎掉了。

在我拒绝了那个小男孩后,那个小男孩走到他父母那,他母亲牵着他的手走向远处的海贼船。我突然想说我也想跟他一起去的时候,我便醒了过来。

在梦境里跟上去的机会消失了,而在现实世界里的我很可能也没有“跟上去”的机会,但我好像不需要像以前那样一直回避去触碰这种破碎感了。

当看到课程里所说的内容,当看到说无论是自恋人格疾患还是自恋行为疾患,无论是通过幻想还是实际行动,他们都只是为了抵挡那虚无、无意义感、无聊感,但他们的努力似乎并没有完全成功,他们依然会感觉到那虚无、无意义感、无聊感,依然需要通过可能获得高成就的幻想或不被社会认可的实际行动来不断抵御那些感觉的时候,我会想到:这根本就治标不治本,只是在prolonging the inevitability. 然后我在想:那我能怎么办?我还有救吗?我要怎么做才能填补上内心那个对这个世界上的任何事物和人都感到无聊、无趣、空缺的部分?好像我并不能够在短期内解决这个“问题”、补足这个部分。

也许我能通过一次又一次的心理咨询、读一本又一本关于自体心理学的书,去看在这个漫长的过程里,事情是否还能变得更好,我是否还能变得更好。但无论我做什么,都不可能在短期内马上摆脱这样的空缺感。事情不会马上就好起来,我也不会马上就好起来,就只能这样了。

当我追溯到那份破碎感的起源,那个对一个美好的家、有着一对好的父母的家的期望被破碎时,我又一次意识到:我的过去真的充满着很多很多破碎掉的部分,而我也并没有能力去改变过去,没有办法将这些破碎掉的部分拼凑起来,没有办法让过去、现在和未来的事情和自己变得更好。

我能做的就只有意识到那些破碎掉了的美好的事物、期望、他人的存在,意识到随着那些过往的事物和人的破碎,自己也随之broken了。

但逐渐地,我身边的世界开始发生了改变,或者说是我眼中的周围的世界发生了改变。我开始感觉到周围的世界并不是我之前所感受到的无聊、无趣,而是支离破碎\pozhehao{}人与人之间的支离破碎。

那种支离破碎的感觉出现在了不同的人际关系里\pozhehao{}和已经有对象前任的关系、和最近聚餐时的那个喜欢的男生的关系、和最近几天来了这座城市的朋友的关系、和之前喜欢但拒绝了我的男生的关系、和父母那彼此像舍友般的关系、和亲戚只有约饭的关系\pozhehao{}那种被排斥感、被拒绝感、被遗弃感,好像并没有人想要和我深入一段更为深入的关系,而是选择与另一个人去深入他们自己的关系,与我无关的关系,无论是伴侣关系还是开放关系还是舍友关系还是亲子关系。不仅仅是我与他人,在我眼中的他人之间也到处都是裂缝\pozhehao{}那对亲戚和他们的孩子之间的裂缝,聚餐时那个喜欢的男生和他对象之间的裂缝,父母那舍友般的关系的裂缝。

一切都是那么的支离破碎。我、我与他人、他人与他人之间,支离破碎。
