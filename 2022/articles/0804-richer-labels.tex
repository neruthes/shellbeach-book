\chapter{“给他们贴更丰富的标签”}

\ardate{2022-08-03}{PBnaiWUe7uGkSNm7mxjxtQ}





\midnote{\href{https://mp.weixin.qq.com/s?\_\_biz=MzI5NDUxMzkwNA==&mid=2247487918&idx=1&sn=5d1399e24a00e68898b5d3e38c262d59}{《随笔 | “不”,不见面》}}

自从去年开始筛选出一些不再打算见面的人(主要是为了不在这些人身上再自讨苦吃)后,今年我在和不同的人相处时,也开始给他们贴更丰富的标签(有过度标签化的可能性)。

对于有的人,每几次见面我就会换一个标签。这可能是因为他们在随着时间改变,也有可能是我改变了,我能看见他们身上更多之前难以看见的事物。对于有的人,我需要在几次的见面后才会贴上一个标签。而对于另一些人,我从第一次见面就沿用着同一个标签,一直没有改过。

我会发现与其标签为“不”,将他们的各种内在部分标签化反而更能让自己预期自己会面临什么,更好地帮助我去选择自己是否真的要见这个人、对这个人我应该抱有怎样的期望。

所以以下是我对身边的人的不同标签。如果你和我在现实生活里有多次的互动,那欢迎对号入座~(也可以问我我给你贴了怎样的标签\sout{,当然我说不说是另一个回事})


\section*{自我意识类}

这一类别与作为一个人的存在的最根本的自我意识有关。这一类人的自我意识在我看来是被压抑的、被阻断的、被切除的,甚至是被击垮的、打碎的、冲散的。

\subsection*{无趣/无聊}

这是我最常用的标签。这一类人在我看来都没有多少深刻的内在\pozhehao{}对自己没有足够的了解,没有多少属于他们自己的思想和情感,就像是一个只是在实现着自我功能的机器人、工具人。这一类人是我感到最为厌烦的人。

\subsection*{思维兜圈}

这是我最少用的标签,因为这种思维兜圈(有时候带有反刍的性质),在我看来已经开始接近精神疾病性质的程度。有思维飘逸的人,在我看来,说话的时候会一直绕圈,比如说从一件事情说到另一件事情再到另一件事情,说着说着他会回到原来的出发点(第一件事情),然后又开始一次新的循环。这就像是,对方一直被困于一个思维循环里,永远都逃不出去,一只只能在轮子上奔跑或停下的白老鼠。


\section*{情感类}

这一类别与一个人的情感有关。总的来说,这类别下的人会给我会一种感觉:他们的情感方面是缺失的、破碎的、无法触及的,甚至是丧失掉了的。

\subsection*{回避走深}

每当谈到一些涉及内心事物的话题时,他就会把话题往浅处拉,拉回到一些日常事物性的话题。其中一个特点是,话题的对象不知不觉地从(活)人转移到了事物(死物)上,而且这种转移是频繁发生的,甚至对方是有意识这么做的。

\subsection*{情感隔阂}

这是一个我用得比较多的标签。最近我逐渐留意到,当我向有的人进行深度的自我表露时,他们会表现得很无动于衷,就像是听平常事一样。所以我会怀疑他们可能无法触及到他们自己的情感,甚至无法意识到我是在深度地自我表露,而可能认为我说的内容没有多特别、没有与其他话题里的其他内容有多少区别,无法区分他人什么时候在进行自我表露。正因为无法触及他们自己的情感,他们也更加无法理解他人的情感。

\subsection*{回避情感}

每当我问他,他对此的感受是怎么样时,他可能会说一两句感受(有时候甚至不说),然后开始说一大堆想法,而那些想法还是不带任何情感的。

\subsection*{不(主动)触碰情感}

他不会主动谈起自己或问起对方的情感。当我问他他的感受时,他会说他自己的情感,而不会回避情感或不表露情感,但他不会主动去谈一些涉及内心情感的话题。

\subsection*{不表露情感}

他会说不少丰富的故事(甚至是他自己的故事),但这些故事都是以一种平静的音调说出来的,而且当中几乎不会有任何关于情感的用词,例如悲伤、难过、后悔、懊恼等。即使是在讲述他自己的故事,给我的感觉是:他在讲授历史,仿佛他所讲述的事情已经过去、已经死去好几百年了。


\section*{依恋类}

这一类别与一个人的依恋系统有关,特别是在与特定的人而不是任何人的互动方式上体现出与众不同的一面。这个类别也是我需要多次的见面才能够感受到的,因为依恋方式需要足够的时间和次数才足以与新的人建立并呈现出来

\subsection*{亲密匮乏}

一开始我会误以为这是一种情感匮乏,甚至是情感类别下的“情感隔阂”,但后来我发现他对情感的感知是敏锐的,但往往在一些涉及亲密加深的时刻和场景\pozhehao{}特别是我在进行深度自我表露\pozhehao{}时,他会切断一切回应(包括切断对自己的情感的表露,就像是情感功能突然死机了)。这也让我意识到他更可能是在有意识或无意识地回避亲密。

\subsection*{恐惧型依恋}

这一类人给我的最直观感受是:在物理空间上,他会一直粘着我……特别当分离将近时,会变得异常粘人。比如说,当见完面准备离开时,会利用性行为来拖延分离的时刻。我猜想这一类人在过去的分离议题上受到了重大的创伤。(在社交软件上这一类人还真不少……)

\subsection*{渴望但无触碰}

他会从他的语气、眼神和肢体语言表露出他渴望身体的接触,但他会回避真正的心灵接触。这就像是:只进入身体,不进入生活。

\subsection*{胆怯}

说话的语气柔软,音调微弱,眼神回避性飘忽。甚至在说一句话的时候会突然停下来,不确定是否要继续说下去。


\section*{投射类}

这一类别的人喜欢将自己的主观现实投射到他人身上,特别是那些他们自己不想要的部分。

\subsection*{盖过(他人)主观性}

喜欢用自己的主观现实来盖过去他人的主观现实,而这些盖过去的部分往往是一些他们内心感到痛苦、想要回避的部分。这一类人是让我感到最为憎恨、最不可能再次见面的人。(详见“他更想说的或许是:你很悲哀”)


\section*{自闭类}

这一类别的人让我想到自闭症/孤独症的其中一些特点\pozhehao{}心智化能力(由外观己,由内观人的能力)低、对人际关系冷漠和被动。

\subsection*{自闭}

他的话绝大多数是陈述句,而且话语的对象绝大多数是指向他自己或他身边的事物,而不会指向他人,因此也更不可能用到疑问句或反问句。他给我的一种感觉是:他一直沉浸在他自己的世界里,以至于他看不见别人了。毕竟在自己的世界里,一切都是可控的、已知的,但如果要踏出自己的世界,那么将会面临很大的不确定性,甚至是危险。

\subsection*{很少投注精力于他人}

他说话的内容绝大多数是关于身边事物的发展逻辑和规律、他对身边世界的看法和观念,但他很少会问他人的感受,很少问对方对此是怎么看的、怎么想的、有怎样的感受,而是一直在说他自己的事情。


\section*{自恋类}

这一类别与自恋有关,此处的“自恋”更多指的是自体心理学理论下的自恋。(“自恋是一种藉著胜任的经验而产生的真正的自我价值感,是一种认为自己值得珍惜、保护的真实感觉”,详见百度百科)

\subsection*{只在乎自己}

和这一类人相处时,几乎每一件事情他都只考虑这是否是自己想做的、想要的,而不会考虑对方的想法和感受。

\subsection*{(把他人当)工具人}

同上。但除此之外,给我的一种更强烈的感觉是:对方只把我看成是一个工具,只是在利用我来实现一些他此时此刻所缺乏的自我功能,而当对方的状态有所好转后,自己就会被对方所遗忘、遗弃。比如说会在心情不好时找我倾诉,心情好的时候就不会想到有我这个人的存在。

\subsection*{自恋行为疾患}

需要不断通过付诸行动\pozhehao{}尤其是一些不被社会所接受甚至是反社会的行动(比如说一些违反法律和/或道德约束的性癖好)\pozhehao{}来短暂填补自体感匮乏而导致的严重的空虚感和无意义感。

\subsection*{自恋人格疾患}

需要不断通过构建幻想或取得成就\pozhehao{}通常是一些能被社会接受的幻想或成就(例如写小说)\pozhehao{}来短暂填补自体感匮乏而导致的严重的空虚感和无意义感。

\subsection*{自恋性暴怒}

当他体验到自己的自体(自我)破碎时,会激发暴怒。这一类暴怒与普通的愤怒不同之处在于,他并不只是想利用愤怒来宣泄不满或保卫人际边界,而是他会想要跨越人际边界去摧毁对方的存在,以此来维护自己脆弱的自体感。同时,处于自恋型暴怒的状态下的人不会将对方体验为一个独立的、有主体性的他人(活人),而是会将对方体验为一个可以任供他操控、摆布、毁灭的客体(死物),就像是手里的玩具一样。

比如说当被羞辱时,普通人更可能会说:“凭什么这样说我?”(保卫边界)但处于自恋性暴怒的人更可能说:“我要搞死你。”(摧毁对方)

\subsection*{过于渡人}

以自己主观设想的方式去帮助他人,但这样的方式更多是出于满足自己的自恋需求(例如在帮助他人的过程中自我感觉良好、提升自我效能感等),而这样的帮助未必是对方想要的、需要的。有时候即使对方已经提出对此的反感,这样的人也未必能够完全停止这样的过于渡人的行为,因为在这背后的自恋需求依然未被满足。

\subsection*{只谈自己}

这一类人很像自闭类别下的“自闭”和“很少投注精力于他人”,但有所不同的是:自闭类别下的“自闭”和“很少投注精力于他人”的人在谈论自己的时候,一般不会有太多的情感波动,而更多是以一种平稳、不带情感的态度谈论着他们眼中的事实。但自恋类别下的“只谈自己”的人能通过谈论他们自己的事情来短暂增加他们的自体感\pozhehao{}他们的情绪会变得高涨,他们会对自己所说的话更加确信,也能够从中获得更多自我满足感。(就是苦了被工具人化的听众就是了)

\subsection*{不表露(对他人的)在乎}

他不愿意表露对他人的在乎,也不愿意表露他在有意识地共情/心智化他人。也就是说,他不愿意表露出他是有向他人投注精力、投注期待的。比如说,如果我问他:“你会觉得TA是怎么想的?”他会说:“我凭什么要在乎TA是怎么想的??”我猜想这是为了避免让自己的自恋进一步受损吧。

