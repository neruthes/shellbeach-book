\chapter{心智化,防御,脆弱,残暴,车流}

\ardate{2022-03-16}{rapxYKq9XQdtmvbozsfo2w}





最近自己的心智化能力开始罢工,而我也越来越抗拒去心智化自己和他人,特别是他人。我回想了一下为什么会变成这样。这种状态的出现是因为上周末和一个朋友在线上“吵”了起来,而我在和对方“吵”的过程中一直试图将自己抽离出来,用心智化彼此的能力去看这个过程里究竟发生了些什么,也用这一能力看到了自己当时的所作所为所伤害到的对方的努力背后所珍重的意义和价值。

直到现在,我都很反感回看那时候我和TA的聊天记录,因为我很不喜欢对方发一大段又一大段的文字,对方最好能像我一样把所有内容整合成一篇文章。因为那种一大段又一大段的文字会给我一种被淹没感,好像我在被对方的话语以及话语里充满着情感的文字所淹没着,我就像是站在沙滩边,而浪潮开始掀得很高很高,而且还一浪又一浪地涌了过来。

不过我也知道自己需要在情绪平复后过一段时间来回顾彼此的聊天记录,特别是当我开始发现自己的心智化能力逐渐罢工的时候。

对方感到受伤的并不只是因为我伤害到了对方的努力背后所珍重的意义和价值,而可能更是因为我没有直接表达我对TA的不愉快的情感、那些攻击性的部分,而是自己承受了那些复杂的情感,心智化自己地处理掉了那些部分。

在我的咨询里,我和咨询师之间的互动也是如此,我对咨询师的不满也总是会被我加工为提建议。我并不想直接向他人表达攻击性,至少是不想向我所重视的人这么做,因为直接表达未经加工的攻击性很可能会破坏关系。

我也在想,为什么我不能直接向对方表达攻击性呢?为什么我不能直接对对方说:“操!你!妈!”呢?现在的我会感到很害怕,很害怕对方用语言直接攻击我;而当我回想起那时候很愤怒的自我时,我则是害怕自己愤怒的一面会伤害到对方。好像无论自己处于怎样的状态下,那份害怕都在驱使着自己运用心智化来处理彼此之间发生的事情。心智化就像是一道防御,防御着自己暴怒的一面利用攻击性来伤害到对方和彼此的关系,同时也防御着愤怒的对方伤害到自己。心智化就像是一个buffer,或者是一个飞行器,能让我从更高的视角看待发生的事情。但同时,我也感觉心智化就像是一个壳,自己保护着自己的壳,而现在这个壳开始硬化了、固化了。那这个壳在保护的是什么?

我感觉这个壳在保护着自己内心更脆弱的自我,因为那个自我很害怕自己会和在乎的他人的关系受损,也很害怕来自在乎的他人的攻击。那个自我想要时刻与在乎的他人保持联系,因为那个自我很害怕孤独,很害怕一个人,很害怕无力。

这会让我想起以前和前任相处的时候,当我通过写作将自己的内心的想法和感受呈现在他面前时,他将我的文字武器化,然后惩罚我将我写的东西用各种语言大声向他读出来。从那时起,我便对那份关系彻底丧失了安全感,同时也不敢将自己的内心直白地表露出来,至少在表露之后要经过一定的加工,将那些脆弱的部分通过心智化或其他方式防御起来。

这可能也是在周末的“吵架”里,对方依然觉得我依然像是一块“石头”、只有当对方表露出消极情绪的时候我才会视对方为鲜活的人的原因。对方感到不愉快之处可能是因为对方一直看不见我极力用心智化来自我保护着的那部分更为脆弱的自我,而当对方表露出消极情绪时,我用心智化也用得越“用力”,但这时我不仅仅在心智化我自己,也在心智化对方,而心智化对方的这个过程可能也让对方感受到我有把对方视为鲜活的人。

我想起我经常跟Neruthes说的一句话就是:“你好像把自己保护得很好”,现在看来,我好像也把我自己保护得很好。

我也会想到为什么周末和对方的“吵架”最终会停了下来。那时候我极度睡眠不足(大概只睡了三个多小时),一边和另一个朋友逛街聊天,另一边在线上和对方“吵架”的同时,还要试图心智化在两个场景下的彼此。就像是我要分身成四个人地去代入各自的内心世界,来试图应对两个不同场景里的不同情况。那时候我感觉自己开始透支心智化能力了,感觉就很崩…… 在那一天下来,我觉得自己已经很疲惫了,不想再心智化任何人了,包括我自己。然后我在线上跟对方说:“看完你说的事情后,我觉得彼此的互动里有很多议题/杂质是超出了我自己的处理能力、时间和精力范围。而且我也越来越不想像处理问题一样处理这些事情,好像就让它们在那里就可以了。”

我不再使用心智化能力来保护我自己和保护对方了,而是用一种冥想的方式来看待周围的事物和他人的流动,因为我累了。这也会给我一种平静感,因为当我不再心智化彼此的时候,我也不再有意识地继续卷入其中,而是抽身出来什么也不做,就看着周围的事物和他人从自己身边流过,不再试图挽留些什么、阻止些什么、改变些什么。就像是冥想的时候,不再跟着脑海里的思绪和情感走,而是选择“下车”,站在路边继续看着各种思绪和情感般的车流奔驰而过,而我只是站在路边看着它们。

当我不再用心智化能力保护彼此的时候,好像对方也消停了下来。我想,这可能是因为心智化本身才是一开始的问题所在,它让对方无法看到我极力运用心智化能力保护着的那个内心脆弱的部分。可能当我不再“纠缠”于彼此的关系的时候,对方也不再“纠缠”了。

自从和前任的相处后,我就一直自我苛责着,不能让自己再那么无力了,不能让自己再表现得那么脆弱了。因为当自己身处于无力和脆弱的时候,自己谁也保护不了\pozhehao{}保护不了自己、保护不了对方,更保护不了彼此的关系\pozhehao{}只能看着一切的消亡。如果我不心智化我自己,他人就没有能力看透我了,但当我这么做的时候,对方可能又会因为这一方式而觉得我没有在表露出真实的自我,像块石头般一直在自我保护着。

其实我成长的环境并不缺乏语言和肢体暴力,所以在长大后,我更不想成为那时候的亲戚和家里人恶魔般的面孔。那些没有经过加工的直接的攻击性的表达,比如说用衣架打、扔字典、“废物”、“生块叉烧都好过生你”,那些东西让我很害怕自己会成为像他们那样的人,害怕自己充满愤怒和攻击性的那一面,那一面在经历了过去的事情后变得对暴力和血腥充满渴望,想要向所有人施暴,想看着对方慢慢地在痛苦中死去,想享受他人的痛苦和折磨给自己带来的快感。以至于当现在在大街上看见有小朋友被打骂,或者是马路上出现了什么事故时,我的第一反应是感到兴奋\pozhehao{}心跳加速、头脑发胀、全身皮肤发热,那一刻的我渴望看见他人受尽折磨,最好是既死不了,又活不成。

写到这里,我发现,好像我的心智化能力并不只是在保护着自己内心脆弱的部分,也是在保护着自己内心充满残暴的部分,正如最开始所写的,“防御着自己暴怒的一面利用攻击性来伤害到对方和彼此的关系,同时也防御着愤怒的对方伤害到自己”。不过,当意识到这一点时,我感到释怀多了,因为我看见了自己更为脆弱的自我和更为残暴的自我,同时也看到了自己之前一直在用心智化来保护着自己和对方。

在我最近学的自杀危机心理干预课程里,有提到一个痛苦耐受技术,其基础是一个“温度计”:当受到一定刺激时,人的情绪从情绪基线(什么也感受不到)到情绪波动(感到不舒服、难受),当刺激难以承受时,人的情绪就会抵达情绪激荡的状态(失去理智、更失去心智化能力)。我想我的心智化能力一直将我保护在了情绪波动而不至于达到情绪激荡的状态,以至于在电话热线督导的时候,我甚至说:“我感觉自己没有需要被督导的感觉,因为在接热线的过程中,我有一定的情绪起伏,但它们都是在一定的安全范围内,而那些超过安全范围的情感我能自行处理。”

\useimgsmall{aimg/2022-0316-1}

这种释怀感让我想到,我可以允许我自己有意识地放弃使用心智化能力,允许自己有意识地进入情绪激荡的状态,无论是言语攻击也好、肢体攻击也好。

同时我也发现,那个充满残暴的自我并不是无缘无故地出现,他的出现往往是因为自己那个脆弱的自我受到攻击后,残暴的那部分自我才会随之出现。就像是残暴的那部分自我的出现是为了保护脆弱的那部分自我,而我则利用心智化来将两者相分隔并保护了起来。

让不同的自我自然地“流动起来”,而不是刻意地将他们阻断并保护起来,这会是我还没有尝试过的体验。就像是在冥想时看着车流里各种各样的情感和想法甚至是画面流过一样,我想我也可以在日常生活里看着不同的自我自然地流动,不去试图阻止哪一个自我,不去试图改变哪一个自我,想上车(想成为某个自我)的时候上车,想下车(不再想成为某个自我)的时候下车。


