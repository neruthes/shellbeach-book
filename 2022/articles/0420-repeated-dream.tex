\chapter{团督后的重复梦,丧失感和幻灭感}

\ardate{2022-04-20}{9M0paXcTrooYDdnsI2FYaA}



半夜,我睡醒了。我感觉到眼角边有一点黏稠感,想起刚刚睡觉时我哭过,哭完后又无意识地睡了过去。我试图回想起刚刚的一个梦境:我在我所生活的街区附近,在手机上刷到前任在朋友圈里发了一篇推文,标题为“被迫离开XX(前任公寓所在的小区名)……”,我联想起自己被迫搬出前任公寓的经历,我感到很心痛、很痛苦。很难过很难过,我在梦里一直在哭,然后半清醒地发现自己在现实世界里也在哭,然后又无意识地睡了过去,便忘记了这件事情的发生。

睡醒后,在发现刚刚自己在梦里和现实里哭过后,我才突然回想起,我昨晚也做过一模一样的梦,昨晚梦里的我也和现实的我一并痛哭着,但在昨天早上睡醒时,我却全然忘记了自己哭过这件事。我想到,重复梦很可能暗示着未被处理、未完全浮现在意识里的内容。

\blockquote{%
	在冥想的时候,当引导语说扫描身体,聚焦于悲伤的情绪在身体的哪个位置的时候,我感觉到那种悲伤是在双眼后方的一个球体那。那个球体一开始很硬,随着自己不断哭泣的同时,那个球体开始越来越软,越来越多的泪水也涌了出来。同时,我的脑海里闪过很多想法(其中很多是关于自体的表征/负向认知的),比如说“我没有能依靠的人”、“我很孤独”、“我很无力”、“没有人会爱我”、“没有人会在乎我”,以及到最后我只想带着泪水怒吼出来。

	\blockquotesource{白色灯塔先生}{团体督导,过程,自我表征}{2022}
}

这个重复梦都是在热线团体督导后连续两晚出现的,所以梦中的内容很可能和团督后的冥想里唤起的悲伤感有关。

我试图回想起梦里的那种悲痛感。梦里的我在看到推文后想起了自己曾经丧失了在前任公寓里的生活以及和前任那本可能成为未来的可能性,那一刻的幻灭感和丧失感,就像是小时候丧失了对家的期望感后的幻灭感\pozhehao{}长大后丧失了对前任家的期望感后的同样的幻灭感。

当一年多前得知前任搬离了那个曾经的公寓,去了小区里住后时,那一刻的我感到蛮悲伤的,同时想知道:他会感到伤心和难过吗?或者说他会像自己一样对此感到伤心和难过吗?我渴望知道另一个人是否能感受到我所感受到的情感,但我又知道前任是不可能(至少是在我面前)表露出任何情感的。好像自从一年多前,我就一直将这份渴望\pozhehao{}渴望前任也能感受到我所感受到的情感\pozhehao{}搁置在一边,直到这份渴望所淡忘。

前任的公寓在我的主观现实里是我的上一个家,而现在,现在我并没有一个我能称之为家的家。这可能也是我需要不断购买、不断参加新的课程的原因,因为我会在发现自己没有新的东西可学、没有新的历程能去踏上的时候感受到淹没般的虚无感、无意义感和抑郁感,这些感觉也是从前任公寓搬走那天半夜搬着行李走在公寓楼下小区路上的感觉,而这些感觉又和丧失了对前任家的期望感后的幻灭感紧密地联系在一起。

我会回想起一个朋友在一个月前留下的一段评论:

\blockquote{%
	对啊,既然那么想,那干嘛不去借钱买网课呢?它到底是一种习惯性的防御?或是一种重复?即便如此那又如何?谁说有新方式后就必须舍弃旧方式的?还是它能符合你所憧憬的新风景、新未来、新事物这目标?对此我也会疑惑,所谓对内在世界无止境的探索会不会让你更无力?这样的学习到底能在多大程度上成为你抵御现实中所体验到的无能?
}

各种的课程学习、书籍阅读和自我探索都能帮助我防御着现实世界里没有一个家的处境\pozhehao{}正如大学时的自己每天都会花一到两个小时去背英语单词,打了一千多天的卡。这种感觉就像是,如果自己一直往前走、一直有一个能继续往前走的方向,那我就不用担心如果自己停下脚步的话,我就会发现其实自己并没有一个能休息的地方、没有一个让自己感到足够安全和依恋的家、没有一个自己得以停驻下来的地方;就不需要感受到从前任公寓搬走那天半夜搬着行李走在公寓楼下小区路上的虚无感、无意义感和抑郁感。

我享受不断行走的过程和路程的风景,同时也畏惧一旦自己停下脚步,虚无感、无意义感和抑郁感就会追上来。因为对家的丧失感和幻灭感一直都在、一直都没有离开过,它就在那里、一直在那,like a hole in the heart that could never be filled。

\blockquote{%
	我记得你在第二、三次咨询的时候会提到三种特定的情感:抑郁、孤独、无意义。我想到两种可能性,其中一种可能性是你可能在用这些创作来防御着那些情感。第二种可能性是那些情感还是在的,但你能看见更多其他的情感、更多其他的方面。

	\blockquotesource{白色灯塔先生}{过程,情绪的起伏,没有剩什么,拼图}{2022}
}

我会回想起还在读大学时的自己在美剧《高堡奇人》里看到的一句那时候的自己很认同的关于那时的境地的台词:“I have no one left and nothing left. There's no way out!”我想到,那时候的我和现在的我好像并没有多大变化,依然没有一个能视之为家的家,依然挣扎着寻找way out。


