\chapter{团体督导,过程,自我表征}

\ardate{2022-04-19}{2jQq4h4UAy25DnX\_LHiT-A}



在上一周的热线团体督导里,我将我接热线的一个接了几分钟就被挂电话的案例报给了督导师,然后督导师在团体里讨论这个案例以及我的回应和做法当中做得好的地方和做得不好的地方。

整个被团督体验对我而言绝大多数都是糟糕的\pozhehao{}我并没有感觉到这个团体氛围里有支持,而在督导师说我的不足之处时,我会感到很悲伤、愤怒和受伤,会想到:我已经被来电者这样对待了,为什么现在还要这样对待我。

在案例团督结束后,我很想哭,但又哭不出来,因为在整个团督过程自己一直在压抑着情感。而且另一个不想哭的原因是,如果我哭了,那么接下来我就什么也干不了了,我只想躺在床上盖着被子在一片黑暗里哭,哭到睡过去。我不想自己的剩余时间都在床上度过。所以后来我选择了一个折中的办法:去做了一组关于处理悲伤情绪的冥想练习,也哭了短暂的几分钟。

在冥想的时候,当引导语说扫描身体,聚焦于悲伤的情绪在身体的哪个位置的时候,我感觉到那种悲伤是在双眼后方的一个球体那。那个球体一开始很硬,随着自己不断哭泣的同时,那个球体开始越来越软,越来越多的泪水也涌了出来。同时,我的脑海里闪过很多想法(其中很多是关于自体的表征/负向认知的),比如说“我没有能依靠的人”、“我很孤独”、“我很无力”、“没有人会爱我”、“没有人会在乎我”,以及到最后我只想带着泪水怒吼出来。

在冥想结束后,双眼后的那个球体已经松软得难以被轻易觉知。但我依然记得那些悲伤、孤独感、被遗弃感以及暴怒感。我开始在想冥想时脑海里出现过的那些话,那些话对我而言曾经是多么的熟悉,那是我在学生时期里经常“听到”的话,比如说考了一个在父母眼中很糟糕的成绩后他们对我的不满、责骂和家暴。这可能就是对我而言的“易感性”了吧。

然而在团督的过程中,我并没有获得多少支持。不过,我也会想到,在有限的督导时间里,督导师是否就应该提供支持?还是在ta眼里应该提供更多技术、能力层面的指导?如果是支持和安慰的话,我只能从团督里获得吗?我能从自己的咨询师和同辈团体里获得支持和安慰。但我依然有一种感觉:那个督导师缺乏对人的关怀\pozhehao{}包括对来电者的关怀和对接线员(被督导者)的关怀,而是几乎完全聚焦于对纯粹的技术、能力层面的指导。但不可否认的是,ta的知识和阅历让他在案例里能看见很多初学者看不到的方向和潜在议题,同时在我看来也是个asshole。

不过我想到,(处理情绪的)过程相比内容而言更为重要。

\blockquote{
    尽管高水平人格病症患者总体上能够进行自体反思,但他们在冲突区域的自体反思能力较弱。也就是说,当冲突被激活、情绪变得强烈时,他们的思维会变得更加具象化,体验也会变得更加紧迫。随着思维变得越发具象化,患者理解心理表征的象征性质且反思它们的能力会受到损害。在治疗中,当无意识冲突被激活时,观察中的治疗师将会加入患者变弱的观察性自体中,推动患者进行自体观察和自体反思。上述过程在每次会谈与治疗的进程中反复出现会帮助患者发展出更好的自体反思能力,即使患者正在面对焦虑和无意识冲突。在治疗过程中,随着冲突被修通,患者的自体反思能力会加强,其自体探索也会更少依赖治疗师的推动。

    \textbf{临床范例:自体反思能力增强}

    一名研究员前来接受治疗,其主诉为与自尊有关的问题。虽然就其内在世界的许多方面而言,他都完全有能力进行自体反思,但是,当谈到其自卑和残缺感时,他的思维会变得更加具象化。与他对自已的态度一样,在治疗早期,这名男性私下认定自已是治疗师遇到过的“最差的患者”。事实上,他太过确信这一点,以至于犹豫了好几个月,才把自己的担忧告诉治疗师。虽然他能够明白这可能不是真的,但与此同时他又十分相信,自己一定是该治疗师所有患者中最难被治愈的。随着时间的推移,治疗师请患者探索他这样看待自己的意义,而不是仅仅作为客观现实来接受这种看法。随着患者与自尊有关的冲突被修通,他不再认为自己是治疗师遇到过的“最差的患者”。

    然而,几个月后,患者发现自己再次出现了相同的体验。这时,与治疗刚开始时相比,他采取了不同的态度来看待自己的自责。现在,他能够维持这样的认识:他思考和感受到的东西,只反映了他当前的心理状态,而非客观现实。也就是说,他正在体验一种特定的自体表征,该表征是在特定时刻、由于特定原因而被激活的。这使他能够反思思维的意义,而不是像以前那样把思维体验成一种具体现实。

    \citebook{人格病症的心理动力学疗法}
}

我发现自己一直以来也在做同样的事情:区分哪些想法和感受只是反映着当下的心理状态、被特定刺激源所激发的心理状态(比如说被遗弃和被批评)。以前的我就真的把这些自体表征(对自我形象的表征)当作现实情况,比如说:“我没有能依靠的人”、“没有人会爱我”、“没有人会在乎我”,这也是读小学时的我在边做作业边哭泣的时候对自己说过无数次的话。有些话说得多了,自然就成为了现实\pozhehao{}自己认同了这些话就是客观现实。

但现在,我的心智化能力更强了(当然冥想也有很大的帮助),不再是“我没有能依靠的人”,而是我认为“我没有能依靠的人”/我感受到到一种“我没有能依靠的人”的感觉,能够有意识地察觉到自己和那些情感和想法(包括自动出现的自体表征)在何时是毫无距离可言的,并能够有意识地保持距离,而当需要去探索情感和想法背后的事物的时候,又敢于将自己投掷于其中。

可以追着各种各样的情感和想法的车辆跑,也可以回到路边看着它们经过。

