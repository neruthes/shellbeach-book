\chapter{自我探索 | 14}

\ardate{2022-01-04}{Jx-QEWQm96UFDAmA3jMllg}

\dialoguelist{咨询师}{
\dialoguesepline{咨询师}{进入咨询室后,我确定了一下我最想要说的是什么}
\dialogue{我}{在上一次咨询结束后,我感觉到一股悲伤,一种像是放弃了自我的悲伤。我感觉在上一次咨询里,我只是遵循着你在上上次咨询的提议\pozhehao{}无意识地说些什么,而没有做我自己。
我甚至会有一种被操控的感觉。在上上次咨询,你问我:‘如果不带准备地来咨询,你设想会发生些什么?’我说:‘可能我会继续无意识地说下去。’在我说完那句话后,我才突然意识到,这更像是你想说的话,但却通过我来让我自己说出这句话。其实在之前的咨询里,我也发现了类似的可能带有操控性的引导,比如说在之前我好奇你的穿着服装背后是否意味着你的生活里存在着某些高光时刻时,你引导我对你的生活进行了设想,一些你既不会去承认也不会去否认的设想。我想这可能是你的一种防御方式,但我能理解那时候为什么你想要这么做\pozhehao{}毕竟你想避免暴露自己的生活。但我难以接受你在上上次咨询用这样的方式来引导我说出你自己想说的话。然后我想了一下上上次咨询的背景:那时候我提出我想要结束咨询,而你不想‘放我走’。当想到那个背景时,我大概能理解为什么你会有这样的意图\pozhehao{}将咨询引向新的方向。但我觉得你可以直接提出这个建议,比如说:‘不如我们可以试着让你不带准备地来咨询,看之后会发生些什么’,但你并没有直接这样提议,而是通过一种隐晦的方式来让我说出你想说的话,所以我才会有一种被操控的感觉。}
\dialogue{我}{那种被操控的感觉会让你联想到什么吗?}
\dialogue{咨询师}{那种被操控的感觉会让你联想到什么吗?}
\dialogue{我}{嗯,我会联想到小时候被母亲、被外婆操控着我穿什么、吃什么、喜欢吃什么、不喜欢吃什么、我对外界的冷热的感知、我对自己的性格的认知。比如说,读小学时有一次去观展,我妈在我出门前让我穿了很多衣服,但那天很热,不过我并没有脱衣服,而是一直闷热到放学回家。一方面,我觉得很热,但另一方面,我妈说我不热。后来读到心理学方面的书籍时,我才知道这种认知操控所导致的矛盾心理是会引发精神疾病的。而且小时候我妈和我外婆都很喜欢在饭桌上塞东西给我吃,我那时候的反抗方式是在吃得很饱得时候把胃里的食物直接吐在饭桌上。}
\dialogue{咨询师}{这种吐的方式频繁吗?}
\dialogue{我}{只是小时候的偶尔几次,后来我就开始学会拒绝了。}
\dialoguesepline{咨询师}{我想咨询师可能是想到厌食症和催吐行为的可能性吧}
\dialogue{我}{Em……会像是塞,但不是硬塞,而更像是一种很隐晦地暗自地塞。}
\dialogue{咨询师}{我有留意到刚刚你在描述的时候经常会用一个词:‘硬塞’,我在想,你在咨询一开始所说的我的‘引导’,会不会也给你一种像是我在‘硬塞’一些东西给你的感觉?}
\dialoguesepline{咨询师}{我停顿了下,回顾了一下刚刚的过程后}
\dialogue{我}{其实现在当我将这些秘密说了出来,我感觉很释怀,而且好像我也并不需要知道事情的对错。我在前一两个月也向前任坦白了一些我所知道的他的秘密……(此处省略某些关于亲密关系背叛的秘密)……当我把这些事情告诉前任后,我好像也不需要他的承认或否认了。}
\dialogue{咨询师}{我会有点好奇为什么会释怀?}
\dialogue{我}{Em……就好像我承担了很多不属于我自己的部分,比如说一些关于前任的破事,然后我将这些不属于我的部分排了出去,扔回给了对方。无论对方是自己解决还是怎样的,我不再需要为对方承担责任、不再需要承担那些不属于我的部分。这让我感到自我完整性。我不需要担心彼此的关系会变成怎样,我只是做我自己就足够了。}
\dialogue{咨询师}{听到这里,我还挺为你感到高兴的。同时,我也会好奇,为什么你会选择现在把这些秘密说出来,因为你好像在咨询的一开始就能看见很多我没有看见的事物。}
\dialogue{我}{以前的我会不敢说出这些秘密,因为担心这些秘密会破坏关系。而当和前任分手后,就更加没有说出这些秘密的必要性了,因为关系已经无法挽回。但现在我开始意识到,我保留着的这些秘密其实会损害我的自我完整性的,就好像我默许着这些事情的发生一样。而当我把这些秘密排出去后,我感到更加释怀了。而且,凭什么要我为对方的破事负责!\\
当我在咨询里把这些秘密说出来时,我不需要担心被你所攻击或无视或怎样的,而如果我在咨询之外的人际关系里这么做,很可能会遭到对方的攻击或无视或冷淡地对待。\\
这些秘密就像是咨访关系里的刺,如果我不将这些刺挑出来,不去面对这些东西,那么我们能说的事情也会越来越少,就像和前任的关系一样,能和他说的事情越来越少,因为我很怕触碰到那些我不敢去面对的雷区\pozhehao{}有关亲密关系的背叛的那些秘密。后来我们能说的话越来越少,而且两人的关系早已说不上是亲密了。我不想我们的咨访关系也变成这样,我不想像之前好几次的咨询那样只是聊一些表面的内容。而当我能看见这一点时,我便开始向你提意见,想将关系变得更深更广。而且这么做的效果好像还蛮好的?至少我的自我感觉很好~\\
我想,在咨询里,我开始逐渐看见我所做的事情,原来我之前就有这么做过,就有对前任这么做过,而且我开始逐渐看清这样的做法的脉络。}
\dialogue{咨询师}{那你之前会有这样的经历或感觉吗?}
\dialogue{我}{嗯,有的。之前我在写作的时候能深挖到感觉背后的经历、起源和脉络。但感觉本身是不受我控制的,而这种(将秘密说出来)的做法是我能够控制的。\\其实我在咨询一开始的本意只是想澄清在上上次咨询里可能存在的操控的部分。而在你的引导下,我好像将这个部分和自己的童年与前任的经历串联在了一起。}
\dialogue{咨询师}{这会给你一种怎样的感觉吗?}
\dialogue{我}{一种很强的效能感。}
\dialogue{咨询师}{好像一种你在掌控着主权的感觉。}
\dialogue{我}{嗯。不过我想之后的咨询我们还是会回到平等的地位。因为好像只有在平等的地方,甚至是像上上次咨询我提出想要结束咨询时,我们在彼此抗衡的过程中,一些真实东西才会冒出来,一些很隐晦但又很真实的东西冒了出来。}
\dialogue{咨询师}{好像你很看重‘真实’。}
\dialogue{我}{嗯。因为如果不真实的话,彼此的关系很可能会像和前任的关系一样越走越窄,在不知道发生了些什么事情的情况下关系就结束了。我并不想这样,我想在关系里做真实的自己,无论这段关系会变成怎样。\\
我也会想到,之前我说我感觉我在咨询里没有什么话好说,所以想结束咨询。这让我想到,这会不会是因为在咨询的环境里没有足够的空间让我创造出属于我自己的东西。\\
当回顾我们的咨询历程,在咨询的过程中,我好像越来越擅长处理一些看不见的东西,或者说越来越能够觉察或认知到这些部分,觉察或元认知的能力越来越强,而且应对环境的能力也还不错?同时,我也开始发现,在咨询的历程里,咨询的焦点好像逐渐从我的身上转移到了我们彼此的关系上,而且这一次的咨询完全起源于上上次咨询的那个带有操控性的引导。而我在应对我们彼此之间发生的事情的能力也在不断提升,开始在创造一些属于我自己的应付方法。}
\dialogue{咨询师}{好像你能看见很多我没有看见的部分,然后你会将它们拿出来说。}
\dialogue{我}{嗯,是的。但我也会在想,这不是咨询师的工作吗?这不是你的工作吗!}
\dialoguesepline{咨询师}{咨询师笑了笑,我也笑了起来}
\dialogue{我}{不过我相信这不仅仅对于我,对于你而言,你也能在这个过程中有更多关于自我方面的成长。不然如果其他来访者只是一味跟随你的引导来回答,我想你的工作对你而言可能更像是一成不变的。}
\dialogue{咨询师}{其实在这五个月的咨询里,通过你,我也对我自己在咨询里呈现的状态有了更多的了解。}
}

