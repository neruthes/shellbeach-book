\chapter{“眼前的一切就像是电视机屏幕里的画面一样闪过”}

\ardate{2022-07-26}{rpupBkAVXotc4MAQr0NUHA}





最近接电话热线的时候,遇到一些边打热线边干其他事情的人,比如说边打热线边刷抖音的。在我看来,边打热线边干其他事情的这一方式有可能是在回避痛苦,因为当我邀请对方多聊一聊刚刚难受的感觉时,对方聊了几句又扯到刷抖音看见XXX的事情上。不过,我也想起之前和一个朋友聊语音时,他会边聊天边刷社交软件,说哪个男生看上去怎么样怎么样。他甚至会在彼此沉默的时候开始哼歌。我想这是他抵挡无聊和尴尬的方式之一,人与人之间的无聊和尴尬。

第二天,我突然想到我妈在我小时候打完我之后经常说的一句话:“家里人怎么会有隔夜仇”。但正是这份不被她所正视过的仇恨,一直持续了近二十年,还在持续着。我身边也有一些朋友会很本能性地回避深入的话题,甚至很有意识地将走深的话题往浅处拉(也是一种很有趣的技巧)。例如,在谈到群聊里进群又退群的人时,那个朋友会说:“进进出出是正常的”(性暗示)。

其实回想起自己认识的几个朋友当中,并没有多少朋友是真正交心的,超过一半的朋友都只是一群吃喝玩乐的朋友。而当难过、不开心的时候,我并不想找他们聊天,因为会感觉:自己从来都没有真正认识过对方,而我的自我表露也不会得到回应。所以想到这一点时,即使在平时的心情并不糟糕的时候,我也不怎么想联系那些只会吃喝玩乐的朋友。毕竟在我眼中,他们都是同质的,并不独一无二的。

最近几天,我发现之前那个有好感的男生和他对象又发了一次动态,动态里又是拍着同一处地方,只不过是拍出了不同的角度。我猜想这是他们“秀恩爱”的方式,而这样的方式又激发到了我的被遗弃感。一开始,我屏蔽了他们的动态,不想去看他们的动态,不想再被激发到自己的被遗弃感。但后来又想到,我总不可能躲着这种感觉一辈子,所以后来还是解除了动态屏蔽。

我想也许只有正视痛苦本身,才能够拥有选择放下还是留着的自由吧。

最近的美剧《西部世界》里有一句台词:Sometimes the things that feel most real are just stories(有时候我们感到最真切的反而只是故事)。这让我想到,我们对他人的认知、对自己的认知似乎绝大部分都只是由各种各样的故事拼凑而成,例如上一次见面时发生的故事、对方的过往经历(故事)。难道我们仅仅只是我们所讲述的故事的集合吗?

最近自己又有那种感觉:自己眼前的一切就像是电视机屏幕里的画面一样闪过。与电视机屏幕的唯一不同之处是,我好像换不了台,最多只能开机和关机。一种不真实的感觉,但又说不上不真实在哪里。感觉自己的意识被打烂了,变成了一片一片的碎片\pozhehao{}某一时刻想到一处地方,另一时刻想到另一个处地方;某一时刻想到一件事情,另一时刻又想到另一件事情\pozhehao{}碎片与碎片之间毫无关联。

我试图放手,让自己的意识随着这种感觉迷失在其中。首先是一种很强烈的被遗弃感,然后这种感觉开始消失,自己的意识不断闪过回忆里的各种场景和画面,最后回到一个场景:还在读幼儿园时,有一次我从二楼的午休室的楼梯滚到了一楼。我不记得自己是怎么摔的,甚至没有意识到自己是在往下滚,只记得自己打算从一条很长的楼梯走下去,结果一下子就到了楼梯底部的地面上。我记得在地面上时,我感到很迷惑,不知道自己在哪,然后看到好像也没有人发现我从二楼滚了下来,然后我就去洗手间了。回想起那时候滚下来的过程,我并不是有意识地脚软,但那种滚下去的感觉就像是脑海里的意识在不同的记忆里滚动的感觉,放弃自己的身体和意识,屈服于自然的规律,一直往下滚、往前滚,滚往一个自己无需控制的方向。那种感觉会很舒服,但落到地面的时候,好像会有一种隐约的被遗弃感\pozhehao{}在那天以及之后,我没有跟身边的幼儿园老师或家里人讲这件事,因为如果跟他们说的话,当时我能设想到的只是被打和被骂。如果我下一次从二楼滚下去的时候死了,估计也不会有人发现我。如果有人发现了我的尸体,也只是会继续骂我打我。


