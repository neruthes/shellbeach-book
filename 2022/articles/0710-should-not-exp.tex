\chapter{“本来就不应该有人要去经历这样的事情,感受到这些感觉”}

\ardate{2022-07-10}{ME78O06g-oOaUvy3Qp9GZQ}



进了咨询室的我,看了看咨询师穿的衣服上的文字,然后我看见咨询师笑了笑。我开始说我进咨询室之前打算说的话。

\dialoguelist{咨询师}{
	\dialogue{我}{我记得上一次咨询结束时,你问我这种无聊、无趣的感觉好像在我与生活当中的很多事物和人都会出现,你问我为什么会有这种无聊、无趣的感觉。然后在结束了咨询后,我去思考了一下,但想不到为什么,然后我尝试把注意力放在这种无聊、无趣的感觉上,试图去回溯这种感觉的来源,但是我依然回溯不到些什么。当最近在上自体心理学的课程时,我突然想到我可以讲注意力聚焦在破碎感上,然后当我回溯这种破碎感的时候,我回想起:我每次放寒暑假都会去亲戚家(他们是一对夫妻)生活个大概一个月的时间。在和他们相处的时候,我会感觉到蛮好的,他们不会打我骂我,也会顾及到我的感受。我记得从一年级开始就有去,在那之前可能也有。但当9岁的时候,他们费了很大的精力终于怀上了他们的孩子,然后那时候我就想起:噢,他们有自己的孩子。在那之前我一直都有一个幻想\pozhehao{}如果他们没有他们自己的孩子的话,那我是不是能够成为他们的家的一部分。但当意识到他们有了他们自己的孩子后,我就感觉,那种期望、那个幻想破裂了,而我要回到我生父生母那继续生活。这会给我一种破碎感、被遗弃感。\\
		而且在上周周末聚餐的时候,我看见一对情侣,那对情侣的其中一个男生我还是有好感的。看着他们相处得很好的时候,我会有一种冲动\pozhehao{}我想要成为他们之间的一部分。同时在上上周和亲戚(那对夫妻)聚餐后,和他们一起走去地铁站,看着他们牵着手的时候,我突然想起我从来没有见过我父母牵着手的样子。而看着他们牵着手的时候,我也有那种同样的冲动\pozhehao{}我想要去成为他们之间的一部分。}
	\dialogue{咨询师}{你能说一下“成为他们之间的一部分”具体会是一种怎么样的方式或画面吗?}
	\dialogue{我}{我会想起在读大学时我做过一个关于塔的梦:当我抵达那个塔顶的平台时,我看见近处有一个小孩子,他想走过去远处的他的父母那。在我看来就像是一个小孩子在他的父母之间牵着两个人各自的手,一起走向远方的画面。}
	\dialogue{咨询师}{你还记得和你那对亲戚相处的时候会有怎样的回忆和感觉吗?}
	\dialogue{我}{我记得有一天晚上,我在他们的房间里看着他们打游戏,那是一个滑雪的游戏。他们坐在床上,我坐在在地上的毯子上。我们在吃着哈密瓜。就是在那一刻我好像感觉我能够融入他们,好像我是他们之间的一部分。}
	\dialogue{咨询师}{(沉默)}
	\dialogue{我}{在这周,我都有为此而哭过几次。然后好像就没有了那种很无聊、很无趣的感觉,取而代之的是这种破碎感。而现在我好像更能看见自己内心破碎的部分,就是从九岁时的那一刻开始,我好像就有了这种破碎的感觉,然后在后续的人际关系里这种感觉越来越强烈,越来越破碎。当能看见自己内心破碎的部分的时候,我眼中的周围的世界也发生了改变\pozhehao{}我好像越来越能看见身边的他人内心的破碎的部分、裂缝,更能看见自己和他人之间的裂缝,也更能看见他人和他人之间的裂缝。}
	\dialogue{咨询师}{这种出现在你自己、你身边的人和裂缝会带给你怎样的感觉吗?在你的生活里是怎么呈现出来的?}
	\dialogue{我}{就是一种裂开的感觉吧。比如说昨晚一个很久没见的朋友见面聊天。聊了一段时间后,我跟他说:我发现好像在聊了那么多,你好像都只是在描述一些故事、一些事件,从一个事件到另一事件。当然我会听到你说一个表达情感的词,比如说困惑。除此之外,我好像就没有听到太多描述情感的词。而且在你讲述事件的时候,你的语音语调都是保持不变的。就是很单一的语调、很monotone。好像你在讲故事的时候,你将你的情感保留了起来,没有直接表露出来,而只是通过一些非语气词(比如说“啧”)来表达。就像是一杯已经很满的水(情绪),水多得已经溢了出来,溢出来的那些边角料就是“啧”的这些非语气词,但好像并没有直接表达、直面情感。\\
		然后他说:你会是一个很厉害的咨询师。但之后就没有了。我就在想:我提出来是因为想要知道他所说的事件、故事背后的那些情感。比如说他在年初尝试去做一名咖啡师,后来失业了,但过了一段时间现在又重新去做一个咖啡师。这个经历本身本应该有更多的情感,甚至是在我听他说的时候,我自己就已经代入那个场景去体验到了很多情感。但是当我回过神,我意识到他好像并没有表露什么情感的时候,我就反馈这一点。但是当他说:你会是一个很厉害的咨询师之后,他就没有把那些经历里本可能存在的情感表露出来,而是把焦点转移给我就算了。\\
		所有我会看到好像我和他之间存在着一道裂缝,说不定在他自己内心,他也和他的情感之间存在着一道裂缝。那道裂缝一直都在那,只不过这个裂缝之前被他所讲述的事件本身所遮盖了,好像看似我们之间是有联系的。但当把这些事件拿开,我就看见两个人之间的存在好像存在着一条裂缝,将这个裂缝呈现了出来。好像我们之间并没有太深层的连接。那个裂缝就在那里,一直都在那里。\\
		所以当我能够看到我内心的裂缝、他人内心的裂缝、我与他人之间的裂缝、他人与他人之间的裂缝的时候,我会对他人有更多的同理心。就好像他们也和我一样,只是这个支离破碎的人际环境里面的其中一个产物,也和我一样地支离破碎。而我就没有像以前那样的judging,会去评判说为什么他们看不见自己的情感、为什么他们那么的不同频。现在的我好像会更多地包容他们,因为能看见我们之间的共同之处。虽然这种融入感并不是连接,但也会让我感觉到我们之间是有相似之处的\pozhehao{}我们都是那么的破碎,虽然他们可能看不见这一点。}
	\dialogue{咨询师}{我接下来这么说,可能听起来会有点像攻击或者是挑战。你好像也不怎么会表露自己的情绪。}
	\dialogue{我}{嗯。不过我不太确定我在之前咨询是否也是这样呢?也会给你这样的感觉吗?}
	\dialogue{咨询师}{这次咨询的状态会有所变化,但上一次咨询我记得是这样的。上一次咨询好像你所呈现的情感只有无聊、无趣,而没有其他的情感了。不知道你会不会留意到这一点呢。}
	\dialogue{我}{其实我会有留意到这一点,就是我和身边的两三个朋友聊的时候,他们会告诉我说,他们看不见我不会表露出任何情感。甚至就像是我现在在讲述的时候,我的语气都是很平静的、没有什么改变的。像昨晚见面的那个朋友说我就像是一杯水,无论往往里面扔什么或者是怎么摇,那杯水依然都很平静地在那里。其实我是有情绪的,即使现在我保持着一个很平稳的语气,但我内心其实是有情感的,我能看见那些情感的起伏,它们的起与落。但好像这些情感不会通过语音语调或一些非言语信息表达出来。}
	\dialogue{咨询师}{你会过为什么会这样吗?}
	\dialogue{我}{嗯,我会想到小时候被我妈打,然后我不能表现出自己很受伤、很难受很想哭、很脆弱。因为只要我表现出情绪的话,她只会打得更厉害。所以我不能够去表露自己的情绪、不能表露自己的脆弱,否则只会让自己更加受伤。直到后来上了大学,我首先是尝试通过写作来表达自己的情绪,后来我也尝试去用言语来表达自己的情绪。但是我发现我好像依然不能够在非言语层面自然地去表达一些情绪。\\
		好像从小我就把自己的情绪有意识地封闭了,然后当我逐渐找到一些办法\pozhehao{}无论是写作也好,言语也好\pozhehao{}去表达自己的情感后,我发现自己的非言语信息(比如说语音语调)依然无法自然地表露出情感。这可能也是我的局限之处。就好像小时候知道了那对亲戚有了他们自己的孩子,但现在的我回过头来看这些事情,都已经过去了十几年了,我也不可能说回到过去去改变这件事情,去让现在的自己不那么的破碎。我也找不到什么办法能将自己统整在一起,也就只能这样了。\\
		所以现在我也会慢慢接受这种裂缝的存在。我这周和另一个朋友过夜了,然后那个朋友说起他他对象的事情。他说他不敢在工作和日常生活去表露情绪,因为他对象在他看来是那种很容易受到负面情绪影像的人。就是当他向他对象表达一些负面情绪后,他对象受到的影响好像比他还要严重。所有他就不再敢跟他对象表露负面的情感了。\\
		当我听到这里的时候,我下意识就想去纠正他,想跟他说:你可以表达你的负面情感。但我并没有这么做,我并没有尝试去填补他与他对象之间的那条无法表达负面情绪的裂缝。因为我想到:这并不是我的事。那条裂缝就在那里了,我只是需要看到它们的存在就可以了。}
	\dialogue{咨询师}{我会留意到一开始咨询的时候,你看了一看我的衣服,然后也没有选择说什么。}
	\dialogue{我}{一开始咨询的时候,我看了看你衣服上面的字,然后我看见你笑了一笑。就是在那一刻,你知道我在看你衣服上面的字,我也知道你知道我在看你衣服上面的字。就是这种你知道在干嘛,我也知道你在干嘛的这种默契感。\\
		同时你笑了一笑的时候,我在那一刻好像也隐约感觉到了一种坚定感:就是你知道如果你穿了这件衣服来到这里,我肯定会留意你衣服上面的字,但你依然选择穿这件衣服出现在了这里,所以好像会给我一种坚定的感觉\pozhehao{}即使这一刻会发生,你也并不回避、不逃避,而是让这件事自然而然地发生。而当我能隐约感觉到你的那份坚定感的时候,我也能够让自己更坚定地开始说我一开始进来咨询之前就打算说的内容。}
	\dialogue{咨询师}{这种坚定感会让你想到些什么吗?}
	\dialogue{我}{我会想到我的生活里好像并没有人会有这种坚定感,就是很坚定地知道自己想要什么、自己想干的事情是自己想干的。我记得好像就只有在前任身上看见过这个部分,然后当和前任相处的时候,我也很想去内化这个部分。比如说我现在说话的这种平稳的语音语调就是想要将自己呈现得很坚定,让自己听起来很确信我现在说的这句话就是我想要说的。}
	\dialogue{咨询师}{那这种确信会给你一种怎样的感觉吗?}
	\dialogue{我}{一种稳定、安全的感觉吧。其实我的一生都没有过什么安全感。这周一起过夜的那个朋友,我也有跟他说过这一点。然后他跟我提温尼科特的理论,说:一个婴儿一开始从乳房那获得安全感,后来转移到像奶嘴,然后到玩偶这样的过渡性客体,最终能够从他自己身上获得安全感。我说:那如果没有这个过程呢,如果没有这个过程,那这个个体是不是就不能存活了呢?因为对我来说好像我一直都没有一个这样的过程。自己一直以来都没有多少安全感。}
	\dialogue{咨询师}{而且好像你也会在不同的人身上去测试这个部分,想要去找到这个部分。}
	\dialogue{我}{嗯,因为这个部分自己一直以来都没有。有时候我会把这种特质投射在别人身上,但是当我看见对方好像真的没有这个部分的时候,我也就不会投射,如果放在自体心理学的角度来的话,这些坚定感、安全感本是父母应该有的特质,但是我好像并没有从父母的关系里内化到这样的部分,所以只能通过现在的人际关系去获得。但现在我和他们的关系并不是父子或母子关系,我也不可能将我和他们的关系扭曲成父子或母子关系。但我依然只能从这样的人际关系里去获得、去内化这些我还是个小孩子的时候本该从父母身上获得、内化的部分。}
	\dialogue{咨询师}{嗯。其实我也会留意到,在这么长的咨询过程里,你都没有提到过你的父亲。}
	\dialogue{我}{可能因为父亲离场了吧。在我看来那只是我母亲的丈夫,而不是父亲。}
	\dialogue{咨询师}{离场是指?}
	\dialogue{我}{他人在那,但他不再参与我的生活、不再和我一起做一些爱好。就像是一个空壳。所以当说起内在父母的时候,我内心并没有一个父母的形象,或者说是一个好的父母、好的客体的形象。如果让我自己成为一个父亲或母亲的话,我也不知道自己应该呈现成怎样的样子,因为我并没有一个这样的形象在。}
	\dialogue{咨询师}{那你和你的亲戚之间的相处,你还有怎样的画面或感受吗?好像在你和他们的相处里,你依然获得了一些东西,才成为了现在这个样子。}
	\dialogue{我}{我不太确定。当我回想起和他们相处的时候,我的回忆里好像并没有多少的画面和感觉,只有一些零星的碎片。所以当我试图回想起他们相处的回忆,想要从中获得一些稳定感、安全感、力量感的时候,我是没有办法获得这些部分的。}
	\dialogue{咨询师}{好像他们也根本称不上是好的客体。}
	\dialogue{我}{是啊。}
	\dialogue{咨询师}{这听起来会蛮令人遗憾的。}
	\dialogue{我}{我会留意到你的用词是“遗憾”,我会觉得你在弱化情感的用词。因为在我看来,这是崩溃、绝望,就像是每次被打都感觉自己要濒临崩溃了。不过这是我的情感,你的情感是你的情感。}
	\dialogue{咨询师}{可能因为我真的难以想象没有父母地生活是一种怎样的感觉。}
	\dialogue{我}{嗯,本来就不应该有人要去经历这样的事情,感受到这些感觉。}
}
