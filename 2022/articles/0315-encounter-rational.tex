\chapter{相遇,想法和感受,纯理智,机械体}

\ardate{2022-03-12}{LRKCCroMI9Mgi2OEqStsMA}





在之前上的人本主义课程里,我经常看见“相遇”一词,但一直把“相遇”视为两个人的相遇,而且课程里也没有对“相遇”一词有任何释义方面的界定和解释。不过我在昨晚读到了以下一段:

\blockquote{
Stern强调当下的时刻是需要给予特殊关注的,通常持续几秒钟:这些时刻的新异性、投入度、不可预测性以及可能产生的问题使之具有特殊的治疗价值,因此需要一定的心智活动和心理工作。潜在的最具变化性的当下时刻就是相遇的时刻,他指的是解读心理:我们称之为移情中的相互心智化。

最能引发我们兴趣的当下时刻出现在两个人发生某种特殊的心理上的接触一一换句话说,是一种主体之间的接触。它包含了心理之间相互的渗透,使我们可以说:“我知道你知道了我知道的”或者“我感觉到你感觉到了我感觉到的”。这是对他人心理内容的一种阅读。这种阅读是相互的。两个人至少在一个时刻看到和感受到了大致相同的心理图景。这样的相遇是心理治疗最主要的部分。

\citebook{心智化临床实践}
}

所以,“相遇”似乎更强调的是心智理论,即我知道你知道我知道你知道……或我感觉到你感觉到我感觉到你感觉到……“相遇”强调的是彼此的心理世界的相通,而不是或不仅是物理现实世界的彼此在场。这也很容易让我想到,有很多人的相遇并不是“相遇”,即很多人在物理现实世界的在场并不代表彼此的心理世界就相通了,比如说聚会时各玩各的手机。

在上周周末,我和朋友Rue和他对象Yang一起约饭。约完饭后,我们回到朋友Rue的住处,他对象Yang给他订了一个生日蛋糕。在我们一起吃蛋糕的时候,Yang坐在地上的垫子上,Rue坐在沙发上,Rue把双腿放在垫子上的Yang的大腿上,然后旁边还有其中一只他们养的猫,在Rue坐上沙发前,那只猫还在沙发上躺着。这一幕激发起了我对家的温馨感的向往,同时也回想起,以前还住在前任公寓时,我在客厅的地板上铺了被子,然后我和前任有一晚就坐在被子上,背靠着沙发,通宵打游戏,不过基本上是我看着前任在打游戏。那时候前任养的那只猫也躺在被子上犯困。这两个场景好像都让我感觉到了一种“相遇”、一种连接。

家对我来说意味着什么?当我在咨询室里提起“家”这个主题时,咨询师并没有止步于此,而是问我家对我来说是什么、意味着什么?那时候我说,家对我来说是一个对方愿意为我停驻的地方。但现在我想,家更是一个彼此“相遇”的地方,一个彼此的想法和感受能够相互连通的安全的空间。

但在这个空间里,并不需要太频繁且刻意的心智化,而是让事情自然而然地发生,自然而然地内隐心智化自己和他人。就好像在看着Rue和Yang吃着生日蛋糕的时候,我好像能隐约感觉到他们同在着,就好像曾经的我还住在前任公寓时,看着前任打游戏的样子,感觉到他和我同在着。那是一种很自然而然的连接感的涌现,而不是自己刻意想要去思考和感受对方在思考和感受着些什么。

这好像也解释了,为什么我在面基和与朋友相处的人际关系方面一直以来都在追求那种相处得很舒服的感觉,就像是对方总能get到自己在想什么、在感受着什么的感觉,感觉对方能感觉到我的感觉,因为那是一种被自然地心智化的感觉\pozhehao{}对方好像毫不费力就知道了自己在想什么和感受着什么,无论是以一种更内隐或更外显的方式的心智化。

不过,我也会想到另一种很僵硬的方式,僵硬地试图理解另一个人在想什么、在感受着什么的方式。

上周末和另一个朋友见面时,他总是在问我“为什么”,因为他在上周看了我写的推文后有很多疑问。但我对这样的轰炸式“为什么”感到很烦躁,因为这并不是我想要聊的方向,而更像是被另一个人拽着去解释为什么以及下一个为什么。

我想到我在接电话热线时和在生活里都很少刻意地向对方僵硬地提问,因为真正“流动”的提问是那些看似不像是提问的提问。比如说:

\dialoguelistthin{我朋友}{
    \dialogue{我}{你今天走路的速度好像比平时快了。(为什么你今天走那么快?)}
    \dialogue{我朋友}{噢,可能是因为我今天穿了跑鞋,所以会习惯走快一点。}
}

我和Nerthes的大多数聊天也像是一个这样的过程:Neruthes先设想一个用于理解我的理论,然后我再解释为什么这个理论work或不work。同时我也会对此感到很烦躁,因为这样的纯理智交谈\pozhehao{}去解释为什么\pozhehao{}并不是人与人之间的“相遇”,并不是“我知道你知道了我知道的”或者“我感觉到你感觉到了我感觉到的”,而更像是对方设计好了一套渔网般的理论,然后往我身上套,看能套得住多少、看有多合身。或者是对方没有提前设计好一套理论,但会用“为什么”来试图探究我的内在运作机制、我大脑里的齿轮是怎么运作的。两者的共同点都是,这仅仅是用纯理智甚至是机械化的方式来试图理解另一个人的存在,这种理解他人的方式并不把对方看作是一个灵活的、充满未知的生命体,而是把对方看作是一个可以被理论化、被解剖、被量化、被预测的死物。

我无法和一个机械体相处/交融得很舒服,因为一个机械体并不能理解和感受他人的想法和感受\pozhehao{}他人的想法和感受并不是相互分离而是相互交织的事物。同样,当对方试图以一种纯理智的方式来理解我和看待我的时候,我感觉对方也只是一个机械体,因为这个机械体只是在寻找着另一个机械体。
