\chapter{“哪些哲学理念和你人生观较为契合”}

\ardate{2022-04-11}{oC\_ZNuNgo-U\_wTsk\_GG5fQ}





\midnote{以下来源于我的本周课程作业(经修订)}

\textbf{课程中提到影响完形疗法的哲学有哪些?分别重要的影响是什么?而你觉得哪些哲学理念和你人生观较为契合,为什么?}

人本主义把人看作是一个整体,看重人本身。存在主义强调死亡、自由、责任、意志和孤独。现象学强调悬置、描述和均平化。场域理论强调事物(figure)和背景是相互影响、相互牵连的。以及一些与禅宗有关的正念觉察。

我觉得存在主义和我的人生观比较贴近。

我一直觉得人都会死,每个人都是毫无选择地被投掷于这个世界,也毫无选择地被迫离开。“一切存在都是毫无道理地诞生,因软弱而延续,因偶然而死亡。”(出自萨特所著《恶心》)但同时,有的人又会因为自己终有一死而珍惜现在短暂的存在。

人是自由的,但由此产生的自由意志又会让人感受到存在孤独。束缚着自由的是责任,但责任又是能够让人与人之间构建起连接的桥梁,也是与自我构建起连接的桥梁。

每个人都是毫无道理、毫无选择地被投掷于这个世界,并注定无法真正触碰到另一个同样孤独的人。但每个人依然继续在这短暂的一生里停驻着,因偶尔而死亡,但又是一种必然。自由且孤独着,因此才需要责任将自由意志与他人和自我相连,对他人负责、对自己负责。既渴望着自由,又渴望着连接;既活着,又死着。

我会想起生活里曾经出现过的一只猫,并目睹到它在面前窒息而死的场景。就像是小说《刀锋》里的男主人公在战场上目睹自己的战友突然被杀后,我也开始好奇,存在是什么?那只猫的死亡真的是必然吗?一个活生生的生命、充满着自由意志的生命突然变成了一团僵硬的肉体,意识究竟存在于哪里,又去了哪里?我有对它负起责任吗?它也有对我负责任吗?就像《小王子》里的那只狐狸、那朵玫瑰,我和它有彼此“驯化”吗、视作唯一吗?它还停驻在这个世界的时候孤独吗?因为我好像并没有给它留出太多的时间,而是忙于自己的事情。即使无法真正触碰到彼此,也许单纯的触摸就已经足够了呢?亦或是自己依然做得不足够呢?没有在那个当下成为更好的人、作出更好的改变、让事情变得更好而不是更糟呢?

有很多提问并不会有一个确定的答案。我想这可能也是作为存在的一部分吧。

“人类拥有丑陋复杂的恶意\pozhehao{}用谎言拼命挣扎、排挤他人、相互争夺、重复着借口\pozhehao{}即便如此,还是要越过山丘,向彼方前行。”(出自《黑执事·马戏团篇》)
