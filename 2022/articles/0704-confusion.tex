\chapter{“怎么理解一个人的困扰形成的?”}

\ardate{2022-07-04}{6PODLxr9GRZDGYNEhcxNgA}



\midnote{以下来源于课程作业(经修订)}

\textbf{第4次作业:每个人都可能遇到困扰,你认为自体心理学是怎么理解一个人的困扰形成的?}

自体心理学会结合发展心理学的角度去看,个体在发展的过程中,首先是从一种很分散的自体感,通过从外界得到不同的自体客体需要的满足,在自己的内心构建起不同的自体\pozhehao{}比如说一开始是核心自体,之后是夸大自体、理想化的双亲影像、孪生自体等不同的自体,并通过和这些自体的互动来满足自己的自体需求,而在满足自己的自体需求的同时,建立起更为稳固更为内聚性自体。在人的一生中不断地投射和內摄。

而题目所说的困扰,在我的理解里可能更像是个体面临的挫折。如果这是一种恰好的挫折,或者是不会带来创伤性体验的挫折,那么这能够让个体调动他之前所获得的自体客体得到满足的经验,来帮助他应对面前的困扰、挫折。但是如果这个个体的自体感匮乏,或者是他过往并没有这样的自体需求得到满足的经验时,那么这个个体可能就缺乏应对这个面前的困扰、挫折的能力。而另一种困扰可能是当个体的自体感匮乏时,个体所应对的方式(比如说垂直分裂和水平分裂)也会给个体在未来带来不同的困扰,比如说一些挫折性的体验,被垂直分裂或水平分裂处理后,也有可能在后续的发展进程里显化为症状地“冒”出来。

我会留意到题目所说“每个人都会遇到困扰”,在我理解里,这个“困扰”更多是客观层面的困扰。但题目的后半段所说的“一个人的困扰”,我会认为这种“困扰”更多是内心层面的困扰。那么这个困扰是如何从客观现实转变为主观现实的呢?

如果放在自体心理学的角度来看,这种从客观现实到主观现实的跨越、这种困扰的跨越可能是来源于自体感的匮乏\pozhehao{}也就是个体在面对客观现实的困扰时,无法运用调动他已有的自体感或自体客体需求得到满足的经验来应对、来抵抗这种客观现实的困扰时,那么这种困扰便可能从客观现实跨越至成为个体自己的主观现实的困扰,开始侵入、扎根了下来。

在现实生活里,当遇到一些困扰时,我也会自问为什么这件事情对我的影响那么大?而不是理所当然地认为这就是一个困扰而已。我也会做客观现实的困扰和内心现实的困扰的区分。

在最近的生活里,我发现自己一直在与各种各样的事物和人相处的过程中很容易感觉到无聊、无趣。这一点在我的心理咨询师看来也非常的奇怪,会好奇地问我这种无聊的感觉的来源是什么、是怎么出现的。但我并不知道。这种无聊、无趣的感觉似乎是我一生以来都有的感觉\pozhehao{}对这个世界里的事物和他人感到失望,感觉这个世界并不应该仅仅只是如此,应该有更多的东西。

如果放在自体心理学的角度来说,也许在面对这种无聊的感觉的时候,我并没有足够的自体感去抵抗、去应对这种无聊、无趣,以至于这种无聊、无趣成为了我内心现实的一个困扰,而不仅仅是客观现实的困扰。但这也只是可能的解释之一。

\useimg{aimg/2022-0704-1.jpg}
