\chapter{零碎的想法 | 31}

\ardate{2022-01-16}{L\_2pR9SX3eh-Fvcm87Zq6g}


\section*{1}

在周初的微信群友聚会中,我加了一个有好感的男生的微信,然后在临近周末时问他这周末会有空出去走走或吃点东西吗?两个人那种,不是一波人。他说他“这周不太行,之后再说咯”。然后我在微信群里看见他说他这周是有空的,想看一看另一个群的群友打不打算约他。这好像也让我更加确信,他并不是周末没空,而是婉拒了我。这让我感到伤心,一种被有好感的人所拒绝的伤心。然后当看着群聊里那个男生也参与到了有关性的话题上时,这让我感到更加难受,并想在周末约个炮。

我知道我只是想利用性来回避一些我不想去面对的事物和感觉,就像现在的我依然会在自己心情很难受的时候买一杯很甜的饮品或吃几颗糖\pozhehao{}一些能够分散注意力的事物。但另一方面,我又不想让自己只是出于回避痛苦而去约炮。性本该是一件愉悦的事情,而不应该是一件用于回避痛苦的事情,就像和有的人相处时经常被提议去看电影,就好像对方在利用看电影来打发时间并回避深入的沟通。

\tristarsepline

一开始心情蛮难受的,但当在脑海里将自己想说的话“说”出来后,我感觉自己舒服多了,所以我决定将这些话写下来。我意识到,只是将那些让自己感到痛苦的内容表达出来本身就已经具有“疗愈”效果。这也让我想起了一年前读过的那本《书写的疗愈力量》里的内容,以及最近在看的《The Human Elements of Psychotherapy》里关于nonmedical model的其中两个原则:“Humans are evolved to develop, maintain, and restore their emotional well-being through supportive relationships with others”(人类被进化为通过与他人的支持性关系来发展、维持和恢复他们的情绪健康)以及“Humans are evolved with the ability to give and receive emotional healing through social means”(人类在进化过程中具有通过社会手段来给予和接受情感治疗的能力)。我想,当自己很不开心的时候,我真正需要的并不是各种各样的distraction,而是需要向另一个人去倾诉那些使自己深感痛苦的事物。



\section*{2}

大概在半个月或一个月前,我做了一个梦:我梦见自己在一辆公交上,这辆公交开在一个日落下的山坡小路上。这部公交要开往一个很远的地方,因为这条线路是由一个机构的总部开向一个已经完成了的任务的地点。这条公交线路的目的就是为了隐藏那个任务地点曾经的存在。

我是这个机构里的工作人员,而坐在我旁边的男生也是我的同事,车上的人都是我们的同事。我坐在公交车靠前的左侧窗边,而他坐在我旁边靠走廊的座位。我在其他座位的遮挡下牵着他的手,我们的手放在了他的左侧大腿上。我记得他的手是修长且铜色的,但我不太记得他的外貌。公交开到一半时,我和他下车去清理路障。在把路障清理好时,我醒了过来。

睡醒后,当我回想起和他牵着手的那个时刻,我感到很幸福,一种我很久都没有在现实世界里感受到的幸福感。

\tristarsepline

在前几天睡醒时,我发现我做了另一个梦:我梦见自己在收拾一个场地,在将场地的桌椅收拾完后,我发现这里变成了高中教室。我在教室里见到了一个大学同学,也遇到了一个高中女同学。那个女同学的样貌变丑了(在我的记忆里,她高中时的外貌还蛮好看的),而且她怀孕了,肚子表面是一凹一凸的骨架的形状,就好像她怀上了一个畸形的孩子。

然后我梦见自己坐着公交,窗外是一条沿海小路,车上都是同学,因为我们刚从教室里出来坐上这趟车。我看了看手机,想到现在已经坐了近半个小时的车,但还没有到站,我想到我快要迟到了。我坐在公交车靠前的左侧窗边,坐在我旁边靠走廊座位的是一个高中男同学,他在高中时的女朋友是那个刚刚在教室里遇见的那个女同学,那个怀着一个畸形婴儿的女同学。那个高中男同学的身材是瘦高的,皮肤是铜色的。我把右手靠在他的肩膀上,他的左手牵着我靠在他右边肩膀上的右手。我感到很平静,一点都不需要担心迟到的事情。

睡醒后,我回想起牵着他的手的那种感觉,那种感觉就是上一个梦境的那种幸福感,原来我在上一个梦境里牵着手的人是他。



\section*{3}

在去年11月底和这个月月初,公众号群聊里有个群友说我有的只是共情知识,但没有共情体验,也更加不会共情。他说我大可以不去学习(他认为是有关共情的)这类课程。在和他的后续私聊里,他说:“你没有必要改变自己。我之所以会选择指出你的这个问题,只是说,你前阵子看共情的课程,某种意义上是希望自己能够做到这一点,但是你又做的不好。那不如不改变。你不是鱼,有如何去体验鱼对在水里的感觉的知识呢,只有理论,却永远无法得到真实的实践,又有啥用呢?如果不合适自己,不如放手。”

当时我感到一种被侵犯感和一种愤怒:对方凭什么有权告诉我什么才是适合我自己的、我应该干什么、学什么、怎么做?对方凭什么有权践踏我所珍重的事物?

当那份被侵犯感和愤怒随着彼此距离的拉远,拉远到我能感到安全的距离,拉远到不会被他所一连串地攻击的安全距离后,我开始思考,什么是共情的知识?因为我所学的课程里根本没有有关共情的知识,或者说并没有任何一种能做到共情的技术,因为共情不能被简化为一项技术,而是一种对待他人的生命和存在的态度\pozhehao{}去试图感受他人可能感受到的事物和可能会有的想法。而且我们永远无法做到真正理解他人,这是一个永远无法达成的目标,所以我们需要做的是不断努力去更好地理解对方、更贴近对方的内心世界。

\useimg{aimg/2022-0116-1.jpg}

\useimg{aimg/2022-0116-2.jpg}

\citebook{The Human Elements of Psychotherapy}

当最近读到这一段时,我会想到,那个群友所说的“共情的知识”背后好像是一种硬科学的态度,有着这种态度的人或许会认为共情是可以被技术和知识所囊括的。但那个群友并不完全是这样,他认为我学了知识但缺乏体验,并且利用“缺乏体验”这一点来试图否定我的努力,试图让我“放手”我的努力。

我也会想到,他和我前任有一个共同点:他们都很擅长看见他人所珍重的事物并将其武器化地践踏他人。所以我逐渐发现,他的话的伤人之处并不在于他言语里的内容,而在于内容之下的那种态度,那种可以随意否定他人所珍重的事物、否定他人的努力、甚至近乎于否定他人的人格和他人的存在的态度。
