\chapter{“更期望能和有对象的人深入关系”}

\ardate{2022-07-29}{iFKwllDD-6d0T5Q3FM4o\_Q}




昨晚做了一个梦。梦里的我和朋友A走在江边,走到一片由集货箱改建而成的酒吧街。朋友A是我之前喜欢的有对象的男生。然后我在手机上看见之前有好感的另一个男生B发了一条动态,动态里又是一张风景图(我记得这是他和他对象特定的秀恩爱方式\pozhehao{}以不同角度拍同一片景色或室内物品来暗示两人的在一起)。风景图下面的文字大概意思是:现在的生活过得很好,没有必要再回顾7月X号的事情(那是我和他见面的那天)。

和朋友A走在江边的我想到,现在我身边的朋友A也不差。虽然朋友A已有对象了,但起码朋友A这个人还在这里,而不像那个男生B连人都不在了,约不出来见面了;起码我还有朋友A这份陪伴,比那个男生B给自己带来的虐心感要好。

当醒了过来后,我马上去社交软件看那个男生B以及他对象的动态,发现看见他们的动态没有更新过,梦境只停留在梦境。在理智上,我会认同那个男生B不再约出来见面的处理方式,但在情感上,我依然感到很伤心、难过、被遗弃感。

在梦里看见那个男生B的动态时,我会有一种设想:自己也能和他一样有人爱、有人陪,能够被爱、被陪伴。所以好像只要我能和那个男生B在一起的话,我就能获得这些东西了。但我真的只能通过伴侣关系获得爱和陪伴吗?在梦境里,走在朋友A旁边的我依然感到被陪伴。

在现实生活里,我和朋友A在前一天才刚见完面。我们见面时聊的一个话题是,我最近遇到了一个对我而言,肉体很有吸引力但内在很无聊的男生C,而我还在思考要不要尝试和那个男生C继续深入关系。朋友A说可以试试。当他这么说的时候,我也会感到一种陪伴感\pozhehao{}对方不会因为我可能会找另一个人在一起而离开我,这段朋友关系并不是排他的。我想起以前经历过的恋爱关系里,对方总是想要彼此的关系是排他的,甚至在两人约会的过程中不想让其他朋友加入进来。

但我真正想要的,只是想被对方视为独一无二,想要被爱、被陪伴,但不意味着这样的关系必须是排他的。

我想起之前和另一个朋友D相处时,对那个朋友D又一次离开了这座城市而去和他对象生活这件事,我也会感到有一定程度的被遗弃感。而朋友A会说,“遗弃”这个词他通常只会用于很深的羁绊,例如需要对他自己的人生负起责任的某段亲密关系。然后他问,为什么这个朋友D会激起自己这么强烈的情绪。他的回应让我想起一件事情:被遗弃感的另一面是曾经的拥有感\pozhehao{}拥有对方、拥有一个家、拥有一个对美好未来的憧憬。这些好的事物在被遗弃的过程中都破碎掉了。

其实每当我发现自己喜欢的人已经有了他们自己的对象时,我内心的一部分都会期望对方能够放弃他们现在的现任而和我在一起。虽然在理智上,自己会说要尊重对方的选择、也许这与对方的道德观相冲突之类的话,但在情感上,我依然想要占有对方,至少是占有对方心目中的那个对方会视其为独一无二的位置,那个属于亲密关系伴侣的位置。我对这个对方内心的独一无二的位置的期望是:能够被对方视为独一无二、被爱、被陪伴。

我有问过朋友A:几年前,你是出于怎样的原因选择了和你对象在一起?朋友A说他那时候就是想要有一份陪伴,对方只要不太胖、不太丑就行。我问他,那你和你对象在一起后,你能获得这一点(陪伴)吗?他说能的,毕竟这又不是什么很高的要求。后来我说:我会蛮羡慕那时候的你选择和你对象在一起的(择偶)标准是那么的简单。

好像我总会向那些已经有对象的人投射那份期望,期望被对方视为独一无二、被爱、被陪伴,但却不会向单身的人投射这样的期望。结果便是,自己更期望能和有对象的人深入关系,而不会想和单身的人深入关系。相比于看见别人的恋爱而想发展自己和另一个人的恋爱,我更想占有那个已经身处于恋爱关系的人……因为我所遇到的单身的人在我看来并没有拥有那些我所期望得到的东西\pozhehao{}被视为独一无二、被爱与被陪伴,只有身处于恋爱关系的人才拥有。

为什么在我看来,单身的人并没有拥有我所期望得到的那些东西呢?好像我所遇到的单身的人都没有这份力量感,这份能够视对方为独一无二、能够去给予爱和陪伴的力量感。而在我的身边,单单是能够做到不仅仅是吃喝玩乐而是能够互相谈论内心事物的朋友就已经少之又少了。


