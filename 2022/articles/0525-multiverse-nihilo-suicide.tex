\chapter{《瞬息全宇宙》\pozhehao{}虚无,自杀,参与生活}

\ardate{2022-05-25}{Lw7YW944mm9wAjzzv8C\_Kw}




昨天看完《瞬息全宇宙》后,我的第一印象是:这是部蛮糟糕的电影。一个中年妇女面临和丈夫的离婚、和女儿关系的濒临破裂、和父亲关系的束缚,甚至连经营多年的洗衣店也因税收问题而面临被收回。而在她令人崩溃的一天里,更崩溃的是,多元宇宙在她面前展开。最终,这位母亲牺牲了自己的稳定意志并成功利用母女之情拯救了想要自杀的女儿。

如果将多元宇宙从剧情中抽离出去的话,甚至可以写成:女儿因出柜和母女问题而离家出走,母亲也同样离家踏上寻女之路,而女儿则在离家出走路上强迫性重复地找了一个又一个的女人试图成为她们的女儿。在找遍一个又一个的城市后,母亲最终在路边的一间公路旅店里找到静脉上还插着针管的充满绝望和无力感的女儿。女儿最终选择回家,并摆脱了毒瘾。

不过,粗糙的剧情并不妨碍一些超乎细节的东西在细节里绽放。

从时间线上来看,Jobu在Alpha宇宙里被她母亲的实验推得太过而导致意识分裂,然后她创造了充满一切事物的黑洞bagel,希望以此自毁。但同时她又并没有这么做,而是开始在无尽的多元宇宙里寻找她母亲,并将她们杀死。但Jobu并不是为了杀死她们而杀的,而是为了拓宽不同的多元宇宙里她母亲的意识,希望能在无数个宇宙里找到一个能和她一样意识分裂后还能成功活下来的母亲\pozhehao{}Jobu想要找到那个能够感同身受到她的感受的母亲,并和母亲一起走入那个黑洞bagel里。

Jobu想要自杀的原因涉及一定的存在主义议题\pozhehao{}虚无:一切可能的结果都只是统计学意义上的必然结果,没有什么特别的;只有片刻的那些能让生活充满意义的时刻,剩余的时刻都被无尽的可能性所冲散了。体验过无尽个宇宙里的痛苦和失望后,Jobu视黑洞bagel为唯一能够找到平静的方式。

Jobu看待现实的视角在一定程度上能被视《存在主义心理治疗》里提到的“宇宙观点”:“从这个位置来看,我们和所有其他生物变得渺小而愚蠢。我们只是无数生命形态的一种。生命中的种种行为变得十分荒谬。那丰富、充满体验的片刻在时间的无限延展中变得微不足道。我们感到自已是微小的尘埃,生命的全部也不过是弹指一挥。”而在书中,应对这种情况的其中一个办法是去参与生活:“参与生活并不能在逻辑上反驳宇宙观点所提出的问题,但是它能够让这些问题变得不再重要。……不管是什么带来了无意义感,治疗的答案就是参与。全身心地参与生活不仅可以消除宇宙观点带来的无意义感,还可以提高个体以某种和谐的方式完成生活的可能性。组建家庭,照顾他人,构思和参与项目,去发现、创造、建设,所有这些以及所有其他形式的参与都能够带来双重回报:它们能够丰富个体,并且可以缓解由存在的残酷现实直接造成的强烈不安。”

当Evelyn打算和她女儿一起走入黑洞bagel时,她注意到她的丈夫正在将现实搞得一团糟,她放不下她丈夫(保护因素:与重要他人的联系),所以她决定不跟Evelyn走入黑洞bagel。而Jobu则打算自己一个去自杀,不再尝试说服Evelyn和自己一起去自杀。

“参与生活”恰恰也是Jobu的母亲Evelyn所用的方法\pozhehao{}通过与丈夫的人际联系,延伸到母女间的人际联系。Evelyn在最后巧妙地追溯到了Jobu当初为什么不顾一切地寻找她,并说:“无论如何,我都想和你呆在当下。我一直一直都会想和你呆在当下。”Jobu说:“在这里,我们共度的只是片刻的那些让生活充满意义的时刻。”而Evelyn说:“那我就会珍惜这些时刻。”这会让我想起在之前学的自杀危机干预的课程里的讲师有说道,要去思考为什么进入了咨询室的来访者没有直接去自杀,而是在跟另一个人诉说着自己的自杀意图和计划,当中的保护因素是什么?是什么保护了来访者活到现在?对于Jobu来说,那个保护因素是一份希望\pozhehao{}希望能找到一个能感同身受到她的母亲,并且希望能够不孤独一人地走向虚无。而Evelyn为她女儿找到的另一个保护因素是她与女儿的人际联系:她一直都会想要和她女儿呆在当下。

但这种想要并不是一种控制。当Evelyn奋力阻止Jobu走向虚无时,Jobu说她累了,让她走吧。Evelyn同意了,但又回头继续挽回\pozhehao{}她愿意接受她女儿想要选择另一条道路,即使这意味着死亡,但她依然想她留在当下。Evelyn这么做的同时也划清了人际界限\pozhehao{}你可以选择自杀,而我依然希望你能留下来和我呆在当下。(我无法操控你的生命,你想要自杀的自由依然属于你)这也涉及了另一个哲学议题、一个没有正确或唯一的答案的问题:我们在多大程度上应该去操控、限制或影响另一个人对于ta自己的生死的自由?如果那个人并不是个陌生人,而是自己身边的朋友甚至是自己所深爱的亲人呢?即使深深地爱着,我们真的愿意为了尊重对方对ta自己的生死的自由和选择而放手吗?

更重要的是,无论怎样的选择,都有好有坏,但没有十全十美。正如《存在主义心理治疗》里所写:“参与生活并不能在逻辑上反驳宇宙观点所提出的问题,但是它能够让这些问题变得不再重要”,Evelyn和Jobu的意识依然分散在无尽的平行宇宙里\pozhehao{}一切可能的结果都只是统计学意义上的必然结果,没有什么特别的;只有片刻的那些能让生活充满意义的时刻,剩余的时刻都被无尽的可能性所冲散了。但这些问题至少不像以前那么的重要了,因为他们能够与现实生活里的他人grounding and binding。

那片虚无依然在那,但那片虚无不再是生活的全部。

