\chapter{“I'm lonely.”}

\ardate{2022-04-26}{vnO4-on\_vszO\_jtFvFYDAA}


周日晚,在结束了两个小时的线上会议后,我出门逛了一个小时,但我并不太记得那一小时是怎么度过的,因为只是到附近(路程5分钟)的便利店买了盒柠檬茶坐在店里喝了一会儿就回去了。在路上,我感到很不开心,然后下意识地说:“今天又过去了”。当听到自己说出这句话时,我开始意识到自己是在惋惜周末的逝去,因为自己在周末几乎什么事情都没干。

第二天早上的通勤路上,我遇到了另一个部门的同事。一开始大家并没有打招呼,而我想到:这是一个互动场,并不只是我没有打招呼,ta也没有打招呼,所以就这样吧。然后ta就跟我打招呼了,而我也热情地和ta聊了几句。当我想继续聊的时候,我看ta把耳机摘下,但又想戴回去的时候,我就没有继续发起话题了,而是ta继续听ta的歌,我继续看我的推文。

当我发现自己很热情地想跟ta多聊几句的时候,我意识到这种热情并不是平常的自己会有的。当我停下来不去发起话题时,我会感觉到一种郁闷感,这种郁闷感好像在说:“I'm lonely.”为什么我会lonely?我开始回顾这个周末是怎么度过的\pozhehao{}周六本来约好去隔壁城市找朋友到处逛逛,但后来那个朋友有事爽约了,而我也没有一个人去那逛。我有想过自己一个人去逛,但又给了自己很多的“理由”,比如说今天的时间不多了、今天可能下雨、路上来回的时间不少、去喝杯咖啡也不便宜。不过真正的理由是,我不想一个人孤单地走在本可以两个人一起享受的地方。所以周末最终还是宅在了家里,也没有干些什么。

然后今天我收到了一年前“喝茶”的一位朋友的回顾(这段回顾是ta半年前写的):
“在喝茶聊天后卖掉或送人相伴已久的书本,且开始了心理咨询的相关培训……内心心疼这么个独特且优秀的年轻人看似快被孤独吞噬,却又无力提供任何协助的自我抚慰。……”

当读到“相伴已久的书本”时,我会感到很强烈的丧失感\pozhehao{}并不只是丧失了那些书,还丧失了曾经将那些书当做是一个家的替代品、一个锚的那个自己,而且还是自己主动如此丧失。一年前的自己意识到自己的内心开始转变,同时也开始渴望通过一些实际行动来改变自己的生活,比如说学习新的知识、腾出房间的杂物来创造一个自习的空间。那时候的自己处理了堆积在房间里的许多书\pozhehao{}大概有一千多本吧,卖出去的起码有五六百本。那些书是从大学四年以及大四实习时不断累积下来的,只要有书的地方,我还能有个escape。An escape from the reality,一个并不值得我活下去的reality。

我甚至觉得那时候舍弃书的行为很有自毁的色彩\pozhehao{}通过舍弃书来毁灭内心的那个曾经的自己。那个曾经的自己在大学里依靠各种原著小说在宿舍里营造了一个家的氛围,以及会大半夜在宿舍里看小说看得很晚很晚(有时候甚至是看到天亮),将自己沉浸在小说的世界里。但在一年前的自己看来,大学时的我只是在利用书籍来逃避现实,只是让自己在逃避现实的同时变得越来越无力、越来越无能。

但其实从现在往回看,我并不觉得大学时的自己就真的那么的无力和无能。正因为那时候的自己每天都背1-2个小时的单词还看了不少原著小说,所以现在的自己读英文的专业书才不会有多少困难。而且在最近的一次课程小组同学练习里,当小组同学各自自我介绍、互相认识时,他们会羡慕我的英语能力(能直接读英文的书籍和文献)以及我的跨专业程度之大(大学学的专业、现在的工作和我所学的课程处于三个截然不同的专业)。我一直以为能够读懂英文书籍是一件很自然的事情,但事实上是,这并不是理所当然的,也不是不经努力的。

不过,现在的我依然会对此感到很强烈的丧失感,因为现在的自己终究不是大学时的自己了,没有了那份“自我”。或许我也是在为过去那些回不去的回忆和地方和人而感到哀伤。

那段回顾的另一个让我很在意的点是“被孤独吞噬”,这也是我在今天开始感受到的感觉。其实自己一直以来都有这种孤独感,只有身边有个喜欢的人,或者更应该说是自己能视对方为重要他人的人、能建立起依恋的人\pozhehao{}一个伴侣,这种孤独感才得以有所消缓。

而现在我的生活里并没有这样的人,没有这样的伴侣。

我会觉得“伴侣”这个词很美好,a companion。它的动词是company,即伴随。一个伴随着的人。伴随着什么呢?伴随着人生的起伏、喜乐悲伤、人事变迁。在充满变动、充满未知、充满不稳定性的世界里的一份稳定、一份已知。

我会想起过去很多年的自己会因为亲密关系的丧失而悲伤上很久很久\pozhehao{}因和初恋的关系丧失而悲伤了三年多,因前任的关系丧失而悲伤了两年多,以及在这些年里一些陆陆续续短暂相处的男生。回想起那份亲密关系丧失的悲伤,那时候的自己真的是紧紧地抓着那份悲伤不放、抓着过去的回忆不放。因为如果我放手了,我就一无所有了,什么人也没有了。悲痛总比虚无好,因为除了和亲密关系对象相处的那些短暂的时时刻刻外,自己的生活一直以来都是那么的虚无,在绝大多数的日子里并没有一个重要他人、并没有一个依恋对象的存在\pozhehao{}无论是父母也好、初恋也好、前任也好、这些年来陆陆续续短暂相处的男生也好。在绝大多数日子里,自己都是一个人,一直都是一个人。

当前段时间那个有夫之夫的男生拒绝了和我的关系的进一步深入后,我有尝试去找另一个自己能喜欢上的人、另一个能产生依恋的对象,但并没有那么容易。上周末在和一个男生面基后,我发现他并不是自己会突然间就神奇般地喜欢上、神奇般地能够成为依恋对象的人。面基的时候我就在想,也许就是这样吧,我并不可能视任何一个人为唯一、并不可能随随便便就对另一个人产生依恋。而我也不需要冲着找依恋对象而去面基,让自己停下来休憩也是可以的,直到下一个人的出现。

并不是每一个体验循环都能够得以完整,并不是每次都要选择开启新的循环,更不需要逼着自己在一次又一次的不完整的体验循环里不断折腾,直到折腾出个完整为止。

