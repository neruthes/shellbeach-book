\chapter{自我探索 | 17}

\ardate{2022-01-18}{lx\_W9ZcTcXF5UJgHeQ4bnw}


\section*{1}

今天上午,我陆续收到了两个反馈,关于工作的反馈和关于热线的反馈。

对于工作反馈,我能大概理清反馈里哪些部分是属于同事/上司的,哪些是属于反馈者的,哪些是属于我自己的。当中有不少对我而言不适用的,甚至和我的自我认知有所冲突甚至让我感到反感的。在看同事评语时,我一眼扫过了好的反馈,然后聚焦于那些有待提高甚至是让我自己感到不安、难受的反馈。即使在工作上的反馈时间结束后,那些负面的反馈依然让我感觉很糟糕,那种糟糕的感觉一直停留在我的内心。

不久,我收到了热线反馈。我同样也是一眼扫过了正面的反馈,然后将注意力完全聚焦于待提高的部分。这让我感到更加糟糕了,而且我也更不相信正面的反馈,因为那些正面反馈似乎纯粹是为了之后的负面反馈而作的铺垫,甚至会给我一种欲抑先扬的感觉\pozhehao{}只是为了让负面的冲击更加强烈。

我想起了在有咨询室里我描述的那种感觉:我在一部很挤的公交里,周围的人都在挤着我\pozhehao{}周围的人都试图挤入我的内心空间。我试图聚焦那种被挤的感觉上,那种被挤感似乎在对我说:“你就是个一无是处的废物,a fuck up no good”。我马上意识到这是一种自我攻击,然后思考为什么会出现这种自我攻击。这种自我攻击的苗头最先出现在我的聚焦点上:我更聚焦于负面的反馈,而对于正面的反馈,我只是一眼扫过。或者更应该说,我似乎会很本能性地立即判断哪些反馈是正面的,哪些是负面的。这样的“认知扭曲”似乎预示着这背后的认知模式/认知图式:我本来就是一个一无是处的废物,a fuck up no good。这符合我的自我认知,而这样的认知又会反之强化我在各种信息里筛选出符合我的自我认知的信息\pozhehao{}那些符合自己是个“一无是处的废物,a fuck up no good”的信息,进而正强化这样的认知。

这样的认知是我在学生时期(特别是小学)里被重复性灌输的,那个每个人的学习能力和天赋甚至是存在本身都被成绩分数所异化的学校环境,以及自己从未被看见、被理解、被认同的家庭环境。

但另一方面,那种自我攻击背后又会是些什么?我回想起之前在咨询室里描述的那个画面:把自己按下水面地去直视自己所恐惧的事物。那种自我攻击就像是把自己按下水面的力量,只不过这种自我攻击是想让我符合自己过往的那一部分自我认知。这种想要让自己符合自我认知的背后又会是什么?我想这可能是想要保护自己\pozhehao{}如果我确实是自己所一如既往地糟糕,那么how worse could it be?但如果我并不是自己以前所一直设想的那么糟糕,而是周围各种各样的人所硬塞给我的认知、评价、反馈、批判、否定,那么我需要面对的则是周围的那些他人,那些比自己糟糕得多的他人,需要承担反抗他人对自己内心世界的入侵的责任。这似乎也是为什么在那个当下我会感觉到那种被挤感\pozhehao{}反馈里有属于他人的部分试图挤入我的内心世界,试图让我内化本属于他们的部分。


前几天,在和一位微信好友聊天时,TA说:“(我)怪不得被人说什么共情知识了,……小心共情共着共着共到沟里去了”。我想,如果真的是共情共到“沟”里去,那么这个“沟”很可能是个人议题,一些深度足以无意识地影响认知能力、觉察能力、判断能力甚至可能丧失理性的过去的认知/思维模式以及那些固化的模式背后潜藏着的创伤。

这也提醒了我,无论是否共情到“沟”里去,我依然有着许多个人议题以及议题背后的过往创伤需要去面对。这并不是个一蹴而就的过程,而更像是在漫长的路程中,需要一次又一次地直面的进程。


\section*{2}

近几个月里,公众号推文的评论区里一直会频繁“出现”一位资深心理咨询师(虽然她一直试图脱掉去咨询师这一“上衣”),她会用心理咨询的视角来看我的文字,并留言她自己的看法。但最近她很少出现在评论区了,所以在过了一段时间后,我在微信里和她私聊这件事。

聊着聊着,我似乎逐渐看见了一种可能存在的互动模式或进程:

\begin{compactitem}
    \item 我的文字让她感受到了一种我的vulnerability;
    \item 这份vulnerability使她想要代入咨询师的角色,并通过留下评论来照顾我;
    \item 这种带有照顾意图的评论使我想要回避她;
    \item 这份回避最终也由我对她的评论所进行的回复中让她感受到了我的疏离感;
    \item 她决定通过不再出现在评论区来保持彼此的距离;
    \item 她的不再评论也触发了我的保护距离行为\pozhehao{}尊重她的选择而没有去询问和确定她为什么会这么做、这么做背后的是什么。
\end{compactitem}

在聊完后,她说这回让她感到有点小开心,我在那个当下也有开心的感觉,因为彼此的关系因此而更近了,而不是聊天开始前彼此保持着距离的那种疏远感。我回应说:“还好我们彼此都有这样的能力去改变关系的原状”。

当说完这句话后,我意识到,其实我一直都在干这样的事情,一直都在改变关系,只不过之前这种改变关系的情况只出现在咨询室里,而没有发生在咨询室外。在咨询室里,我在试图改善我和我的咨询师(不是那位资深咨询师)的过程中,我似乎也在提升自己走入(广义的)亲密关系的能力。我从来没有过一段深入的亲密关系,即使是和初恋还是前任的恋爱关系也都并没有走深。在这近半年的心理咨询个人体验历程里,似乎有一些事物开始潜移默化地影响着我在咨询室外的生活,像是某些充满生命力的植株从咨询室里长出了咨询室外。

\tristarsepline

另一方面,我也开始看见我的所谓“尊重对方的选择”的背后,其实是我的一种拉远彼此距离的防御方式\pozhehao{}试图利用利他的理由来合理化自己对人际关系安全感的需求,担心和对方的再次触碰里会又一次受伤。

同时,我也会在意自己那份vulnerability的部分。敞开甚至是深挖自己内心的事物时,必然会将一些vulnerability的部分展露出来。这份vulnerability和她想要照顾他人的需求似乎像是拼图般的凹凸相嵌在一起。昨天,在和课程同学进行倾诉练习时,我说到了在上周末去另一个城市的社区街道里我所感受到的宁静感,我想知道那份宁静感是什么。在倾诉练习结束后的互相反馈环节,对方说TA在一开始不敢打扰我,因为那个场景对我来说好像很private。同时TA说TA在一开始感受到了我的vulnerability,这会让TA不敢靠得太近,但TA在整个倾听过程中都感觉到我好像不想TA靠得太近,但又不要离我太远。

我想,这份vulnerability似乎会对不同的人产生不同的效果、发生不同的互动,就像是同一片拼图,不同的人看到了不同的形状,看到了他们想要相互嵌合的形状,那个形状正好和他们自身原有的形状相嵌在了一起,一些可能属于他们的个人议题,也可能是属于我自己的个人议题,或两者都有。

\tristarsepline

最后,我留意到被照顾感会让我想要回避对方。这种抵触感来源于我妈和外婆在我的成长历程里的“照顾”,但那种“照顾”并不是真正的照顾,而更像是一种借照顾为由的操控,一种否定我的能力、认知、感知、想法、情感甚至是存在本身的操控。这样的“照顾”并不是我想要的,而是我被施加的。我想要的照顾、我需要的照顾仅仅是一个拥抱或是一场倾听便足够了。

而且,我会认为这种“照顾”背后还有着一份不信任:不信任我有能力解决我自己的问题,不信任我能处理我自己的情感,不信任我能照顾好自己。这种不信任也在一定程度上没有看见我有能力的部分在,甚至一定程度抹去了我有能力的部分。对方眼中的那个需要“照顾”的、没有能力的“我”,并不是我。

这让我想到这周的人本主义课程里的其中一点:比起给对方提建议、替对方找办法,甚至可能因为自己没有帮到对方什么而认为自己很没用,与对方一起呆在那个无奈甚至是绝望的境地里反而更需要勇气和力量。而在那个境地里陪伴着对方呆上一段时间后,对方自然而然会有属于自己的力气,并开始去为自己做些什么。

不过,在倾听另一个人时,我也会时不时地想要“照顾”对方的情感,试图帮对方找到一些出路。所以,之后的我会试图仅仅只是陪对方呆在那里。因为如果当事人是我的话,我更希望、更需要的也只是那一份“呆在当下”,仅此而已。

