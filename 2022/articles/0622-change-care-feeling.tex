\chapter{“我改变不了对方的想法和感受,所以根本没有必要在乎他们的想法和感受。”}

\ardate{2022-06-22}{FiM7xQjVxpKsYqEAzXuxVg}

% \dialoguelist{咨询师}{
% 	\dialogue{咨询师}{是啊。}
% 	\dialoguesepline{咨询师}{(沉默)}
% }


最近几天上班都能看见不同的上司会在其他同事都在的公共群里指出我的工作“问题”。第一次,我会觉得很不爽,想到自己下次要多注意。但第二次、第三次的时候,我就已经躺平、摆烂了\pozhehao{}反正无论我做什么都没有办法改变别人对我的工作质量的看法,都会认为各种有问题、糟糕的话,那我做什么、做得怎么样事实上并不重要,毕竟我也改变不了他们对我的看法。这么想之后,我会感到轻松,因为我不用在乎他人的想法和感受了,也不用在意如何去改变自认为无法改变的事情和人。

然后我会联想起前几天收到的一个接热线的反馈。在反馈里,那个来电者表达了ta认为倾听者(我)对ta没有帮助、完全不理解ta、不会向别人推荐热线,同时也表达了ta认为倾听者做不到接纳、做不到真诚地告诉ta我无法理解ta,而是逃避ta的问题。当我往回看回忆稿的时候,ta并没有在当下直接表达ta的不被理解,也没有表达ta认为我逃避了ta的问题。否则我会直接问:是我的哪个回应让你感受到不被理解吗?你觉得我逃避了你的哪一个问题?同时我内心也没有不理解ta或逃避ta的问题的感觉。

当我第一次看到这个反馈的时候,我会感到很愤怒和难过,觉得我在热线时间里关心了对方,但却得到这样一个反馈。但后来我想到:为什么我要在非热线时间里在乎ta的感受?我不是在变相加班了吗?再说了,我也改变不了什么,我能够确定我内心并没有ta所表达的东西就足够了,并不需要在非热线时间去处理ta的情绪。然后我的心情就舒服了点。

不过我也留意到我会很容易地把一件给我带来受挫感的事情(无论是工作还是热线)引入到一个特定的分类:我改变不了对方的想法和感受,所以根本没有必要在乎他们的想法和感受。当我好奇为什么我会设立一个这样的分类的时候,我很快意识到,这可能是来源我从小到大的经历。

家里的奶奶有一个习惯:她会在任何人面前说除了对方之外的所有人的坏话。结果是,每个人都从别人口中听到了奶奶说自己的坏话。逐渐地,没有任何一个人愿意和她说话,甚至连她的任何子女和孙子(我)也是如此。今天吃晚饭的时候,我妈说她周末要去哪里哪里玩,然后说等回来之后奶奶肯定又会说她坏话了。我说:反正你去不去她都会说坏话的。

当我联想到我的回应时,我意识到“我改变不了对方的想法和感受,所以根本没有必要在乎他们的想法和感受”是我从小到大对奶奶的应对方式,同时也延伸至我在和他人的互动里对受挫经历的应对方式,而这个应对方式直到现在都很有用,能让自己抽离出一些困境甚至是情绪旋涡。

我也会在想:什么是“在乎”?在乎意味着我愿意将自己的时间和精力投注于对方或彼此共同参与的事情上面,无论是工作和热线还是对方的情感和想法。而当把这部分原本投注于对方或彼此共同参与的事情的能量撤回时,这一方式很好地保护了自己,及时止损。而我可以选择将精力投注于其他事物或人身上,比如说一些会感谢我表达了真诚并且感受到连接感的来电者。

能够向他人\pozhehao{}特别是对对方一无所知的人\pozhehao{}投注精力很重要,但懂得何时、如何撤回也同样重要。而且我越来越觉得,“在乎”真的是一件很有力量的事情,也是一件不是理所当然的事情。没有人理所当然地在乎自己,自己也不需要理所当然地在乎他人。无论是普通人际关系还是亲密关系,如果一个人不再在乎对方,也不再在乎彼此共同参与的事情,那也就离彻底“撤回”不远了。而另一方面,这份在乎、这份精力的投注也因此建立了个体与另一个他人或事物的连接,是个体(包括我)与外界世界的桥梁。

