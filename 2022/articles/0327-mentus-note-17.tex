\chapter{白色灯塔先生的情感笔记 | 17}

\ardate{2022-03-27}{9Hh9Xo0idRMZOi7Slb4G-g}




\subsection*{对作业可能被攻击的恐惧感}

然后我试图去觉察自己的感受,并对冬裔说:“在你问‘看看你的观点’时,我会很抵触,因为我不想去触碰那个‘伤口’。我会怕被攻击,因为我的成长经历里很多这样的被攻击的经历\pozhehao{}自己的爱好、感受和想法被否定、被打压和被抹去。我不希望又多一个人(你)来判断我的‘对错’。我可能刚刚有点过度防御了,就好像助教没有评优一样,会很容易将这些信息和以前的经历带来的感受联系在一起。我可能也很不相信一些没有看见我的情感的人,会极度地不信任对方,因为我不知道对方会不会突然来攻击我。”

\blockquotesource{随笔 | “即使我学了那么多,但在处理自己的情感时依然很蹩脚”}{白色灯塔先生}{2021-11-20}


\subsection*{“凭什么”的愤怒感}

我回想起在这次的经历里的最开始,我的脑海里有一个很本能性的声音:“凭什么?”,并试图追溯这个声音的来源。

这个声音最开始出现在我刚上小学的时候。在读幼儿园那段时间,我一直在买各种的玩具,然后又很快地扔掉被我玩厌了的玩具。但在不断地买和扔的过程中,有两箱我既喜欢又不舍得扔的玩具留了下来。那两个装着我的玩具的纸箱就放在我房间门对面的架子上,纸箱上还写着我的名字。在上小学一年级时,我妈说我已经上学了,不能再玩玩具了,所以就连写着我的名字的纸箱连同里面的玩具都扔了。当发现我很喜欢的玩具都被扔了后,那个声音便第一次出现了:“凭什么?”所以,比起“认为自己的努力、自己的精力、自己的存在被否定”,“凭什么”这个声音背后的情感似乎更想要表达的是:我所珍爱的东西被他人扔掉了。如果是放在这一次的作业没有被评优的经历里,这种情感似乎暗示着,我的文字就像是自己小时候的“玩具”,而当助教没有给我的作业评优时,我会感到自己的“玩具”被他人扔掉了。不过我也会想到,我早就已经不是那时读一年级的我了,我所珍爱的东西也不会那么轻易地就被他人所“扔掉”。但自己内心那个脆弱的我依然存在着,那个脆弱的部分也是现在的我能感觉到自身的各种正向和负向情感的原因。内心的那个脆弱的部分依然会对自己所珍爱的东西很敏感,敏感得会对此时此刻的经历作出一些属于他自己的解释。更重要的是,那时候玩具被扔掉的经历给那个更为脆弱的自我留下了一股很强烈的攻击性。那种攻击性背后的那份愤怒、那份憎恨一直以来都埋藏在自己的内心深处。

为什么有的人能把一个小孩子所珍爱的东西给扔掉?

\blockquotesource{随笔 | “即使我学了那么多,但在处理自己的情感时依然很蹩脚”}{白色灯塔先生}{2021-11-20}


\subsection*{自杀意图不被看见的孤独感}

我回答说:“我记得我向我前任表露过我的自杀意图,然后前任的回应是:‘不尊重生命的人……请滚出我的亲密圈’。那时候我真的感到很受伤。我会期待对方能承接我这个部分,能表达他对我愿意表露出我的这一部分的勇气,并愿意承认我的这一部分。但他并没有做到,他只是将我尽力地推开。不过现在我知道他这么做可能更多的是为了自我保护,保护自己不去触碰他自身的那些脆弱的部分,就像我的父母一样,他们会去回避彼此之间不愉快的情感。但我好像就成了他们的自我保护机制下的受难者、孤儿。”

咨询师问到:“成为了他们的自我保护机制下的受难者,这会是一种怎样的感觉吗?”我说:“我会感到很孤独,感到他们压根就没有看见过我,我一直都很孤独。

\blockquotesource{自我探索 | 11 \pozhehao{}“一个由黑白灰交织而成的漩涡”}{白色灯塔先生}{2021-11-29}


\subsection*{逐渐流入虚无的黑白灰交织的漩涡感}

咨询师问:“那你的那个更深处的自我会是怎样的吗?你会用怎样的文字来表达吗?”我沉默了一会儿,然后说:“那个部分对我来说很难用言语来形容。那个部分更像是各种悲伤、失落、孤独等情感混杂而成的集合体。”然后我又沉默了一会儿,继续说:“如果是用画一幅画的形式的话,我会画一个漩涡,一个由黑白灰交织而成的漩涡,漩涡的中心是什么也没有的虚无,外界的一切色彩在被卷入漩涡时褪去了颜色,变成了黑白灰,然后最终流入虚无。”咨询师说:“即使你用一幅画的形象来进行表达,但我依然不怎么能理解那个画面。”我继续说:“比如说从生到死吧。一个人不断成长,开始展现出自身的各种各样的可能性,直到自己的身体开始衰老,自己身边的朋友开始离开,自己的亲人开始逝去,最终自己的身体也会消逝,自己的自我意识和记忆也会丧失。”咨询师说:“这个过程对我来说会有其他的收获。我会没有了以前年轻时的那股干劲,但现在我也会收获到了一些其他的东西,比如说知识和阅历。当然这只是我的看法,但在你看来生命好像只是一个走向消逝的过程。”我认同了TA的看法的角度,并表示:“嗯。这可能是因为在我的成长历程里的各种各样的关系和客体都是在消逝的\pozhehao{}我遇到的很多人都消失了,与相对而言更亲密的人的关系也消逝了。我除了在年初“抓住”的学课程这个客体之外,直到现在,我身边的他人和环境依然和以前一样,having nothing left and no one left。”

\blockquotesource{自我探索 | 11 \pozhehao{}“一个由黑白灰交织而成的漩涡”}{白色灯塔先生}{2021-11-29}


\subsection*{一切都回不去了的悲感}

在这一愤怒之后的是悲伤,一开始我还不太明白为什么我会感到悲伤,后来我逐渐感受到这份悲伤来源于年初认识的那个现在已经不在这座城市的男生的离开。当准备要上白班时,我回想起以前经常下班都会找他见面\pozhehao{}去他下班的地方等他或者是去他住处附近找他,回想起以前我和他一起去过的地方\pozhehao{}江边、面店、街道等等,回想起和他认识的那段时间里,我总是对外界充满好奇,想和他一起尝试去做、去分享各种各样的事情和经历。我还蛮喜欢和他相识那段时间的我自己的,那时候的我更像是a better version of myself。那时候的我既喜欢他,也更喜欢我自己。

当他离开这座城市后,我开始将自己投入了另一种生活\pozhehao{}学课程、参加心理咨询、上夜班。现在往回看,上夜班让我拒绝了很多人际关系上的可能性,因为我能在白天忙学习、晚上忙工作。上夜班的我似乎无意识地利用这种生活来过度甚至是回避他的离开所给我带来的情感,包括那份悲伤。

\blockquotesource{零碎的想法 | 28}{白色灯塔先生}{2021-12-06}


\subsection*{对与自己的价值观和珍重的事物相冲突的事物而感到的扭曲感和难受感}

在下午逛公园时,我们找了个阴凉的地方休息。喂鸽小哥聊起他说他被家里人催婚,而他会考虑结婚,并谈了些他对未来生活的设想:他的对象在另一个城市,在疫情前他对象还打算来到他所在的城市定居,但因为疫情原因而耽搁定居一事,而且也因为疫情,他们很久没有见过面了。他说他还蛮喜欢现在这种异地的状态,因为他很难想象两个人生活在一起的场景。这会让我感到很奇怪,因为他似乎并不会设想彼此在未来的共同生活,而是他有着他对自己的未来生活的设想,设想着一个对方几乎不复存在的生活。既然他对未来生活的设想是打算结婚,我问他:“那你会设想之后你和你对象的将来会是怎样的吗?”他说他想到女生会查看手机,所以之后他或许不能保留着他对象的联系方式了,可能只能偶尔打一次电话聊一聊天。听到这里时,我会对这样的生活感到扭曲和难受,但我没有把我对于这样的生活的想法和感受说出口。后来,当我回顾这件事时,我想到这种“扭曲”更多是出自我自己的价值观。这种价值观来源于那些我更想要亲近但无法亲近的人,而我在羡慕他们的能力和优势的同时,似乎也不知不觉地认同和内化了他们的价值观。

……同时,我试图去感受那种扭曲感背后的是些什么。那种扭曲感似乎是在“告诉”我:“我坚决不能过那样的生活”。我意识到,那样的生活和我所珍重的事物是相悖的,那一事物便是开放性。我不能接受自己的开放性变得越来越窄,不能接受自己被困在原地。这种对于被困于困境或停滞在原地而感到的恐惧和排斥来源于我自己的个人经历,那段我不知道自己应该朝哪个方向走的经历\pozhehao{}在尝试的过程中不断地挫败以及因此被更为亲近的人所嫌弃甚至是弃之不理的经历。

\blockquotesource{零碎的想法 | 29}{白色灯塔先生}{2021-12-07}


\subsection*{对表露更为快乐的自我而感到难受、抵触和排斥}

我发现在和他相处时,我会真正地笑起来,而且还会笑得很频繁、笑得很久。我已经很久没有感觉到自己是在真正地笑,所以我会很喜欢那个在他身边时会笑得很开心的那个我。笑起来的时候的那种开心的感觉也让我意识到内心深处的那个更为快乐的自我的存在。但另一方面,我会对这一更为快乐的自我的表露和展现感到难受、抵触和排斥,因为我并不习惯在另一个人面前表露出这一部分的自我\pozhehao{}更为快乐的那一部分的自我。我会害怕更为快乐的那一部分的自我会遭到攻击、会遭遇失望,特别是因为对方已经有对象了,对方的内心已经有一个占据了很大的位置而且也占据了很久的人。我会害怕在他面前呈现那一部分的自我时,我会越过界限或者他会认为我越过界限而做出一些让那一部分的自我感到失落甚至是受伤的事情。我也会想到,为什么我会预设更为快乐的那一部分的自我会遭到攻击或遭遇失望。我想这是因为在很久以前,当自己刚开始用和之后继续使用社交软件来交友甚至是试图开始恋爱关系的历程里,我遇到过不少人会因为有了对象而排斥其他一切人际关系,或者说不愿意去深入其他人际关系的深度或可能性,就好像只要有了男/女朋友之后,其他人就只能是泛泛之交。

\blockquotesource{零碎的想法 | 29}{白色灯塔先生}{2021-12-07}


\subsection*{从内心情感和现实世界中飘走的抽离感}

咨询师继续说:“这种让它们就这样就行的方式,好像和之前的咨询里的你很不一样。之前的你会刻意地准备很多东西带进来说。”我回答说:“是啊。那时候我总是想自我探索,想去剖析一些东西。但现在我不想再趁自己有伤还把自己往死里抠。虽然在自我剖析的过程中我发现了、收获到了很多不同的情感以及情感背后的经历和故事,但这最终还是为了获得最后的那个平静感\pozhehao{}‘噢,原来是这样的’,我知道自己为什么会这么做。但现在我发现我不通过自我剖析也能找到那份平静感。”

咨询师说:“你能再多描述一下那种平静感吗?”我说:“就是我感觉我慢慢地飘走了,我不再感受到、不再停留于那种无力感。这好像也和我之前不知不觉地离开了那个漩涡(的情况)一样,我突然发现自己突然就走神地离开了那个漩涡。(现在的)我好像看着那个理智化的声音和那份无力感,远远地看着它们,(我)就像是个局外人,然后我就飘走了,就像是漂流在河流里,我随着河流飘走了。”

咨询师有点困惑地问到:“我想你会不会是想说,你一个人已经能很好地处理那些情感了,不需要咨询了?”我回答说:“Em...不完全是。我好像不只是不需要咨询了,我也不需要其他东西、其他他人、其他客体了。我感觉我离他们(他人和客体)越来越远。我感觉我离我所学的咨询,那些理论和知识,离那些由人所构建出来的东西越来越远。”

咨询师再次困惑地说到:“你能再多描述一下那种感觉吗?”我也开始感到困惑地回答道:“其实我也是刚开始感受到这种感觉,所以我可能还没有办法能很好地描述这种感觉。”我又继续沉默了一会儿后继续说:“这种感觉……有点像是性高潮后或者是吃饱饭后的那种无欲无求的状态。当然这种状态不会一直持续下去,接下来我可能会感到孤独、无意义和虚无感。但当我处于现在这个状态下的时候,我没有任何欲求,我不需要任何他人,也不需要任何客体。我不需要去感受些什么,我也不需要我的思维和意识。这可能也是我在这次的咨询里越来越沉默的原因,因为我觉得我好像不再需要去说些什么、不再需要去感受些什么、不再需要去思考些什么。就只是这样就可以了。”

咨询师问:“你会是有意识地这样做吗?你会对此感到更喜欢还是不喜欢?”我思考了一下,然后说:“在不同的经验,我都会有喜欢和不喜欢的部分。我喜欢现在的这种平静感,也不喜欢这种平静感,因为它让我难以感受生活的波澜。而我也会不喜欢之前的那种无力感,但也会喜欢那份无力感让我感受到的生活的丰富性。而且我也并不是有意识地这样做,而是自己的大脑好像让我逐渐地过渡到了另一个地方,离开了原来的地方,就像是随着河流飘走了,飘到了一个什么他人、什么客体也没有,只有我一个人的地方,而我对此还蛮ok的。我不需要去感受些什么,不需要去思考些什么。”

咨询师对我的这种“感觉”感到越来越困惑,想让我继续多描述一下这种“感觉”。我说:“这可能和我最近在上一些冥想的课程有关。这种状态可能有点像是一种无我的状况,就像是冥想时,我的自我意识只是黏附在了这副身体上。”

咨询师笑了笑说:“嗯。其实这一路下来,我也感觉这像是一种无我的状态。”我回答说:“嗯。可能是最近的(冥想)经历让我意识到,原来之前我在很多情况下不知不觉地走神地离开了原来的境地,比如说那个漩涡,原来那种走神其实是一种无我的状态。”

原来在几个月的咨询历程里,我逐渐开始不断地放弃对咨询的目标和方向的控制、对自我的控制、对自己的情感和想法的控制、对自己的经验的控制。

我也会想到之前和一位微信好友聊天时,TA的聊天内容里总会隐约地涉及或显明地提及“分裂”这个话题。我想,人可能本来就是“分裂”的,有着不同的经验,因此也有着不同的自我。对我而言,我好像不再试图去整合那些“分裂”的部分了,无论它们是经验也好,还是自我也好,而是随着河流drifting。

我想,有很多人(包括我自己)可能都曾经被扔入不同的河流里,在河流中挣扎求生,而我开始逐渐随着河流飘走,just drift。

\blockquotesource{自我探索 | 12 \pozhehao{}“just drift”}{白色灯塔先生}{2021-12-12}


\subsection*{从内心情感和现实世界中飘走的悲伤感}

其实我会发现,当以前发现自己飘走了之后,我又把自己拉回来,拉回到之前的那份悲痛,拉回到以前相处的那几个月的回忆里。好像只要我不呆(回忆)在那里,如果我没有了那些悲痛和回忆,我就是对这份关系的背叛。而没有了那些悲痛和回忆,我觉得自己什么也不是,没有值得我活下去的事物。但现在,我好像有了值得我停驻于这个世界的一些东西,一些对其他事物的热爱和追求。现在的我能停留在现实的世界里,而不只是活在过去的回忆。”

咨询师回答说:“好像你现在的情感会变得更为流动。”我点了点头,补充道:“其实那种飘走的平静感的深处还暗藏着一丝悲伤,看着过去的重要他人和客体离自己而去的悲伤。好像在飘走的过程中,我要被迫去接受(在我内心里的)他们的离去。”

咨询师说:“被迫?你愿意多说一说吗?”我继续说:“Em……在我最近的一次冥想里,引导语说感受身体的感觉时,我感受到一股悲伤。然后当引导语说想象一片蓝天时,我想象自己站在一片平原上,远处的是蓝天,而近处的是那些曾经的重要他人。我看着他们的身影背对着我,在离我远去,而我也没有跟上他们,只是看着他们离开,看着他们离开的脚下所留下的脉络。后来当引导语说发散自己的思绪时,我想象着自己的脉络延申到很远很远,延申到未来那些我想要触碰的事物那,但没有延续到那些曾经的重要他人。

就是一种感觉,一种我的未来好像和他们再也无关了的感觉。

但我也知道这只是一种感觉,这种感觉终将过去。而我在飘走的过程中,也看着他们逐渐远去。”

\blockquotesource{自我探索 | 13}{白色灯塔先生}{2021-12-20}


\subsection*{追求自己所珍重的价值的受挫而感到的匮乏感}

我会认为“优秀”和“美好”背后暗藏着来源于外界的价值观,以及被我所内化的来源于自己的价值观。他人的优秀和美好可能是经济能力强、有对象、长得高、内心强大等等。这些价值观的背后意味着我所珍重的价值,以及我在追求这些价值的历程上所经历的受挫。这些受挫不仅仅让当下的我无法拥有自己所珍重、所渴望的价值,更是在内心留下了一个个的空洞\pozhehao{}那份匮乏。那份匮乏的背后是一种受伤的感觉,因受挫的经历而感到的受伤感。

这让我回想起在读书时期,我总是很羡慕考试得高分的同学,总是对自己的学习成绩低和学习能力的匮乏而感到自卑。

\blockquotesource{他人的优秀和美好只能让自己感到自身的匮乏}{白色灯塔先生}{2021-12-23}


\subsection*{羡慕或嫉妒之下的无法得到自己想要的事物和人的绝望感}

但当自己因此而感到很难受时,我会想有一个能拥抱的人,一个对方能get到我的拥抱不是因为性欲,而是出于对连接和触碰的渴望的拥抱。不过在现实里,并没有一个在自己身边,能get到我对连接和触碰的渴望,又能腾出时间来拥抱的人。这会是我在那个当下想要的事物、想要的人。而当没有这样的人、得不到这样的事物时,那种感觉确实蛮不好受\pozhehao{}更加清晰和明确自己想要的是什么,而那个事物是自己所无法获得、无法达到的。那份痛苦比最开始、最表面的羡慕或嫉妒要更为深入和绝望。The inevitable incompleteness.

\blockquotesource{他人的优秀和美好只能让自己感到自身的匮乏}{白色灯塔先生}{2021-12-23}


\subsection*{对被操控感而感到的愤怒感和背后的无力感}

我感觉自己被操控着,而我却默许地遵循着这种被操控。被操控感的背后还有一种愤怒\pozhehao{}愤怒于咨询师可能作出的操控,愤怒于咨询关系里可能存在的操控成分,愤怒于自己为什么没有直接提出这一操控成分的可能存在,而是默许了这种操控。

我也会想到我在咨询里所说的来自前任的操控,但当咨询师要我说出具体事实时,我能想到的事实并不能让我理性地认为那是一种操控。

……这种被操控感的源头来源于我在上小学时甚至是在上小学之前就被我妈操控着生活的各方各面:她决定着我能吃什么、能穿什么、外界天气的冷热感知、我喜欢吃什么、不喜欢吃什么、我是否内向、我是否自卑、我是否听话、我是否勤奋、我是否懒惰等等等等。她不仅操控着我的穿着和饮食,还操控着我对外界环境和他人的认知,以及我对自己的认知。

我就像是一个空无的容器,任由他人往我身上和内心灌入任何东西。我感觉自己无论是在肉体上还是心灵上都成为了一个名副其实的妓男。但另一方面,我在这个过程里收获的又是什么?我收获了安全感、收获了被供养、收获了存活下去的机会。但我并不想去责怪小时候的那个自己,因为那个我只是为了生存、只是为了少挨打、只是为了活下去而已。不过现在的我并不需要那些了,也不再是小时候那么的无助、无力。

……而且,我觉得操控之所以让我感到非常反感,不仅仅在于被操纵者的结果\pozhehao{}被操控者的自由意志的丧失,更在于操控者的用意\pozhehao{}操控者从一开始就没有打算给予对方自由意志的余地。

\blockquotesource{零碎的想法 | 30}{白色灯塔先生}{2021-12-27}


\subsection*{生命/存在的连接感}

这也让我想起第一次去前任公寓的晚上。那天晚上他把窗帘拉上了,两人在一片漆黑里坐在沙发上看着电视,他的猫应该躺在了旁边的黑暗角落里。坐在左边的他把手放在了沙发的靠背上,并握着我的右手小臂。在那片漆黑里,只有电视散发着唯一的亮光,但我能感觉到他就在我的身边,能听到他的呼吸声,闻到他的气味。

虽然我们彼此没有对话和交流,但那时候的我依然有一种彼此相互连接的感觉。虽然我和他都身处于黑暗(这片黑暗不只是现实意义的黑暗,还是作为一个存在需要面对的“黑暗”\pozhehao{}孤独、未来的不确定性、虚无等存在既定),但我们在那一刻的当下依然有着彼此的那段关系、那一联系,由他的呼吸声和气味所紧紧维系着的联系。

\pozhehao{}““生命影响生命”,我—汝,相遇“ 2021-12-28


\subsection*{对没有看见自己的他人感到厌恶}

无论我说什么话,他都只会按照他计划好的剧本继续说下去:即使我说我家附近吃的店不多,他也依然按着他自己的剧本\pozhehao{}“你家附近不是蛮多街边小店的吗”\pozhehao{}继续说下去。为什么我要和一个我无法对对方产生影响的人进行单向且无反应的聊天?

……在我看来,这就是一场独白伪装成对话,无法超越自己、无法转向他人。即使有另一个人在身边,他也依然是独自一人。即使有对话,但也依旧是一场独白。

如果找另一个人聊天只是为了自言自语,如果找另一个人相伴只是为了干一些自己一个人的事情,如果另一个人的存在只是为了沦为工具、沦为客体、共同陷入无尽的虚无。我会在想,何必要把另一个人卷入一个如此孤独的世界?Misery loves company?

但拥有了另一个人的相伴,真的能让当事人没那么孤独?真的能将这场独白化为交流?真的能让他不再按着自己的剧本走?真的能让他愿意敞开心怀地接受他人的影响、看见他人的存在?

\blockquotesource{随笔 | “我—它”交流}{白色灯塔先生}{2021-12-29}


\subsection*{想要投入一段熟悉的关系的过山车感,以及要放弃重复的经历而感到的悲伤}

在短暂的沉默后,我继续说道:“这也会让我想起以前和前任、和初恋的经历。和他们的经历就像是坐过山车,有点像是双相障碍患者会跟随着自己的情感波动而波动,而我就像是在坐过山车:先经历高峰,再落入低谷。”

咨询师回应说:“那种坐过山车的感觉会是一种怎样的感觉?当时准备坐过山车时,你会有怎样的感觉?”我回答道:“一种get high的感觉,就像是买醉。这种感觉背后也有一种熟悉感,好像我又能赶上一趟过山车了,又能开始一段难忘的历程了。这好像能为我的平淡生活带来something new,但当我这么说时,我也意识到,这并不是something new, 而是something old, 一种旧的模式。当然我也会想到,这种重现是否也是出于我的潜意识想要通过不断地重复来冲破那个原有的困境,毕竟只需要一次的成功便足够了。这似乎也是人类的本能之一。我也会感到一种悲伤,一种要放弃自己所熟悉的事物(模式)的悲伤,因为无论好的部分也好、糟糕的部分也好,这样的经历依旧是我会深深铭记的经历。我会觉得很遗憾,这一切本可以再次发生。不过,当我看到喜欢之情背后的拯救欲后,我开始怀疑这是否真的是自己想要的。”

\pozhehao{}“自我探索 | 15"2022-01-09


\subsection*{在梦里牵着手的幸福感}

我梦见自己在一辆公交上,这辆公交开在一个日落下的山坡小路上。这部公交要开往一个很远的地方,因为这条线路是由一个机构的总部开向一个已经完成了的任务的地点。这条公交线路的目的就是为了隐藏那个任务地点曾经的存在。

我是这个机构里的工作人员,而坐在我旁边的男生也是我的同事,车上的人都是我们的同事。我坐在公交车靠前的左侧窗边,而他坐在我旁边靠走廊的座位。我在其他座位的遮挡下牵着他的手,我们的手放在了他的左侧大腿上。我记得他的手是修长且铜色的,但我不太记得他的外貌。公交开到一半时,我和他下车去清理路障。在把路障清理好时,我醒了过来。

睡醒后,当我回想起和他牵着手的那个时刻,我感到很幸福,一种我很久都没有在现实世界里感受到的幸福感。

……然后我梦见自己坐着公交,窗外是一条沿海小路,车上都是同学,因为我们刚从教室里出来坐上这趟车。我看了看手机,想到现在已经坐了近半个小时的车,但还没有到站,我想到我快要迟到了。我坐在公交车靠前的左侧窗边,坐在我旁边靠走廊座位的是一个高中男同学,他在高中时的女朋友是那个刚刚在教室里遇见的那个女同学,那个怀着一个畸形婴儿的女同学。那个高中男同学的身材是瘦高的,皮肤是铜色的。我把右手靠在他的肩膀上,他的左手牵着我靠在他右边肩膀上的右手。我感到很平静,一点都不需要担心迟到的事情。

睡醒后,我回想起牵着他的手的那种感觉,那种感觉就是上一个梦境的那种幸福感,原来我在上一个梦境里牵着手的人是他。

\blockquotesource{零碎的想法 | 31}{白色灯塔先生}{2022-01-16}


\subsection*{对他人践踏我所珍重的事物而感到的被侵犯感和愤怒}

在去年11月底和这个月月初,公众号群聊里有个群友说我有的只是共情知识,但没有共情体验,也更加不会共情。他说我大可以不去学习(他认为是有关共情的)这类课程。在和他的后续私聊里,他说:“你没有必要改变自己。我之所以会选择指出你的这个问题,只是说,你前阵子看共情的课程,某种意义上是希望自己能够做到这一点,但是你又做的不好。那不如不改变。你不是鱼,有如何去体验鱼对在水里的感觉的知识呢,只有理论,却永远无法得到真实的实践,又有啥用呢?如果不合适自己,不如放手。”

当时我感到一种被侵犯感和一种愤怒:对方凭什么有权告诉我什么才是适合我自己的、我应该干什么、学什么、怎么做?对方凭什么有权践踏我所珍重的事物?

……我也会想到,他和我前任有一个共同点:他们都很擅长看见他人所珍重的事物并将其武器化地践踏他人。所以我逐渐发现,他的话的伤人之处并不在于他言语里的内容,而在于内容之下的那种态度,那种可以随意否定他人所珍重的事物、否定他人的努力、甚至近乎于否定他人的人格和他人的存在的态度。

\blockquotesource{零碎的想法 | 31}{白色灯塔先生}{2022-01-16}


\subsection*{通过逻辑和理论获得的安全感}

我回忆了一下,回应道:“嗯。我也会在想,为什么我写作的内容会那么高度逻辑化和理论化。我身边有的人也会在语言和文字的沟通上用一些高度逻辑化和理论化的术语、一些属于他自己的事物、他人很难懂的理论体系来表达一些浅显的内容。有一次我跟他聊天时,我就跟他说,他好像是在用这种给人际关系“建模”的方式来回避与人相处的未知性,好像只要呆在自己的理论里,就能保留着属于自己的那份安全感,而不敢去面对属于他人的未知。”我思考了一下,继续说:“我会想到,我的写作内容之所以会那么高度逻辑化和理论化,可能是因为我需要去面对那些我周围环境里的未知,比如说那些虚无和无意义的部分。一旦能够命名那些未知的部分,那些未知好像就变成了已知,我好像也就能从这些逻辑和理论里找到一份属于自己的安全感。但当然,并不是每个人都会拥有和我相似的经历,所以每个人所构建的对于如何理解他们各自的周遭世界的理论和框架都很不一样。

\blockquotesource{自我探索 | 16}{白色灯塔先生}{2022-01-16}


\subsection*{表达了自己不被理解的情感和想法后感到的轻盈感}

离开咨询室后,我感觉自己很轻松。在表达了自己不被理解的情感和想法的这些部分后,我感觉自己更加轻盈了,那种不被理解的感觉没有像以前那么的沉重。我想到,那种不被理解的感觉所真正渴望的并不是真正地、完全地被理解,而是渴望向他人去表达自己不被理解这一部分。仅仅是表达本身就已经足够了。这也像是痛苦:我并不是要去消解痛苦或原谅他人、原谅自己,而只是需要将这份痛苦表达出来。这就足够了。

\blockquotesource{自我探索 | 16}{白色灯塔先生}{2022-01-16}


\subsection*{街区的宁静感,以及对这份宁静感可能不再存在的恐惧感}

在一起走去吃饭的店的路上时,走在一片安静的街区里,我突然感到很宁静。那种宁静感唤起了我过去的回忆,我回想起还在读大三时,当时的我也去了另一个城市Z,在经过城市Z里的一片街区时,那是我第一次感受到这种宁静感。在这两个城市的这两片街区里,街边都有不少在居民楼一楼开的路边咖啡店,偏雪白色的地面砖,干净的路面,间隔得十分整齐的树木,一列又一列的树荫。

距离上次感受到这种宁静感,已经是四年前的事情了。当我第一次身处于城市Z的那片街区,我的大脑开始自主地产生各种幻想,幻想着这条充满树荫的街道无尽地延伸,延伸至视野的尽头。我感觉这里更像是一个梦境,一个安静得与世隔绝的梦境、一个比现实世界更为宁静的梦境,而我想在这个梦境里一直停驻于此。

在大学毕业后,有好几次,我有考虑去城市Z找回那片自己曾经路过的街区,但我并没有这么做。在以前,还在读大学的我总想着之后如果还去城市Z玩时,说不定就能再次经过那片街区。如果有这样的机会的话,我会立即在路边停下来,到其中一家咖啡店坐着喝杯咖啡,停驻于此。但在那之后,我就再也没去过城市Z了,因为那时候我所认识的一个生活在城市Z的曾经的相识,早已不再相识。

但即使如此,如果我真的很想再回到城市Z的那片街区,即使是独自一人,我也会选择去。但我并没有这么做,所以我在想,我真正感到恐惧和悲伤的似乎不只是人来人往,还是:如果不只是人,连记忆里的场景也在现实世界里不复存在了的话,那好像任何事物、任何人都能消失不见。Nothing and no one can stay forever. I can't stay forever.

如果真的再次回到了那个地方,我可能就要面对那份幻灭,那份知道自己无法停驻于任何人、停驻于任何地方的幻灭,以及那份幻灭背后的无意义感和虚无感\pozhehao{}我又能停驻于什么呢。

……我会很畏惧:如果我所看到的现状和我所记得的样子相差很大怎么办?这好像会进一步证实了我更为畏惧的想法:过去的一切早已回不去了,过去的事物和人早已消失得无隐无踪,抓也抓不住的事物和人。

\blockquotesource{随笔 | 街区宁静感,停驻,畏惧}{白色灯塔先生}{2022-01-17}





