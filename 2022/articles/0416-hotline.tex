\chapter{接热线,无为}

\ardate{2022-04-16}{XUn8WO0L\_ZtIvGZ02UDw4A}




最近一周又开始接电话热线了。我们(和其他热线志愿者)是每月有一个Quota(要接够几个小时)。每个人都有不同的接热线节奏,有人喜欢一气呵成\pozhehao{}一口气接好几个小时的热线,有人喜欢每天接几个小时,有人喜欢天天接一个热线。而我是那种随心接\pozhehao{}想接就接,不想接就拖,拖到最后再算。

在这个月接热线接着接着,我突然发现我接热线的状态变了,和上个月、和之前不一样了。这可能和我这个月接的第一个热线有关。第一个热线的过程是个total disaster:我的几次共情失败和对方最终的盖电话,所以第一个热线只持续了几分钟的时间。在接第一个热线的时候,我的脑海里会闪过很多想法,比如说聚焦事件,评估自杀危机,是否用得上安全计划,短期因素、长期因素和保护因素分别可能是什么,对方的情绪是否会过于激动以至于丧失心智化能力等等。但最终的结果是:nobody cares。被盖电话后,有两种情感涌现了出来,第一种情感是心疼\pozhehao{}心疼自己和心疼对方,第二种情感是憎恨\pozhehao{}憎恨对方的所作所为。

在那通热线之后,我就开始很佛了。比如说如果对方不主动提自杀议题,我就很少会马上主动去问,毕竟这只是个有限的热线渠道。有时候我会留到最后几分钟才主动提,或者是在知道对方有自杀意图后关心地问一问对方是否有能帮助到自己的支持性资源,如果有的话,就不会过多主动引导对方往自杀议题里谈。

共情方面也是。如果自己真的难以共情对方的话,那就一直镜映(对方的情感和想法)好了。如果对方说着说着开始绕圈子,我就回想一下刚刚在绕的过程中是否会有一些新的议题或角度,如果没有的话就继续陪着对方绕圈子,绕多几次说不定我会看见些什么,或者是对方会开始提出一些新的素材、新的方向。

我开始觉得自己越来越无为了。对方不动,我也不动;对方动一点点,我也动一点点;对方很奋力地向前,我就再跟上去。我开始越来越少引导对方,如果对方不想往自己提的方向走,那就算了。躺平~ 摆烂~

我想这也是一种越来越放松的状态,doing but not doing。可能有点像是踩单车,从一开始把注意力死死地聚焦在平衡感和单车的倾斜程度上,直到后来身体开始自然而然地、不费力地保持平衡、改变方向。不过这也会给我带来一种感觉:我不知道自己在干嘛。在接热线的时候,我经常不知道自己在干嘛。这不是说我没有在听对方说话,也不是说我不知道应该怎么去回应对方。这些我都在做,但就像是在踩单车的时候\pozhehao{}既觉得自己在踩单车,但又觉得自己没在踩单车\pozhehao{}既觉得自己在接热线,但又觉得自己没在接热线;既觉得自己在共情,又觉得自己没在共情;既觉得自己在回应,又觉得自己没在回应。

这可能会是一种心流的感觉?至少这种感觉是我在上个月接热线时所没有体会过的。

