\chapter{自己究竟是谁,内在世界和外在世界,探索 \& desire}

\ardate{2022-03-25}{iJIMwBzlLfVFfeBqAsIxHA}


最近(特别是在最近几次的心理咨询里),我发现自己更加能接受内心世界的现实和现实世界的现实之间的差异。比如说在之前的咨询里,我发现自己在心智化他人的时候,在内心世界里的我更像是个读小学时的自我,而现实世界里的我则是个二十多岁的自己,所以当看自己的肢体时,会有一种感觉这副身体不是自己的的感觉;以及在最近一次的咨询里,我发现自己在现实世界里一直有一个家,但在我的内心世界里,我一直没有一个家,那种既有一个家但又从未有过一个真正的家的感觉。内心世界的现实和现实世界的现实虽然有所矛盾、有所冲突,但这并不有损两者各自的真实性\pozhehao{}我接受自己在理智层面上认为自己从未有过一个真实的家,但也接受自己在情感上体验过一次又一次家的温馨感和温暖,以及一次又一次地丧失掉这种家的温馨感和温暖。

最近又看到了一些关于考研的资讯,又触发到了自己一定的焦虑和恐惧,或者弱化地说是担忧吧。自己一直很害怕再次尝试那些曾经受挫过的体验,比如说考研和开车。我记得读大学的时候,自己真的为了让自己去克服一些东西而去做各种尝试,试图把自己撞得头破血流,期望会有哪一次自己能撞得出去,比如说尝试各种学术或业余竞赛。结果就是,在一次又一次的受挫过程中越来越害怕受挫,越来越压抑着自己不去尝试任何事情。哪怕只是一些看起来很小的事情,只要有一丝受挫的可能性我都不想去尝试。有时候,比起害怕失败这一外界现实本身,我更害怕的是自己会陷入内在世界的那种无助感、无力感、孤独感里。而在大学时期,经历了现实世界的失败后,自己就真的陷入了内在世界里的无力感和无助感里,感觉自己什么都改变不了、谁也成为不了。

我会想起小学时期玩的一个网游“冒险岛”。一开始,每个人都是“新手”,在新手村里边打怪升级边往外面的世界走,到了一定等级就去坐船出岛(新手村在一个岛上),出岛去旁边的大陆里的各个城市找师父转职,而我每次都去魔法密林转法师,然后再到一定等级又能转下一个分支,比如说可以转牧师或火毒法师或冰雷法师。而我总是在转职处卡住,因为我决定不了自己想要成为哪个法师,或者在练了一定等级的冰雷法师后,又会回头创个新号去尝试火毒法师,在觉得没有想象中好玩和厉害之后又转回冰雷法师。所以最后我并没有升多少级,也没有去更多的大陆去探索更庞大的世界。

但是在大学时,我并没有去探索更庞大的世界是因为我搞不清楚自己究竟是谁。是个冰雷法师?还是火毒法师?还是XXX法师?还是谁?可能有人会说:“也许探索世界的过程也是不断寻找自己是谁的道路。”嗯,是的,但这并不“合乎逻辑”。这并不符合我从小到大被灌注的逻辑\pozhehao{}一个人要先决定他想要成为谁,然后才去参加各种教育、学习、工作等经历。

那一个人在一开始是怎么决定他想要成为谁的呢?并没有人说要怎么决定这一点,因为就连这一逻辑的灌输者(比如说小时候自己印象中的一些老师和家长)自己都不清楚属于自己的答案。

当然现在,我也在尝试一些不同的东西,但我一直希望自己能像是那个刚转为冰雷法师的曾经的新手一样,不再担忧于自己的选择是否正确,而在于自己是否愿意继续向前迈步、去更远的地方探索。因为无论自己是谁,这都不妨碍自己去探索更广阔的世界。无论是法师还是飞侠还是战士,他们都可以去探索更远更多的大陆、更丰富的人物和故事,The world is for them to explore。(世界就是以供他们探索的。)

内在世界和外界世界并不是分离的,对这两者的探索也不应该是分离的,而是交融在了一起\pozhehao{}在探索外界世界的时候也探索着内在世界,在探索内在世界的同时也探索着外界世界;在新的经历里寻找着新的自我,在新的自我里经历着新的经历。

同时,现在的自己也不像是大学时那么的“受虐”。那时候的自己可能是受到从小到大无论家里人还是学校里的他人的生活观的影响\pozhehao{}人就是为了受挫而受挫的,总是要去为了克服些什么而克服,总要为了努力学习而努力学习。在有关心理学的学习和体验过程中,我发现自己之前所“默认”的这一生活观并不是正确的,或者说并不是唯一的,而是还能有别的活法。

人不是为了受挫而去受挫,而是为了获得自己想要的事物的过程中愿意去承受受挫、耐受受挫。那些受挫甚至是苦难并不是必须的、必经的,并不是单纯要让自己暴露在一次又一次的痛苦和受伤当中,而是为了获得想要的事物自己愿意走多远、愿意承受些什么。

我很不确定将来的自己会在外界世界里走到哪里、能走多远,可能自己走到哪一步就会被迫停下来。但在每一个little steps里,我都充满怀疑且确信地踏出每一个步。这不代表着我对过去和未来就不会充满怀疑和确信,但如果将自己的意识安放在此时此刻的当下,我知道这是我想要的,我是能在当下安定下来、锚下来的。我也很高兴自己没有在一开始因为某个渺茫的目标而害怕得不敢踏出任何一步,比如说自我攻击地说:“我做不到的,别尝试了,不会成功的,哪里可能啊,我怎么做得到那么远的事情……”。我也不是为了一个渺茫的目标而努力,而是为了脚下的每一步在努力,enjoy the view and see where it will lead。如果真的走到了尽头,be peace with it,然后再看看还能往哪里走。

我最近一次充满恐惧的时刻是第一次接电话热线和第一次热线考核的时候,那时候自己感觉身处于极度焦虑和恐惧的状态\pozhehao{}心跳加速、头脑混乱,仅仅是焦虑的状态本身就已经消耗掉很多心理精力了,但依然有一部分的感觉是\pozhehao{}我渴望踏上些什么,渴望踏上a new voyage。那份渴望使得那些恐惧和焦虑在我看来并没有那么的恐怖,因为恐惧和焦虑并不是全部,而且也没有进一步对恐惧恐惧、对焦虑焦虑(二级情绪),而是在焦虑和恐惧的状态下对自己说:“嗯,我很恐惧,也很焦虑,但同时我也渴望着something else。Let your desire guide you.”

