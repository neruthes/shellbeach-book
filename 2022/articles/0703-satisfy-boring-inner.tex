\chapter{无聊,无趣的内在,自我满足}

\ardate{2022-07-03}{-UoNMiW7ahzYHYxALN40gQ}

\dialoguelist{咨询师}{
	\dialogue{我}{其实之前每次我都看不见时钟的后半段,差不多过了30分的时候就看不见了。然后这次我想到的一个解决办法是,把我的手机架在桌子上,用我的手机作为一个时钟。}
	\dialogue{咨询师}{好像你之前都没有提出来,而这次会提出来。}
	\dialogue{我}{我这次会想这么做是因为,之前是打算先放在那里,等一段时间看会发生点什么,看自己是不是真的想那么做。然后过了一段时间后,我觉得我还是想在后半段看得见时间。}
	\dialogue{咨询师}{‘放在那里’,你还会想到些什么吗?}
	\dialogue{我}{我会想到,比如说上周和一个朋友去逛了逛公园。一开始我的设想是去另一个公园,但是他提议说去一个我们之前没去过的公园,我说ok呀,就去看看会发生点什么。因为如果一切都往自己的预期走的话,那也会变得很无趣,所以会尝试就先放在那里,看会发生点什么。\\
		这可能也和我最近的状态有关,最近我会感到蛮无聊的,特别是最近几天。工作很无聊,和朋友约饭很无聊。\\
		昨天我和之前喜欢但拒绝了我的男生去约饭,但我发现在和他的相处里,我好像越来越感到无聊,后来我将这种无聊摊出来跟他说,我说好像我们之间的关系没有继续深入了、话题也没有多少不同了。他说可能是因为他的生活没什么变化,他认识的人也没什么变化。我说可能会是这个原因吧,因为我好像每次见面都会想找一点新的东西,就算是在咨询里我也总会在你身上试图找一些新的装扮。他说他好像满足不了我,因为他的生活并没有什么变化、他认识的人也没有什么变化、他的存在本身也没有什么变化。\\
		当我看到这一点后,或者说当他提出来的时候,我会有一个转变的过程。一开始我会很回避这种无聊的感觉,很想赶紧找一些新奇的东西去做、去聊。但当我把这种无聊摊出来说,而他表露说好像他真的没有办法满足我对新奇这方面的需求、没有办法让我感到不无聊的时候,我就发现自己的状态会有改变,好像我突然间能够接受这种无聊。当接受了这种无聊后,我就会在想:我下一步能怎么办,我还能做点什么。好像我就会更有活力地去做点什么或说点什么,去用自己的力量激发一些东西。}
	\dialogue{咨询师}{当一开始你说把你的手机作为时钟的时候,好像你也是创造了一些东西带了进来。}
	\dialogue{我}{嗯。}
	\dialogue{咨询师}{那咨询也会给你一种无聊的感觉吗?}
	\dialogue{我}{会的。}
	\dialogue{咨询师}{为什么咨询会给你一种无聊的感觉?}
	\dialogue{我}{这种感觉好像从一开始、第一次就会有。所以那时候我才会用一些自己的文字、卡片这些东西来填满咨询的空隙、咨询的时间。但后来当我意识到咨询蛮无聊的时候,我也会有一种转变的过程,就是当我接受了:噢,好像咨询就真的那么无聊了,那我还能做点什么吗?后来,当跨过了这种无聊的状态后,我好像就能更自然而然地回应、去说些什么,而不需要太刻意的准备,比如说之前的纸条啊文字啊。}
	\dialogue{咨询师}{那之前的咨询会有给你带来不无聊的时刻吗?}
	\dialogue{我}{有时候会有。}
	\dialogue{咨询师}{那时候通常会是些什么东西呢?}
	\dialogue{我}{好像会是一些我没有预料到的东西,比如说在自我探索的过程中发现了一些我之前没有设想到、没有看到、在我的视野范围之外的事物。就像是昨天和朋友约饭聊天时我才意识到我对待无聊会一种状态的改变一样。}
	\dialogue{咨询师}{那这种无聊会带给你一种怎样的感觉吗?}
	\dialogue{我}{就像是一种空白的感觉,好像一张空白的纸张,然后我需要在上面写一些东西、创造一些东西,就像是咨询一样,我要去说些什么。}
	\dialogue{咨询师}{刚刚提到你能够去接纳这种无聊,接纳这个过程是怎么发生的呢?}
	\dialogue{我}{比如说昨天和朋友约饭聊天的时候,当他说他没有办法满足我对新鲜事物的需求、没有办法让我感到不那么无聊的时候,我会很伤心和难过,也会很感到很失落,好像不只是他,好像我对外界的很多事物和人的期望都会落空,都会让我感到很无聊。\\
		好像就是在那一刻,我突然接受了这样的现实:这个世界上的各种事物和人就是那么无聊的了。当跨过这种无聊的空白的时候,下一层好像是一种活力感、力量感,好像自己能够创造点什么。因为在过往的经历里,自己一直都有这样的自我满足的能力,比如说去写作,特别是写短篇故事,好像自己一直都有这份力量和创造力去自我满足。}
	\dialogue{咨询师}{你能想到些什么例子吗?}
	\dialogue{我}{你是指在现实世界吗?还是在我想象的世界里?}
	\dialogue{咨询师}{都可以。}
	\dialogue{我}{我会想到我读大学时,大概大二大三的时候,写过一篇短篇故事(《暗世界》)。那个短篇故事里的世界是另一个世界、一个充满光亮的世界,然后整个城市都被电力所照亮着。但是只要一停电,当灯光关闭的时候,城市就会陷入黑暗,然后角落里的阴影就会将人拉进去,所以不断会有人失踪。但是当那些灯光重新被点亮的时候,角落里又什么也没有。就是那种未知的感觉才充满着魅力、才更具吸引力,同时也会让人感到更加恐惧,因为不知道它的起源、它的来源是什么,但它就一直在那里。}
	\dialogue{咨询师}{我留意到当你说到这些内容的时候,你会有点皱眉。}
	\dialogue{我}{嗯,好像现在当我说起这些我过去写的故事的时候,我依然会有那种兴奋的感觉,好像我是能够自我满足的一样,会有那种自我满足的兴奋,能够满足我对新奇事物的渴望,会让我感到很兴奋。}
	\dialogue{咨询师}{你会有对这种新奇的事物有怎么样的设想吗?}
	\dialogue{我}{我好像设想不到。正是因为自己设想不到,所以才会感到新奇。正是因为设想不到、意识不到、摸不到,充满着未知,才是乐趣所在。就像是一个求知的过程,乐趣所在并不是那个结果,而是这个过程本身。不知道最终的目的地会是什么,才更让自己感到更加兴奋。但如果自己一开始能够设想得出来的话,就不是未知的了,就会没有了那种乐趣。\\
		就像是一个小朋友在玩玩具。当他发现外界现实很无聊、没有办法满足到他的时候,他就转向了内部世界、想象中的世界去自我满足。}
	\dialogue{咨询师}{当你提到玩玩具的时候,好像你也在说你所做的事情与他人无关,你能很好地自我满足。}
	\dialogue{我}{其实我并没有那种想要回避他人、想要推开他人的那种感觉。我好像只是在一个人自己玩,但并没有想要推开他人。因为好像只要我能够自我满足,那对我来说就已经足够了。但是我不知道其他人会有怎么样的反应,比如说当一个小朋友在玩玩具的时候,他的母亲可能会有各种反应,比如说他母亲可能会只是在一旁看着他;也有可能想要插一条腿进来和小朋友一起玩;也有可能会把他的玩具挪开,跟这个小朋友就说:你别玩玩具了,来和我玩。\\
		我能设想到很多可能性,但这些关于他人的可能性是我不能控制的,所以我好像也没有太顾及他人,更没有想要去推开他人。}
	\dialogue{咨询师}{为什么你会想到母亲?会是因为我说话的语气吗?}
	\dialogue{我}{不是不是。是因为我在说的时候,我就回想起了一些小时候大概是上小学前的回忆,比如说在外婆家或者是在父母家,我坐在地板上玩积木或者是玩具的回忆。比如说一个拱形的玩具,我会拿起来看着它,然后设想这是一个建筑,然后我会设想建筑周围会有一些怎么样的场景,就是在现实世界的基础上去构思自己的想象世界可以是怎么样的。}
	\dialogue{咨询师}{但当你这么说的时候,好像都是你一个人在玩,并没有其他人和你一起玩,并没有人陪着你,好像他们把你留在了那里。}
	\dialogue{我}{不完全是。因为我的印象里,大人都会在旁边看着我玩,只不过他们没有参与,但他们依然会在旁边看着。\\
		当你这么描述的时候,好像你在说:其他人把我留在了那里,好像是其他人把我给遗弃了。但好像我并没有被遗弃的感觉。因为当我在玩玩具的时候,在我的印象里那些大人都是有在旁边的,只是没参与。}
	\dialogue{咨询师}{但他们精神上呢?他们在精神上好像遗弃了你。他们好像身体在那里,但他们的精神并不在。}
	\dialogue{我}{Em……我好像也没有那种在精神上被遗弃的感觉。因为好像他们并不是有意藏着这个部分。不是说:比如说他们看电视的时候,不是他们有意识地将他们内在很丰富的部分移开了、遗弃了我。而是他们从一开始就没有那个有趣的内在,好像他们从一开始就是一副空壳。如果他们从一开始就是一副空壳的话,我也不会因此而感受到被遗弃感,因为他们本来就没有丰富的内在,而不是他们藏着不给我看见。\\
		但我依然会感到不足够,就是这个现实世界里的事物和人都不足够。}
	\dialogue{咨询师}{我在想,相比于他们藏着丰富的内在的那一面,如果他们原本就没有丰富的内在话,那后者的情况会不会更加的无力?因为好像你只有自己一个人,需要将他们填满。}
	\dialogue{我}{会是的。但这种无力一直都是我的状态。\\
		但我也很确定,他们并不是藏着内在的丰富不让我看见。因为自从学咨询,特别是学咨询之后,我能够看到很多人的内在。我发现他们并不是刻意要藏着什么东西不给我看(虽然对于有的人,我确实有那种对方藏着什么东西的感觉),而是他们本来就没有什么丰富的内在,都很普通、很平凡,然后当他们将这部分摊出来的时候,比如说昨天约饭的那个朋友,当他将他的内在摊出来的时候,当我看见他的内在好像真的很平凡、很无趣,很没有办法满足到我的时候,我就看见,好像他们并不是刻意隐藏着一些很丰富的自我,而是他们本来就没有这个部分,本来就是那么平凡、无趣。所以我也不会感到他们在躲着我,也不会有他们藏着那丰富的部分的那种被遗弃感。}
	\dialogue{咨询师}{在最后,我也会很好奇,这种无聊的感觉的来源会是什么?因为好像你会在很多事物和人身上都感觉到很无聊、很无趣。也许我们能留到下一次咨询再继续说。}
}
