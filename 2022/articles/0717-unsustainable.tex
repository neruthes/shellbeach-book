\chapter{“这样的好事并不会一直持续下去”}

\ardate{2022-07-17}{LmeKgbfxdYmBJ-vLAa8sdA}


\dialoguelistthin{咨询师}{
	\dialogue{我}{我现在的状态可能不太好。我昨晚没有吃晚饭,因为下班的时候心情很不好,感觉绷不住了,只想躲在被窝里自闭,所以我也就那么做了。然后我半夜睡醒了,但没有起床,只是躺在床上一直刷视频,刷到天亮,直到时间差不多才出门来咨询。}
	\dialogue{咨询师}{噢?}
}

其实是和一个男生有关。我上周和一个男生见面了,就是上次说在之前聚里看见他和他对象在一起的时候我会很想融入他们的一部分的那个男生。在之前我就发现我会对他有好感,但我不知道为什么我会对他有好感。有可能是因为我们两的长相很像?那次聚会有朋友将聚会的合照发在一个更大的群聊里,然后马上有人说我们长得蛮像的,我才突然意识到这一点。所以外貌的相似可能产生好感。但直到见完面后,我依然不知道为什么我会对他有好感,仅仅是因为外貌吗?

然后我在这周记录了三个梦境。

第一个梦境是和他见面那天晚上就梦到的。梦里的我在大学宿舍和其他舍友在等大学考试的最后一门科目\pozhehao{}考研政治。考试原本定在下午3点开始,但没有正式的通知,所以我们都在宿舍里等考试通知。但我没有准备任何东西,甚至连考试是开卷还是闭卷都不知道。等过了下午三点,我们依然没去考试,然后宿舍的同学们提议说要不要去市中心打火锅、吃点东西,我们就一起去坐公交了。公交里近一半的座位坐着的都是我们宿舍楼的同学,而我在不同的人群里聊着天。但在这个过程中,我依然很焦虑,我会想到万一突然通知说要考试,那我们是不是要赶回去。后来我决定一个人找了个靠窗的位置坐了下来,去翻我带的书。我在书的最后一章找到了关于课程设计的内容,知道应该如何着手开始做课程设计。这时候我的焦虑程度就下降了,而我也随之醒了。

醒了之后,我想到,无论是考研政治还是课程设计,这两者都会给我带来很大的焦虑,而这种焦虑感也是我和那个男生在相处、互动的过程中经常体验到的。我会感到焦虑之处是,他还在读研,而且他说他之后打算去读博,甚至是出国留学。如果他不止是和我不在同一座城市,甚至不在同一个国家的话,我并没有能力去跟上他,留不住想要共处的人。

我会想跟他表露这个部分,想跟他说我只是普通本科学历,拿着一份在我看来很低的工资。而我之所以想表露这个部分,是希望自己感到自卑的这个部分能够被他接纳。如果他能够接纳我的这个感到自卑的部分的话,我会设想这能带来一种很强烈的被接纳感。

同时我也想他在乎我,因为他带我逛他经常逛的地方这一方式似乎也是一种他表露内心事物的方式。我有留意到他好像很擅长发现身边事物的规律,比如说个体店和连锁店与人流之间的关系。想到他是做研究的,擅长这一点也不足为奇了。但我会留意到他很少将精力投注于他人、很少投注于我,比如说他很少问我的感受、看法。当我跟他提出这一点时,他说他会向面前的人表达他内心的事物。我猜想这是他表达他在乎面前这个人的方式,所以我也会想他在乎我。如果我在乎他,而他也在乎我的话,我会设想这能带来一种交融感。

虽然我知道我想从和他的相处里获得被接纳感和交融感,但我依然不知道为什么我会对他有好感。

第二个梦境是我和他在一栋教学楼大概三、四层楼高的教室里。我们打算做一个实验,但是好像缺了点东西,所以还没有办法开始做实验。我留意到很多人在走廊那,我走了出去,看到有很多同学和一个老师在扶墙那。我看见他们在用绳子拉着一些东西,我靠着扶墙往下看,看到楼下也有很多学生,绳子挂着一个篮子,篮子上面放了一些东西,可能是实验要用的器材,等那些实验器材拉了上来后,我和那个男生就能开始做实验了。看着他们齐心协力的样子,我一直在说:没有用的,没有用的,没有用的,没有用的……”

\dialoguelistthin{咨询师}{
	\dialogue{咨询师}{“没有用的”,你能想到些什么吗?}
	\dialogue{我}{我会想到很多事情。比如说和前任的相处,以及一次又一次地去初恋曾经出现过的地方,希望能和那个地方产生连接,以此来连接他,但每一次都会失望,因为并不可能偶遇到他。我甚至回想起小学一、二年级的时候,我想买点卡充值网游,但父母并不会给我这样的钱,所以我只能存着一块、两块钱,直到存到十块、二十块、五十块才能去买点卡。好像过去的很多事情都会让我感到一种无力感\pozhehao{}我的努力是难以改变现实的,是没有用的。}
	\dialogue{咨询师}{(沉默)}
	\dialogue{我}{睡醒之后,我会想到,拿到实验器材和他一起做实验似乎象征着我和他能够进一步地深入关系。但当我看见那么多的人想要促成这件事情的发生时,其实我并不相信这能够成功,我一直在说:没有用的,没有用的,没有用的。而且我会注意到在上一个梦境里,我的工具是书本,这是一个人的事情,而我并没有向其他同学询问关于考试的事情。当我从上一个梦境里察觉到了这一点后,我跟自己说:其实我是可以接受别人的帮助,甚至是可以向他人讨要帮助的。然后到这一个梦境时,即使有那么多人在努力地帮助我、努力地想要促成这件事情的发生,但我依然不相信这是有用的,依然觉得这是“没有用的”。\\
		在第三个梦境里,我在这(咨询室)附近的那个地铁站,我出站的时候看见了你牵着一个小朋友。那小朋友就是这几次咨询前我都会在走廊里看见的那个在隔壁咨询室的小朋友。当第一时间看到你们的时候,我下意识就躲开了,因为我并不想在咨询室外碰见咨询师。等你们离开后,我看了一下时间,距离咨询开始只剩下10分钟了,而我还没吃早餐。我去找了家面店,但又想到耗时太长,不可能的。然后我打算去便利店买个包子就马上赶过来。当我到便利店门口时,我就醒了过来。\\
		这好像也是我周五的状态\pozhehao{}开始有点照顾不好自己,比如说吃饭、睡觉的时间都打乱了,这些生活里面的规律事件开始变得支离破碎。这好像一直以来都是我应对情绪的方式。\\
		睡醒的时候我会在想,为什么我会作这样的一个梦?其实在梦里第一眼看见那个小朋友的时候,我会感觉那就像是自己内心的小朋友的部分\pozhehao{}那个弱小、脆弱的部分。但我会不明白为什么咨询师要牵走我内心的小朋友的部分。而当那个部分被牵走后,我整个人的状态变得很慌张失措\pozhehao{}我不知道自己的早餐要怎么解决、不知道应该怎么及时赶到咨询室。我想到,这种慌张失措的感觉也是我在这周经常感受到的,因为我在周初的时候就跟那个男生表达了我对他的喜欢之情,因为我说我不想隐藏这个部分\pozhehao{}一方面是如果隐藏的话,好像关系就会变得不那么真实;另一方面是我也担心他会察觉到我在对他隐藏着些什么。所以在跟他表达了我的喜欢之情后,我不知道他可能会怎么想、可能会怎么做。}
	\dialogue{咨询师}{当你在梦里看到我和那个小朋友牵着手的时候,会给你一种什么样的感觉吗?}
	\dialogue{我}{会感觉到一种连接感,甚至是一种融合的感觉……(沉默)}
	\dialogue{咨询师}{你会想到些什么吗?}
	\dialogue{我}{我会想到其实我还蛮警惕这种融合的感觉。可能在我看来,两个个体不应该是融合在一起的,而是两个独立的个体,有时候牵着手、有时候分开,而不是一直牵着手,就像是我在走廊第一眼看到那个咨询师和小朋友是一直牵着手走进咨询室的。这种对融合的警惕感也会存在于我和上周那个男生的相处里,比如说我周初跟他说我想约他周末见面,但是那时候他说他还不知道周末有没有空。然后我在周四的时候问了他,但他没有回复,直到周五晚他才回复说这周还是不见了。我会想他更早一点告诉我,因为我想在周末约自己的朋友见面,而并不想腾出整个周末的时间去等他是否会周末有空。我不想去讨好他,因为如果这样做的话,我就不再是我自己了,好像要放弃自己的一部分。\\
		但另一方面我也会渴望这种融合的感觉(咨询师:“是啊”),就是能够什么事情都和对方一起做,比如说以前和前任相处时,他说他想下楼吃个宵夜,我说那我和你一起去,想要对方和自己待在一起。}
	\dialogue{咨询师}{(沉默)}
	\dialogue{我}{同时我也会想,为什么梦里的那个咨询师要牵走我内心的那个小朋友的部分?这让我想起上周那个男生,他就像是一个咨询师,那个能提供自体客体经验的人。在这一周,我都会感到一种若隐若现的力量感\pozhehao{}我会想到他能够去做他想要去做的事情,比如说读研、学术上的成就,而我会设想我好像也能够像他一样做到自己想做的事情。但这种力量感只是若隐若现地存在,当他不回复我信息的时候,我会感觉这种力量感消失了,而当他回复我信息时,这种力量感又出现了。\\
		我会留意到我可能更像是在他身上投射了力量感,然后在和他的相处过程中,我好像能收回、摄入这种力量感。但他真的能够成为那个给我提供自体客体经验的人吗?因为他好像并不会即时回复我信息,所以我也不确定他是否是这样的人。但我确实渴望从他身上获得这样的自体客体经验,渴望获得这份的力量感。}
	\dialogue{咨询师}{你会提到好像和他的相处当中,你并不总是能够获得这样的体验。}
	\dialogue{我}{嗯。我是在社交软件上给他发信息而不是在微信上。一方面是因为社交软件上有我的头像。我在上周和朋友聊天的时候发现我依然不相信他人眼中的我是独一无二的,但如果连我自己都不愿意相信他人眼中的我是独一无二的,那么我又怎么能令对方相信他们眼中的我是独一无二的呢?所以我会想在社交软件上以呈现自己的头像的方式来让他视我为独一无二的。\\
		另一方面是,社交软件上有已读未回的功能。我周初问他周末还有没有空见一面,然后我看到他已读未回,直到那天晚上才回复我说他还不知道周末有没有空。后来我周四的时候去问他,但这时候我发现信息依然是未读,但我是能看到他的上线时间的,也就是他已经上线了,但没有点进去我的信息。但信息会有简介栏,所以他可能是知道我的信息,但没有点进去,而是周五晚才点进去回复我。可能是他已经察觉到了我在利用已读未回的功能来试图侵入他的边界。}
	\dialogue{咨询师}{那你会怎么理解他已读了你信息但没有及时回复,以及他后来没有点进去回复你呢?}
	\dialogue{我}{我好像觉得我和他处于一场拉锯战。我想和他拉近一点距离,然后他就会拉远一点。所以可能下次我就会在微信上给他发信息,而不是在软件上,那就不会有那么多的揣测和困扰。那种感觉就像是很on edge、很战战兢兢\pozhehao{}我会一直上软件看他回复我信息了吗?他没有回我信息是因为他太忙而没有上软件吗?还是他上软件了但没有回复我信息?}
	\dialogue{咨询师}{其实我会留意到当我问你你是怎么理解他没有回你信息的时候,你好像没有提及感受。}
	\dialogue{我}{我会感受到一种不被在乎的感觉,当他没有及时回我消息的时候。但我想这种感觉可能还是要带到线下去和他聊,而不是在线上。因为我好像在亲密关系里会很难去理解言语当中的信息,会有很多误会和误读,因为我好像会将自己的情绪灌注在其中,会扭曲一些文字原本的意思。}
	\dialogue{咨询师}{好像这和之前你的做法不一样。}
	\dialogue{我}{嗯。……其实刚刚提到和他线下见面时,我会有一种恐惧\pozhehao{}恐惧于如果我和他没有下一次见面怎么办?因为我会想到和前任相处时,有时候我想和他有一些面对面聊天的机会,但并没有这样的机会。他总是没有空在家,或者是即使他回到家,他也只是躺在沙发上看综艺,看到一定的时间就去睡觉。我有跟他说我想我们能找个时间去约会,无论是找个地方逛逛或者是坐下来好好聊聊天,但他说他并不想这么做,他工作了一整天已经说了很多话了,他不想回到家还继续说话。\\
		不过有些时候,当我想和他就一些线上的文字或语音的误会或冲突而好好面对面聊天时,但当他回到了家里,我就会觉得那些冲突、误会都不重要,只要他人在这里就足够了。那些冲突和误会就能一笔抹去了。}
	\dialogue{咨询师}{为什么能够一笔抹去?}
	\dialogue{我}{因为好像只要他还愿意在场,这就代表他还在乎我。因为如果不在乎的话,他可能晚上就不回来了,可能会去另一个地方过夜,而且他不会告诉我他去哪里过夜,就是他会退回到自己的空间里。但是他愿意回来、愿意为我而在场、愿意在那里(公寓)的时候,就意味着他依然选择和我有交集、依然在乎。}
	\dialogue{咨询师}{我会想到之前你提到你父母在你看来只是人在那里,但好像并没有真正在那里。}
	\dialogue{我}{嗯,就像是一个空壳,没有人与人之间、灵魂与灵魂之间的连接的时刻。}
	\dialogue{咨询师}{我能理解为,当他想愿意为你而在那里的时候,你是想要追求那种人与人之间的连接的时刻吗?}
	\dialogue{我}{并不完全是。好像我想要追求前任在那里,是因为我和前任的交流越来越少了,灵魂之间的交流越来越少甚至开始消失了。所以好像我留不住他的魂,起码我能留住他的人、他的肉体。好像留住他的肉体能够一定程度上弥补他的灵魂的缺失。}
	\dialogue{咨询师}{(沉默)}
	\dialogue{我}{联系到上周的那个男生,上周和他见面时,我会感觉到我和他是能够有深入交流的。但当我发现这周可能约不到他见面时,我好像就想要留住他的人,想着如果能见上面就好了,想要从他身上获得那种确信感。}
	\dialogue{咨询师}{你能多说一说这种确信感吗?}
	\dialogue{我}{就是确信自己喜欢的东西是自己喜欢的、自己擅长的东西是自己擅长的。}
	\dialogue{咨询师}{如果无法确信呢?}
	\dialogue{我}{那会给我一种困惑感,对自己的能力的困惑、对自我的困惑\pozhehao{}我究竟擅长些什么、喜欢些什么。}
	\dialogue{咨询师}{好像你期望自己一直能够确信。}
	\dialogue{我}{嗯,是的。比如说接电话热线,一开始是怀着期望、期待,然后慢慢耗竭、倦怠,后来有怨恨、憎恨,但最近又有开心和成就感。结果就是不确定自己真的喜欢做这件事情吗?就像是我渴望做一件事情能够一直有好的体验,然后就希望能够继续追求这种好的体验。}
	\dialogue{咨询师}{我会想起你说到你和你叔叔阿姨的相处时,好像你也希望能够一直有这样的好的体验,能够一直这样生活下去。}
	\dialogue{我}{是啊。这也是很正常的期望,不是吗?}
	\dialogue{咨询师}{(点了点头)}
	\dialogue{我}{(沉默)……当我提到这一点时,我的思维马上就发散了,以至于我不知道现在应该说什么。我的思路会发散到,比如说这个个体能够怎么应对将来的挫折、这个个体能够内化多少这样的好的体验的部分。}
	\dialogue{咨询师}{我也会有相似的感觉,好像当你提到这一点时,你转向了理性的思考。}
	\dialogue{我}{可能在我的经验里,这样的好事并不会一直持续下去,会马上面临一个危机,然后一切都会破碎、崩解,然后又会回到一个人的生活里,要一个人去面对很多难受的感觉,一个人继续生活下去。所以需要去思考一个人要如何去面对这些事情。}
	\dialogue{咨询师}{好像你并不相信这是有用的,这是“没有用的”。}
	\dialogue{我}{可能因为我并没有这样的体验,并没有这样的好的经历能够一直持续下去的体验。我有的只是一些零碎的片刻,比如说和前任一起去逛商场或者是到楼下吃宵夜的时刻,而这些时刻很快就转瞬即逝了,又回到了一个人的状态。}
	\dialogue{咨询师}{而且好像你也没有办法在这些时刻里内化到一些有用的资源。}
	\dialogue{我}{其实会有。比如说这周当上司说我的工作质量不行时,我会感到很难受,然后我的脑海下意识就回到了之前和一个朋友在隔壁城市的一个村庄里一起找了个角落坐着的画面。那个场景很安静,只有我和他在那个小巷里,在那里我会感到很安全。虽然我们最终还是离开了那里,但那个场景依然留在了自己的内心,能够在我难受的时候下意识地回到那里,获得一份安全感。}
}
