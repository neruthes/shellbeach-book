\chapter{活着的意义,心智化}

\ardate{2022-03-01}{xTnivKnLgjF6ndjDAeDDxQ}

活着的意义是什么?

我记得以前的我曾经迷失在这一提问里很久很久,试图以自己的方式去寻找答案。我曾经拿这个提问去问不同的人,去看不同的人对此写的书。我记得最让我感到满意的书是欧文·亚隆写的《存在主义心理治疗》。但这依然不足够。我想这很大程度上是因为没有人能取代自己去走要走的路,没有人能取代自己去找属于自己的答案。Paths I had to walk alone. Answers I had to find for myself.

我想起我在大学时的写作都是无方向的\pozhehao{}想到什么就写什么,就算是胡乱写一通也能写出不少字,也能在其中找到思路的脉络。但现在的写作几乎都是带着方向的,比如说写咨询的回忆稿,写与谁谁谁见面的经历,写最近自己被读到的书、遇到的人所激发的思绪。

和Neruthes的聊天里,我一直认为Neruthes在用过于理智化的方式来理解他人,就像是一个物理学家或数学家试图用公式、框架、结构、模型等方式来理解心理学。Neruthes会让我想起过去的我,但我区分不出来“过去的我”具体是在哪一时间段里的我。过去的我也会用过于理智化的方式来试图理解他人。具体的记忆我已经不记得了,但我记得那种无法理解他人的感觉,而看待他人的话语的方式就像是在看固定的公式表达,而且我也不能理解对方为什么会生气、为什么会开心、为什么会哭泣。

这一切的改变是从和初恋的关系开始。和初恋相处的那三个月里,我开始感受到爱的温暖,而这种温暖开始激发了一些自己内心的东西,自己开始感受到更丰富的情感,就像是自己的情感感知光谱被扩大了。当然,和前任的关系也是如此。

最近我在读《心智化临床实践》:

\blockquote{
在心理治疗中,我们心理工作者大多鼓励外显心智化,举例来说,就是将感受转换成言语:“当她那样说的时候,我认为她是想要胜我一筹,而这惹恼了我。”外显心智化是象征性的,绘制图画或创作歌曲以表征某一种心理状态\pozhehao{}正如人们在艺术治疗中常常做的\pozhehao{}也可以归为外显心智化。然而,通常把语言视为外显心智化的媒介,而大部分心智化都采用了叙述的形式\pozhehao{}故事。Holmes曾经非常贴切地将心理治疗比喻为培育讲故事的功能,他描绘的心理健康的特征是,取决于“构建故事与打破故事之间的辩证关系,形成叙述的能力与在新的体验基础上瓦解叙述的能力之间的辩证关系”。外显心智化是以准确、丰富和弹性的融合为其特征的,这是最佳状态。正如心理治疗实践所证明的那样,安全依恋是人们阐释心理状态的源泉。

……正如我们之前所阐释的,外显心智化是较为有意识的、深思熟虑的,并且具有反思性的。但外显心智化不过是冰山的一角,在人际互动中,占据主导地位的是内隐心智化\pozhehao{}自动且无反思性。举例来说,共情必然包含着对面部表情和肢体姿态的某种程度的非意识镜像。如果某位治疗师试图将这种镜像建立在外显推理的基础之上,那么其结果一定是僵硬不自然的。在一场唇枪舌剑中同样需要内隐心智化,正如我们所经历过的景象:某人谈到另一个我们并不知道的人,但却不考虑我们对此的一无所知,也不提供所需要的背景信息时,我们肯定会被激怒。在理想情况下,我们交替说话并且留意着他人的看法,这并不需要明确地思考该怎么做,因为负责调控这一过程的心理活动实在是复杂得令人难以理解。

外显心智化与内隐心智化的区别也许可以体现为记忆领域的一种平行式区分,换言之,也就是语义(外显)记忆与程序(内隐)记忆的区分,又或者,更简单地说,是知道什么与知道如何之间的区别。外显记忆是我们在考取驾照过程中需要通过的笔试部分;内隐记忆则是我们所需要完成的路考本身。内隐心智化是在程序上知道如何操作;外显心智化则是某些可以用象征化形式表达的东西。

……一旦进入“谈话治疗”,我们这些临床医生以及我们的病人,都会有意识地关注外显心智化。我们也常常通过心智化过程,让那些本来不太有意识的事物变得更加意识化。尽管如此,我们也不能将心理治疗理解为仅仅专注于解释。恰恰相反,我们同样会采用外显过程(更高阶段的意识)来关注内隐领域,最重要的是,关注于对自我和他人的感受,这样做的目的是,这种关注可以变成不断增加的自动化和直觉,就像省映性过程可以逐渐转化为反射性的过程一样。正如Karmiloff-Smith所说的那样,发展会同时沿着两条路线进行,一条是从内隐到外显,另一条则恰恰相反:

\itshape 发展与学习,似乎会沿着两个互补的方向进行。一方面,它们会采取逐渐自动化的方式(即,让行为变得更加自动,也更不容易被理解)。另一方面,它们则会采用“外显”的过程,让可理解性增加(也就是说,将那些在行为组分中的内隐性程序表征为外显的信息)。这两个方面都与认知的改变相关。
}

大学时的那些无方向的写作,现在回顾起来,似乎一直在外显化自己的内隐心智化,试图用文字以表达和组织一些之前没有被意识到的事物。外显心智化使我能够和他人进行情感上的交流,而不是仅仅停留在纯智力层面;内隐心智化使我能够感知到他人的情感,而不是仅仅停留在字词上的解释和理解。但在大学后,特别是在学习心理学后,我的写作似乎一直停留于外显心智化的层面,停留在文字的表达里,而没有像大学时试图将一些未被意识到的事物通过文字以外显出来。

这也会给我一种感觉:感觉我自己像是一个机器。我不断地接收各种输入,并在经过自己的处理后进行输出,但好像从来没有一些自发的输出。就好像,如果我没有了输入,那我就不会输出了。如果没有了外显心智化的事物的刺激,那我就不会外显化自己的内隐心智化。

大学时的那部分自我更喜欢去外显化自己的内隐心智化,去写一些有的没的的文字,总是去试图表达一些难以表达的事物;现在的自我则更喜欢停留在外显心智化的层面,在文字上运用逻辑、联想等方式进行进一步的加工处理。我好像从来没有看重外显化自己的内隐心智化的过程,而是更看重如何组织和加工外显出来的外显心智化内容\pozhehao{}不怎么看重如何表达难以表达的事物,而更看重如何组织要表达的内容。所以自己才更像是一个机器,只是在不断组织和加工着手上的材料,而丝毫没有留意自己是如何获得材料的,更重要的是,如何“无中生有”\pozhehao{}外显化自己的内隐心智化。

\blockquote{
    在等电梯时,我试图继续往下挖,我好像能挖到一个更深处的自我,那个自我身处于黑暗和痛苦当中,像是在水下被一层膜裹成了一团。那个自我好像在说:我还依然身处于痛苦当中,LOOK AT ME!  在毕业之后,我就理所当然地认为读大学时的那部分自我消失得差不多了,但好像那部分的自我依然存在着,那个之前被我遗忘的自我,依然身处于痛苦的自我。
    \blockquotesource{Life still goes on, I still go on.}{Shell Beach}{2022}
}

我能感觉到,在现在的自己里,依然有一部分自我,那个大学时的自我,渴望着大学时期的生活方式:喜欢在教室上课时看原著小说、喜欢半夜在宿舍里写有的没的的文字或者是看原著小说、喜欢在校园里闲逛、喜欢每天背单词、喜欢和朋友去饭堂约饭。大学时热爱着生活的那部分自我好像在我毕业之后就一直被我遗忘了,这可能也是为什么在上次心理咨询结束时我能感觉到大学时的那部分自我依然身处于痛苦当中。大学时的自我好像被困于我现在的生活里,现在这一自我所选择和渴望的生活。

现在的自我渴望现在的生活,但大学时的自我渴望着另一种生活,而这两个自我依然存活在自己的内心里。因此,我似乎需要照顾好大学时的自我,在现在的生活里融入一些那部分的自我所渴望去做的事情,比如说看原著小说、用英文写作、背单词甚至是写原著小说,并在这个过程中外显化自己的内隐心智化。而现在的自我所渴望去做的事情,比如说在文字上运用逻辑、联想等方式进行外显心智化的进一步加工处理,在这个过程继续“构建建筑的高度”。

活着的意义是什么?我想这一提问的精彩之处在于,it could be the beginning of anything and everything.
