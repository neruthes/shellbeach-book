\chapter{随笔 | 唱K,呆在当下}

\ardate{2022-01-12}{yQbuSvbpzlqDJeRBGnW-Qw}


周一晚基友H过生日,晚上我去了他组织的唱K局。

进了K房后,加上我,在场大概有个十个男生,其中大多数的男生我并不认识,我只认识基友H和他的对象,以及在之前的聚餐里也出现过的另一个男生。期初,大家的话都比较少,更多只是在各点各的歌。在等了一会儿后,基友H开始介绍在场的每个人,但介绍的内容只是各自的其中一方面,比如说星座是什么、工作是什么、哪里人等。在被介绍完之后,大家又继续各自唱各自的歌。

随后,基友H暂停了音乐,并提议每个人说一个最近让自己印象深刻的人。大家便建议发起的人自己先做个示范,基友H便说了两个男生的故事。之后他点名想让另一个男生分享,但那个男生说还需要时间想一下,而另一个男生则说唱K不要停下来,还是继续唱吧,之后再停下来让另一个男生分享。后来大家便继续唱K了,而这个说故事的提议也没有继续下去。这让我感到蛮可惜的,好像终于有了一个大家能够深入地分享一些事情的机会时,有的人反而会很抗拒这种“走深”,只想回到一些表面的事物,一些不需要触碰到深处的事物。

后来,我在手机上打了些字:可能唱K不太适合在一开始建立连接,然后给基友H看,他说他也发现了这一点,觉得可能唱K这一活动更适合有一定的相识基础的朋友们。他继续说,这是他第一次组局,觉得大家还是需要些能参与的活动。我说也许可以尝试去清吧聊天,让大家面对面地坐下来聊天。他说他之前有试过,但效果不太好,觉得还是有一些能参与的活动会好一些。我会认为,活动可以有,但活动的内容本该是让彼此产生连接,而不是让每个人在一个吵杂得难以进行聊天的环境里自high。(当然我在那个当下没有跟他说这一点,不然有点扫兴)

\tristarsepline

我也发现了另一件事情。在场的十位男生里,坐在最左边的两个男生和最右边的两个男生(共四个男生)会比坐在中间的男生更频繁地玩手机,而他们也是最早选择离开聚会的。看着他们玩手机的样子,那个画面会给我一种他不想呆在当下的感觉。我会想到,他们是因为一开始就不打算呆太久,所以才坐在最靠边的位置方便离开,并通过频繁玩手机来打发时间?还是因为频繁玩手机这一行为让他们更难以呆在当下,所以才更想要离开?

另一方面,过生日的基友H和他的对象都很能呆在当下。基友H会在别人唱K时自己摇晃着身体或用手模仿着节拍的韵律。由于我坐在基友H的左边,基友H的对象坐在基友H的右边,所以我只是不时会看到他对象。基友H的对象经常能和他聊起一些话题,以及在和他聊天时,他对象的坐姿是斜四十五度对着他的。(中间一行的座椅都平着一列的)

后来,我发现那些还没有选择离开聚会的人,都是在别人唱K时,自己能够呆在当下的人,比如说看着屏幕里的视频和歌词、看着唱K的人、和身边的人闲聊等,而不是频繁刷手机。我也发现自己更能呆在当下了,发现K房里的环境并没有我之前所记得的那么吵。虽然有喇叭声和唱歌声,但如果仔细地听,依然能“听到”背景里的那片宁静和平静。

\tristarsepline

我会认为,呆在当下的能力对于人际关系而言蛮重要的。当然,这不意味着即使身处一个自己不想呆下去的环境也要逼自己呆下去,而是意味着可以尝试用另一个状态来身处于当下,而不仅仅是回顾过去或计划将来。

这也让我想起最近在公众号的倾听渠道的反馈。对方说他在倾诉时的短暂停顿里,看着我只是不断地点头,所以就继续说下去了,他会好奇我是否真的能理解他。我回答说,那是因为在短暂的停顿后,他又会继续说下去,所以我不想打断他的表达。而且,我也能发现他在讲的过程中会不断组织自己讲过的内容,就像一些零零碎碎的点开始织成一张网,而那个网的中心也越来越清晰,他说的内容也越走越深。他说我不需要担心是否能理解他,因为我的不断点头已经能让他想要继续说下去。

我会想到,其实在我们的日常生活里,即使是朋友之间的相处里,我们很少有什么都不“做”的时刻。我们很少不去思考些什么、不去做些什么、不去说些什么,不去回忆过去、不去计划将来。所以,无论那是一个怎样的当下\pozhehao{}这个当下可能是在通勤的当下、在朋友说着话的当下、在自己准备入睡的当下\pozhehao{}即使每个人时时刻刻都处于当下,当下似乎总是被人所遗忘。

当自己越来越能呆在当下,而不是在倾听的过程中将大量时间花在自己的思考、自己的分析、自己对倾诉焦点的计划等思绪或情感或过去或未来时,当只是听着、看着、投入于那个当下时,我似乎能在吵杂的环境里“听见”宁静,也不需要过于在意对方是否会很介意我是否能懂他。因为对方好像能“看见”我有在听,无论我是否听懂了,或者说对方也许能感受到那份“投注”。
