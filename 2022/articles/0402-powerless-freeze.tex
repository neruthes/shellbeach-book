\chapter{无力感,僵住}

\ardate{2022-04-02}{mNI5QJAQ0vum\_rFqHrKU\_g}



\dialoguelist{咨询师}{
\dialogue{我}{其实刚刚我在地铁上看见你了,就是在扶手楼梯前。那时候我在想,那会不会是你?然后我看了看外套,就知道是你了,因为有种熟悉感。之前我在早上去一些店吃早餐的时候我也会留意(哪个人)会不会是你,但看了看袋子就发现不是你,因为你这一年来都是带这个袋子。}
\dialogue{咨询师}{那你在地铁站看到我的时候,你会想到些什么吗?或是有什么感受吗?}
\dialogue{我}{我会感觉你很渺小,就是人群里的其中一个渺小的人。然后我也会感到很渺小,因为自己好像无法与你连接,无法与那时候那个被人群所包围的你连接。会有一种无力感,觉得自己无法依靠另一个人,无法与对方建立连接,只有自己一个人在。\\
    我可能会期待你呈现的样子是很独特,格格不入但独一无二。不过我想这种期待更像是我对自己的期待。}
\dialogue{咨询师}{我记得之前的咨询你也会用到‘格格不入’(一词),但这次你好像会和独一无二放在一起。}
\dialogue{我}{嗯,因为凡事都不是全好全坏。有好的一面,也有坏的一面。}
\dialoguesepline{咨询师}{短暂的沉默,我开始将手里刚刚擦完眼镜的纸巾无意识地不断地揉搓}
\dialogue{我}{其实那种无力感会让我想起昨晚我看着看着书就想去搜一些关于考研的信息。考研一直是我的焦虑源,每当提起考研的时候我就会心跳加速、头脑发胀、头晕头痛,还会过度思考。我昨晚就是想到,如果我想去考研的话,我要报考研班,那钱怎么办?如果我之后不工作了,呆在家里备考可行吗?我要考本硕还是专硕?……我的大脑会停不下来思考,然后又会因为过度思考而头晕头痛。\\
    以前我看待焦虑的含义就是‘对未来的恐惧’,但现在我对焦虑有另一层看法:一方面,一部分的我很渴望去改变生活、去开启新的生活方式、去探索新的世界,另一方面,一部分的我害怕现在稳定的生活会被破坏,害怕失去现在这份稳定的安全感。然后这两部分的自己就开始抗衡、开始内耗。我见过有的人难得度过了环境对他的打击后,他就开始用“不知行合一”\pozhehao{}明明知道自己应该那么做但又做不到\pozhehao{}来自我攻击。我会觉得蛮心疼的。所以当我身处于相似的境地,当我发现自己在自我攻击的时候,我就会有意识地不让自己自然而然地进入这个自我攻击过程。而且这种自我攻击更像是一种自我保护,毕竟把自己先打击一轮,那么就不需要等待外界的打击了,毕竟自己都已经自我攻击过一遍了。\\
    现在当我发现自己在焦虑的时候,比如说过度思考或用手无意识地在抠脸上的毛孔的时候,我就会跟自己说‘噢,我的头脑开始停不下来。’,然后在意识到这一点后,试图让自己往后退一步地去看我能怎么办,或者转移自己的注意力,比如说看一看窗外的风景。但当自己真的处于焦虑的时候,虽然我能意识到自己内心的各个部分是什么,能意识到他们的相互抗衡,能意识到自己的身理反应和无意识行为,但我依然觉得很难耐受住那种焦虑的状态,会很容易觉得很累很困。}
\dialogue{咨询师}{好像考研这件事情对你来说真的很恐惧。}
\dialogue{我}{我会想到我能报一个考研班,这样就不会有孤军奋战的感觉,就不会有那么强烈的无力感。我也想到我可以找另一个有考研方面的资源的朋友。那个朋友在以前有劝我要不要和他一起去考研,我说不了,因为那时候我还对考研很焦虑很焦虑。不过现在我想做出改变了,但我不想找他,因为他会给我带来很大的焦虑感。以前当我开始就业的时候,大概是大四的时候,他会跟我说很多找工作方面的信息,但他越说我越焦虑。我想到,如果现在我去找他要这方面的资源的话,我会更加焦虑。我已经有很大的焦虑了,我不能再承受多一份焦虑。我宁愿先以我自己的方式,去寻找一个与他人建立联系、获得支持的方式。}
\dialogue{咨询师}{好像你会很强调先以自己的方式去这样做、去获得支持。而且也会幻想一个离你很远的群体是能给你支持和安全感的,但反而身边的一个能把你的诉求转化为资源的人没有办法给你这些。}
\dialogue{我}{可能因为之前有一个朋友也是这样。那时候我想和他聚,但聚在一起的时候,他反而开始倾倒他自己的焦虑感, 比如跟我说我现在的生活是不行的,要改变,比如说他那时候就开始去健身了。我知道这不是我想要的生活,但我改变不了啊!而他只是通过投射他自己的焦虑来使他自己没那么焦虑。所以在和他聚完之后我就回家躺在床上,在被子里裹成一团地抱着被子睡了过去。\\
    当我想从其他人身上获得一些陪伴或支持或抱持或依靠的时候,对方总是没有办法给我这些东西,所以好像这样的经历也加重了那种无力感。\\
    其实我并不擅长与人打交道,只是学咨询后开始越来越擅长甚至精通。就好像我终于找到了自己擅长的东西,知道自己是个怎样的人。这好像才给了我力量去面对一些以前会感到很焦虑的事情。}
\dialogue{咨询师}{这好像也贡献了那种无力感。}
\dialogue{我}{嗯。}
\dialogue{咨询师}{那种无力感会给你一种怎样的形象吗?}
\dialogue{我}{形象?}
\dialogue{咨询师}{嗯。}
\dialogue{我}{好像会像是一个婴儿。要靠自己的力量去获得自己想要的东西,如果不去这么做的话,其他人就不会看见自己、不会给我我想要的东西。觉得很无力。}
\dialogue{咨询师}{我听起来会蛮难过的,就好像一个婴儿需要独自面对很多很多的事情和遭遇。那你会具体地设想你恐惧的或者是觉得无力的会是一些怎样的情况吗?}
\dialogue{我}{我有设想过,但我没有想到。我会设想万一考研失败了、辞职后只能呆在家里准备考研。关于现实的设想并不会让我感觉有多恐惧和无力。但那种无力感一直都在那里。所以有时候比起现实,我更加恐惧那种无力感。\\
    以前的我会很无意识地进入一种过程,比如说我想到考研,就很恐惧,很焦虑,很无力,我好累,我好困,我只想睡过去。但现在我会想到,其实我设想不到任何会让自己感到很恐惧或很无力的情况,反而是,我会对那种无力感感到很无力、很恐惧。}
\dialogue{咨询师}{好像你现在能有意识地不让自己进入那种无意识的状态。}
\dialogue{我}{嗯。}
\dialogue{咨询师}{那种无力感会说些什么吗?}
\dialogue{我}{那种无力感会说,我不能靠他人,我只能靠我自己。然后当我只能靠我自己的时候,我又会感到很累。我会回想起这种无力感的源头。那是小时候在儿童医院的门口,因为我小时候老是生病,那时候儿童医院门口总会有小摊贩卖玩具,而我又总是很想买新的玩具。但每次我想买的时候我身边的大人、我爸妈都会说没钱。}
\dialogue{咨询师}{他们就真的只是说没钱吗?}
\dialogue{我}{我不太记得那时候他们具体说的是什么话,但大概的意思就是没钱。}
\dialoguesepline{咨询师}{咨询师点了点头}
\dialogue{我}{然后一次又一次这样的受挫的经历后,我每次看见小摊贩卖的玩具,就会跟自己说,我买不起的,我穷,我不会获得自己想要的东西的,别尝试了。所以后来我经过小摊贩的时候都不会去看玩具,然后那时候的大人就会夸我懂事。}
\dialogue{咨询师}{那时候你会有怎样的感觉?}
\dialogue{我}{我会感到愤怒、悲伤、难过、被辜负、被打击。好像我也因此丧失了很多乐趣,因为没有办法获得外界一些我想要的东西。}
\dialogue{咨询师}{好像你不只是丧失了玩具这一件事。你还会想到其他什么事情吗?}
\dialogue{我}{我会想到,其实我以前是个左撇子,但是上小学的时候被班主任和家里人说,所以后来我就变成右撇子了。我现在的左手写不出右手写出来的字。觉得丧失了自己的一部分,丧失了那充满可能性的未来。}
\dialoguesepline{咨询师}{咨询师点了点头}
\dialogue{我}{这种无力感让我想到当我搬出前任公寓时,距离现在都快三年前了,那时候我一个人拖着行李走在公寓小区路上的时候,那种无力感。}
\dialogue{咨询师}{其实这个场景、这种无力感也在之前的咨询里出现过很多次。}
\dialogue{我}{我感觉自己好像走在悬崖边,我觉得自己很危险,但我还要保持着平衡继续走下去。}
\dialoguesepline{咨询师}{一段沉默}
\dialogue{我}{我发现我的大脑停了下来。我的大脑好像僵住了,就停在了那里。就好像战斗或逃跑的状态,现在我的状态好像就是僵住了。我动不了了。我在悬崖边僵住了,走不下去了,动弹不得了。}
\dialogue{咨询师}{其实之前你提到你小时候被打的时候也是僵住的状态。}
\dialogue{我}{嗯,是的。就是僵住了……就好像面前有一块大石,我被挡住了,我跨不过去,看不见那后面会是些什么东西。但我依然想去看那后面有什么东西,因为如果我能看到的话,说不定我就能有更多的灵活性了,能更灵活地应对生活的改变。}
\dialogue{咨询师}{也许过去的这些经历真的给你留下了很大的创伤。也许在未来的某一刻,或者是在未来发生些什么,会让你想到一些事情,会让这些僵住的东西松动一点。}
\dialogue{我}{(叹气)也许吧。}
}

结束咨询后,我意识到自己超过了情感阈值,或者说一直保持在很高的情感强度以至于到了应激状态,以至于大脑停止了部分的功能,停止了思考和感知情绪。但我依然不可能控制自己的状态。

在过了半天后,当状态逐渐恢复过来后,我试图将焦点不放在现实世界里,不放在现实世界有多么危险、多么无力,因为好像每次想到现实情况的时候,我都是在无意识地安慰自己:这并不是现实,现实并没有自己所设想的那么可怕、那么令我恐惧、那么令我感到无力。就像是一个父母安慰自己的孩子,黑暗里并没有怪物,这不是真的。

% 自我探索 | 1

我试图把焦点放在主观世界里、情感世界里,试图去挖掘这种无力感,去体验这种无力感。这种无力感会让我再次“回到”那个无尽黑暗楼梯间的梦境\pozhehao{}永远没有出路,还有小时候被我妈赶出家门的画面\pozhehao{}坐在台阶上在门外敲打着铁门哭泣着。我突然将这两者联系在了一起,梦里在无尽黑暗楼梯间里的那种无力感和被我妈赶出家门的感觉是那么的相似,而父母家门外、那个我被赶出去的家门外就是一个楼梯间。被遗弃的无力感。这就是从前任公寓搬出来时,走在公寓小区路上的感觉。这种感觉一直贯穿着我过往的无数经历。

被赶出家门的时候,我有想过,要是我走出去呢?我会活不下去的,我会生存不下去的,我会死掉,我好害怕。所以我没有走出去,没有像梦境里的那样不断寻找出路并最终走出去,而是僵硬地凝固在了楼梯间的台阶上,我什么也做不了了,什么也改变不了,没有人会来开门,没有人会发现我,没有人会要我。我想走出去外面,但我又害怕外面的黑夜。我想死,但我又害怕死。我只能这样了,只能在这里了。


