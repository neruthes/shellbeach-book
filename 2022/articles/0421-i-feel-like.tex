\chapter{“我觉得……”}

\ardate{2022-04-21}{h9fa\_LgoDCfgwS\_-1EuVRg}




在上周的课程作业里,我和其他两位课程同学一起进行了三界觉察的练习。在练习时,我发现了其中一个课程同学在表达觉察上的习惯或模式,而这种模式对于每个人而言(包括我)相信也并不少见,那就是:“我觉得……”。

每当ta说“我觉得……”时,我就下意识想把之后表达的觉察内容记录在内界(身体感觉、当下感知、言语表达和身体感受)里,但紧跟“我觉得”之后的往往是中界的内容(逻辑层面的思考、分析、推测、推理、想象和记忆),比如说:“我觉得白色灯塔先生在不耐烦。”

当时我给ta的反馈是:也许你可以用“我感到……(内界内容)”或“我认为……(中界内容)”。但当我在个人练习的时候,我同样也会说:“我觉得还有一段时间。”而在日常生活里,我也同样会表达:“我觉得他并不在乎我的感受和想法。”当我在说“我觉得”时,我在那个当下的感觉(对感觉的感觉\pozhehao{}心智化)更像是一种直觉般的知识或内容。而这让我猜想到,或许这种看似(或更应该说是感似)直觉般的洞见正是内隐心智化。

\blockquote{
我们可以将内隐心智化解释为直觉,Lieberman将它定义为“一种现象和行为,与我们通过内隐学习获得的知识息息相关”,这其中包括“人们在靠近其他社会目标或是情境时产生的感受、判断或是预感,这些体验出现的时候往往缺少一种可以被清晰表达的理由”。内隐学习建立在重复暴露于与奖赏有关的刺激的模式基础上,在它出现的时候,往往不会带有觉察,也不会就所学到的东西产生外显的知识,对于“如何”,所能学到的东西就更少了。直觉,就建立在这种内隐学习的基础上,它是我们对于非言语情绪交流进行恰当反应的能力根基,而大部分这类反应往往在发生的时候脱离了外显知觉。

……很难在内隐心智化与外显心智化之间划定一条清晰明确的界限,就像我们很难清晰地区分什么是直觉和自动化的,什么又是有意识省映性的。……当我们进行心智化的时候,我们不断地在更内隐的加工和更外显的加工这两者之间来回摇摆。通常来说,如果事情进展顺利,我们并不需要对此进行解释。只有当事情出错的时候,外显心智化才会发挥它的作用一比如当我们对自己或是他人的行为感到困惑时,就会扪心自问:“我当时怎么会那么做?”“她到底是怎么想的?”

……在心理的领域中,内隐心智化同样是一个意识过程,只不过它反映的是更低水平的意识,即基本觉知。相反,外显心智化则体现了更高阶段的意识,即体现出了自我觉知。更高阶段的意识让我们可以应对新奇的事物,同时也让我们可以更为灵活地进行问题解决。

\citebook{心智化临床实践}
}

当我说“我觉得……”时,我在将内隐心智化的内容表达出来(外显化出来),将内隐的东西放在台面上,带入有意识的觉察当中\pozhehao{}我觉察到了我觉得……。这样的外显化内隐心智化无疑体现了更高阶段的意识\pozhehao{}指有别于其他哺乳动物的属于人类的自我觉知,也能让我们更为灵活地判断这样的内隐心智化(这样的“我觉得”)是否符合现实,抑或者只是来源于我们自身的部分。

比如说“我觉得白色灯塔先生在不耐烦”的背后是否是当事人(“我”)自己觉得不耐烦,而又防御性地将这种不耐烦投射/分裂了出去、投射/分裂给了对方。如果我真的觉得不耐烦呢?这种不耐烦是否是属于我自己的部分,抑或者我只是在投射性认同对方?如果是的话,彼此之间的动力又是怎样的?

从此之后,每当我听到“我觉得”时,我便会猜想对方是否是在外显化内隐心智化。当然,这篇文章的整个过程也是在外显心智化的其中一部分。

同时,我也会想起一句曾经蛮常听到的话:“我不要你觉得,我要我觉得”。如果将这句话里省略的内容写出来的话,那可能会是“我不要你觉得(的主观现实),我要我觉得(的主观现实)”。这句话当中凸显了各自的主观现实中的差异,有时候这种差异甚至是个鸿沟。所以紧接“我觉得……”之后表达的内容不仅仅包含着内隐心智化的内容,这一内容背后更暗含着每个人独一无二的ta的主观现实。比如说学习能力强的学生在面对某道题目时可能会说:“我觉得这道题蛮简单的”,而学习能力差的学生在面对同一道题目时可能会说:“我觉得这道题挺难的。”

同时,我也会想起在接电话热线的时候,我偶尔会听到来电者表达ta觉得我并不理解ta,这时候我就会问:“那会是因为我具体的哪一个回应让你感到不被理解吗”(试图鼓励对方外显化内隐心智化),以及试图进一步探索对方的这种内隐心智化背后是否可能存在着一个更庞大的主观现实。如果是的话,那又会是一个怎样的主观现实\pozhehao{}比如说在对方的日常生活里充满着不被理解的经历和体验。


