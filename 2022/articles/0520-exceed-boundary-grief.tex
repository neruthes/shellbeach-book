\chapter{冲破界限,悲伤,幻灭}

\ardate{2022-05-20}{UWqKmvaMm3IurGX9fajtfQ}




前几天,我和一个很久没见的男生见了个面。和他见面的整个过程中,甚至是在见完面后的一整天里,我整个人的状态都很糟糕。之前,我在他的微信昵称后备注上“分离”(我的一个习惯:在微信名后备注上我对那个微信好友的刻板印象或议题),因为他会触发到我关于分离的议题。见面时,我的感觉依然和以前一样,我感觉我看不透他,感觉他隐藏了很多很多东西。这种感觉让我想到,三年前和前任在一起的那段时间里,我也有这种看不透对方的感觉。这种看不透的感觉(现在我意识到这是我在心理咨询里时刻会从咨询师身上感受到的界限感)很大程度上影响了我和那个男生、和前任的相处。

三年前和前任相处时,我一直试图冲破这一界限感,但我的“冲破界限”行为在他看来是一种对他的隐私的侵犯,比如说偷看他的手机屏幕。这种“冲破界限”一定程度加剧了两人关系的破损。而和那个男生相处时,这种“突破界限”也给他带来了一定程度的不适。虽然我能通过自己的方式来“冲破”对方的界限,能发现不少对方不想让我知道的部分,甚至是在我看来属于背叛的部分的存在,但那种界限感依然没有减弱,甚至有所加强,就好像:正因为我看见了那些对方不想让我看见的部分,我越来越确信对方是有所隐藏的,那个对方所隐藏的部分很庞大,我看到的只是冰山一角,而我需要、我想要看到更多。因此这种“冲破界限”的行为后来也逐渐强化。

在前几天和那个男生见面时,我一直很小心自己的提问会不会触碰到对方的警戒感。我有很多提问想问,比如说想问对方的对象是怎样的、对方的生活是怎样的、他在乎的是什么、他在乎的人是怎样的、他在乎我们彼此的关系的程度、他之前为什么消失了那么久。但另一方面,我又知道对方是不会回答这些问题的,因为在见面的过程中我也有用过一些细节来试探对方表露程度的广度和深度,而他的表露程度一直只停留在广度,而不会回答有所走深的提问,甚至会用一些话来转移话题,回避掉走深的提问。

见完面后,我意识到自己对他人边界的敏感并不一定是一件好事,尤其是当我无意识地感知到对方的边界时,我会无意识地做各种事情去试图突破对方的边界。在之前,这种对对方的界限感一直驱使着我去找各种办法去突破对方的边界,想看到更多对方的部分,尤其是那些对方不想让我看到的部分。而在和我的心理咨询师的互动里,我开始看见这种试图突破对方的边界的行为是来源于一种渴望感,渴望看见对方更多的部分、渴望探索、渴望更深入的交流、渴望两个灵魂的相互交汇。

但我依然不知道这种对他人边界的敏感这一特质是从何而来的。

在那个男生见完面后的一整天里,我都感到很悲伤。冥想时,我追随着悲伤的感觉,想到,我一直都不是他们家的一部分,不是那个男生家的一部分、不是前任公寓家的一部分、不是父母家的一部分。我想起小时候被父母赶出家门的画面,被cast out。他们只是想要一个他们想要的孩子,一个能时刻满足他们的期待和要求的没有自我意志的培养品,他们想要的不是我。我从来都不是任何人的家的一部分,同时我又渴望成为任何人的家的一部分,好像如果有哪一次自己真的成为了另一个人的家的一部分的话,过往经历里的那些悲伤、难过、被遗弃感、丧失感、绝望都会统统被抹去,替换成家的温暖感和被爱感。

但,自己眼中的家一直没有存在过,而之前的我只是将自己心目中的家的形象投射到不同人身上、投射到他们的家身上。那个家的形象,那份温暖感和被爱感从来都不是真实的,只是我幻想出来的,幻想一个温暖的家、充满着被爱的家来给予自己温暖感和被爱感。对方的存在、对方的家的存在就像是一个载体,而之前的我所感知到的所有内容\pozhehao{}对他们家的形象和对他们的形象以及从中感受到的温暖感和被爱感统统都来源于我自身。

而和前任的见面、和那个男生的见面,一次又一次地证实了自己的幻灭\pozhehao{}我从来都不是他们的家、他们的生活里的一部分,被cast out。但更残酷的事实可能是,之所以会被cast out,是因为我从头到尾拥有过的,只是一个幻想\pozhehao{}幻想中的家、幻想中的生活、幻想中的对方。

