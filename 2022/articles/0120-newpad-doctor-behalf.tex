\chapter{随笔 | 换平板,老中医,“替”}

\ardate{2022-01-20}{uhVOxgl7h0R4B-Ixj4CFmA}

这个月初,我想换一部安卓平板电脑,原因是在元旦假期到另一座城市玩时,我用 iPad 的办公效率让我想砸东西,而元旦假期没有上的课程最终也导致我在那周的某个工作日下班后连续上了6小时带1.25倍速的课程。

当我将想换平板的想法告诉一个朋友后, 他一如既往地给我想好了接下来的流程:去回收旧平板—拿回收的钱加多一点点钱就能换一部新的—哪里有回收店\pozhehao{}新平板买哪个型号比较好。一开始我是蛮兴奋和充满期待的,直到我发现旧平板的回收价只有三百块时,他说:“所以我都是用了半年至一年就马上换新的,就是为了防止你这种情况”时,我的第一反应不是想砸东西,而是想用东西砸他。

当回顾这件事情的过程时,我回想之前早就有不少类似的情况:比如说我想换手机时,他也会替我想好一个他认为最好、最划算的方式来达成我的想法;比如说我想在手机上实现什么功能时,他会帮我找好几个方法。不仅如此,他的语气里还带有既催促又鼓励的语气\pozhehao{}carrot and stick。如果说他所提的方法里有任何共同点,那便是\pozhehao{}它们对我而言都是不适用的。

我想起几个月前和一个课程同学的倾诉练习时,对方说到她很想马上作出某个改变,但却做不到。我对此提出的解释是:“我能不能这样理解:在理性层面上,你知道这个模式的转变是需要一定的过程的,但在情感层面,你依然很想马上转入一个新的人际互动模式里,很想不再经历同样的事情。”她在那个当下认同了这个解释,然后说要用拳头喜剧性地“打”我。在结束倾诉练习后,我问她为什么在那个当下会想“打”我。她说她想起小时候去看中医时,开药的老中医会跟她说只要坚持吃药,病就会好。所以她特别讨厌那些老中医都说这样的话。我对此作出的回应是:“好像一方面,他们告诉了你你接下来应该怎么走,但另一方面,他们好像没有陪你走下去。”她也立即认可了这个解释。

我会想起这个几个月前的倾诉练习是因为,那个朋友就像是倾诉练习同学口中的老中医:只是指了个方向,但并没有陪我走下去。当我还没有动力和勇气去按他所设想的方向走时,他会push着我或拿他所设想的结果来“诱惑”我踏出脚步。而当我走着走着发现前面有个障碍时,他就马上躲开了,并说:还好他自己怎样怎样。

这应该也是为什么最近学的课程内容里几乎一致强调不要替对方做决定(特别是有关改变的决定):如果成功的话,对方很可能产生依赖;如果失败的话,对方很可能产生憎恨。我会认为,这样的“替”,更多是对方自己想要的,而不是我想要的。我更像是对方满足自我效能感、自恋的工具。

