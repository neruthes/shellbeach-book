\chapter{零碎的想法 | 32}

\ardate{2022-02-07}{jliR4AMemlljiUyh3AfTdg}



\section*{1}

在春节和前任见面前,我回想起两年半前在和前任的关系濒临破裂时,他曾说他还没有爱上我。那时候的我会感到很伤心,并自责为什么自己那么快就那么深地爱上了他。但当春节和前任见面之前,当那时候的我回想起这件事时,我反而会替他感到惋惜,惋惜于他在那段短暂的亲密关系里并没有我所感受到的那份曾经处于热恋的爱意,惋惜于他没有好好地体验那段短暂的彼此的共同人生历程。


\section*{2}

但我依然想和他有更深的关系,所以我说我在和他分手后的这两年半以来见过很多人,但都觉得(和他们的相处)蛮无聊,不过和他聊天不无聊。他说,我感到无聊是因为我和他人的互动方式以及我自己都没有怎么变化吧?我回答说:“不是的。我自己一直在变,和他人的互动方式也在变。我觉得和你聊天不无聊是因为我‘看见’了你背后的那个人,那个带着漠然、超然的态度对待身边的事物和人甚至是用这一态度对待你自己的那个你,也看到这一态度在你生活中的各方各面的延伸。”他有点不相信地说,难道不无聊不是因为我没有预料到这次的聊天(的内容和进展)吗?我说:“不是的,我在来之前就设想到现在这一步。我之前的那些隐隐约约的想法和感觉都在这次的聊天里慢慢地展开了。但我没有设想到的是你背后的那个带有着漠然、超然的态度的那个人。看见了这个人让我感觉我们之间的距离拉近了。”

\blockquote{
	他问我:“这就足够了吗?”我说:“我会想知道我们之后的关系可以变成怎样?我会对未知充满好奇。”他说我们之间的关系已经没有其他可能性了。两人关系的顶峰就是共同生活,之后就会走向分离和结束,所以还不如在“无聊”之前就停在那里。我说:“如果你已经知道了两人的关系会走向的阶段,那不会让自己难以呆在当下吗,既然你都知道事情会怎么变化?”他说他能呆在当下的状态。

	之后我试过几次试图突破他的自我保护,想要问出他会想要些什么,但每次都问不出来。我感觉他把自己有所欲求的部分隐藏地很深,而且他还不断问我:“这就足够了吗?”同时,我也感觉他在用这个提问促使我走得更深,或结束这次的聊天。我说我脑海里有一幅画面:一个观察者的画面。你在这副身体的背后一直观察着事情的走势。他说我如果这么觉得就这么觉得。

	\blockquotesource{这就足够了吗?}{白色灯塔先生}{2022}
}

自从春节和前任见面后,我发现自己又在无意识地内化他的形象。在春节时,我给自己安排了两天的居家自学网课。当因天气很冷而懒得起床,或者是因为学得很累而想睡觉休息时,我脑海里会响起一个声音:“这就足够了吗?”我马上意识到这是春节和前任见面时他重复过好几次的话语,现在这个话语开始在我的脑海里“扎根”了。

“这就足够了吗”似乎带出了自己更具有欲求的那部分自我,我会在想:自己在学习课程方面的努力真的足够了吗?自己在经济能力方面真的足够了吗?自己的自我成长和能力方面真的尽自己全力了吗?我真的尽力地去尝试触碰前任内心深处可能还存在着的那个有着欲求,甚至可能欲求着我的自我了吗?而我每次对此的回答都是:并不足够。


\section*{3}

最近在学一门关于心智化的课程,我开始意识到,在这大半年来的心理咨询历程里,我一直在培养自己的心智化能力,以及在亲密关系当中的能力。我之所以开始意识到这一点,不仅仅是因为课程里有关心智化的知识,还是因为在春节遇到了一个有好感的男生,而我能够在和他的交流里不断走深,不断感受到更深的亲密感。我知道自己有能力和另一个人建立更为深入的亲密关系,而且这一能力也早已逐渐变强。

我会回想起以前在咨询历程里的梦境、和咨询师的对峙、跟咨询师提的建议,那些一次又一次的事件都在逐渐培养我建立亲密关系的能力,毕竟我在现实生活里根本没有这样的人能建立起亲密关系。

在那个关于心智化的课程里,讲师说来访者来找咨询师就是为了让咨询师失败,来访者从而能从咨询师如何处理TA自己的失败当中学习如何处理来访者自己的失败。但在我的视角里的咨询历程中,我去找咨询师的过程反而通过让咨询师经历失败,从而能够让我在咨询师的失败里学习如何处理咨询师的失败。当能够处理对方的失败时,我也不再担心自己的人际失败,也更能勇敢地去面对咨询室外那些人际关系里隐而未说的事物,面对更多人际关系的可能性。


\section*{4}

某天躺在床上冥想时,我看见自己在一个井底里。我抱着双膝坐在井底,井底在不断下沉。井底外面是现实世界里各种画面的闪过,而自己所在的井底在慢慢下沉。在不断下沉的过程中,我开始能看见地面上还有很多井底,每个井底里都坐着一个人。那些人都在很浅的位置,但自己一直往更深处的黑暗里沉。我开始离其他的井底越来越远,他们开始离开了自己的视野,消失在了黑暗里。我还看见在我旁边还有一个井底也下沉得很深,井底里的那个人试图跟上我下沉的速度,但还比我浅一点。

我看着他下沉的速度越来越慢,我想用手穿过井壁去触碰他,但我穿不过去。我看着他的身影,很担心他最终会跟不上我下沉的速度,我担心自己会一个人在黑暗里继续下沉。我想大喊:“不要!”

最后这种想要大喊的渴望将自己从冥想的状态“唤醒”了过来。在回顾这个意象时,我意识到那个我想要触碰的在另一个井底的人似乎象征着前任。


\section*{5}

\blockquote{
	大概在半个月或一个月前,我做了一个梦:我梦见自己在一辆公交上,这辆公交开在一个日落下的山坡小路上。这部公交要开往一个很远的地方,因为这条线路是由一个机构的总部开向一个已经完成了的任务的地点。这条公交线路的目的就是为了隐藏那个任务地点曾经的存在。

	我是这个机构里的工作人员,而坐在我旁边的男生也是我的同事,车上的人都是我们的同事。我坐在公交车靠前的左侧窗边,而他坐在我旁边靠走廊的座位。我在其他座位的遮挡下牵着他的手,我们的手放在了他的左侧大腿上。我记得他的手是修长且铜色的,但我不太记得他的外貌。公交开到一半时,我和他下车去清理路障。在把路障清理好时,我醒了过来。

	睡醒后,当我回想起和他牵着手的那个时刻,我感到很幸福,一种我很久都没有在现实世界里感受到的幸福感。

	在前几天睡醒时,我发现我做了另一个梦:我梦见自己在收拾一个场地,在将场地的桌椅收拾完后,我发现这里变成了高中教室。我在教室里见到了一个大学同学,也遇到了一个高中女同学。那个女同学的样貌变丑了(在我的记忆里,她高中时的外貌还蛮好看的),而且她怀孕了,肚子表面是一凹一凸的骨架的形状,就好像她怀上了一个畸形的孩子。

	然后我梦见自己坐着公交,窗外是一条沿海小路,车上都是同学,因为我们刚从教室里出来坐上这趟车。我看了看手机,想到现在已经坐了近半个小时的车,但还没有到站,我想到我快要迟到了。我坐在公交车靠前的左侧窗边,坐在我旁边靠走廊座位的是一个高中男同学,他在高中时的女朋友是那个刚刚在教室里遇见的那个女同学,那个怀着一个畸形婴儿的女同学。那个高中男同学的身材是瘦高的,皮肤是铜色的。我把右手靠在他的肩膀上,他的左手牵着我靠在他右边肩膀上的右手。我感到很平静,一点都不需要担心迟到的事情。

	睡醒后,我回想起牵着他的手的那种感觉,那种感觉就是上一个梦境的那种幸福感,原来我在上一个梦境里牵着手的人是他。

	\blockquotesource{零碎的想法 | 31}{白色灯塔先生}{2022}
}

昨晚,我又梦到了那个高中男同学。

在梦里,他打算去一个地方,但在去之前他想先去一条路逛逛。我想到时间可能不够,但没有告诉他。后来我们一起去了那条路上的一个书店,那是我之前实习过的书店。我们一起看了看书店里的活动,我还和前同事聊了聊天。后来,我们一起离开书店,走到附近的公交站。他发现时间不够去下一个地方了,那个他一开始就打算去的地方。他很失落,我也感到有点内疚,内疚于没有提前告诉他时间不够。但我内心有一部分是感到庆幸的,庆幸于我和他的关系还没有走得太深。

然后我就醒了过来。


\section*{6}

随着生理年龄的增加,我发现自己的性欲越来越强,强得开始影响我的感受和认知。

当在大街上看见好看的男生时,我的脑海里会激起性幻想。但更影响我的是,我发现这种性欲开始影响我看待身边的他人的感觉和视角。比如说,我在最近的面基时很容易喜欢上对方,而我也开始警惕这种喜欢是否仅仅是单纯的性吸引力,而没有更深层次的喜欢。

在今天和面基的男生说起这件事时,他说我可以找个人have sex。我说我并不想只是单纯地have sex,而是想和自己喜欢的人享受性的快乐和幸福,但我身边还没有一个能这样做的人,所以这种欲望无处宣泄。那个男生说他回答不了我的问题。我继续说:“我感觉我在被自己的身体驱使着去做一些我并不想做的事情。”

