\chapter{小组练习,三界觉察}

\ardate{2022-04-17}{Hu8P1LJM4vxk8c3L52odtw}





\midnote{以下来源于我的本周课程作业(经修订)}

\noindent\textsf{小组作业:\\
找伙伴,尝试一个人、两个人、三个人的三界觉察练习。并在平时多加练习,全面提升自己的觉察力。练习结束后请写下你的感受。\\
三界分别是外界、内界和中界。外界指的是感觉通道的视、听、触、嗅和味,内界指的是身体感觉、当下感知、言语表达和身体感受,中界指的是逻辑层面的思考、分析、推测、推理、想象和记忆。}


这次的作业,我和课程同学A和B一起练习三界觉察。在一开始的互相了解环节,我们了解到了彼此的性别、年龄、阅历、学习经历都很不同。

在各自的5分钟三界觉察练习并在记录员的记录和反馈下,我发现自己的三界觉察比较平均,外中内界分别为12、17和16\~17次。而课程同学A和B则有更偏向于中界的,或更偏向于外界的。

另一点让我感到很好奇的是,她们在偏向其中一界的时候,另外的那两界也呈现出一种倍数阶梯式的变化,比如说外界是其他两界的两倍、中界是外界的两倍而外界又是内界的两倍。我会在想,这背后是否也暗示着某种觉察脉络。不过,仅仅通过三界觉察的比例来估计各界的脉络会过于片面,但我依然会有这方面的好奇。

在自己练习完三界觉察后,三界觉察的过程的感觉非常像是在冥想时的感觉\pozhehao{}时而流动,时而僵住,时而跟着情感或思绪跑,时而把注意力拉回到觉察上。当然也有不像之处,比如说当把脑海里的感受和想法说出来时,反而更容易延伸出新的内容,而不是像冥想时仅仅让这些事物“飘过”。

和小组里的课程同学A和B一起练习时,我发现这有一个好处:正因为彼此之间的差异,每个人反而更能看见对方ta自己平时难以觉察到的一些习惯或模式。

我想到,可能三界觉察练习只有一次的记录是很有限的,所以我一个人练习了第二次,同时通过录音来试图发现一些自己的模式或习惯。第二次练习的外中内界分别为20、17和14次,而我也隐约知道在这次的练习里自己很少谈及内在的感受和情感。我发现我经常会在留意到外界(比如说鸟鸣)或内界(比如说紧张感)时会去思考中界的内容,比如说现在的鸟鸣和清晨的鸟鸣的区别、为什么自己会感到紧张。同时有一些话也让我感到难以区分这到底是中界还是内界还是介乎于两者之间,或者更像是两界的整合,比如说:

“我感觉到我的双脚又在晃(内界),\\
(我看到)我所坐的椅子在晃(外界),\\
我感觉在我晃的时候(外内界\pozhehao{}看着自己晃和感觉自己在晃),我的无聊感或者是焦虑感变少了(内界),我好像能更自然而然的诉说(中界)。”

同时,在回头听自己的录音的时候,我感觉到声音另一头的人(几分钟前的自己)会给我一种很稳定、安然于此时此地的感觉。虽然TA的思绪会有所飘离、情绪会有所波动,但最终还是会回归到一种平均感,就像是泛起的涟漪最终又回归到平静。

