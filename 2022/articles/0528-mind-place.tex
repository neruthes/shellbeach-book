\chapter{“自我意识究竟是什么,难道只是情绪和想法和念头飘过的地方而已吗”}

\ardate{2022-05-28}{vVNhMvRlrCRmPj1ONyUFJg}



\dialoguelist{咨询师}{
	\dialogue{我}{其实上次咨询结束后,我的那个你所说的资深咨询师看了上次的回忆稿,然后跟我说可以去了解下“嫉羨”(看了看手机上的发音),我的中文不太好。我去搜了下,嫉羨是二元关系里的那种羡慕与嫉妒,而且最近我也在看那本书,了解到嫉羨其实也是人类共有的一种情感之一。我会感觉到它所描述的那种情感和我有相似之处,但又不完全一样。它的理论场景是婴儿认为母亲不给自己奶水,是个坏乳房,所以想要占领和毁灭那个坏乳房。我会想占领自己更为亲近的对方,但我并不想毁灭对方,只是想占领而已。
		我还蛮喜欢有人能提一个词,比如说亲密关系,然后我就能去刷好几本这方面的书,去探索。因为我无法了解自己所不知道的领域,我也去不到我所不知道的地方。我会想赶在这次咨询之前把书看完,但后来还是没有,还是顺其自然吧。
		所以我提出来是想看看你对此会不会有什么想法或念头。}
	\dialogue{咨询师}{为什么你会将回忆稿给那位咨询师看?会是想通过这种方式得到一个解释吗?}
	\dialogue{我}{Em……不完全是。因为并不是每次咨询都会留下一个疑问,所以也不是每次结束咨询后都会想去得到一个解释。我是会写下来,然后放在一个自己的空间里,想看的人随时都能看见。身边有的朋友会选择不看,因为我写的东西会触发到他们很强烈的情感或者是让他们感到不适;有的朋友会去看,然后私聊或线下和我交流。}
	\dialogue{咨询师}{那为什么你会想这么做?在写下来的时候你会对我们咨询的过程做怎样的处理吗?因为我不知道是以一段文字的方式还是对话的方式。}
	\dialogue{我}{我会隐去你的个人信息。之前我会习惯写下来,不过我也会想为什么我写下来之后还会想让其他人看见、还会想发出来。可能是因为我能通过这种方式和他人保持一种更深层的连接。如果我不写出来并敞开给其他人看的话,日常生活的我就是现在这个样子了,和其他人没什么区别。但如果我将自己内心更深层的事物敞开的话,起码那些想看的人能够看得见。}
	\dialogue{咨询师}{那些看过你写的东西的读者和那位咨询师的互动里,你会有怎样的感觉吗?}
	\dialogue{我}{Em……好像每个人都不同。有的人看了会说“别想那么多”或者是“清醒是很痛苦的”,然后我就会觉得很愤怒和不被理解。那个咨询师的话,之前ta的留言里会去分析我的言行举止里背后的事物,之后我有向ta提出了我的不适,我说我不喜欢别人分析我,所以后来ta的评论里都是看到我写的东西后一些ta自己的感受、想法和联想,这我感觉还好。上周那个我有好感的男生和他约饭的时候,他会看完我写的东西后和我线下讨论,然后会问我很多为什么会这么做为什么为什么为什么,然后我再解释,体验还好吧。}
	\dialogue{咨询师}{我会想起上一次咨询,你会想去占据对方的内心空间。但这次好像你会反过来,想为对方腾出一部分自己的内心空间,让想进来的人进来,和他们产生互动。}
	\dialogue{我}{嗯,是这样的。因为起码我做了我能做的。}
	\dialoguesepline{咨询师}{(沉默)}
	\dialogue{我}{其实我这周一直在脑海里徘徊着一句你在之前的咨询里讲过的话:“这是你的咨询”。我会感觉那背后好像有着不满和愤怒,就好像那句话只说了上半句,下半句可能是:“但这并不完全是你的咨询。”这周会不停想到这句话可能是因为这周接热线的时候,我会遇到有的来电者会做一些热线范围之外的事情,然后我会强调热线的功能和范围是舒缓情绪和梳理困扰,如果你想要做热线以外的事情的话我会愿意倾听,但我真的帮不到你什么。然后那时候其实我就很想对对方说:“这是你的热线”,但我更想说的是“但这不是你的热线”。}
	\dialogue{咨询师}{其实这次我也有说过这句话。}
	\dialogue{我}{哈?“这是你的咨询”这句话吗?}
	\dialogue{咨询师}{嗯。}
	\dialogue{我}{是吗?我印象里好像没有。}
	\dialogue{咨询师}{那当时我说的这句话会对你有什么影响吗?}
	\dialogue{我}{我会退一步,就是之前我会把自己写作的内容带进来咨询,但那些内容对你来说好像很刻意,而且好像你也很难理解我经过梳理和组织后写下来的内容。所以后来我就自由发挥了,就好像我放弃掉了一些在咨询里的控制权。所以我也会想在这次咨询问一问你会对咨询有怎样的期待吗?}
	\dialogue{咨询师}{如果是之前的我说这样的话,我想表达的意思是你可以在咨询的任何时刻表达任何你想表达的内容。}
	\dialogue{我}{噢,这样……我会觉得很奇怪,为什么那时候的我会从中听到了攻击性和愤怒?因为虽然是同一句话,但在不同的人眼里会有不同的含义。}
	\dialogue{咨询师}{嗯,我也会很好奇。}
	\dialogue{我}{好像,这是你的咨询,但……又不是你的。就像是现在我穿的衣服是我的(扯了一下眼前的裤子)、我的手机是我的,但我又感觉这不是我的,没有什么是我的。}
	\dialogue{咨询师}{能解释一下为什么你的衣服、你的手机是你的,但在你看来又不是你的吗?我有点不太理解。}
	\dialogue{我}{嗯。就好像它们在现实世界是我的,但它们在我内心世界里并不是我的。我会回想起自己的成长经历里,好像从来都没有什么是我的。比如说小学的时候,我妈会给我十块或二十块,那时候外婆还会给我一百块,那时候的一百块。但我妈会说外婆给我的钱不是我的,也会收掉过年我收到的红包,因为她说她是需要给别人封红包,所以要拿我的红包补回去。所以从那时候我就感觉没什么是我的,我的钱不是我的,没什么是我的。直到成年后,我的宿舍、校园和写作空间才是我的空间。但一直以来都没有人这样表达过,没有人说什么东西是属于我的,也更加没有人倾听我,说这段时间是属于你的,无论你说什么我都会愿意倾听。这些都没有。}
	\dialogue{咨询师}{听你说到这里,我会感到自己好像在沙发上沉了下去,不知道你的感受是怎么样的。}
	\dialogue{我}{我会感觉到我的鼻子在充血,因为鼻子好像通了一些。我也会感到蛮悲伤的。
		我会想起自己在梦境里通常是没有实体的,有时候甚至会像是灵魂一样在飘。因为在现实世界里我也没有一个对自己的形象。我看见镜子里的自己,但并不感觉那就是自己,我认不出镜子对面的那个人是谁。我认识的一个双向障碍的朋友会在内心里有一个自己的形象,然后看着镜子里的自己就会觉得自己胖了很多,因为他在拿自己内心的形象和镜子里的形象去对比。所以我是知道他在内心里有一个对自己的形象的,但我没有。所以如果有哪一天我毁容了或者是衰老了,我看着镜子也不会有多大反应,因为我好像从一开始就认不出自己的样子。我也经常会感觉不到自己的存在,我也不知道其他人是怎么确定自己的存在的。}
	\dialogue{咨询师}{那你会有怎样的设想吗?其他人是怎么确定他们自己的存在的?}
	\dialogue{我}{Em……我不知道,我想不出来。好像身边的人都是很理所当然地知道自己是存在的,但我而言这种东西并不是理所当然的,而且我也很讨厌有的人会把这样的东西视作理所当然。
		其实你在之前的咨询里会问我,我会担心在你的眼里的我不是独一无二的吗?那时候我说我并不担心,因为我视我自己是独一无二就足够了。但后来我发现我更像是在回避这个部分,我在回避他人眼中的我,因为我并不相信他人眼中的我是独一无二的。就像是这周的工作评估,我会自然地过滤掉那些好的评价,剩下的就只是坏的评价。因为如果只剩下坏的,那自己就不会有失望了。但如果接受了好的,那必然就会有坏的,就会有失望。所以,好像自己也没有在他人眼里看到什么自己的形象。所以我也会觉得自己和他人很不一样、不被理解。}
	\dialogue{咨询师}{我会有个猜想,那些部分你不想让他人看见,所以你也从中看不见你自己?}
	\dialogue{我}{Em……我不太确定,可能并不完全是。当我跟你这么讲的时候,我是能感觉到我自己的存在感的,我能感觉到自己是存在的。但如果我不说、我不写的话,时间久了,我会感觉自我在消融,特别是孤独一人的时候,消融到不复存在,不知道自己是谁。这也是我想脱单的原因之一,因为好像我一直需要另一个人来确定自己的存在,就像是一个个体需要他人才能确信自己是个“个体”。就像是萨特写的那个剧本《他人即地狱》,在里面的其中一个场景是地狱里的人没有镜子,所以他们要用另一个人来作为自己的镜子。比如说,你来看看我,我的脸有没有脏,我要涂口红,你来看看我有没有涂歪。但另一个人会说,那如果我不看你呢?如果我说谎呢?你没有了我这个镜子怎么办?所以他人好像充当着一面镜子的角色,而我又不敢相信他人眼中的镜子,因为每个人都有他们自己的扭曲。所以我不断地说、不断地写也写了那么多年了,但我拥有的也只是一些转瞬即逝的存在感,这种存在感并不稳固,而我需要继续去说、继续去写才能继续保持这种短暂的存在感。而当身边没有他人的时候,我就会把这些东西投射到纸张上,不断地去写,从文字里确定自己的存在。}
	\dialogue{咨询师}{从文字里确定自己的存在,这是一种怎样的感觉?}
	\dialogue{我}{可能是一种真实感吧。因为我能往回看自己写的文字,觉得那就是自己。但时间久了再回头看自己写的东西,又会觉得并不是自己,那种存在感又消失了。
		我也会在想,自我意识究竟是什么,难道只是情绪和想法和念头飘过的地方而已吗?}
	\dialogue{咨询师}{当你说了那么多你的这些感受和想法,我好像比以前更能理解你一点了。我不知道当你说了那么多,你会感觉我有更理解你、更看到你吗?}
	\dialogue{我}{我好像暂时没有这样的感觉,因为我好像一直以来都不期待他人能理解我、能看见我。跟你说了那么多,就好像是写作,我会把属于我自己的部分呈现出来,我做了我能够做的了,至于对方怎么做是对方自己的事情。
		我也不知道这种一直在写、一直在说的方式是否健康。}
	\dialogue{咨询师}{你能多说一说在你看来健康的方式是怎样的吗?}
	\dialogue{我}{可能更应该说是普通,就是身边的普通人是怎样做到的,怎么做到确定自己的存在,怎么自然而然就知道自己就是自己的。}
	\dialogue{咨询师}{我想,我不确定这是否是你的感受,这只是我的猜想:可能是在你的生活里一直都没有一个能够确定你的存在的人在。}
	\dialogue{我}{Em……很可能是。好像母婴关系就是这样,婴儿就是需要通过母亲这面镜子来确定自己的存在是怎样的。而我好像一直以来都没有这样的一个部分在。}
}
