\chapter{很烦,渣男,《房子》}

\ardate{2022-03-19}{UL1uCliah8mRZ6z4wIwyBA}



最近我觉得很烦。特别是想到要心智化自己和他人就很烦,除非那是我想要做的事情。

去理解他人的话语里的潜台词很烦,因为这会让我感到很累。我不想去理解对方究竟想说的是什么,有什么想说的直说不好吗?比如说:“那所以作为一个投标单位,虽然在理解这件事情上可能很难做好。但这不并不排斥这个投标单位,在亲近感上面可以是一个供应商。就要这么去理解吗?”“一方面是这种所谓的分析错了,你会不开心。另外一个就是,就算分析对了呢?那么这就是会是你欢迎的吗?如果你欢迎别人来做合理的分析,那么我会是这个列表中的其中一员吗?我要去过早的建立起这种信心吗?我觉得还是放松一点好。至少在有更明确的欢迎信号之后再做。”“这个事情跟我的理解好像是稍微体现这些差异,我所关注的点是分析、解释、探究的这样的一种去知晓、去思考的过程,但是你刚刚说到的改变。这个好像和我描述的那种对你经历的研究的那种行为,好像是不太吻合。”(\pozhehao{}《聊天 | 与 Neruthes, 22 Feb 2022》)

或者是想到将会在周末和另一个男生聚一聚的时候,自己可能又会被提问到一大堆问题的时候\pozhehao{}比如说“为什么你会这么想?”“这两者之间有什么联系?”“是什么让你这么做的?”之类的提问\pozhehao{}我就觉得很烦。

在这两者的过程里,我要么要分析对方话语的背后究竟在说的是什么,我要么要分析我自己的思维和情感脉络。我要么在解码对方,要么在解码我自己。我就觉得很烦。

\midnote{《理智和情感,不确定性》}

在读完昨天这篇文章后,有位微信好友跟我说我前任就是个渣男。这让我想起之前面基的一个男生也说过我前任就是个渣男,然后告诉我如何识别渣男,以及他识别渣男的一次“成功案例”。在那一刻,我觉得他只是用“渣男”一词来塞住我的嘴,并且自我保护了起来。我向那位微信好友回复:“但还是要保持一个好奇的心态去探索(如果自己真的想要得到更多答案的话)”。

\midnote{《“这就足够了吗?”》}

我会想起今年过年和前任见面的时候,我确实做到了悬置自己的憎恨之情而带着好奇地去探索前任这么做背后可能存在的情感和想法。如果我在那个当下真的对他破口大骂或大发脾气的话,我可能就不得到更多答案了。毕竟骂一个人是渣男很容易,但要试图看到对方为什么会“渣”、对方的行为背后的那些内心事物本身很难。


\midnote{豆瓣:\href{https://movie.douban.com/subject/35496211/}{https://movie.douban.com/subject/35496211/}}

昨天看完了一部黑色喜剧电影《房子》(2022),里面有三个小短片。在第三个短片里,女房东从小在那间房子里长大,但父母已经不在了,而她仍然坚守着这栋房子,即使这栋房子所在的街区都被淹没成了一片大海,租户都走了,除了两个没钱付房租的长期住户。其中一个女住户带来了一个朋友,试图促使女房主离开(move on)。在他们最后的午饭时,女住户问女房东:“你究竟想要的是什么?”女房东说她想要重新创造过去美好的回忆,想让这个房子重新热闹起来(就像以前她的父母还在这里时一样热闹)。但女住户跟女房东说她必须离开,不然水就要淹没房子了。然后女住户说了一句话:“试着回顾你做过的好事,爱你的过去,然后继续前进。”

\tristarsepline

我之所以会觉得分析别人和分析我自己会很烦,是因为那根本不是我想要的,而是别人施加给我的,一种硬要我去分析这个分析那个的施加。而当我真的想去分析、去心智化的时候,虽然我能用简单的一个词“渣男”来概括、简化对方整个人,但我仍然希望自己能看见他背后的那些事物,那些想法和感受,那些让他做出过去那些行动的原因和动力。因为我相信人的灵活的,而不是一个标签,不是一个以供分析的死物,也不是一个只会去分析他人的机器。

在今天上的关于自杀危机干预的课程里,讲师说的下面这段话让我有一丝感动:

\blockquote{处理危机和面对人我们是要一起做,这个并不是两个分开的东西,并不是当我今天听到,来访提到自杀,提到某些字,某些词的时候,我内心有一个反应机制启动,听起来很像机器人,很像我们在跟Siri对话,很像我们在跟一些现在AI的这些程式对话的时候,当我出现关键字,关键字会跳出相对应的回应。不是,我们是人,我们要做的事情是,当我听到这些关键字的时候,我心中有些考量,这些考量包括我对法律与伦理的认识,包括我自己在临床上学到的知识和技术,包括我对这一个来访的概念化的理解,与其关系的体验是什么,当我听到这些讯息,这些危机的讯息的时候,我的心中有一个这样的理解。与此同时,我尝试,我也继续稳稳地和对方有联结,我想要了解是什么样的时刻,什么样的情境会让你有想要自杀或轻生的念头,无论这是不是一个危机干预,这是,但是即使你不想危机干预,这也是一个很真诚的想要理解对方伤痛以及脆弱的时刻,这个就是在面对人,这个就是我们作为治疗师,能够带给来访最重要的事情,就是一个新的体验,一个新的联结的体验。这样的体验是没有办法在文字上学到的。}

有一些很自然而言甚至让人有所触动的与他人连接的时刻,恰恰是那些无法被程序化、逻辑化,无法通过分析而“计算出”应该怎么回应的一些话,甚至可能是一些带有着自己的价值观、人生观、世界观的话,一些将自己活生生的存在呈现在对方面前的回应,比如说:“试着回顾你做过的好事,爱你的过去,然后继续前进。”

\useimg{aimg/2022-0319-1.jpg}
