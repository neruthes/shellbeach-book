\chapter{像是个临时救助站,被遗弃感,憎恨}

\ardate{2022-04-24}{SX4TjZeXVgMYd6LoaE2TNg}




昨天面基的时候去了一个之前并不怎么去的地铁站。我不记得我有去过那个地铁站,但当我出站到地面时,我发现自己曾经来过\pozhehao{}大概是两年半前,自己刚开始工作,然后认识了一个男生。当时的我搬离前任公寓没多久,所以心情状态很糟糕、很糟糕。那个男生的出现让我看到了希望\pozhehao{}不仅仅是对爱情的希望,更是对重新拥有一个家的希望。

在那短暂的两三个月里,我经常去地铁站附近的他住处一起过夜。他的住处很破旧,是旧小区的居民房,还是和他那时候的舍友合租的。在那时候的我眼里,相比于前任的公寓,他的住处就像是个临时救助站,但起码也是个能让自己暂时安放着的地方。那时候的我还在为前任的关系而悲痛,但同时也有所缓解,因为那个男生的存在、他住处的存在就像是个抗抑郁剂,直到抗抑郁剂也终究变成致郁剂,直到后来他说他还是想找一个能在事业上扶持他的人做对象。之后我们的关系便逐渐稀疏至消失。

直到现在,我依然很憎恨那个男生,憎恨他抛弃了我。虽然在理智上我知道这并不完全是一种抛弃,但在情感上,他依然激发了我的强烈的被遗弃感,就像被前任遗弃、被初恋遗弃、被父母遗弃。想到这里时,我也会想,自己失败了又如何,we can begin again。而在那之后,那时候的我也确实又重新开始了,和另一个男生相处了半年后又一次经历失败,又一次重复着相似的模式。在那之后,我就完全放弃了,完全放弃通过依靠另一个人去找到一个家,放弃去找一个家。但我又会想到,一个真正的家又缺少不了对方。

然后自己又去做了关于悲伤的冥想练习。当在扫描身体并聚焦悲伤的部位时,我感觉到那个悲伤的部分就像是个肿胀的气球,那个肿胀的气球就在心脏的位置,甚至比心脏范围还要大。我的呼吸变得艰难,需要大口地吸气、大口地呼吸,试图让这个充满悲伤的肿胀气球松弛下来,但它一直就是肿胀的,阻碍着呼吸。逐渐地,我开始回想起从那个地铁站走去那个男生曾经的住处的那段自己所熟悉的路,那段自己在面基时走过的有所不同但又没有多少变化的路,还有那附近的街道。我感觉它们越来越遥远,越来越沉淀在了过去。

冥想结束后,我马上编辑好文字向那个男生表达了我对他的憎恨。表达完之后,自己的感觉也轻松了些。

