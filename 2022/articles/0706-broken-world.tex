\chapter{“如果有问题的是这个支离破碎的世界”}

\ardate{2022-07-06}{Xy1r\_TLcQ2YNcYomFfbN5w}


\blockquote{
	科胡特认识到,成年的W先生会继续从事类似的活动,这类活动也是消除碎裂恐惧的统整形成(cohesion-producing)的方法。治疗期间事先安排的中断之前,W先生有一个详细列出他口袋内物品的习惯,无论是多么地微不足道一硬币的确切数目、一小团羊毛球、一张揉皱的小纸片,等等。他平静地、“自娱自乐”地数着这些存物。科胡特的理论指出,W先生在这些时刻专注于口袋内物品的心理意义是“在一个已经变得不安全、不可预测和不熟悉的世界中\pozhehao{}和他的碎裂自体一样碎裂\pozhehao{}他在一个封闭空间中寻求庇护,这个空间完全被他的心智所掌控,因为他知道这个封闭空间里的一切,这一切都是熟悉的且在他的控制之下”。让科胡特印象深刻的是,在W先生的一生中他是如何已经非常适应性地使用这些强迫类型的活动,以便在孤独、无支持的环境中维持他的自体统整。

	\citebook{自体心理学导论}
}

当读到这一段的时候,我笑了笑,因为当看到W先生的这种强迫性的数存物的方式,我马上就联想起自己的写作方式,“以便在孤独、无支持的环境中维持他的自体统整”。

这本书是我在一年半前就已经读过一遍的,但最近因为在上自体心理学的课程以及最近自己的情绪状态,所以又打算把自体心理学的书单刷一遍。当时读的时候是一知半解(毕竟那时候没读过什么心理学方面的书),但现在我能将书中内容结合课程内容组织成一定的架构地去理解。

\begin{itemize}
	\item 镜映自体客体需要:“需要感到被肯定和认可、感到自己是被接受和欣赏的,尤其是当(个体)展示某些有关自身价值的事物的时候。”“镜映中“恰到好处的失败(optimal failures)”促使孩子发展内在的方式,以维持自尊、忍受不可避免的挫折并追求恰当的抱负。”
	\item 理想化自体客体需要:“需要感到与钦佩他人相连接,产生一种平静、抚慰、安全、有力量和/或有激情的体验。它和我们需要与我们信任的某人融合或亲近有关。这样我们就会感到安全、舒适和平静。”“儿童对他所钦佩的双亲人物的全能全知的深信不疑,慢慢地转变为内化的目标、价值和力量。这些目标、价值观和力量作为源自成熟的镜映自体客体需要的抱负和自体-奋斗(self-striving)的组织者。”
	\item 孪生自体客体需要:“另我自体客体概念是关于整合与另一人的“相似性共鸣(resonance of alikeness)”的情感体验。并且她指出这个“相似性共鸣”是感到被理解的关键组件。……情感同调让婴儿“知道"他的妈妈知道他的体验。随时间发展,同调的这个体验\pozhehao{}假设在母亲体验中有足够程度的相似性\pozhehao{}被孩子整合或结构化为持续地感到能够被认知和被理解。这个感觉是确信个体的情感生活可以和另一人分享的关键。因而,另我自体客体功能本质上是情感同调并且与人类全部情感体验有关。”
\end{itemize}

我记得大概一年多前和一个心理咨询师喝茶聊天时,ta问我为什么我会对前任那么的念念不忘,我说可能是一种对孪生自体的渴望吧。我猜在那一刻,那个咨询师心里说不定在想:这家伙是不是在用心理学知识来自我防御……

现在回想起来,那时候我的回答并不完全错,但也并不完全准确。在前几个月和前任喝茶聊天时,我发现自己会将他理想化为一个很独立、全能、什么样的问题都有唯一的答案、对自己的事情很确信的形象。当我意识到这一点后,我也会去想:为什么我要这么做。如果我这么做的话,他在我眼里就能够成为一个港湾、一个依靠了。就像是“儿童对他所钦佩的双亲人物的全能全知的深信不疑,慢慢地转变为内化的目标、价值和力量”。不过他并没有让我这么做,而是在我面前呈现了一副“任何事物和人在他眼里都没有差异、都一样,他都毫不在乎”的形象。

但现在我会往回想:好像我的一生并没有多少这样的自体客体经验\pozhehao{}没有镜映的人,没有理想化的人,没有“孪生”的人。现在的我的内心甚至连一个很亲近的重要他人都没有。

这似乎解释了,为什么每隔一段时间,我就会进入一种状态\pozhehao{}不知道自己是谁、自己要做什么、自己是怎么从过去延续到现在的,甚至会有一种脱离了现实世界的感知感。而我确定自己的存在的唯一方式便是写作,当然这种方式能带来的只有转瞬即逝的存在感,不过也正如自体客体需求是持续一生的,“科胡特坚持,对自体客体需要的回应是我们心理存活和成长的基本营养物。用他的话来说:“自体-自体客体关系构成持续一生的心理生活的本质”。”

不过,最近几天的我开始更能够看见我自己内心的破碎之处,同时,我也开始更能看见我与身边的他人的裂缝以及身边的人与人之间的裂缝。我甚至好奇身边的人是怎么在这样的环境里活下来的。难道他们就不会感觉到自己的存在、自己的生活、自己的人际关系里充满着虚无、无意义、无聊、无趣、无方向感、裂缝和崩塌吗?

我越来越感觉自己只是诞生于这个支离破碎的人际环境里的其中一个支离破碎的产物(死物)。但与此同时,我也并不只是像个死物般单纯地与环境互动,而是创造性地发展出了属于自己的方式(写作)去维持自身那转瞬即逝的存在感,而我也确信自己是能够享受这一方式所带来的成就感的,不会在完成那一瞬间或之前就感到无意义感、无聊、无趣、虚无。

但意识到这一新的(主观)“真相”并没有让我感觉更好。事实上,我感到更加糟糕、更加伤心。我开始认为怪异的不是我,而是我周围的人、我周围的这个人际环境,我变成现在这个样子并不是我的错。但这也会让我感到更为无力\pozhehao{}如果有问题的是我,起码我能通过自己的努力去改善我自己、去找一个解决办法,但如果有问题的是这个支离破碎的世界,而我只是这个支离破碎的世界里其中一个支离破碎的个体呢?

之前我会把身边的不少人看作是“正常人”、“普通人”,而我则是那个不太正常、不太普通的人。但现在在我看来,他们和我一样都那么的支离破碎,只不过他们好像更难以看见人与人之间的裂缝以及他们自己内心的裂缝。


