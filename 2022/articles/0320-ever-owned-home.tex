\chapter{我真的拥有过一个家吗}

\ardate{2022-03-20}{aJkGIjgRARcpvnarparEqQ}



\dialoguelist{咨询师}{
	\dialogue{我}{在之前的一次咨询里,我有提到自己在一年半前的自杀意图变得很强烈,以至于有差点就实施的自杀计划。当时你问我那时候为什么自杀意图会变得那么强烈,当时发生了什么?我说我不记得了。其实我一直都知道,但我不知道为什么那时候想不起来、不记得了。在昨晚看书的时候,我回想起那时候的事,那时候之所以会有一个难以忍受的情绪的‘高峰’,是因为一年半前一个相处了半年的男生突然无故消失,再也约不出来见面了。所有那时候的短期因素是人际关系的破裂,特别是又再一次重现了前任的无故消失。那件事情给我带来了一种对人际关系的绝望感,就好像自己终于从前任的经历里恢复过来的时候,又遭遇了又一次的无故消失,又一次的失败,又一次的绝望。我会想到,那将来自己所依恋上的人说不定也会同样地无故消失,那活着有什么用呢?既然我不能找到自己真正能够依恋的人,而只能一次又一次地经历重复的痛苦。
		直到现在,我也不知道那个男生为什么会无故消失,我也不知道当时前任为什么也无故消失。
		昨晚读关于心智化的书的时候,里面有讲到心智化能力的提高在于个体能够用文字或言语来陈述创伤。但当我想起我和前任以及和那个男生的经历时,我依然不知道这到底发生了什么。是我做错了哪件事情吗?这一切的变故究竟是为什么?为什么会变成这样子?我不知道。我将来也很可能不会知道。
		不过,那本书里也写到,心智化能力提高的关键不在于心智化的内容,而在于心智化这一过程。也就是,寻找变故、创伤发生的具体原因并不重要,重要的是在寻找答案的这一过程里提高的能力,比如说现在的我能找到无数个可能存在的答案,但依然没有任何一个答案是确切的、唯一的。心智化能力的提高帮助个体调节之前难以承受的情绪,并且阻止事情的再次活现(发生)。而现在的我也很确定过去的经历不会再重演了,因为自己有足够的能力去察觉这样的事情可能的重演,并从中跳出来。但这并不意味着我能在过去的事情里找到安心、找到平静感。我在过去的往事里找不到平静感,因为我永远不可能知道对方无故消失、对方为什么要这样对待我的原因,那个唯一的、确切的答案。
		如何能在过去的往事里找到内心的平静呢?我不知道。
		之前的自己能找到平静是因为自己突然发现了很多可能的答案,而现在无法找到平静是因为现在的自己已经习惯了那些可能的答案,而我一直想得到一个确切的答案。
		就像是犹太人在经历了大屠杀和集中营的经历后,其中有一个人在逃脱后创建了意义疗法,来帮助其他经历了这样残酷的经历的人。我也想到,如果我能找到一个意义\pozhehao{}比如说他们的无故消失让我在心理学和学咨询路上越走越远,但这个意义对我来说并不work,或者说并不足够。}
	\dialogue{咨询师}{你说道‘不足够’?如果对方真的能给到一个答案呢?你会设想那个答案是什么?}
	\dialogue{我}{我不知道。我设想不到。我有试过问前任为什么他无故消失了一年多,他说是因为工作忙。我也问过那个男生为什么他无故消失了,他说他不想聊这方面的事情,还说再继续问下去就不怕他又消失了吗。我从他们那里得不到答案,或者说得不到我信服的答案。如果靠我自己去想的话,我想不到,因为设想究竟还是来源于我自己。我已经做不到回答这样的问题了。}
	\dialogue{咨询师}{你有说到前任的不确定性和那个男生的不确定性。你会联想到什么吗?}
	\dialogue{我}{我会联想到……(我在读)六年级的时候,我看着窗外的日落,我想要走出窗外,走向那片日落。但那时候我没有那么做,是因为我不确定未来会不会有更美好的事物、会不会有值得我活下去的事物。在和前任分手后,我真的很想自杀,很想逃避那份痛苦,但我没有这么做,因为我想到如果自杀了的话,我就再也没有机会见到他了。我不确定未来还能不能再见到他。}
	\dialogue{咨询师}{你能说一下从小学到一年前甚至几年前的你会有怎样的变化吗?}
	\dialogue{我}{Em……最大的变化是现在我几乎没有自杀的念头了。而且也没有了那种攻击性,无论是对前任、对那个男生还是对我自己的攻击性。和前任分手、和那个男生的联系消失后,那时候我会责备自己,为什么没有在六年级的时候就走出窗外,就结束这一切?为什么我还要承受一次又一次的无故消失。那时候我会觉得很绝望,觉得未来没有希望了,即使自己再遇到想要依恋的人,对方也只会无故消失。这个世界也就这样了。但现在没有那份攻击性了,但依然延续下去的就是那份不确定性,从六年级一直延续到一年半前。}
	\dialogue{咨询师}{如果你能把那份从六年级一直到一年半前的不确定性进行浓缩的话,那会是什么?}
	\dialogue{我}{Em……好像……Em……像是……我会想到……家。不确定自己是否有一个家。我一直没有一个家,有一段时间我试图去将家的缺失放在一边,因为我有点害怕去面对那份缺失。因为那份缺失是让我之前想要自杀的很大一部分原因,因为一次又一次地失去家。我会想到,如果未来遇到了一个我可以视之为家的人,那么过去的无故消失就可以翻篇了。我就可以接受过去一次又一次的家的丧失了。}
	\dialogue{咨询师}{你愿意多说一点,帮助我理解你的话吗?}
	\dialogue{我}{我发现身边并没有人提起这样的事,好像每一个人都有一个家,好像这是理所当然的。就像是自杀,身边从来没有人提及过这样的事,或者大家都默认了这件事情但不会提及。所以我一直感觉自己是个异类,因为我能看见很多东西,比如说之前会陷入一些关于存在主义的思考里,觉得身边的一切都没有意义、都充满着虚无。我能看到事物背后的东西,但好像周围的人都看不到,都觉得事情表面就是理所当然的。所有无论我在哪里,我都是一个人,我都是格格不入的。虽然我知道有这样的群体,有无家可归的人,有充满自杀意图的人,比如说接热线的时候,但我生活里一直没有这样的经历,没有这样的部分在。
		我会想起,初恋和前任都是那种家有所破损的人。比如说初恋的母亲在他几岁的时候和另一个男人出国私奔去了,留下他和他父亲,所以算是半个家破碎了。前任的话,他更像是个留守儿童。而我现在喜欢的男生,他也是这样,他和他爸爸的父母(爷爷奶奶)那边更亲,而不是和他父母。}
	\dialogue{咨询师}{我会想起在有一次咨询快结束时,你提到了脆弱感。你会对此感到脆弱吗?}
	\dialogue{我}{好像不会。我感受到的是一种被排挤感。而和他们在一起,就像是在异类里找到了同类,甚至会有一种无意识的熟悉感。但因为他们,自己经历了一次又一次的家的丧失。我觉得很awful耶,为什么都是有类似经历的人,却要让对方重复着过去的创伤??}
	\dialoguesepline{咨询师}{(短暂的沉默)}
	\dialogue{我}{我现在会感到很悲伤,好像我需要去哀悼自己没有一个家,想去哭一场。也许哭完下次的情感就没有那么强烈了。我最近在上的自杀危机干预里有一个比喻,它说情感就像是海浪,海浪总会一波又一波地拍打过来。而痛苦耐受技术就是在海浪最高峰的时候让当事人能够度过最困难的时候,让TA跌回海浪的低谷,直到下一波海浪的到来。这很像是之前的咨询里我提到的那个漩涡的意象,就好像我一次又一次地回到那个漩涡里。所以我知道这种情感还会有下一次,和下一次、下一次、下一次。我会觉得很崩溃,觉得自己很无能。我没有什么能做的,我也控制不了情感。我能做的只能去面对情感,去应对情感。我好像一直以来都只是在应对情感。我只能守着我自己的一小片空间,试图找到一份宁静,但现在的我是找不到的。我能控制的也就只有那么多,剩余的我无能为力。}
	\dialogue{咨询师}{这会让我想起一个画面,就是你妈将你遗留在街上的画面,会很迷茫。}
	\dialogue{我}{嗯。好像我总是会回去以前的那个场景,总是回到那个迷茫的时刻。即使被无故消失地对待已经过去起码一年多了,但自己依然感觉好像被无故消失地对待又在自己脑海里重演了一边。觉得很难受,很想哭。}
	\dialogue{咨询师}{我在想,为什么你会一次又一次地回到那个迷茫的场景呢?}
	\dialogue{我}{我不太知道之前,但这次的刺激源是因为我昨晚读的那本心智化的书。里面有一个案例,就是一个女性,她小时候经历家暴,然后她的丈夫也同样对她家暴。她没有足够的心智化能力意识到这一点,同时她的孩子也无法幸免。她会无意识地让她孩子处于危险,处于一次又一次的创伤,因为她认为这是让她孩子得以应对未来的危险的方式。而她孩子也有解离的症状,这让她孩子更容易置身于危险的情况里了。在读的时候,我好像就会不经意地代入自己,想起被无故消失对待的经历。
		所以未来很可能也会有这样的刺激源,无论是书上的内容还是来电者还是他人的经历,都会让我一次又一次地回到过去。}
	\dialogue{咨询师}{好像你不能确定未来还会不会遇到这样的刺激源。}
	\dialogue{我}{嗯。}
	\dialogue{咨询师}{但我也留意到,好像你的焦点从他人转移到了你自己。比如说你现在没有去想对方怎么样,而是在想自己怎么样。}
	\dialogue{我}{可能是因为现在的我没有把家投射到某个人或某个空间里。之前我会在想,为什么这个我视为家的人能这样对待我?为什么对方能无故消失?但现在,我觉得即使关系能延续到现在,即使我和前任直到现在都还好好的,但这真的就是家吗?我好像只是把家投射到他身上,并且美化了和他相处的经历。这真的就是家吗?我更加不确定了。
		我不确定……我……真的……拥有过一个家吗……}
	\dialogue{咨询师}{在这种不确定性里,我好像也看见了一种确定。或者是一种期望,好像你会期望在未来拥有一个家。这种期望好像也帮助着你继续走下去。}
	\dialogue{我}{我不确定……以前的我会对未来有期望,这种期望确实帮助了我继续走下去,而不是停在原地或自杀。但现在我对这种不确定性只有迷茫感和不安全感。
		现在在我面前好像只有两条路可以走。一条是面对不确定性,不确定未来的家是否真的存在,以及更大的不确定性是,家真的有存在过吗?但我确确实实有过家的感觉,虽然那只是一种投射和美化,但我确确实实感受到家的温馨感和温暖。就像是自己在做梦,我把自己置身于梦里。梦里的感觉很温暖,但梦依然破灭了,而我也梦醒了。但我不会去抹去梦里的感觉,那种家的温馨感和温暖,因为那是我所仅有的了。
		另一条路是去哀悼,去悲伤,去哭泣\pozhehao{}我只有现在这样了,未来从未存在,过去就是这样了。我从未拥有过一个家,我有过的只有那片刻的家的温馨感和温暖。
		……好像比起面对不确定性,我更想去抓住这份悲伤,因为这份悲伤本身起码是一种确定性。}
	\dialogue{咨询师}{你最后的这个察觉真的蛮有趣的。我们的时间也差不多了,我想下次我们可以在这里继续。}
	\dialogue{我}{嗯。下周见。}
}


我是应该接受未来的家的不确定性,还是选择去哀悼家的一次又一次的丧失?我不确定。我是应该接受现在就是现在,未来是不存在的,还是接受过去的无故消失?是抱着一个渺茫的希望,希望未来有可能再次体验到家的感觉,无论是投射也好,美化也好,还是接受现在就是现在了,我就是这样了?

不过,我也开始能看到了这两条路的折中之路:接受事物本来的样子,接受过去并没有一个真正的家,而只有家的感觉,以及一次又一次地丧失了家的感觉,接受只有现在,并面对未来的不确定性。
