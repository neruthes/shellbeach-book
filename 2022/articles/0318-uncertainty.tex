\chapter{理智和情感,不确定性}

\ardate{2022-03-18}{vkPyplA7UQcQqhH702XjVQ}



有一天失眠的时候,翻了翻云相册,往回看了一些三年前和前任在一起那段时间的照片\pozhehao{}有在公寓拍的照片,有在楼下餐厅、快餐店拍的照片,有去海岛玩拍的照片。在看照片时,脑海里有一个声音特别响:“我为什么还要活下去?我失去了那么多,我失去了好多好多。为什么离开那样的生活后还愿意继续活下去?”在这段时间里,我也开始习惯脑海里的这个声音了,以及这个声音伴随的复杂的情感\pozhehao{}丧失感、遗憾感、悲痛感、被遗弃感等情感交杂在一起。

经过这段时间的探索,在理智上,我知道那时候我和前任在一起背后的那些属于自己的心理动机,比如说每次亲密关系的出现都是因为生活的变迁,因为我需要另一个人、另一个家来让自己在这个世界里找到下一个能够让自己停驻下来的地方和人。我也会想,也许对方究竟是一个怎样的人并不重要,毕竟现在的我也终究还是没能看清对方究竟是个怎样的人。

但在情感层面,自己现在也依然对前任带有着很强烈的爱意和恨意。即使在理智层面上知道自己并没有认为他有多特殊,但情感上对他的感觉依然是独一无二的。情感和理智的“冲突”并没有阻止我跟随自己的感觉去靠近他,也没有让我在和他的相处过程中失去理智。

以前的我有一段时间会为这样的冲突而自我责备,比如说:“既然他并不是那么特别,为什么自己还那么想念他、想见到他?”“既然自己那么爱他,为什么那时候自己还选择离开他公寓?既然自己那么恨他,为什么还想一次又一次约他见面?”就像是内心里的两、三个自己\pozhehao{}理智的部分、情感的部分(爱的部分、恨的部分)\pozhehao{}互相撕扯着,谁也不愿放过谁,而且也不会撕扯出一个结果,永远没有胜负,只有不断地受伤和再次受伤,还是心甘情愿地受伤。

但现在我会觉得,one does not diminish the other。对他的情感强度并不会因为理智上分析得有多透彻而有所削弱;对彼此的理智化分析也不会因为情感的强度而有所丧失。两者就像是两个不同的方式,一个是用眼睛去看,另一个是用心去感受。两者互不替代,也互不影响。

而且在这段时间里(特别是最近一年),自己慢慢习惯于游走在不确定性里。不确定下次和他的见面是什么时候,不确定是否还有下一次见面,不确定他内心的感受和想法,不确定自己的内心感受意味着什么,不确定彼此关系的未来会是怎样的。

我想起以前在亲密关系里,那时候的我总会因为追求确定性而做很多事情。比如说想要确定对方是在乎我的、爱我的,想确定他什么时候会回公寓,想确定他每天可能会去哪……想确定的事情有很多,以及背后那逐渐加剧的不安全感。但在和前任相处的过程中,如果说有什么是不变的定律的话,那就是:nothing is certain。我无法确定彼此的关系是什么,无法确定这段关系里彼此能做什么、不能做什么,无法确定这段关系会延续多长,无法确定他的想法和感受。即使是生活在一起的时候,连他几点会回公寓,今晚是否回公寓都是不确定的。

在经历了这样的不确定性后,我好像也更能承受现在的不确定性。比如说现在的学习和发展的不确定,人际关系的不确定性,和自己在乎的人的关系的不确定性,对他人和自己的内心世界的不确定性。

在和前任重新联系后(之前他无故消失了一年多),我开始去接触一些心理学的知识和体验,那时候的初衷就是想去减少对人、对他的不确定性。我想知道那时候的他为什么会这么做、为什么会这么对我,而这些我所渴望的答案是无法从他口中问出来的(现在依然也是如此)。但在寻找答案的过程中,这种不确定感并没有减弱,而是不断增强。我更不确定对方和自己内心究竟有什么、究竟是什么。每一个对对方的提问、对自己的提问都会延伸出无数个可能的答案,无数个无法去证实对错的答案。比如说那时候他很晚才回公寓甚至不回公寓,可能因为他想回避那时候的我;他有其他更值得回公寓的事情好干,比如说工作或/和他人;他在重复着小时候经历的事情,只不过现在他是施加者而不是被迫者等等等等。我更不确定自己内心感受和想法的出现是为什么,更不确定对方行为背后的想法和感受是什么。没有任何答案是确定的、唯一的,没有任何一个提问是能得到确定的答案的。所以到头来,我手里的提问越来越多、越来越庞大,一堆没有答案的提问。

所以,承受着不确定性会让我很疲惫,无论是在和前任的关系还是和在生活里其他的人际关系,没有什么是确定的、已知的,而自己只能继续游走在不确定性里。没有真正触碰到些什么,也没有真正看见过谁。有时候自己会自我欺骗地说:“起码自己能获得平静感,因为自己现在用能各种可能的答案来解释事情的发生和经过了”。但有时候这种自欺欺人的方式并不work,这种平静感并不会last long。自己依然感到很不安全,我怎么就能确定对方不会离开呢,我确定不了。是啊,我谁也确定不了,什么事物也不确定不了。Nothing is certain.

在最近看的关于心智化的书里有写到,边缘型人格障碍者的心智能力普遍很低,所以在面对痛苦时,比起心智化他人和自己,他们更会选择一些看得见的“行动”,比如说通过自伤或自杀来获得他人的关心和照顾,或者只是单纯为了逃避痛苦。我大概能理解为什么他们会这么做了,毕竟看得见的行动是确定的,而思考以及对思考的思考(心智化)本身是看不见的、抓不住的、不确定的。自己和他人的内心也是如此,特别是他人。

但痛苦总是真实的。

