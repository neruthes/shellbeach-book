\chapter{“眼里的光消失了”}

\ardate{2022-07-07}{4Vw9gQeBOwDMyddRT8PCmA}



昨晚和一个有一段时间没见的朋友聊天,有聊到了彼此看待同一个行为(例如约晚饭后见面、是否过夜等)会有不同的解释,有聊到他和他的咨询师的相处方式,有聊到咨询方面的专业知识,有聊到和他约见面时我会有不安全感和危险,但我依然还是和他约见面,也见上面了。

但让我印象比较深刻的并不是一些纯理智的聊天,而是他说的其中一句话。他说我和几年前刚认识我的时候的样子发生了很大的变化,是一个不好的趋势,“眼里的光消失了”。后来我跟他说了我在这几年的经历,从大学到大学毕业遇到前任到前任之后的几个短暂相处的男生,以及我在这个过程中的内心变化。

然后晚上我做了一个梦,梦的前半段我已经忘记了,后半段是我在走一个破旧的老社区里,走在不规则的社区通道,那里的社区通道连接着建立在不同高度的山坡上的居民楼。通道的路面和周围的护墙都是灰色的混凝土,地面上很脏,看似并没有人打扫过这里。通道外是一棵又一棵的深绿色树木,我所走的社区通道似乎并不在地面上,而是架空在了空中。

我和一个朋友走在错综复杂的社区通道里,他走在前面,我跟在他后面。我的鞋子的后半段走着走着就脱落了,只有前半段是完好的,而我后半只脚则光着脚走在很脏的地面上,这让我没有了一半的安全感。但我并没有叫走在前面的朋友停下来,而是继续跟着他走。

后来我们走到了一个空旷的地方,前面的两侧是树立在两个高度不同的山坡上的居民楼。在两个山坡之间凹下去的地面上,是一条混凝土建成的有护墙的道路,远处能通往更远处的居民楼的拐角处。但在面前的是一片凌乱,许多树干倒在了路上,砸裂了路面,连同树上长的榴莲一同砸在了地上,地面上有不少榴莲壳和榴莲刺。我想到自己的后半只脚是光着脚的,所以我怕踩到榴莲刺,而且路上四处倒下的树干也让我不知道应该怎么跨过去。当我还在犹豫的时候,走在我前面的朋友早已找到一条往下走、能绕过去的路。

后来我们在这片社区兜兜转转又回到了这个地方,但倒下的树干和榴莲都消失了,只有裂开的路面。路中间站着一个穿着破旧的睡衣、瘦骨嶙峋的老头。那个朋友说,这个人是一个父亲,但他投资失败了,落魄了。我看了他一眼,看见他眼神呆滞地望着前方的地面,整个人都呆滞住了。我们很快地走过了他身边,继续往前走。

睡醒之后,我就在想,这个老头是谁?我想起昨晚朋友说我“眼里的光消失了”。

那片老社区(也许是内心世界)很破旧,就好像被遗弃了,只有几个居民在附近坐着聊天、打牌。然后我跟着朋友走到了一个更空旷,能通往更远的地方的(人生道路)道路,但道路被倒下的树干(冲击)砸裂(被挡住、破裂),地上也有着会刺伤脚的榴莲刺(刺痛)。我的鞋子并不是从一开始就破了,而是在跟着朋友走的时候在半路破了,就像是跟着这个朋友走会让我的鞋子破掉一半(安全感消失一半)。但这个朋友也带我绕过了倒下的树干和地上的榴莲刺。

不过我们还是没有找到出路,兜兜转转又回到原来的那个空旷的地方,就像是之前我知道现实世界的那个朋友在上我一年前学过的关于咨询的课程,看似有着和那时候的我同样的热情,但现在的我在上完那个关于咨询的课程以及后面的实践后,我反而丧失了当初自认为的方向感和对这一方向感的确信感。看着他有着当时我学这门课程时的热情,也有一种跟着对方回到了原地的感觉。但那个空旷的地方已经没有了掉落的树干(冲击)和榴莲刺(刺痛),只有裂开的地面(破碎的道路)。

那个穿着破旧的睡衣、瘦骨嶙峋的老头让我联想起现实世界的那个朋友所描述的他眼中的我\pozhehao{}“眼里的光消失了”。不过在梦境里的我看来,那个老头更像是生命力都枯竭了。然后梦里的朋友说他投资(精神投资)失败了、落魄了。

我尝试去代入这个老头的位置,去设想如果我是他,我可能会说些什么。我可能会说:我在这条路上差点被倒下的树干砸中,仅仅是清理掉路上倒下的树干(冲击)和榴莲刺(刺痛)就已经耗费掉我全部的精力,我已经没有生命力去做任何事情了。

但我会好奇,为什么那个朋友说“这个人是一个父亲”?在我仅有的印象里,父亲这个角色的作用是保护孩子以及为孩子探索前面的道路。但这个穿着破旧的睡衣、瘦骨嶙峋的老头很显然没有了任何保护他人的力量,也失去了探索前面道路的动力。

但让我感到庆幸的是,梦里的我不是那个老头,而是一个跟着我朋友还在探索前面的路的人。梦里的我并没有将自己认同为:我就是那个老头。

当昨晚那个朋友说我在往一个不好的趋势走、“眼里的光消失了”的时候,当时我的内心在想:也许是,但起码现在自己每走的任何一步都是自己想要走的。同时我也会感到哀伤,不一定是为了朋友口中那“眼中的光”的消失而感到哀伤,而是回想起几年前的自己对未来的期望、对有着一个亲近的重要他人和一个温馨的家的期望在这几年来一次又一次地落空、破碎和粉碎而感到哀伤,直到逐渐接受现实原本的模样。


