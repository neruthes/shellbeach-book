\chapter{不在乎的自我,更干净的边界,轻松}

\ardate{2022-06-26}{PKlvhEm6licGZUGs6InkAw}

% \dialoguelist{咨询师}{
% 	\dialogue{咨询师}{是啊。}
% 	\dialoguesepline{咨询师}{(沉默)}
% }

\dialoguelist{咨询师}{
	\dialogue{我}{待会儿我走的时候,我会直接说“下周见”,然后就直接走出去,不会像之前那样走到门口再180度转身说“下周见”,因为我会觉得那样太刻意了,很不自然,甚至觉得好奇怪。}
	\dialogue{咨询师}{噢?好像之前你都没有将这件事提出来?}
	\dialogue{我}{嗯,我不记得第一次咨询是怎样的,是我站在门口你才走过去门口说“下周见”的?还是你站在了门口,我才回头跟你说“下周见”的?我也不知道你对待其他来访者的过程是怎样的。}
	\dialogue{咨询师}{为什么你会想在门口回头说一声才离开呢?}
	\dialogue{我}{不知道呢,可能是社会习俗?比如说如果在人多的地方也会打声招呼再走。}
	\dialogue{咨询师}{那在一开始,如果你直接走出去的话,你会有怎样的设想吗?}
	\dialogue{我}{我会想到你可能会感到被拒绝、被落在了原地。}
	\dialogue{咨询师}{好像你会设想到我的感受。}
	\dialogue{我}{嗯,可能是我会无意识地代入到你的位置去感受你会有什么感受。但这当然只是我的投射。\\
		我会想到这种代入最初是从初恋那习得的。和他相处的时候他会很粘人,每次我要去见朋友或者是一个人去逛逛的时候,他都会很不开心。所以时间长了,我就会开始代入他的感受,去看他可能会在分离的时候会有什么感受。然后这种代入也延续到了他之后的其他人身上。}
	\dialogue{咨询师}{那你呢?你会有什么感觉,面对分离的时候。}
	\dialogue{我}{Em……我好像什么感觉都没有,我通常不会对分离有什么特别的感觉。}
	\dialogue{咨询师}{听到这里,我还感到蛮惊讶的。}
	\dialogue{我}{我会想起我最近接的一个热线,临近(热线)时间到的时候,ta说ta感觉我不理解ta、不在乎ta,我说我内心感觉了一下,我好像并没有这样的感觉,那我这边先挂电话了,再见。\\
		然后我会想起最近见的一个之前很久没联系的男生,我去了他家坐一坐,然后看到他家的变化很大,然后我立即感到难过。后来离开之后,我感到更加难过了。我去觉察了下,这份难过是因为之前的我一直渴望成为他家的一部分,然而当我看到他家里有了那么大的变化但这些变化都与我无关后,我意识到这份想要成为他家的一部分的期望破灭了,我不属于任何地方。但这种难过是在见他的时候就已经出现了的,只是在分离的时候加剧这种难过了。就像是见对方的时候我报着一个期待,然后见上面的时候这个期望落空了一级,离开之后又落空了一级,会想到:我们之间的关系也就这样了,不可能有深入下去的空间了,也不可能成为独一无二的关系了。\\
		我还想起之前喜欢但拒绝了我的男生,和他约饭最后离开时,我会感到伤心,但也会想到之后如果想约他的话我可以过两周再问问,我还能再去约他,而不是人不在了。}
	\dialogue{咨询师}{好像我们的咨询里也会有这种感觉,你在咨询里感觉到我在拒绝你、在隔开你。}
	\dialogue{我}{Em……此时此刻的我没有这种感觉。就像是我跟那个来电者说,我此时此刻没有这种感觉一样。如果你有的话,我想这可能是属于你自己的部分。}
	\dialogue{咨询师}{当你说“这可能是属于你自己的部分”的时候,你会有什么感觉?}
	\dialogue{我}{我会感觉很轻松吧,不需要为对方的部分承担什么责任,也不需要承担一个咨询师的角色帮对方去探索、去觉察那些部分背后会是些什么。就像是在热线时间到了的时候,那个来电者说我不理解ta、不在乎ta,我内心感受了下,我并没有这样的部分,而我可以选择去处理ta的情绪,也可以不去处理。既然热线时间到了,那就不处理了吧,毕竟这又不是属于我的部分,所以就挂了电话。}
	\dialogue{咨询师}{我会想到一个词:边界。}
	\dialogue{我}{嗯,会有一个更干净的边界。}
	\dialogue{咨询师}{“干净”。你会想到什么不干净的例子吗?}
	\dialogue{我}{我会想到自己之前,特别是在学咨询之前,会看别人是怎么看待我的。但别人眼中的我通常是很混乱的,甚至是同一个人看到我的样子都很混乱,比如说一时觉得我勤奋,一时觉得我懒惰。然后当我将所有人看待我的样子融合在一起的时候,就更加混乱了,而我也看不清自己到底是怎样的。}
	\dialogue{咨询师}{好像从其他人的镜子里怎么看,都很混乱,都看不见自己是怎样的。}
	\dialogue{我}{是的,所以当上大学开始写作时,我就选择放弃了他人眼中的我,而是选择了一个更干净的镜子,用纸张来作为一面镜子来看自己。这样会更加干净,没有其他人各自的主观性。}
	\dialoguesepline{咨询师}{……}
	\dialogue{咨询师}{我记得你在之前,就连上一次咨询都会有提到我的一些穿着、我带进来的袋子。}
	\dialogue{我}{其实这次我也会留意到你身上的不同,但这次没有特别想提出来。}
	\dialogue{咨询师}{这次你看到的不同会是?}
	\dialogue{我}{比如说你腿上有个红色的印子,不知道是刮伤还是什么的,还会看到你会刻意用裙子盖住膝盖。}
	\dialogue{咨询师}{但这次你没有想提出来是为什么?上一次你还会提出来。}
	\dialogue{我}{可能是因为我在彼此的关系里能感觉到安全感吧,所以不需要刻意去捕捉一些细节。我看见了这些细节,好,我可以把它们放在一边,不需要特别提出来,只是在想提的时候才提出来。可能因为有这种安全感在,我不会去猜想这种表象背后可能会有什么东西,有时候甚至这些猜想或剧本是自然而然地呈现在我面前的。}
	\dialogue{咨询师}{我记得之前你说这些剧本都和被拒绝、被隔开有关。}
	\dialogue{我}{嗯。}
	\dialogue{咨询师}{那为什么这次会有这么大的转变呢?}
	\dialogue{我}{可能是因为我最近的状态?我不太确定。最近我开始觉得自己那不在乎任何事物和人的自我逐渐沉淀了下来。之前这部分不在乎的自我和那部分在乎其他事物和人的自我会产生冲突,然后现在这两者好像能够和谐地共存下来,不会起什么冲突。而这好像会给我一种更安全、安定、稳定的感觉。\\
		比如说和你的相处里,我不需要担心你的这些穿着或袋子或肢体语言背后可能暗藏的东西,因为我可以把自己那部分不在乎任何事物和人的自我自由地拿出来防御自己,毕竟如果我不在乎任何人的话,任何人也伤害不到我的内心。\\
		好像现在的自己没那么容易被伤害到了,所以感觉自己更加安全。知道自己能有办法去处理情绪,比如说通过去看自己内在的情感和情感背后的事物来从关系中撤回。}
	\dialogue{咨询师}{那你之前呢?之前会是怎样的。}
	\dialogue{我}{之前,比如说小时候,我应对的方式会很绝对、很非黑即白,可能因为那时候我无力保护我自己,所以更加要完全避免伤害,所以那时候会完全封闭自己的情绪。}
	\dialogue{咨询师}{听起来好像现在你也依然是通过往内去处理一些东西。}
	\dialogue{我}{嗯。不同之处可能是,小时候的我根本不知道内心受损了要怎么修复,所以会很绝对化地说:我一定要保护好我自己,绝对不能受伤。而现在是,如果受伤了的话,我知道怎么修复。\\
		我也会想到这部分不在乎事物和他人的自我是从前任那内化而来的。自从和前任喝茶聊天后,我就开始变得越来越像他,因为他看待事物的角度是:任何事物和人都是一样的,没有什么独特性,都是同质的。但后来我发现我不只是单纯地变得更加像他,而是内化了像他的那部分自我,并融入了其他部分的自我,特别是那部分在乎他人的自我。\\
		我会留意到你的姿势变了,会想到你是不是想站起来走走,因为时间好像快到了。}
	\dialogue{咨询师}{嗯,时间确实快到了。但我是在想你刚刚说的话。我脑海里有一个画面,就像你说的,你的边界好像变得更干净的。}
	\dialogue{我}{嗯,我想这很大程度上是由那个不在乎他人的自我所维护的。}
	\dialogue{咨询师}{我在想你这么做是不是因为在乎他人会带来很强烈的情感。}
	\dialogue{我}{Em……可能此时此刻的我没有这样的感觉。}
	\dialogue{咨询师}{那你此时此刻会有怎样的感觉?}
	\dialogue{我}{我会感觉更加轻松。我可以变得更加真实,在乎的时候真正在乎,不在乎的时候我也不需要假装自己看似有点在乎。不然这只会像是在演戏,只会让一切看似更加毫无意义、更加虚无。}
}

