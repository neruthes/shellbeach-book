\chapter{过程,情绪的起伏,没有剩什么,拼图}

\ardate{2022-03-28}{\_M5Q8oUj17A-dBjlufI69Q}


\dialoguelist{咨询师}{
	\dialogue{我}{我突然想起我们好像是要继续谈上一次咨询结束时的事情。我记得上一次咨询结束时,我面对着两条路,要么是面对过去的不确定性,不确定过去的重要他人为什么会无故消失;要么去哀悼过去的丧失,那一次又一次的家的感觉的丧失。然后我在咨询后找到的一条折中的道路是,哀悼过去那些发生的事情,并认为现在就是现在了,未来是不存在的。好像当放弃了对未来的期望的时候,过去的那些事情也没有那么束缚着我了。比如说以前我会想要在未来找到一个家,去想象两个人之间相处时的温馨的场景,找到一些能平衡过去所经历过的糟糕的事情的东西或人,但现在我放弃了那种期待了。当我放弃这么做的时候,过去的事情也没那么束缚我了,过去就过去了,现在就只有现在,未来是不存在的。大概就是这样,然后这一周也没有因此而有太大的情感波动。}
	\dialogue{咨询师}{好像你说的事情都是这两次咨询之间发生的。那这个过程会发生了什么?}
	\dialogue{我}{过程……我找一下我写的文字(拿出手机翻推文):
		我是应该接受未来的家的不确定性,还是选择去哀悼家的一次又一次的丧失?我不确定。我是应该接受现在就是现在,未来是不存在的,还是接受过去的无故消失?是抱着一个渺茫的希望,希望未来有可能再次体验到家的感觉,无论是投射也好,美化也好,还是接受现在就是现在了,我就是这样了?
		不过,我也开始能看到了这两条路的折中之路:接受事物本来的样子,接受过去并没有一个真正的家,而只有家的感觉,以及一次又一次地丧失了家的感觉,接受只有现在,并面对未来的不确定性。
		这好像和我刚刚说的差不多,就是这样的过程,并没有什么太特别的。但好像也觉得很奇怪,因为这一过程总不可能只是我写的这两段话而已。噢,我想起咨询结束后我哭了一场。}
	\dialogue{咨询师}{那时候的哭是什么时候的事情?}
	\dialogue{我}{上周X的事情,就是咨询结束后我回家准备睡觉的时候发现睡不着,然后就哭了一场,哭完起床把东西给写了。}
	\dialogue{咨询师}{我记得之前你说过你有一次在进咨询室之前在走廊哭了,把自己的情绪处理好才进来做咨询。现在你说你上周结束咨询后在家哭过了。而且还是我问这个过程发生了什么问了几次你才说你上周咨询结束后哭过了。会不会你不敢在咨询室里哭?在咨询室里哭会让你感到不安全吗?}
	\dialogue{我}{Em……我之所以没有一开始就说我上周哭过了,是因为哭对我来说只是一种很平常的经历。它会有分量,但只是一点点分量。至于为什么没有在咨询室里哭,好像我的情绪每次都是在咨询之后才抵达高峰,然后再回落下去。像是咨询里的聊天触发了我的一些情感或个人议题,然后这些情感并没有随着咨询的结束而结束,毕竟咨询时间有限,而是在之后继续加剧,而之后我也能通过写作或其他方式来处理这些情感。而且如果我一开始进来咨询就说我上周哭过了,如果是我面对这样的一个人的话,我会觉得很奇怪,我会在想:那你是想干嘛?你是想从我、从中获得些什么吗?我想我之所以没有一开始就说我哭过了,是因为我没有想这么做的背后的那个动力、驱力,没有想通过在一开始说我哭过了从而获得些什么,没有需要通过这么做来获得些什么。}
	\dialogue{咨询师}{那如果在咨询室里哭呢?你会想到些什么吗?或感受到些什么吗?会不安全吗?}
	\dialogue{我}{我会想到一部美剧,其中的一个女生的一个朋友被神附身了,而她在找各种办法试图去救他。和她在一起的一行人问她她难道就不对此感到伤心吗?她一直否定和回避着(提问)。然后当她自己一个人和另一个人交谈的时候(那个人不是那一群人里的),她说她当然伤心了,但她不能哭泣,因为只要她开始哭泣,她就什么也干不了了,除了哭泣。我会想到,如果我在咨询室里哭5分钟的话,我这五分钟里什么也干不了,只能一直在哭,而且哭完我会有一段待机时间,那段时间我不想去面对任何人和事,只想一个人呆着。所以如果在咨询室里哭的话,对我来说好像不太适合,因为我还需要在咨询里继续往前走。而且如果我真的要哭的话,我会想一个人呆着,我不想身边有其他人在。}
	\dialogue{咨询师}{如果在你哭的时候有另一个人在,比如说我或另一个人的话,你会有什么感觉吗?}
	\dialogue{我}{我会觉得很不自在,觉得不安全,不是一个适合我哭的空间。我会想起以前我在对方面前哭的时候,比如说父母、初恋和前任,我在他们面前哭的时候,他们会用各种方式来应对我,比如说打我或无视我或骂我或攻击我。但现在我会想找自己现在在相处的一个男生哭,会想抱着他哭,如果是我遇到什么受挫或经历的时候。我会想以我自己想要的方式哭。但每次和他见面的时间都只有周末了,所以当我想抱着他哭的时候,要等到周末,那时候我很可能都已经把自己的情绪处理完了,哭也自己一个人哭完了。毕竟现实世界还是现实世界。所以还没有过这样的经历,但我会想这么做,如果之后有想哭的时候。
		其实现在想起来,我并没有因为过去的那些经历就不敢在另一个人面前哭了,并没有阻止我想去做我想做的事情。过去已经是这样了,现在就是现在。}
	\dialogue{咨询师}{嗯,这好像也和咨询一开始你说的主题很吻合。}
	\dialogue{我}{我想这可能也和之前很多次咨询里我没有什么想说的有关。因为一周下来,没有留下多少特定的情感和想法。他们都流走了,没有什么停在了这里。我想这种状态就像是自杀危机干预课程里说的那种情感基调,就是自己并没有感到不开心,也没有感到开心,就是什么感觉也没有。我会想起大学时期的我的规律是大概有一两周会感受到强烈的情感,比如说悲伤和抑郁,然后就会有一两个月是什么情感都没有的状态。那时候更像是自己身体的规律,但现在是由外界所触动的情感,而这些情感波动完之后就又沉了下去,什么也没有了。我想起之前我在热线团体督导的时候提的一个提问就是:如果觉得自己没有需要被督导怎么办?因为其他同学想要迫切地被督导是因为他们内心有强烈的情感驱使着他们去这么做,但我没有。我的情感会有所波动,但他们都在一个安全的范围内,或者是有一些超出了安全范围的情感是我能处理的。}
	\dialogue{咨询师}{(笑了笑)那你的督导师怎么说?}
	\dialogue{我}{TA说,那我是处于一个正常的工作状态里,不像是有的认知行为学派的咨询师掀不起什么情感。TA说如果只是将我觉得满意的一次热线拿出来分享,让其他人看一下我会不会有什么没有发现的、没看见之处呢?我说,嗯,可以呀,蛮好的,就是为了增进咨询或接热线的能力而这么做。}
	\dialogue{咨询师}{我想起之前的咨询里你也提到你觉得自己不需要咨询,好像这两者会有相似之处?}
	\dialogue{我}{嗯,其实我在对那个提问进行详述的时候,我身后就有一个自己在看着我自己说话,然后那时候就意识到,这好像就是我在咨询里说的话,没有什么太大的情感波动,不需要咨询。}
	\dialogue{咨询师}{所以你依然来做咨询,你是想从咨询里获得什么吗?}
	\dialogue{我}{我想就是自我探索吧。}
	\dialogue{咨询师}{那从那一次咨询你提出说你并不需要咨询,直到现在的这么长一段时间里,你会有什么感觉吗?}
	\dialogue{我}{我觉得自己就像是个小朋友,小时候的我会买完一个又一次的玩具。每次买新玩具都会很开心,但也并没有止步于此。而是继续买继续买,买完新的就渐渐忘记旧的。就是那种很开心的感觉。就像是一个又一个的新玩具,咨询里的一个又一个自我探索的收获。}
	\dialogue{咨询师}{你能多描述一下那种感觉吗?我有点不太理解。}
	\dialogue{我}{就像是一个画家,TA画完一幅画,觉得很好看,然后再画下一副下一副下一副,就是那种很开心的感觉。虽然一直在做重复的事情,但依然觉得很开心很开心。所以我也不太确定那种开心的源泉是什么。}
	\dialogue{咨询师}{我记得你在第二、三次咨询的时候会提到三种特定的情感:抑郁、孤独、无意义。我想到两种可能性,其中一种可能性是你可能在用这些创作来防御着那些情感。第二种可能性是那些情感还是在的,但你能看见更多其他的情感、更多其他的方面。}
	\dialogue{我}{嗯\~ 其实这两种可能性都存在着。我会用各种创作来防御或回避那些情感,但同时我也会体验到那些情感,那些情感依然还在,但现在有其他更多的情感。以前我只是仅仅体验着那些情感、身处于那些情感,但现在我能利用它们来创作出一些新的东西,而不仅仅是那些情感而已。比如说我可以写虚无感、无意义感、孤独感,写下那些情感本身就已经是一种创作,就已经可以让自己眼前的世界变得鲜艳起来。无法让我令眼前的世界鲜艳起来的是那种虚无,那不是一种情感,而是什么都没有,而我也无法利用那种虚无来创作些什么。就像是之前的咨询里,我想不到说什么,你也想不到说什么,然后我们两个人就在这片虚无里,我也无法令这个咨询室、周围的环境变得鲜艳起来。}
	\dialogue{咨询师}{那当你能推动咨询进行下去、当你能让周围的环境变得鲜活起来的时候呢?你会有什么感受?}
	\dialogue{我}{一方面,我会觉得很累,自己总是在推动些什么。另一方面,我觉得有新鲜的事物蛮开心的,也充满好奇和期待。这两者交织在一起的话,我的感觉就是还好。\\
		我会想起最近的雨雾天会让我感到很抑郁,会想到活着没有什么好干的、活着没有什么意思。然后我脑海里就会想起之前听过的一首纯音乐,然后再去看外面的雨雾天的时候,就会觉得雨雾天其实也蛮美的。活着没有什么好干的又如何?又能糟糕到哪呢?没有什么好做就没有什么好做咯。}
	\dialogue{咨询师}{我好像一时间找不到什么词来形容我现在的感受。我好像会觉得蛮感动的,你会有这样的变化。好像你不再急于去做些什么,就像之前的咨询里你会主动带一些东西进来,主动地推动咨询的发生。但现在我会越来越感觉到更加自然。}
	\dialogue{我}{嗯,我也在想我为什么会变成现在这个样子。就像你在咨询一开始想要捕捉的那个过程,我也想不到那个过程究竟是怎样的。并不是我写的哪些特定的文字,也不是哪次特定的咨询,好像重要的不是这个过程里经历的哪些具体的东西。而是在这个过程里,文字也好,咨询也好,生活的经历也好,而是这些一次又一次的经历。}
	\dialogue{咨询师}{也许这就是过程本身呢。也许现在这样的情感流动的状态更加重要。}
	\dialogue{我}{嗯,也许吧。而且如果我过于执着于过程的话,说不定我也会失去掉这种流动的感觉。}
}

然后我在咨询那天晚上作了一个梦。梦里的我在我自己的房间里拼拼图,然后用封箱胶将拼好的拼图固定,然后粘在墙上。因为刚拼好的拼图要和之前拼的拼图粘在一起,所以它们是连成一条的(因为它们是一个渐变的主题,像是壁画一样)。那一条封箱胶上的拼图就很重,从墙上粘好的位置上脱落了下来、垂到了旁边放的东西上。我觉得自己很狼狈,把脱落的地方粘上去的同时还要小心不能把之前拼好的拼图给弄脱落了。不过后来还是被自己给蹭掉了一些之前拼好的拼图。我看了一下靠前的拼图,那是一些有花纹和图案的砖块,还有一部由布满符文的材料制成的飞机(平面的拼图上贴着一个立体的飞机),靠后的拼图则是各种立体的街景。我蹭掉的就是靠前的拼图里的那些砖块。后来我索性拿剪刀把连接着靠前的拼图的封箱胶直接剪掉,把之前拼图的部分去掉,不把它们贴墙上了。然后我继续去沉浸于贴在墙上的新的拼图(街景)。

我会突然想起这个梦是因为,那些拼图就像是我在咨询里的收获也好、写作的内容也好、心路历程也好,我会在一直拼新的拼图的过程中非常快乐,但同时那些之前的拼图开始难以粘在墙上、开始脱落(记忆的遗忘)。而当我看见那些之前的拼图开始拖累到后来新的拼图的时候,我尝试去保留整个封箱胶的拼图带(这一路的过程),但并不成功,太沉了,粘不上去。后来我索性拿剪刀把过去的拼图给剪掉,不把它们粘在墙上了(接受过去的事物的被遗忘,接受过程的不完整)。然后去沉浸于贴在墙上的新拼图(享受咨询里新的收获、新的写作内容、新的心路历程等)。

