\chapter{神经质概念化模型,改变,内驱力,平静}

\ardate{2022-04-30}{bgY64kEAe0vT7AcGo9c5MA}



\begin{figure}
	\useimg{aimg/2022-0430-2.jpg}
	\midnote{五层次神经症模型}
\end{figure}

最近学的关于完型心理治疗的课程里又讲到关于神经质的概念化模型,这个模型是从完型的角度去看一个个体的状态是怎样的。

\blockquote{
	如果我们丢下角色,则到达第三层或者僵局层(impasse layer)。我们可能体验到我们的两部分卡在冲突中:想要完成未完成事件的健康的一面,以及想要避免痛苦的另一面。我们中的很多人想要避免体验僵局,因为我们想要避免为我们卡住的状态承担责任。我们倾向于否认存在性恐惧或焦虑,而这二者是当我们意识到自己的自由和局限时不可避免地体验到的。僵局的特征是感觉被卡住、混乱和焦虑,经常体验为非常不舒服。

	在僵局层后面,并且支持僵局层的是死层(death layer)或内爆层(implosive layer)。这一层是对立力量的崩塌。我们把自己拼凑起来,我们收紧肌肉,我们向内爆。我们相信如果我们向外爆的话,那么我们就活不下去,或不再被爱。

	如果个体真的面对僵局层的卡住和混乱的存在性焦虑,和内爆层死亡的感觉待在一起,那么最终他在外爆层(explosive layer)重获新生。在这一层,这个人是真实的,他可以体验和表达自己的真实情感。他外爆进入哀悼、愤怒、喜悦、大笑或高潮。他采取行动,他活着。“据我看,这是变得真实的必要一步。主要有四种类型的外爆:外爆为喜悦,外爆为哀悼,外爆为高潮,外爆为愤怒。有时候,外爆非常温和,这取决于向内爆层投射了多少能量。”

	\citebook{格式塔咨询与治疗技术(第三版)}
}

其中最外层的是虚伪层\pozhehao{}类似于人们为了满足社会角色而戴上的虚伪面具,但这样的虚伪面具并不是个体的内在本质,而只是为了适应社会的某种功能。

\blockquote{
	不过我逐渐意识到他的很多话语都是经过处理和包装的。我说:“我发现你在表达你自己的内容的时候都没有即兴发挥或及时化的东西,好像这些话早就对自己或对他人说过很多遍的。”他说每个人都有准备的痕迹,有的人多,有的人少。我说:“那你不会觉得这不真实吗?”他说:“每个人本来就在生活里扮演着许多不同的角色,最终这些角色加起来的就是那个人。”我说:“好像你没有办法分清真实和虚假?”他说(这)本来就没有真实和虚假之分。

	\blockquotesource{白色灯塔先生}{“这就足够了吗?”}{2022}
}

当课程讲到这里时,我回想起过年和前任见面时的聊天。现在回想起来,我会认为前任在那个当下所呈现的那种有所准备的状态就像是处于虚伪层的状态。他在说“每个人本来就在生活里扮演着许多不同的角色,最终这些角色加起来的就是那个人”的时候,我就在想:Are we merely the sum of our characters? 我们仅仅只是我们所扮演的角色之和吗?起码对我而言并不是这样的。如果要真正地活着,那么在真正地活着的那一刻的自己不能仅仅只是扮演着外界所要求自己扮演的某个角色,而是真正地活出自己原本的样子。

虚伪层的下一层是恐惧层\pozhehao{}恐惧于自己原本的样子,恐惧于自己无法完美地扮演虚伪层的角色。所以虚伪层和恐惧层两者经常会有冲突,一方面自己需要扮演某个特定的角色,另一方面自己的内在本质并不完全符合这一角色,而又对此产生了恐惧\pozhehao{}恐惧于他人会发现自己并不能时刻地完美扮演这一角色。

两者的冲突的背后就是第三层的僵局层\pozhehao{}自己处于进退两难的境地,个体可能会停滞在那里或选择去逃避或钝化自己的感觉。

我会想起一个之前认识的男生最近在朋友圈里评论道:“我有时候会看到你的公众号,感觉你状态不是很好,很多事情看得粗放点可能会更好哈,人间清醒是很痛苦的”。我记得我在刚开始写公众号的一两年里,总会有人(无论是在公众号还是朋友圈)评论一些“想开一点”之类的话,而当现在往回看,我会觉得那些话更像是他们在回避着某些议题,某些会激发他们的负面情感的议题。但回避这些议题并不代表它们就不存在了,所以他们还会去回避那些与这样的议题有关的人。

僵局层的下一层是内爆层。在看到僵局层的矛盾性后,通过这种矛盾性,自己开始思考真实的自己想要的是什么、是否有某些需求没有被完成(未完成事件)。当自己不准备逃避或逃无可逃的时候,便才能够接触到了真相。

内爆层的下一层是爆发层。个体开始展现勇气,去思考自己究竟应该怎么办才能有所改变,然后真的去做、去达成自己想要的、去处理那些未完成事件。

我会想到,要从虚伪层“走到”爆发层并不容易,也并不简单,特别是我的身边并没有一个这样的模范。

大概半年前,另一个曾经认识的男生问我是怎么做到改变的,那时候我回答不上来,只是觉得事情就这么顺其自然地发生了,比如说去学课程或去做个人体验或准备开始接电话热线。那时候的他想要改变,却改变不了,但其实在我看来,他拥有着我所羡慕的改变\pozhehao{}去另一个城市生活,但他并不知道这一点。如果从完型的这个概念化模型来看的话,他可能处于僵局层\pozhehao{}自己想做出改变但又做不到,就像是很多人(包括我)都会经历的一种状态:“道理我都懂,但就是做不到”。

最近的我在考研方面也处于这样的状态,既想要这么做,但又做不出改变。我记得一年前当自己想开始学习咨询时也是这样的状态和心情:觉得自己被卡住了,自己想要往那个方向去,但另一部分的力量又在把自己往回拉:“你不会成功了”“万一失败了怎么办”“现在的稳定生活万一被打破了怎么办”“这不值得”“过去都没有成功过,难道现在的你就会奇迹般地成功吗?不可能的”。后来自己真的踏上了学习咨询的历程,还越学越上瘾……

现在的我不完全像是一年前的自己,有相同的部分,但也有不同的部分\pozhehao{}同样是被卡住了,但自己并没有感到那么的卡住。与之相反的是,自己内心会感到很平静\pozhehao{}因为自己不止一次经历相似的境地,而且改变这种事情不是通过逼自己而达成的,而是通过倾听自己内心的声音、内心的欲求(内爆层),并将这种欲求实体化(爆发层)。在实体化欲求的过程中,改变作为一种副产物也将顺其自然地出现。而现在的这种平静感可能是因为我知道自己的内心空间是平静的,因为能时刻听到自己内心的情感和想法,而那些暂时的情感和想法又会随之消失,又会恢复平静,因此是否改变对我而言并不是那么的重要。改变很重要,但更重要的是:去倾听自己是个怎样的人、自己在此时此刻所欲求的是什么、又有什么在拉扯着自己不去满足自己的渴望。

这可能会是一个朋友跟我提起的“内驱力”。在我看来,内驱力就是在内心空间里自然而然地冒出的欲求,一股很强烈的desire,就像是饥饿感以及对性爱的渴望般日常。但当那些desire自然而然地冒出来后,是否能被知觉、识别、动员、行动、接触、满足、消退和休憩,那则是另一回事了。

\begin{figure}
	\useimg{aimg/2022-0430-3.jpg}
	\midnote{接触循环圈八步模型}
\end{figure}

尊重自己的内心、让自己的内心感到舒服和平静比是否改变重要得多。就像是在做冥想的时候,当察觉到自己在跟随着想法和情感跑的时候,自己有两个选择\pozhehao{}是选择继续跟随着各种想法和情感跑,还是离开他们回到“路边”享受路边的平静。在现实生活里,自己是选择跟随着“我一定要改变”“为什么我就不能改变”“为什么我那么无能”的想法和情感走?还是选择回到内心一直都存在着的平静里,看着那些想法和情感的流过?我会选择后者,即使这意味着不选择改变。如果躺平、摆烂是自己想要的,I will do just that。And I will be in my inner peace. 就像我昨晚跟另一个朋友说的:“为什么要逼自己做自己不想做的事情?”

