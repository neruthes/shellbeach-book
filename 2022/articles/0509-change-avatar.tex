\chapter{换头像}

\ardate{2022-05-08}{YD2fh125rBblwFZHBf5OaQ}



\useimg{aimg/2022-0509-1.jpg}

最近把上图的头像换走了。这张照片是三年前还和前任同居时拍的,是书桌旁固定着的一盏灯,然后前任把一张纸卷了一卷插进灯座里,于是便变成了火炬般的模样。当时的我就觉得这这样的创意很棒,所以就马上拍了下来。而之前我会想将这张照片作为头像是因为它就像是个白色的灯塔,a beacon of home,至少这对我而言是上一个家,或者至少是在我眼中的家,而不一定是前任眼中的家。

\midnote{\href{https://mp.weixin.qq.com/s/YD2fh125rBblwFZHBf5OaQ}{退群,投射}}

上周周末和前任喝茶时,我逐渐意识到,有很多东西都是我的投射,包括一些我对他的看法(独立、自由、喜欢深入交流、能理解他人的想法和感受)。但在喝茶后,我在同一天目睹了读者群里R先生退群,而当我在处理属于自己的部分时,我在用自己所熟悉的方式来处理,但在和前任喝茶聊天后,我的处理方式好像改变了。在处理的过程中,我不仅仅只是停留于看见那些部分就足够了,比如说不仅仅是看见那些自己的投射物本身就足够了,而是进一步地不断将自己内心的事物和他人分离出来,同时将内心事物从他人那、从现实世界里撤回。

在这个处理过程结束后,我会有平静感,甚至会有比以往的做法\pozhehao{}仅仅看见更深层的事物本身\pozhehao{}有着更平静的平静感。但虽然我不再有意识地处理情绪,推动这个进程,但这个进程好像在我的脑海深处依然不断运作着,开始在推翻一些更深层、更基底的事物。

第二天睡醒后,我突然感到很悲伤,以及一些那一刻的我还无法识别的情感,一些情感开始冒出来。我开始有一连串的想法:想到前任公寓的那个在我眼里的上一个“家”只是我的投射罢了,那一切从一开始的就不是真实的,就连前任在那时候也只是在半夜我所想要看见的角色。我不知道自己是谁,不知道自己为什么要活着,活着的意义是什么,觉得未来没有值得期盼的事物,不会有自己所期盼的相同或相似的家。如果不能达成、不能找到这样一个家的话,那自己就失去了做一切事情的任何动力了。没有任何想做去的事情,没有任何值得自己哪怕是挪动肢体的动力和理由。

然后脑海里的想法、思维开始停不下来,一直在转动时,我试图重新聚焦在情绪上,去问自己:自己此刻在感受到些什么?我感受到悲伤、丧失、幻灭、破碎、无意义、虚无、脱离、恍惚。

之前的我并没有这么多的想法或情绪,是因为家对我而言就像是一个锚,一个足以安定于现实世界以及内心世界的一个锚。但当我意识到我眼中的上一个“家”只是我的一种投射、一个幻影时,那个“家”的形象在我内心中开始幻灭。随着“家”里那个曾经的人\pozhehao{}前任在我眼中的形象开始幻灭,那个“家”也随之幻灭了。

对此我找不到任何答案、任何出路,从前任那也更不可能会有答案或出路。

我开始意识到自己可能又一次经历类似于身份认同危机的体验,又一次不知道自己是谁、自己要往哪里走、这一切的意义又是什么。随着心目中的家的幻灭,自己的很大一部分也开始支离破碎,而剩余的部分什么也拼凑不出来,什么都不是,谁也不是。我也更加不知道自己应该相信些什么。

这或许是为什么三年前的自己需要给自己建构一个家,同时将这个家投射到和前任的关系上、投射到前任的公寓上、投射到公寓里的我和他身上的原因。因为如果没有这样的家的话,那么那时候的我很可能就会感受到现在的我所感受到的情绪以及脑海里不断循环的想法,而那时候的我说不定也感受到了、也体验到了,而极力想通过建构一个家来回避掉很多很多自己找不到答案、找不到平静感、只会不断迷失在其中的东西。

自己在三年前丧失的是现实层面的家,而现在丧失的则是心理层面的家。当看清楚那个我眼中的家只是一个幻影而已后,自己也无法继续去维持一个幻想了。之前的自己也还幻想着,现在自己做的一些事情能让自己重新找回那个家或者是一个相似的家,一个能让我找回那盏黑暗当中的灯的道路,比如说去学心理咨询、去自我探索、去学更多的知识。以前的我似乎相信或期待自己的努力、自己所做的事情能让自己找回这样一个家或相似的家。但现在想起来,这两者并不是同一回事\pozhehao{}自己的努力并不会让自己找回那个家甚至是相似的家。

我眼中的家,或者是我所追求的家有一个固定的空间,有一个固定的人在,能有稳定和固定时间相互交流,两人之间能有彼此的在场、投入和连接。当然家里的那个人也要有一定的沟通能力和特质,比如说独立、自由、喜欢深入交流、能理解他人的想法和感受。而我努力的方向并不是必然会指向一个这样的家,而更应该是通过自己的努力去获得一个固定的空间和找到一个有自己想要的特质的人,然后再看是否能有之后的进展。

所以我好像依然“构建”了一个家,一个更为无形的家,不像是前任公寓那般丰富和具体。好像我的内心依旧需要有一个家的形象指引着自己前行,否则,便不会前行、不会改变了,只是单纯地存在着。

那种只是单纯地存在着的生活状态在之前也经历过大概几个月,在一个相识了半年的男生无故消失后自己有了自杀计划又放弃了之后的几个月,沉入到那种状态里,那种自己所珍重的事物和人全部消失不见、完全幻灭掉后的状态里。后来我和前任重新见上面了,自己开始不断想要寻求一个解释过去的变故的答案,同时也遇到了一年前曾经在这座城市的男生,和他相处的过程中,我自己能做、想做很多事情,而不仅仅是单纯地存在着。

在自己所珍重的事物和人全部消失不见、完全幻灭掉后,从那种仅仅存在着的状态里有所变化,我似乎需要一些重要他人,或者是我眼中的重要他人、视对方为唯一的人来kick start自己。


