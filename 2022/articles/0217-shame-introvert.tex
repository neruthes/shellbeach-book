\chapter{羞愧,“内向”}

\ardate{2022-02-17}{BSQnHqNcMvHBy\_5IZJfmCg}


今天在读《格式塔咨询与治疗技术》时,发现了一套解释羞愧的动力学解释,个人觉得蛮有趣的:

\blockquote{
	羞愧并不是个人自我脆弱的表现,而是调整人际接触的方式\pozhehao{}因担心被拒绝而退缩。因此羞愧是一种共同创建的人际关系,而不是个体自身心理弹性的不足或欠缺。

	羞愧可以使人回避不利场合或危险情景,具有接触调节或接触改善的功能,其利弊常常视特定的场条件(即环境)而定。

	内疚是针对“已做的事情”,是有条件的(是可恢复的);而羞愧往往涉及“我是谁”,是无条件的(主观上无法改变的)。羞愧被个体体验为完全不被接纳、无价值或有缺陷的感觉,这种消极感受往往导致个体不遗余力地掩饰或隐藏。随着时间的推移,羞愧变得如此根深蒂固以至个体常常毫无觉察,而仅仅表现出对批评和评判(或对表扬或赞美)的强烈反应,导致个体丧失脚踏实地思考的能力。

	羞愧意味着一种习得的人际边界,来源于儿童早期社会化过程中被要求学习和遵从基本的社会规范时,儿童习得了自己的行为不被赞许的后果\pozhehao{}被自己所仰赖的他人的长期拒绝,进而产生深深的羞愧感。

	判断自己的行为在什么时候是恰当的或不被赞许的,是社会关系中必不可少的重要能力。

	如果社会环境、社会化过程如果过于拘谨、偏激,最终可能导致个体形成严重缺陷而僵化的格式塔。一些儿童养育、教育方式以及宗教信念十分推崇羞愧,并提倡将羞愧作为评估个体的指标:“你应该为自己感到羞愧。”(而不是“你的行为不太妥当” )这种评价方式意味着儿童应认为自己是坏孩子,或者本质上是有缺陷的。

	羞愧与不被接受的需求之间产生了永久的联结,导致个体无法觉察自身的内在需求。换言之,需求丧失了发言权。然而,与羞愧紧密联结的需求并不因此而消失。所以,为了能始终把此类需求当作“非我”,为了能持续地与不支持其需求的环境和谐相处,个体在需求不自觉地浮现时,便会体验到羞愧。

	仅仅是尴尬或羞愧就足以成为调节社会交往和遵守社会接触的基本过程。但是,当这种调节变成僵化而有害的格式塔时,羞愧就成为一种对不被接纳、不被容许的事物的不健康的无条件反应。只要内心出现了微弱的愿望或需求,都可能产生自动的羞愧反应。
}

这首先让我想到的是内向。我的家里人从小到大都说我很内向\pozhehao{}在饭桌上不说话,也不主动认识新朋友。但他们不知道的是我从高中开始就用社交软件见不同年龄层、各行各业的人,他们不愿意去承认小时候对我的言语和肉体攻击给我带来的影响。他们对此的合理化是:“每个人都这样过来的”。

小时候的我(大概是上幼儿园和小学时期)很健谈,基本上是想到什么就说什么,所以会不经思考地说出一些有意或无意伤人的话,不过那些话的具体内容我已经想不起来了。家里人和亲戚当时对我的教养方式是辱骂和家暴,所以后来的我(现在也是)在他们面前越来越寡言,不会在他们面前说什么话,也不会透露多少关于自己生活的什么信息。

在踏上自我探索历程的路上,我开始发现压抑着自我表达的那份力量,是在保护着我自己,保护自己不要因为说错话而遭到言语和肉体上的攻击(虽然童年的经历早已过去了十几年)。这种保护的力量一直都在,而这种力量在意识层面上的感觉则是一种自我表达的自我压抑感。这种自我压抑的感觉并不好受,但无疑是十分适用的,适用于我的生存。但当我可以意识到那份自我表达的自我压抑感背后是一种试图自我保护的意图后,我便可以从以前的无意识自我保护转为现在的有意识自我保护\pozhehao{}有意识地保护起自己,不和家里人和亲戚建立起人际联系,不透露不必要的信息。

\tristarsepline

我记得大概是读小学时,我在家里人和亲戚的饭桌上,亲戚说起我在他们家住宿时(通常是寒假或暑假住一两个月),我说了很多关于家里人怎样怎样不好的事情。在我看来,亲戚的这种行为就是一种背叛:正是因为我无法在家里说这样的话,才会在亲戚家说的,但亲戚却向我的家里人告密。在饭桌上,先是亲戚说我这样做很不好,然后是家里人开始在饭桌上骂我,然后轮到老一辈的亲戚骂\pozhehao{}被全桌人骂完一圈再开始下一轮。最后回家也免不了挨打。

这样的经历重复得多了,我便每次吃饭时都会看书(那时候我还没自己的手机),遇到有亲戚或家里人想找我聊天我也只是听上几句就继续看书,任由对方想说什么就说什么。书籍从那时候便一定程度上变成了我的保护壳。后来不仅在饭桌上,在和亲戚和家里人有交集的场合中我也采取了相同的自我保护措施。

直到现在,我在饭桌上也依然手里有着一本书或 iPad,而当家里人和亲戚到现在也依然说我很内向(合理化我的人际疏离)时,我也不想去反驳些什么,毕竟这个“内向”的保护一直以来都蛮管用的。

\tristarsepline

所以,对我而言,我的“内向”并不是自身心理弹性的不足或欠缺,而是一种在特定的人际关系场中共同创建的人际关系,有我有所贡献的部分,也有他人需要对此承担责任的部分。这一“内向”使我得以回避不利的场合甚至是危险的情景,调节对自己而言存在危险性的他人的人际边界距离。我的“内向”并不是根深蒂固的,而我也不是对此毫无察觉的。在离开了特定的人际关系场(特定的他人)后,我可以自由地卸下这副铠甲,做我想做的人,成为我想成为的人。

我的“内向”意味着一种习得的人际边界,来源于早期社会化过程中被要求学习和遵从基本的社会规范时,我习得了自己的行为不被赞许的后果\pozhehao{}被自己周围的他人的长期拒绝甚至是言语和肉体上的攻击,进而产生了“内向”。因此这一“内向”也属于童年的我对周遭恶劣环境的创造性适应。

我的“内向”与不被接受的人际需求之间产生了联结,它阻止了我从家里人和亲戚身上获得人际关系连接,但也保护了我免遭他们的攻击。不过,我的人际需求并不会因此而消失,而我也能从更为外界的人际关系里(比如说朋友、面基等)满足自己的人际需求。

\tristarsepline

所以我会认为,即使一个个体再无力、再无能,他也能一定程度地创造性适应周遭的环境。但这不意味着他在这个过程中能免遭心理和肉体上的痛苦、折磨和孤独,也不能免遭他人将他的创造性适应视为病态的扭曲。不过,我认为更为重要的是,他还活着,他活了下来。

我活了下来。

