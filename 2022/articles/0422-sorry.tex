\chapter{对此,我很抱歉,但没有那么抱歉。}

\ardate{2022-04-22}{H-0sN8EJEgquJzwKbRKsew}


大概一年前,我去和一个心理咨询师“喝茶”\pozhehao{}差不多就是两个人喝着饮品聊天聊上一段时间。聊到后期,我会(主观地)发现ta会向我提建议,但每次提建议前都会先说优点,再说缺点,然后给建议。这样的句式多了,我很快就对ta的褒奖很敏感,因为我知道这之后肯定有一个“不过”,不然对方不会无缘无故地给出褒奖\pozhehao{}在那次对话里的好几次情形也确实如此。

在一个多月前,我跟ta提出来了这一点:

\blockquote{
我会想到一个词‘不同调’。在你的主观世界里,你付出了在乎和关心。在我的主观世界里,我能看见你的在乎,但那份在乎之后也夹杂了不少分析和剖析。这会给我一种感觉:好像你是为了后续的分析才有先前的在乎的。这种感觉也和我一年前和你喝茶时有所相似,那时候我注意到你会有很多先扬后抑的语句表达。

比如说,先说做得好的,再说‘不过…’。所以后来聊天的过程中,你每当说我的优点时,我都会很警惕,甚至没有在听你说我的优点,因为我知道肯定会有转折,后半句才是重点。这种不纯粹的在乎或称赞的感觉好像从第一次见面的时候就已经在我的内心里出现了。而随着陆续的留言里,这种感觉也一直延续着。
}

然后就彼此在文字上吵了一架,不过大家都没有争吵谁对谁错,而是各自说出自己的感受和想法。所以在那个情境里,我一直都不知道究竟是我的问题,还是ta的问题。

然后我最近上的课程里碰巧有提到相似的情形:

\blockquote{
日常生活当中你看到有人吵架,往往并没有思路那么清晰,是按照这个顺序,而且都涉及到。所以一般最常见的甚至训练有素的一些书里,告诉你说,如果你要对别人表达不满,你先要说欣赏,你还做的怎么不错的。但是我个人会觉得,但凡你沟通的对象是有点智商的,他都会知道一件事情,当你前面无论说得这么好,只要你一说但是,重点就在后面,而且后面通常是一个不好的东西。所以除非他的脑力不够,脑力不够他就会把重点听到前面去,听到前面去也不是你的目标,因为你之所以跟他谈是希望突出一个图形,这个图形是个抱怨……

甚至我听到一个很夸张的故事,我有一次点评,在朋友圈讲了类似的事情,然后有一个人就在底下留言分享,她说她的孩子才三四岁,因为她的企业的训练,所以经常所谓叫三明治法、三明治批评法,所以要先说一个表扬,再说个批评,再说个表扬,然后这样端出去才会比较好。所以她也是学了以后也是经常去运用,所以她就跟她的孩子,她孩子三四岁,有一天就她想教她一个事情,但她没有直接告诉她说她哪里有问题,而是先表扬宝贝,你今天表现真好。她说那孩子就这样看着她,她还在那边讲好的,想还在铺垫,那孩子说,妈妈后面你要说但是了吧,好你就直接说但是好了。那个留言让我印象很深,我后来想说,我原来一直说,但凡对方如果智力不够、你用这个方法三明治法可能有效,但是如果稍微有点智力,在职场当中高层管理者,你再用别人会觉得那是套路。但是那个留言刷新了我新的看法,原来三四岁的孩子都看得懂你在玩什么花招。
}

% “你还不如死了算了,生块叉烧都好过生你。”

这让我想到,虽然我对遭受批评有一定的“易感性”\pozhehao{}来源于学生时期的被言语攻击的经历。但我想对方肯定也有属于ta自己的部分,比如说我会主观地认为ta的句式里有好几次先说做得好的,再说‘不过…’,然后给建议的模式。

% “团督后的重复梦,丧失感和幻灭感”

另一次激发我对遭受批评的“易感性”的是电话热线的团体督导。在处理完自己的情绪、开始看见这些情绪的起源以及这些情绪对主观世界的扭曲后,当我开始以一种更清晰的视角去看督导师时,我更加觉得ta并不像是一个人\pozhehao{}缺乏对人(包括对来电者和接电员)的关怀,ta的话语里只有判断什么地方做得好、什么地方做得不好,没有任何人本的关怀。

因此,现在的我越来越确信,如果在一段人际关系里,我有任何不适甚至是反感或被攻击的感觉,那肯定有属于我自己的部分,但同时也肯定有属于对方的部分。正如在接热线时遇到一些令接热线体验很糟糕的来电者,我也同样相信,我有我的“贡献”,但对方也有ta自己的“贡献”。对此,我很抱歉,但没有那么抱歉。

后来那位心理咨询师说,没准督导师也需要有人反馈,给一个机会让督导师知道自己的心像石头。我回答说:“可能有机会再反馈?也可能不反馈,毕竟那是ta自己的问题,不是我的。” 然后ta继续问,难道我不想成为一面更好的镜子?我说:“不太想,(在这个过程里)我又没钱,又没乐趣,又没新发现,甚至还可能被攻击。”

如果在一段人际关系里,我有任何不适甚至是反感或被攻击的感觉,在探索完属于自己的部分后,再去看哪些部分是属于对方自己的,那么自己就可以不需要为属于对方的部分而负责。但我想,这个过程并不容易,因为这需要心智化他人和自己,同时也需要对自己有足够的自我探索以及愿意去这么做的勇气和决心。

我会想起一年前认识的那个离开了这座城市的男生,他也蛮喜欢说“He/She is not my problem to fix”,但我感觉当他充满反感地这么说的时候,他不仅仅是在划清人际界限,他也是在避免去探索属于他自己内心的部分,那些让他对此如此敏感的部分\pozhehao{}“易感性”。

