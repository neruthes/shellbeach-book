\chapter{“他更想说的或许是:你很悲哀”}

\ardate{2022-08-03}{PBnaiWUe7uGkSNm7mxjxtQ}




最近打算买一部新手机,但等到最后一天才发现我还要再等多一天,当时我就感到很失望,因为在这天之前就已经盼望了一段时间了,然后马上想到:要不不买了吧,要不不干了吧,好像也差不多哪里去,好像也没有什么区别,好像做什么也不会有多少改变。这种失望感让我马上联想到一个月前见了一面的朋友。那个朋友在见面时说我在往一个不好的趋势走,“眼里的光消失了”。

\midnote{\href{https://mp.weixin.qq.com/s/4Vw9gQeBOwDMyddRT8PCmA}{《“眼里的光消失了”》}}

其实在那次见面之前,我本以为他会很理解我,因为我们之间有着一些在我看来很相似的经历,例如都曾经有过自杀意图、有去找各自的心理咨询师、与家庭关系的恶劣、他也在上我一年前上的有关心理咨询的课程。他说他去找的是认知行为流派的咨询师,而且也持续了很长一段时间。他现在寻求心理咨询帮助的目的更多是想让自己不被情绪所左右。在那次见面后的几天,我在他的社交软件主页简介上发现了一句新增的话,大意是:人生最悲哀的,莫过于长大了,但眼中的光却消失了。我猜想在那次见面时,比起“你在往一个不好的趋势走”,他更想说的或许是:你很悲哀。

当发现两人之间相似的经历并不会带来理所当然的互相理解时,我会感到蛮失望、伤心和愤怒的。

同时我也会感到自己蛮孤独的,尤其是发现身边不少人都会用他们自己的主观现实来覆盖我的主观现实。比如说上周末第一次见的男生会认为在亲密关系的相处里两人依然需要遵守性角色的性格,当我说我不想在除了床以外的地方继续扮演这样的角色时,他会评判我的观点的对错,甚至会说“男人就是会出轨的”这样的话。当我说我在学习心理咨询方面的课程和在接心理热线时,他会说:“为什么你会学这个?你觉得你适合做这方面的事情吗?”而且,即使是认识多年的朋友也不例外,对方会认为我朝着一个不好的方向走,眼里的光消失了,甚至可能觉得我是悲哀的。

共情的基础是要以自己为模板,去往他人的心智上套,在无数的共情失败和成功当中筛选出那些与自己相似的人,与这样的人组成更为亲密、更具支持性的人际关系。但我甚至没有觉得他们是在共情我,而是觉得他们是在以他们自己的视角、站在他们自己的角度去看待和评判身边的事物和人。对于这样的人,我还蛮反感的。

当然,我有我自己对此反感的原因。我会想到,我妈就是那个在我的人生历程里一直在用她自己的主观现实试图覆盖我的主观现实的人,例如我听她说得最多的口头禅是:“别老是记着些不好的东西”、“一家人哪有什么隔夜仇”、“别和那些不好的人比”、“你真的是不会体谅人”、“一辈子很快就过去了,别想太多”。

我想起有一次和她说起我记得小时候被她打的经历时,她说“别老是记着些不好的东西”、“你怎么老是记住坏的不记住好的,也不记一记我对你有多好”。在她的主观现实里,她对我的好远大于坏,但在我的主观现实里,她对我的坏远大于好。但相比于承认我的主观现实的存在,她更情愿用她自己的主观现实来覆盖我的主观现实,期望将我的主观现实(坏远大于好)扭曲为她的主观现实(好远大于坏)的模样。

可惜这样的方法并不管用。如果通过将他人的主观现实扭曲为自己的主观现实的样子这一办法是管用的话,那么人与人之间也就不会有那么多的争吵、分歧和关系断裂。所以当看见身边的人(无论是第一次见面的人还是认识了好几年的人)在“故技重施”,我都会感到很恼火。这种对他人的主观现实的操控、对他人的内在的操控。

不过,我也知道,身边绝大多数人都有着这样的用他们自己的主观现实覆盖我的主观现实的行为,只是频率多少的差异而已,尤其是那些让他们感到痛苦的、极力去回避的部分。当我想到那些身边仅有的几个完全不会将他们自己的主观性覆盖我的主观性的人,我能想到他们的共同之处是:他们都愿意甚至是敢于去面对他们自己内心充满矛盾和痛苦的部分。而这也是我会感到和他们相处得更为舒服、更为喜欢的原因之一。


