\chapter{没那么喜欢,匹配,客体,物化}

\ardate{2022-06-08}{w8BJKA34K8\_5HT6v4f3CUA}




今天我在读者群里看到有位朋友在分享他和他对象的近况\pozhehao{}A说自己没有像以前那么喜欢B了,而B却说自己反而更加喜欢A。我说可能是因为两个人不匹配,然后另一个群友说他认为不是这样的,而是……,然后下一个群友继续给出下一个解释。

当那个朋友分享他和他对象的近况时,看完他写的文字,我会感到蛮伤心的,因为这让我想起在和前任分手时,我也是处于那个越来越喜欢对方的位置,而对方却变得(或者说从一开始就)没有那么的喜欢自己。那时候的我会不停地给分手、给前任口中所说的“不合适”找各种解释、找各种为什么,试图去解释这一切,试图将一切都解释为:这都是合理的、有前因后果的、可控的。

前段时间在和前任喝茶聊天后,公众号里有个留言说这是因为“不匹配”\pozhehao{}两个人在亲密关系里想要的东西不一样。我会认为这蛮对的,我在亲密关系里想要的是陪伴、共情和理解,以及我想要视对方为独一无二,同时也更想对方视我为独一无二\pozhehao{}想要对方眼中有我的一部分。但前任对亲密关系的看法是:只是个搭日子的,只是因为在社会层面上需要这么一个人的存在,想对方在身边就在身边、想对方消失就消失。

我也会想起一年前还在这座城市的男生,那时候我蛮喜欢他的,一下班就会去找他,想和他多呆在一起。但后来他说他想回家,我就说那好吧。他问我我就不会想他继续留在这里吗?我说我很想他选择继续留在这里,但很明显留在这里的你并不开心,毕竟你总是在说“我要回家我要回家”。如果回家是你觉得会感到开心和舒服的、是你真正想要的,那你就回去吧。

有时候我会认为,与其说是不匹配,这更像是彼此都不愿意为了对方而放弃一些自己真正想要的事物,比如说前任不会为了我而放弃他自己的部分;我不愿意为了一年前的那个男生而放弃生活在这座城市而跟他去他家那边的城市;那个男生也不愿意为了我而留在这座城市。而一年前的我之所以没有试图留住那个想要回家的男生,是因为如果我是他的话,我也不愿意为了另一个人而放弃自己真正渴望的事物,那份自认为的只要回到家就能够得到的开心和舒服。不过也可能是因为我会想到:对他而言他还有一个家,而我没有,我不应该阻止对方去享受那个我所无法拥有的家。所以好像每个人都选择了一个客体,而不是他人。

我也会想到,这样的解释是否将关系过于物化:只考虑自己和对方在这段关系里想要的是什么客体,而不在乎关系里的对方这个独一无二的人之存在。我想起昨天和一个男生面基时,他说他看重三个条件:身材好、外貌好、性技术好。如果三者能满足其中一个,就能做炮友;如果能满足其中两个,那就可以做对象了。我问他:“那对方为什么要选择你做炮友或对象呢?”如果只是用一种物化他人的角度去看待关系,那自己在关系里也不可避免地被自己和对方所物化。两人的关系变成了一个符号系统,彼此交换着各自想要的符号\pozhehao{}金钱、安定、舒服、安全、平稳、在场、陪伴、搭日子、满足需求……

回到读者群那个朋友分享的近况\pozhehao{}一个人越来越喜欢着另一个越来越不那么喜欢自己的人,在看完不同群友给的各种解释和解决办法后,我不认为会有一个适用于所有人的解释和解决办法,也更加怀疑那些解释和解决办法是否适用于那个朋友,因为那些解释和解决办法很大程度上带着每个人各自的投射。最重要是,当事人本人要靠自己的力量去寻找适用于ta自己的解释和应对办法\pozhehao{}去解释一些之前的自己所难以解释的事物甚至是变故,去应对现实世界的问题和内心世界的议题。

