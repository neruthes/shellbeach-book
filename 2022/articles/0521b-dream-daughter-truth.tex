\chapter{梦境 | 女儿,演唱会,相认,真相}

\ardate{2022-05-21}{RMrpHE61EIEpXUsFwmi7dQ}



我去参加了周杰伦女儿的演唱会。周杰伦女儿在演出的最后宣布自己的身份\pozhehao{}她是周杰伦的女儿,这场演出的目的是想和父亲相认。但在演出结束后,台下的她被周杰伦的粉丝攻击。我到了她的演唱会后台,吃着泡面,泡面下垫着她的合同,我就边吃着泡面边看着合同。周杰伦的女儿真的很想和她父亲相认,但网络上的粉丝都在攻击她,说她是骗子。她很愤世嫉俗,很想达成和父亲的相认,但又做不到。我跟她说:“人并不总是能够找到问题的答案,至少不是那个自己想要的答案。”当我这么说的时候,其实我是对自己说的,因为我想起我想要找到恢复和前任的关系的答案,但最终找到的答案却是幻灭\pozhehao{}自己内心曾经的家、内心曾经的幸福感和温暖感统统都只是内心世界的投射。

醒过来后,我大概能理清楚梦境里的不同元素分别象征着什么。

周杰伦\pozhehao{}我眼中的前任和最近很久没见又见了面的男生,他们都会给我一种感觉:他们的地位是我遥不可及、难以触碰、难以产生连接的。周杰伦女儿\pozhehao{}渴望与遥不可及的依恋对象建立联系的自己,经历幻灭前的自己。我\pozhehao{}经历幻灭后的自己试图跟那个经历幻灭前的自己(周杰伦女儿)说:你想要寻找的答案并不一定是你想要的,因为真相总是残酷的。演唱会\pozhehao{}经历幻灭前的自己(周杰伦女儿)试图通过自己的努力来与依恋对象建立连接的努力本身。网络上的粉丝\pozhehao{}批评自己的努力是徒劳的那部分更为苛刻的自我。合同\pozhehao{}我写下的文字,建立在试图通过自己的努力来与依恋对象建立连接的努力(演唱会)之上,也建立在渴望与遥不可及的依恋对象建立联系的自己(周杰伦女儿)之上。我检视着合同\pozhehao{}经历幻灭后的我往回看曾经的自己写下的文字、自己所作出的努力后,发现曾经的自己付出了努力,但最终找到的答案是多么的残酷、多么的令我崩溃。

\noindent\begin{minipage}{\linewidth}
	\center\bfseries
	“人并不总是能够找到问题的答案,至少不是那个自己想要的答案。

	The truth is always hard and bitter.

	It doesn't care about who you are and your effort along the way.

	The truth will not change for anyone or anything.”
\end{minipage}


