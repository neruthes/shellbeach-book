\chapter{短篇故事 | 咨询室}

\ardate{2022-06-19}{KvkBkxu7qPQkkJXOwJ7jFw}


在咨询室外的走廊里,0619坐在沙发上,等待着咨询室时间的开始,等待着咨询室的门打开的那一刻。走廊看似和以往一样\pozhehao{}挂在墙上的时钟,桌面上的纸巾和水,通道一旁的垃圾筐,沙发上褶皱且褪色的枕头,微弱的亮黄灯管。

等到那一刻真的到了,咨询室的门真的开了,0619就能进去了,就能说0619准备好的那些想说的内容。但0619又想到,他为什么要来咨询?这真的重要吗?有什么是重要的?这真的能改变什么吗?如果不能的话,为什么他还要去做一些毫无意义的事情?意义感又是什么?此时此刻真的还存在意义感这种东西吗?也许在这之前,很久很久之前,意义感是存在的,但现在不存在了,意义感已经消失很久很久了。

那对其他人而言呢?他们真的觉得这一切是有意义的吗?看似是的,或者说“听”似的是的,因为只要集中注意力,0619就能听到其他人脑海里的所思所想和感受。“这道菜真好吃”、“这里的风景真好”、“和谁谁谁聊天真开心”,但0619留意到他似乎不会听到其他人用到“意义”这个词,不会听到他们说:什么真有意义、和谁相处真有意义。

有时候0619并不想要听到他们的声音,并不想感受到他人的感受。但它们就像是白噪音\pozhehao{}它们就在那里,一直在那里,从来没有安静过。即使是此时此刻,那些声音依然响着,各种各样道别声以及同样的悲伤。0619想到,难道他们不知道所有人都能听到他们的声音吗?他们就不能安静一点吗?但0619同样害怕那些声音的消失,因为这意味着人越来越少了,而最近还在的声音越来越少。

0619意识到自己的情绪变得越来越烦躁,而这并不是他想要进入咨询室的状态。0619戴上了耳机,开始播放冥想引导语,开始将自己的意识锚定在呼吸上。随着呼吸的平缓,0619感觉脑海里那些吵杂的声音和情绪开始渐渐飘远,沉入意识的底部,褪为背景里的杂音。

在脑海里,0619开始回顾他打算跟咨询师说的内容:最近的生活、0619所喜欢的人、对这个世界的看法、对即将发生的事情的想法和感受。绝大多数的内容和之前的咨询差不多,以至于0619对于自己今天为什么还要来咨询感到更加困惑和迷茫。

0619想到,他其实可以不来,但又有哪里好去呢?也许其他人还能去见一见他们想见的人,做一些最后想做的事情。但0619想不到有这样的人、这样的事情的存在。如果0619跟咨询师说:“Hi,我今天过来只是因为我没有别的地方好去了。”不知道咨询师会有怎样什么想法和感受。不过,0619确实能知道咨询师的想法和感受,只要集中注意力就可以了。但0619并不想这么做,因为他并不想时刻知道另一个人的想法和感受,同时也不想时刻被人窥探到他自己的想法和感受。但,这并不是0619所能改变的。他想到:又有什么是真正属于我自己的呢?


咨询师的门打开了,咨询师正靠着门把上。在0619眼里,咨询师更像是需要把自己靠在门把上,而不是想要靠在门把上。而咨询师脸上的笑容依然让他感到无比僵硬。0619很快摆脱了这些他并不想让咨询师听到的想法,果断地走到椅子旁坐了下来。

看着旁边的窗外,透过淡蓝色的屏障,看着外面的天空,听着窗外的声音,0619留意到咨询师总是在开门前最后一刻才打开咨询室里的空调,但他没有一次想要问咨询师为什么要这么做。


\dialoguelist{咨询师}{
	\dialogue{0619}{Em……我还在适应这个环境。外面的天空好像越来越紫了,甚至还有不少雷电不停闪过。}
	\dialogue{咨询师}{你需要我把窗帘拉上吗?}
	\dialogue{0619}{不用不用,我还蛮喜欢看着窗外的,特别是看着附近的建筑崩解,看着瓦砾不停地往上浮,浮起来的同时又在不停地瓦解成灰尘,直到什么也没有。看着它们消失在天空会让我感到一种平静感,就像是以前会看着白云飘过或者是看着流水流过一样。}
	\dialogue{咨询师}{这个过程是什么?}
	\dialogue{0619}{可能是消逝的过程吧。比如说生命,比如说我们,比如说这个世界。即使关着窗户,我依然能隔着玻璃听到外面的建筑在崩解时的脆脆的声音。首先是一道很大的裂缝裂开,然后是一些后续许多很小的裂缝继续裂开,就像是将一个石头扔进水潭里,落入水面的声音慢慢扩散,逐渐消失。}
	\dialogue{咨询师}{听起来,好像你很享受这样的过程给你带来的感觉。}
	\dialogue{0619}{还好吧,也会感觉有点无奈,因为自己也改变不了这个过程的发生,什么都改变不了。}
	\dialogue{咨询师}{如果你能改变些什么呢?你会有怎样的设想吗?}
	\dialogue{0619}{如果……(笑了笑)如果我真的有这样的能力的话,我也想不到有什么想改变的。或者说,我想不到有什么改变是有意义的、有价值的,能让事情真的有所不同,能让事情和现在相比有所不同。\\
		我也会想到。不少人对这个世界做出了蛮多的改变,无论是现在的人还是过去的人,但最终依然没有真正改变些什么。这一切依然那么的毫无意义。\\
		也许会想让自己在乎的人还留在自己身边吧。}
	\dialogue{咨询师}{噢?”在乎的人。”}
	\dialogue{0619}{嗯,自己之前喜欢的那个男生死了。他的最后一刻选择和他家里人一起过,回家去了。我在脑海里再也听不到他的声音,无论我怎么集中注意力都听不到。然后我就想到:噢,他真的离开了,真正离开了。}
	\dialogue{咨询师}{据我所知,每个”死去”的人都没有真正离开,他们还活在我们的内心里,活在我们的集体意识的深处,成为了集体潜意识的一部分。但对你来说,好像他确实死了。}
	\dialogue{0619}{嗯。可能因为我再也看不见他的存在了吧。}
	\dialogue{咨询师}{好像能够看见他的存在对你来说真的很重要。}
	\dialogue{0619}{是啊。因为好像只有视觉上地存在,比如说这副身体,我才能将他和其他人区分开来。不然,每个人的内在似乎都是一模一样的,拥有着一模一样的意识。}
	\dialogue{咨询师}{那你此时此刻的感受是怎样的?}
	\dialogue{0619}{我会感到蛮伤心的、蛮难过的。但也会想到,这是他的选择,我又没办法去改变他的想法,也不想阻止他这么做。毕竟他还有一个家能回,而我没有,我总不可能要求他放弃我所没有的事物、我所渴望的事物。}
	\dialogue{咨询师}{那为什么你没有选择跟上去,跟着他去他家呢?而是选择留在了这里。}
	\dialogue{0619}{可能因为觉得不属于吧,觉得自己不属于那里,不属于他的家的一部分。}
	\dialogue{咨询师}{那你会觉得自己属于什么地方吗?}
	\dialogue{0619}{没有,哪里都不属于。}
	\dialoguesepline{咨询师}{(沉默)}
	\dialogue{0619}{其实我不知道自己为什么要来咨询,不知道咨询有什么意义。也不知道其他人来咨询是为了什么。}
	\dialogue{咨询师}{当你知道其他来访者也会来咨询,而你也来咨询,但你不知道其他来访者为什么要来咨询的时候,你会对其他来访者有怎样的设想吗?}
	\dialogue{0619}{不知道耶,好像设想不出来。我也不知道为什么我们现在要交谈,只是因为咨询是'talk' therapy吗?我们明明只要集中注意力就能知道对方在想些什么、感受到些什么了。就像是以前的人明明可以发信息,但却还要见面聊天。}
	\dialogue{咨询师}{是啊,为什么呢?}
	\dialogue{0619}{也许是一种念旧的方式吧。好像只要我们还继续交谈着,还见着面,这一切看似都没有改变过,好像这个世界还是和以前一样。继续自我欺骗。}
	\dialogue{咨询师}{在你看来,我们这次咨询会是一种自我欺骗吗?}
	\dialogue{0619}{不会耶,我觉得此时此刻和你在这里交流着,不知道为什么,我会感到舒服多了,没有那么心烦意乱,脑海里没有那么多吵杂的声音。你知道我们最近听到的声音是怎样的。}
	\dialogue{咨询师}{好像虽然你无法控制脑海里的那些声音,但你依然能控制今天来咨询。}
	\dialogue{0619}{嗯,会有一点掌控感吧,好像来咨询这件事依然是我能控制的,而不是像外面的世界一样崩解,不像外面的人一样失控,在不同的尖叫声中戛然而止。不过那些声音倒是越来越小了,可能因为人也越来越少了。}
	\dialogue{咨询师}{这会让你感受到些什么吗?}
	\dialogue{0619}{我会感到蛮孤独的。因为好像每个人都选择了接受命运,或者是那些不接受命运的人也就只能这样了。我就不是那种轻易就接受命运的人,不然我也不会在一开始来咨询了。}
	\dialogue{咨询师}{好像你依然希望通过咨询改变些什么。}
	\dialogue{0619}{是啊。那时候会想再挣扎一下,会想到:如果这一切真的只是这样了,就这样就结束了的话,那这一生多无趣啊。}
	\dialogue{咨询师}{”无趣”。}
	\dialogue{0619}{嗯,好像自己从很久很久之前就不知道自己是谁了。如果自己死了,那只不过是成为了集体无意识的其中一部分,其中一个微不足道的部分。而且好像一直以来自己都没有被视为一个个体,而只是集体里的其中一人罢了。如果我消失了,这个世界也不会有任何变化,世界上的其他人也不会有任何变化,其他人也随时能代替我,因为他们拥有着我生前的所有记忆和情感,就像他们所说的:”成为集体的一部分”。就连死亡都无法使自己摆脱这个集体,而是终于被彻底纳入为其中的一部分,其中一个与其他人毫无差别的东西。}
	\dialogue{咨询师}{如果你能成为一个个体呢?你会有怎样的设想吗?}
	\dialogue{0619}{我会想到以前的人是一个个个体,有着属于他们自己的独一无二的意识和情感。但他们依然很孤独,甚至比我们现在还要孤独。而且他们也和我们一样重复着,绝大多数的个体都是一模一样的人,只是占据着不同的躯体,但内在都相差无几。也许……以前的情况并不会更好,也许我只是在理想化以前的世界。}
	\dialogue{咨询师}{好像你意识到你在理想化。}
	\dialogue{0619}{嗯,因为这不会那么绝望吧。如果过去的事情真的比现在更美好的话,起码自己还能有一个憧憬,即使是一个完全无法抵达的憧憬。但如果过去真的没有比现在更好,如果以前就和现在一样糟糕的话,那……就真的蛮绝望的。}
	\dialoguesepline{咨询师}{(沉默)}
	\dialogue{咨询师}{我们剩余的时间可能不多了,你会想谈一谈屏障这件事吗?因为我们在之前的咨询里共同商定了在结束咨询的时候把屏障关掉,但我留意到你在这次咨询里好像没有想主动谈这个部分。}
	\dialogue{0619}{可能因为这对我来说不太重要吧。}
	\dialogue{咨询师}{但好像你之前说你不是那种轻易就接受命运的人。}
	\dialogue{0619}{嗯,但我又能做点什么呢?我又改变不了这个世界,阻止不了这一切。}
	\dialogue{咨询师}{我记得这个城市依然有几处避难所是会继续开着屏障的,让有的人把未竟之事完成。至少是在电力耗尽之前。}
	\dialogue{0619}{嗯。}
	\dialogue{咨询师}{而你并没有选择去那里,而是选择在这里。}
	\dialogue{0619}{因为会觉得没什么意义吧。无论是在哪里也好,在这里也好,对我来说都一样,都没有什么意义。}
	\dialogue{咨询师}{不知道在最后的时间里,你选择和我一起在咨询室里度过,你会有怎样的感受吗?}
	\dialogue{0619}{会觉得,这样也不错。}
	\dialogue{咨询师}{那我们只能谈到这里了。}
	\dialogue{0619}{嗯。}
}

0619和咨询师都看向窗外,看着淡蓝色的屏障逐渐消失,看着附近的建筑物随之崩解,不断浮在空中,逐渐消散。他闭上了双眼,感觉自己的身体越来越轻,离开了地面,感觉自己的意识和存在逐渐解离,直至不复存在。
