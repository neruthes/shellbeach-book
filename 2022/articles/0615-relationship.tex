\chapter{关系,贪求和嫌恶,体验}

\ardate{2022-06-15}{OdUrYJ3XsTaSrj7\_22x8jQ}


最近的生活状态又转变了过来。之前的生活状态是每天除了必须要去做的事情(工作、吃饭和睡觉)外,就只是在打游戏。但前几天我把游戏卸载了,因为发现自己投入再多的精力都无法获得多少新鲜的体验,比如说没有了新武器、新技能、新变异、新怪物、新地图等。就像是上周在咨询室里所说的,当新鲜的事物越来越少的时候,我就会放弃那个东西了吧,无论是游戏还是人。

这让我想到,前几天有个朋友分手了,ta的对象说自己没有感情了,就分手吧。我想起在和初恋相处时,我也干过这事,跟初恋说我对他没有了那种情侣的喜欢,也考虑过是否应该就此找另一个人过日子。一旦喜欢的感觉消失,就开始考虑换人,因为这样的感情生活越来越没意思,想永远保持在热恋里,即使是和不同的人的热恋。而当热恋高潮\pozhehao{}就像是orgasm一样\pozhehao{}在褪去后没有任何事物能够填补低谷的空洞、虚无和无意义感,那些本来就存在于自己的生活里,只是因为热恋而短暂退去了的事物。就像是用火柴点亮黑暗,用爱情、用亲密关系、用热恋、用orgasm、用人际关系来点亮那无意义的黑暗,但每根火柴终究会燃尽,所以只好找下一根火柴、下一个人。

前几天在看冥想课程时,讲师讲到贪求(Craving)和嫌恶(Aversion)这两种情感:

\blockquote{
	当我们评判一个东西它是负面的消极的是我们不喜欢的,就会触发像嫌恶这样的体验。那么一旦触发嫌恶体验的认知评价出来触发了嫌恶之后,我们一般的反应都是希望去摆脱当下的状况,我们希望能够尽快的离开这个状况,并且我们还希望未来再也不要出现这样的体验,我们想要规避它,我们想要躲开它。那么嫌恶这种情绪或者说我们讲这种反应,它在行动上或行为上看起来是什么样子呢?它就是一种希望双方拉远距离的这样一种反应,我希望我跟对方拉开。

	(贪求)这种体验在我们喜欢某个东西的时候触发,你要不喜欢一个东西就是嫌恶,你要是喜欢一个东西这个时候你所触发的就是贪求这样一种状态。那么它显然就是你对一个东西认知评估成正面的或者说是积极的,当有一个东西你觉得好你觉得喜欢你会产生怎么样的一个反应?我们会不允许自己释放掉当前正拥有的体验,然后我们会想如果这个好的体验可以一直继续该多好,希望此刻成永远这样一种心情。那么我们实际上说这个是一个很微妙的地方,人究竟是喜欢另一个东西另一个人还是喜欢这个东西这个人所带给自己的这种体验?那么不同的流派不同的这种哲思它可能有不同的理论,但是我们\textbf{从正念的角度来说,正念认为你所喜欢的是它带给你的体验,而可能并不一定是这个人或者这样东西而是这个体验本身,这也就意味着为什么有些时候这个人还是这个人,这个东西还是这个东西,但没有感觉了,感觉不一样了,那么你实际上爱上的是快乐的那种体验那种感觉}。

	那么贪求这种感觉它实际上在你体验上是什么样的?我们讲首先必然你觉得一个东西是积极的是正面的是好的,那么也必然会产生一个兴奋的快乐的一个情绪,我觉得好开心。\textbf{但是如果得不到你就会产生失望不满的情绪,也就是说贪求和嫌恶互相之间有些时候常常是互相转化的,当你贪求一个东西得不到它,立刻你可能就会变成嫌恶}。……贪求它在行为上是企图拉近对方的这样一种行为,我说嫌恶是拉远,贪求是拉近,……在这个贪求模式状态下,你要想背后是不是有嫌恶这种模式出现了,我不喜欢现在的生活,很多人贪求是什么?是我不喜欢现在的生活,我想要别的生活,我想要生活在别处,这也一样是一种贪求。
}

之后我在想,在多大程度上那个朋友的对象以及我只是喜欢那种体验、那个客体,而不是那个朋友这个人本身?以前在那段亲密关系里能收获到丰富的意义感,因此贪求,而之后发现在同一段亲密关系里自己不再能够获得多少意义感了,便嫌恶。但嫌恶并不是什么都没有,而是一种特定的情感。如果自己真的完全不在乎的话,那也就不会有嫌恶了、不会有任何情感了。那种对自我感觉这已经没有多少意义的关系的厌恶更像是一种不被满足所带来的失望、无意义、虚无、伤心、厌倦,还有渴望,渴望再去重新体验一次热恋高潮。正是因为还有渴望,这份渴望本身便是一份动力,一份推动自己去达成分手的动力。如果没有了动力,自己也不会去做任何事情。

昨天夜班加班后回到家躺在床上已经是中午,在床上半睡半醒地听着歌曲“Mystery of Love”的时候,想起了那部电影“Call Me by Your Name”,想到:为什么自己会喜欢那部电影?那部电影里的故事发生在一个遥远的国度的一个更为遥远的小镇上,两个男生在一段有限的时间里从最初的相识和试探到热恋直到结束和离开。一段有限的过程,就像是自己在亲密关系里所经历的那样,在三四个月的时间里就能走完从相识到热恋到结束的过程。

同时我也在想为什么,为什么能够在短时间内走完整个过程,为什么这个过程总是那么相似,为什么自己会在曾经如此热恋的对方身上无一例外地感受到对关系的厌倦,觉得生活越来越没有意义,只是无意义地重复着。

在我的记忆里,对爱的体验,最早是存在于和母亲的互动里。具体的事件我已经回忆不起来了,只有片刻的场景,比如说儿童医院的门口、附近的公交站和公交上、父母的房间里、拜神的佛庙、晚上的楼梯间里,那些场景之所以能够脱离具体事件地存在于我的脑海里,是因为我在那些地方都被我妈打得一直在哭。以至于当现在回忆起来,我搞不清楚为什么那时候我妈要打我,我也不知道为什么我要哭,也不知道为什么现在的自己依然能感受到封存在那些场景里的悲痛和无力,对被爱的憎恨和恐惧、对去爱的无力感。

无论是接受被爱,还是主动去爱,最终自己都会被伤害、被毁灭。所以在重复性地经历爱与受伤的矛盾性体验后,在和她的相处里,每当我发现自己好像靠得太近的时候,内心都会有一个声音:她马上就要变脸了、要打人了。然后就会把自己拉开,这样至少在她变脸、打人的时候,我不会被突然伤害到内心,起码受伤只是肉体,只是这副躯体。

我会想到,好像一直以来我都没有内化过什么富有意义的东西,一些能给自己带来持续的意义感的东西。在寻找一段恋爱、一段亲密关系时,我似乎更多是在寻找生活的意义、活着的意义,或者更具体而言,是那份充满意义感的体验,而不一定是某个特定的人。因为相比于他人,体验本身更加真实、更加安全、更加能够占有、更不那么复杂,更不会因此而受伤。而唯一能够继续感受到转瞬即逝的意义感的方式,也就只有不断追寻一次又一次的体验了。

