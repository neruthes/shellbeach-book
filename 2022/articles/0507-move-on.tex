\chapter{哀悼,并继续前行}

\ardate{2022-05-07}{IbVoy4c0eDO6zmD7B5LLbQ}





\dialoguelist{咨询师}{
	\dialogue{我}{我留意到你今天换了个袋子,(你)衣服上的图案让我想起小时候吃饭时在桌布上看见的图案。}
	\dialogue{咨询师}{当你看到这些变化的时候,你会有怎样的感觉吗?会是担心吗?}
	\dialogue{我}{嗯,会有点担心,可能是出于对规律的生活的变化的担心吧。}
	\dialogue{咨询师}{我会留意到这几次咨询你都会将你观察到的东西说出来。}
	\dialogue{我}{嗯。我会想起以前的我会有很多猜想,然后会信以为真,会把那些猜想当作现实。但现在的我只是会提出来。比如说半个小时前的事情,我和前任约了见面,然后我留意到和他约的两次见面都没有约饭,然后我就跟他提出来了。他说有什么想法就直说。当他这么说的时候,我回想起以前的我、当和他处于亲密关系时的我会把这些设想信以为真,比如说我会猜想他这两次都不和我约饭是因为他要在吃饭的时候吃药,而他不想让我知道他在吃药。不过现在的我会跟他提出来,跟他见面喝茶的时候再提出来,去核实,而不是信以为真地认为这就是事实。\\
		我也会想到,现在和他有喝茶聊天的时刻正是那时候的我想要的东西。那时候和他在公寓里同居时,他每天都上班,不会有休息日,如果我想要有交流的时间或找天去约个会都得不到。然后我就会猜想他会不会去找别人了,会不会有身体或精神上的出轨。然后我也确实向他提了出来,但他说没有。就像是我之前问他为什么他会无故消失了一年多甚至是两年,他说是因为工作忙。但我并不信服他的答案。而且后来我也从某些途径找到了他肉体出轨的证据,这好像也更加确信了我的猜想,我也更加把自己的猜想信以为真。但这就像是个死循环,我继续猜想,继续信以为真,然后又继续地猜想,继续信以为真。后来我分不清哪些是自己的猜想,哪些是现实。猜想是建立在现实之上的,但两者融合在了一起,我分不清哪些是真实的哪些不是。}
	\dialogue{咨询师}{他还说了什么让你不信服的吗?}
	\dialogue{我}{比如说有一次他回家的时候把房卡放在桌子上。我搜了下房卡上的酒店,是公寓斜对面的酒店。然后我猜想他和别人去开房。但为什么他和别人开房了还要把房卡拿回来?他是想激怒我吗?如果是的话,我应该被激怒吗,还是应该视而不见?他是在操控我吗?如果是的话,我就真的应该这么做吗?所以后来我没有跟他提这件事,就去睡觉了。睡醒之后,他问我有没有看见桌子上的房卡,我说看见了。他问我我就没有想问他那张房卡是怎么来的吗?我说没有。然后他就生气了,生气于他认为我不重视彼此的关系。但我不想提出来是因为我知道了他肉体出轨的事情,如果我提出来房卡的事情的话、如果我提出这个小出轨的事情的话,我怕我会想要把大出轨的事情提出来,那我们彼此仅存的关系就会完全破裂了,而那时候的我害怕和他的关系的破裂。不过过了段时间我还是问他那个房卡是怎么来的,他说是他捡回来的,叫我还回去。但我后来没有理他,没有把房卡换回去,他爱咋的咋的。}
	\dialogue{咨询师}{即使他说房卡是捡回来的,你也并不信服。}
	\dialogue{我}{嗯。}
	\dialogue{咨询师}{那你现在会怎么看那时候的情况?}
	\dialogue{我}{我会想到,那时候的我并不能心智化我自己。我能心智化对方,但我没有办法心智化我自己。我可以代入对方的位置去看事物,但我没有办法脱离我自己去看我自己。我没有办法看到那时候的情绪,比如说愤怒以及愤怒背后的恐惧一直驱使着我从逻辑上进行着各种推论甚至是妄想,也没有办法看到是因为那时候的愤怒才驱使着我说了很多、做了很多不像是平时的我会做的事情。}
	\dialogue{咨询师}{那如果是现在呢?你会怎么做?}
	\dialogue{我}{如果是现在的话,我会直接表达说:‘我并不信服你说的话’。我知道这并不能改变些什么,我也很可能得不到我想要的答案,我也改变不了彼此的关系,我更改变不了他。但我依然会这么做。\\
		以前的我不会,因为以前我好像一定要得到一个确切的答案。那个答案对我来说很重要,如果我得不到的话,我就压根不会提出来。而且以前我会觉得即使我提出来了,我也不可能得到一个确切的答案,所以为什么我还要说出来。但最近,我在跟那些无故消失的男生表达我对他们的愤怒和憎恨后,虽然改变不了现实,对方也不会给我一个确切的答案,但我发现自己在表达了愤怒和憎恨之后就不像是以前那样那么地渴望一个确切的答案。在表达完自己的情感后,我会感到一定程度的平静感。\\
		以前的我也意识到我并不是一定要得到一个确切的答案,我只是需要得到这个答案背后的那个平静感。但我靠自己得不到。这也是我开始学咨询的原因之一,因为我想要得到一个能解释对方的所作所为的答案。}
	\dialogue{咨询师}{那你学咨询也学了那么久,你觉得你在学咨询的过程中有什么变化吗?}
	\dialogue{我}{一开始我是找到了很多可能的答案,比如说能用很多咨询里的理论知识来解释对方的所作所为,会有一种平静感。但这种平静感并不稳定,很容易就会消失了,因为虽然我能找到很多可能的解释,但并没有任何一个是确切的答案、唯一的真相。而且我也无法从对方那得到唯一的答案。后来我将焦点转移到自身,去运用那些知识对自己的情感进行工作,比如说冥想的时候去回溯情感,去看那背后会有些什么。当我能看属于自己的部分时,我就能够得到属于自己的平静感。\\
		我会‘易感性’来形容自己的部分,毕竟那些情感都是属于我自己的,而我能找到的也只有属于自己的答案,我无法找到对方的答案,我能找到的也只有属于自己的部分的平静感。我也无法改变他人,就像是我无法叫你提早半个小时来咨询室开门,我也无法占据前任心目中的长期伴侣的地位。但起码我能在我自己的领域里、范围里尽可能找到属于我自己的平静感,我能知道属于自己的部分的来龙去脉。当我能这么做的时候,那些情感本来像是很大的海浪,但我能够将它们变成浪花。而我学咨询的知识本身更像是拐杖,我只是利用他们来写下属于自己的唯一的真相,而不是用那些理论知识来自我说服、自我欺骗。因为那么做对我来说并不work,我并不信服那些理论知识就是对的、就是唯一的答案,所以他们对我来说也不管用。}
	\dialogue{咨询师}{这会让我想到一个比喻:那些知识就像是一个个单词,但你用那些单词组成了自己的语法。}
	\dialogue{我}{嗯。其实当我回顾这次的咨询历程时,我觉得我们在兜圈,但我又没有那种被困住的感觉或绝望感。有时候我接热线或者是听朋友倾诉的时候会发现对方在兜圈,然后我会感到被困住甚至是绝望。有时候我会直接跟对方表露说:‘听起来好像有点无力甚至是绝望’。但现在并没有,就好像一个圆,或者说是格式塔的体验循环,好像这个圆完整了,而我也不再像以前那样试图用一个确切的答案来填补它。}
	\dialogue{咨询师}{你能多解释一下兜圈吗?我有点没理解那个圆是什么。}
	\dialogue{我}{Em……就像是在咨询里你有两三次激发起我的情感,但我都通过自己的方式将海浪降为了浪花,比如说那时候处于亲密关系时的被困住的感觉、不安全感、想要去猜疑、恐惧、甚至是绝望感。但我都通过自己的方式处理掉了那些情绪,比如说我能在属于自己的部分里找到平静感。}
	\dialogue{咨询师}{那我大概理解你所说的圆了,好像你都能通过说出来的方式淡化过去的那些情感。}
	\dialogue{我}{嗯。}
	\dialogue{咨询师}{我看到你皱眉了。你会是有怎样的情感吗?}
	\dialogue{我}{我会有点担心这样的方式会不会变得太程序化了、太方法论了。好像每次我都能熟练地用同一种方式处理好自己的情感。但没有挑战就没有成长。}
	\dialogue{咨询师}{我会想到一个词:新鲜感。好像你在警惕和排斥固化。}
	\dialogue{我}{嗯。我会担心固化,特别是关于自己的固化。会很警惕和排斥关于自己的固化。}
	\dialogue{咨询师}{这次的时间差不多了,我想我们能在后续的咨询里继续谈论这种对固化的警惕和排斥。}
	\dialogue{我}{嗯。}
}

离开咨询室后,我去和前任喝茶聊天。他说我口中的“长期伴侣”只是我的一种投射,我承认是这样的,因为他眼中的长期伴侣和我所设想的长期伴侣并相同,甚至是有很大的不同。他说我在赋权他,向他赋权一个父亲的角色,认为他有着我所寻找的答案。我说我并没有,我需要的是他的在场感,我想和他约喝茶并不是一定要找到那个关于他的长期伴侣是怎样的答案,我更想要的是和他建立起连接,因为这种在场感能让我感到没那么孤独。

我也跟他说了我对他两次都没约饭的猜想:我猜是因为他吃饭时需要吃药,而他不想让我知道他在吃药。他否定了这样的猜想,并说我脑子里总是构建着各种不切实际的世界,说我没有验证逻辑。我说是的,所以我才要拿出来跟对方、跟他核实,因为这只是猜想而已,而我提出来跟对方核实这一步就是在验证自己的猜想。他说我还会有下一次的猜想,我说是的。

我说我想我之所以总是在猜想那么多,是因为我的成长环境里并不会告诉我我想要得到的答案,比如说我问我妈为什么我会诞生,她说我是从垃圾桶里捡回来的;我问她为什么我要上学,她说其他人都上学为什么你不上学。有很多提问是即使我问了也不会得到答案的,所以只能靠自己猜,不然不会有确定感。他也指出了我在他身上投射了确定感。我说是的,在我眼中他是一个很自大的人,总是对任何事物和人都有着确定的答案,而我会想获得那种确定感。他问我那真的是确定感吗?我说是的。

喝茶结束后,当开始回想他所说的投射,我会想起之前读者群里有人问我“怎么对前任还念念不忘”,我说:“因为前任身上有很多自己感到很吸引的特质吧,比如说独立、自由、喜欢深入交流、能理解他人的想法和感受。”但现在我意识到,这些特质绝大多数都只是我对他的投射。他在我面前呈现的部分里并没有独立和自由,他也不喜欢深入交流,甚至和我喝茶都只是因为我“有这样的需要”而他有这样的场地和时间就发生了,这是为了我而发生的,喝茶与他无关,这并不是他想要的,他没有什么想要的。在聊天的过程中,我也感觉到他并不能完全理解我,并没有看到我全部的样子\pozhehao{}他说我和上次喝茶时的状态是一样的,提问的内容也是一样的,只是换了个壳,他甚至觉得这次喝茶是浪费时间。

我意识到他眼中的世界和我眼中的世界很不一样,而我也无法从他身上获得我所想要的和他的连接。世界不一样、彼此想要的东西也不一样,而他在我身上什么也不想要、什么也不需要。我想我需要一段时间来哀悼内心的前任形象的丧失。他的世界在我看来很monochrome,而我想要的世界是更为多彩的、更富有意义的,而他眼中的世界更像是缺乏欲求的。我跟他说,他的世界里好像充满着虚无,他说那不是虚无,虚无是不稳的,总是要找抓手的,而他认为那是稳固,稳固就在那里,稳定不动。

我想到这或许会是虚无的下一层。我在最近一年里都在用持续不断的学习来逃避虚无,如果我真的陷入了虚无里呢,虚无的下一层会是什么,会是前任口中的稳固吗,亦或者会是其他的东西。我会想陷进去虚无里去看下一层会是什么。同时我也知道自己要继续去寻找我所想要的事物了,因为我想要寻找的自我并不在他身上。是时候去探索自己想要成为一个怎样的人,而不再从他人身上、从前任身上去寻找我所想要成为的那个自我。

To accept that I can't get what I want from him, and that he was, is and will never be the kind of person that I want and desire. To mourn and carry on.

