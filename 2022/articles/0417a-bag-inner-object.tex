\chapter{袋子,深层情感,内在客体}

\ardate{2022-04-17}{FybT2GCaOJedrBvym7c1XA}



\dialoguelist{咨询师}{%
\dialoguesepline{咨询师}{(彼此沉默)}
\dialogue{我}{其实我的注意力一直在你的袋子上。我发现你会带几个袋子进来,会把其中一个袋子放在旁边(的座位上),另一个则放在地上。之前你也有带袋子,但不是这种非一次性的袋子,而是那种购物的纸袋。}
\dialogue{咨询师}{好像你之前也有留意到我带了些袋子进来。}
\dialogue{我}{嗯。之前是偶尔会发现有袋子,但最近经常总是有袋子,所以注意力就会被转移到袋子上。}
\dialogue{咨询师}{我把袋子带进来会给你一种怎样的想象吗?}
\dialogue{我}{我会想到……会想到买菜的大妈,提着一袋又一袋的菜。}
\dialogue{咨询师}{(笑了笑)}
\dialogue{我}{虽然每次你都会带袋子进来,但我都看不见里面会是些什么内容。}
\dialogue{咨询师}{我把袋子带了进来,但没有把袋子里的内容呈现给你看,这会给你一种怎样的感觉?会是不安吗?}
\dialogue{我}{Em……好像还没有达到不安,只是隐隐约约的一种介意感。只是觉得你把自己包裹得很好。我最近打热线里的热线志愿者也是这样,会回避自我暴露。但我接热线的时候会有不少自我暴露,所以我也能理解为什么有的来电者总是说‘如果是你,你会怎么做’、‘你真的能理解我的感觉吗’,那时候我会做适当的自我暴露,然后对方就会安心地继续说下去。好像这个领域的人都不喜欢自我暴露。我会想到,如果我在咨询里也不自我暴露,那我们还做什么咨询。}
\dialogue{咨询师}{我更少地表露我自己的内容,而你表露了很多你自己的内容。这会让你有怎样的感觉吗?}
\dialogue{我}{Em……我会感到……有一种感觉开始冒出来,但我还说不上那种感觉是什么……}
\dialogue{咨询师}{(沉默)}
\dialogue{我}{我发现我总是关系里那个推动着关系往前行的人\pozhehao{}面基也是,和前任的相处也是,和最近喜欢的那个男生也是。总是自己在推动着见面的机会、相处的机会。}
\dialogue{咨询师}{你会有怎样的感觉吗?}
\dialogue{我}{我会觉得很累。就是……很累。}
\dialoguesepline{咨询师}{(彼此沉默)}
\dialogue{咨询师}{我留意到了你的皱眉和沉默。}
\dialogue{我}{嗯。其实我刚刚走神了,我突然沉进去了那个场景里。最近喜欢的那个男生说我们还是保持朋友关系,我说:‘嗯,可以呀。’但我会感到伤心和被遗弃感。然后刚刚就是沉进去了他这么说的时候的场景,他的话在我的脑海里就像是单曲循环地一次又一次地重复在我的耳边。}
\dialogue{咨询师}{为什么他的话会一直重复在你耳边,而且你是沉了进去?}
\dialogue{我}{我想可能是因为这当中有很强烈的被遗弃感,而自己也被沉了进去。}
\dialoguesepline{咨询师}{(彼此沉默)}
\dialogue{咨询师}{好像你也就停在了这里。}
\dialogue{我}{嗯,我会觉得有点累吧。\\
	我想这可能和我这周的状态有关。我可能有点抑郁了,不过是从症状方面来看的。我发现自己在这周的状态很奇怪,但又不知道为什么奇怪。在这周的一天晚上,我接了几个热线后打算下线,然后看见有个志愿者的名字叫‘猫叔’,然后就连通过去作为来电者和他聊天。我想知道他的名字为什么叫‘猫叔’,但他也像是你这样通过把焦点转移到我身上来回避掉提问。我想,那我就聊聊我最近的状态、那种奇怪的感觉吧。最近一周都对事情没什么兴趣,不过不是那种表浅的兴趣,因为课程学习和接热线也还在继续着,而是那种深层的兴趣。\\
	最近我参加的一个焦虑训练营里的最后一个作业是需要记录下最近一周里的三个幸福感的瞬间,然后我觉得这也太难了吧,给我几年我都不一定写得出来。他问我还记得什么幸福感的时刻吗。我回忆起三年前和前任去沙滩的时候,我是感到幸福的,但我不知道为什么自己会感到幸福,幸福感的本质是什么。后来他问起亲密关系的时候,我才发现原来自己每次在亲密关系里\pozhehao{}不一定是严格意义上的伴侣,即使是面对自己喜欢的人也可以\pozhehao{}都能感受到很丰富的情感,比如说开心、幸福感或者是伤心或悲伤。但只要这个人消失了,我就会感受不到什么感觉,表浅的感觉还是会有的,但没有了深层的感觉。}
\dialogue{咨询师}{我在刚刚听到的内容,我好像听到两个状态的你,一个是在亲密关系里能感觉到更深层的情感的你,另一个是不在亲密关系里的你会感受得很浅。}
\dialogue{我}{嗯。不过当我想到现在的这种状态是因为一周半前那个男生拒绝了我之后,我又有了更深的悲伤和难过,这让我的情绪又有了更深入的感觉。但除此之外,其他方面就和这周的状态差不多,没有多少情感。所以我也跟那个志愿者说,我好像需要另一个人才能让自己感受到些什么。}
\dialogue{咨询师}{听起来好像你渴望和另一个人有更深的联系或体验。}
\dialogue{我}{嗯,会是。}
\dialogue{咨询师}{为什么会是这样呢?这对你来说的意义是什么?}
\dialogue{我}{因为我会想看见对方更多的部分、更丰富的部分……但我好像总是在失败,对方总是会退回去或者无故消失,比如说前任是这样,之后认识的一个相处了半年的男生是这样,一年前喜欢的一个离开了这座城市的男生也是这样,最近自己喜欢的那个男生又是这样。}
\dialogue{咨询师}{为什么对方总是退回去或无故消失呢?}
\dialogue{我}{(叹气)……我不知道。好像总是因为自己做了些什么而导致了这样的结局。我好像总是很容易就能触及到对方不想被触碰或回避的地方,也许可以说是人格僵化或雷区。而且我需要的时间也很短,通常两三个月就能触碰到这些部分。前任的话,他的雷区是他的个人生活甚至是性癖好。相处了半年的男生的话,是他生活里的有伴侣的那个部分,我从一开始见面就感觉很不对劲。一年前离开了这座城市的男生的雷区是共情,他拒绝去共情他人,虽然他总是在无意识地共情他人,但他拒绝有意识地共情他人,他会说他不care他人,但我感觉他只是在维护着他自己的自恋,因为我能从他的话里听到背后的悲伤和愤怒。至于最近喜欢的那个男生,他是有对象的,所以他会设定界限,我们之间的关系不能太深入,不然他担心他的对象会发现他的心思还放在另一个人身上,他不想失去他所拥有的那段关系。}
\dialogue{咨询师}{我会想起在上一次咨询里,你提到的小时候的毒舌。这也会给我一种类似的感觉。好像现在的你把语言\pozhehao{}那些容易伤害到他人的语言给撤了回来,但依然还这么做。}
\dialogue{我}{……嗯……好像是的,好像即使在言语上撤了回来,但我依然在人际互动里无意识地这样做着。}
\dialogue{咨询师}{嗯,所以为什么会是这样呢?}
\dialogue{我}{我想起以前我都是在脑海里构建起一个场景来给自己安全感,但现在我好像是在内心里构建起一个对方的形象,可能是一种内在客体。就像是雕塑家,雕塑着对方的形象,但当自己发现对方的某个部分空缺了的时候,就会觉得很奇怪,就会想突破对方的界限。}
\dialogue{咨询师}{好像你在通过内心的那个形象来与外界的对方互动。}
\dialogue{我}{嗯,每个人都只能看到自己想看到的样子,那个客体。我想心理学里讲的也是这样,通过和自己的内在客体与外界的他人互动。但我发现哪里缺了就会想填上去。}
\dialogue{咨询师}{你会是在这个过程中追求些什么吗?}
\dialogue{我}{我可能是在追求一个更完整、更鲜活、更生动的形象吧。这会让我感到没那么孤独。但是每次的尝试都走不到终点,只是走到中途对方就撤了回去或者是无故消失了,然后我就会感觉到很强烈的被遗弃感。就像最近那个我喜欢的男生说,他不能跟我一起走到我想要走的目的地。}
\dialogue{咨询师}{一个更鲜活、更生动的形象。你对此会有怎样的设想吗?那个历程的终点会是什么?}
\dialogue{我}{我好像设想不到什么……这可能也是那个男生会说不能跟我走到我想要走的目的地的原因吧,他好像能看到我是朝着某个方向、目的地走的。但那个方向好像就只是一种感觉,就是能和这样的人在一起的时候自己会感到很兴奋和幸福。他这个客体需要是一个容器,能容纳我对他的情感以及我对未来的期望。可能像是课程里说的‘涵容’和‘护持’,能容纳我的情感,并支持我去探索……我想起我从小到大都一直没有过这样的人的存在,现在也没有。我好像一直以来都在追求着这样的人,试图在内心里构建出一个这样的人。\\
	就像是咨询一开始的袋子也是这样,就像是一个完型或者说一个环,看见这个环缺了一处就会想把它填上去,但又一次又一次地受挫。如果失败得太多就会放弃了,会换个人,但自己依然渴望着能有这样的人在,而且我之后说不定也会继续这样做。}
\dialogue{咨询师}{当你说之后也会这样做,会是什么意思吗?}
\dialogue{我}{Em……好像现在的我能意识到这种模式的存在,这种模式一直贯穿着我的人际互动里。但当我意识到这个模式的时候,我并不想去改变它,因为这是我从小到大都一直无意识或有意识想要的,这是我渴望的,而我也会继续追求下去。}
}
