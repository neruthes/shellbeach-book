\chapter{剧本,意义}

\ardate{2022-06-19}{CfPEhtIIUjNTiLOAJpFJ3w}



\dialoguelist{咨询师}{
	\dialogue{我}{刚刚我在穿衣服的时候,留意到你的表情和刚刚不太一样。我会发现每次你开门的动作、坐下的动作和表情都很像,就像是一个流程,呈现出一个很欢迎的状态。但刚刚穿衣服的时候我看到你闪过的一个表情,是和这整个基调不一样的表情。}
	\dialogue{咨询师}{你会有怎样的猜想吗?当你看到那个表情的时候。}
	\dialogue{我}{我会猜想,只是猜想,当你每次都呈现那个欢迎的状态,但刚刚闪过的那个脱离基调的表情时,我会猜想你其实并不想总是呈现那个状态,我也会猜想你会不会很疲惫,一直需要呈现欢迎的样子。}
	\dialogue{咨询师}{这样的猜想,你还会想到些什么吗?}
	\dialogue{我}{我会想起,在和前任的相处里,我也会有很多猜想。前任会将这称之为“剧本”,说我的内心世界很丰富,剧本很多。我很讨厌他这么说,但我确实会有很多设想,比如说之前有一次喝茶聊天时,我说要不要约个午饭。他说不了。我就在猜,他会不会是因为午饭要吃精神类的药物而不想让我知道。他说不是,然后就说我的剧本很丰富,说为什么我没有猜他就是不想和我约饭或者是已经和其他人有约呢。}
	\dialogue{咨询师}{其实我也会留意到你和我之间也有这样的互动,比如说你会看我的袋子里的东西、看我穿的衣服、看我的表情、看我穿的衣服。而你的剧本(虽然你并不喜欢用这个词)会围绕着某个方向吗?如果继续发展下去。}
	\dialogue{我}{嗯,我会注意到我的剧本好像都会朝着一个很不好的方向走。我刚刚在写短篇故事的时候就注意到,我写得越来越黑暗、悲观、无意义、虚无,越写我的情绪越低落,越来越陷进去。}
	\dialogue{咨询师}{那和我的互动里呢,你的剧本会是怎样的主题?}
	\dialogue{我}{好像会是,比如说拒绝、回避、隔阂。}
	\dialogue{咨询师}{这会让你想到些什么吗?}
	\dialogue{我}{我在试图回溯这种感觉,让我想到我和父母的互动。在我眼里,我父母就像是在演戏,他们并不想在一起,但还是在演着戏,维持着表象。比如说小学三、四年级的时候我在他们房间的衣柜里发现性生活用品。那时候我早就知道性生活的存在,所以会觉得很正常,但当看见那些性生活用品的时候,我会觉得他们把我排外了。比如说小时候我问他们我是怎么诞生的,他们说是从垃圾桶里捡回来的。后来我知道自己是怎么来的,但当回想起他们那时候给的解释时,我会觉得他们是选择了一个与他们无关的解释,而不是选择一个与他们有关的解释。就会感觉到我被排除在外了,我不是他们家的一部分。}
	\dialogue{咨询师}{他们好像隐藏了一部分的生活没有表露给你。}
	\dialogue{我}{嗯。好像那时候的他们在演戏,而现在轮到我演戏。我在维系着生活的表象,比如说还在做一些规律性的东西,例如上班、上网课、来咨询。不过我也会想到,如果那时候的他们不维持这个表象的话,那会对那时候的我带来多大的冲击,就像是如果现在的我不维持着生活的表象,那会对现在的我有多大的冲击。}
	\dialogue{咨询师}{但你依然还在维持着这个表象,依然还来咨询、还去上班、去上网课。好像你依然渴望着什么。}
	\dialogue{我}{可能是……渴望生活能有点意义吧,期望能有一个事物或人来到自己的生活里,然后我的生活就突然充满了意义,就像之前喜欢的男生出现的时候,自己瞬间觉得生活充满了意义。但如果我不去维持这个表象的话,那就只能这样了。}
	\dialogue{咨询师}{好像那个渴望让你维持着那么多的表象。}
	\dialogue{我}{嗯,也许那份渴望并没有我所设想的、我所感受到的那么弱小、那么微小,而是力量很大吧。但同时我又很抗拒那个部分,抗拒那份渴望,因为有期望就会有失望。如果生活中的表象都只是在无意识地运行着,看似是它们自己在延续着,而不是我选择让他们继续下去的话,那么这些事情好像都不太费力。但如果是自己选择让他们继续下去的话,我会觉得很累很累。}
	\dialoguesepline{咨询师}{……}
	\dialogue{咨询师}{那你想要看见父母的那个部分是什么?}
	\dialogue{我}{可能是他们性的部分吧。因为他们在我面前就像是一直在演戏,比如说聊着工作怎么样、吃的怎么样,但这些都是很表浅的情绪,没有深入的东西在,而且他们也不会找我谈论内心的事物。}
	\dialogue{咨询师}{那你能想到什么例子是你能看见其他人的内心事物的吗?}
	\dialogue{我}{比如说之前喜欢的那个男生,他会有一种无价值感,然后他从小就用自己的成绩来填补这个部分,但直到现在他依然有这份无价值感。当他愿意将这份无价值感呈现在我面前的时候,我觉得那是深入的东西。}
	\dialogue{咨询师}{好像你能和他产生连接。}
	\dialogue{我}{嗯。因为我能看到这是他所珍重的意义和内心事物。例如之前我在群聊里有聊性话题,然后一个朋友感到不适,后来他告诉我这是因为他学生时期的经历。但是当他在说他的经历的时候,我感觉他只是在说故事,而将情感的部分隔开了。所以我觉得我和他的关系深入了,但没有那么深入,因为他好像把他的情感隔开了,或者说没有呈现给我。}
	\dialogue{咨询师}{好像你并不是想要探究他们的隐私,而是他们内心的情感。}
	\dialogue{我}{是啊 。我也会好奇为什么我想要看见他们的情感、他们所珍重的意义。}
	\dialogue{咨询师}{是啊。}
	\dialogue{我}{可能是因为之前的我都处于情感封闭的状态,然后当我打开自己的情感后,我发现这个世界很枯燥。我会预设每个人都有丰富的情感,但不是每个人都会把自己的情感呈现出来。}
	\dialogue{咨询师}{当他们没有把情感呈现给你的时候,你会怎样的感觉?}
	\dialogue{我}{我会感到很失落。嗯,就是失落。我会感觉不足够,这个世界不足够丰富,所以想去看见更多他们的情感。}
	\dialogue{咨询师}{然后你会感觉到好像他们并没有表露他们的情绪,而是回避了、隔开了。}
	\dialogue{我}{嗯。}
	\dialoguesepline{咨询师}{……}
	\dialogue{我}{我会想起刚刚自己写的短篇故事里,就是外面的世界在崩塌,中间有一个壳,一个可以随时关闭的壳,而壳里面的自己很安全,但那个我依然觉得一切没有意义。所以好像我需要看到其他人的情感,看到他人所珍重的意义,然后试试看能不能将他们的意义搬过来。如果这个世界对他们而言是很丰富的、很有意义的,那说不定我也能像他们一样,用和他们同样的视角去看待这个世界。就像是如果父母看待这个世界是很有意义、很丰富的话,孩子也会沿袭同样的视角,认为这个世界就是很有意义、很丰富的。但对我来说,好像我的父母一直以来都没有这个部分,没有给我一种他们眼中的世界很丰富、很有意义的感觉。然后当我在读大学时试图去找比如说活着的意义、一些存在主义方面的事情的时候,我收到最多的评价是:生活就是这样,别想那么多。所以我也一直无法从别人那里获得一个答案。}
	\dialogue{咨询师}{好像在你看来,其他人都能够回避掉这个问题,但你不能。}
	\dialogue{我}{嗯,对我而言我做不到。当我越回避它,它越会环绕我,越来越觉得一切都没有意义,很虚无。}
	\dialogue{咨询师}{所以你才那么想找一些东西去填补那个空洞。}
	\dialogue{我}{是啊。……但我也意识到,当自己知道这一点,事情并没有变得更好,这并没有让自己变得更好。}
	\dialogue{咨询师}{是啊。}
	\dialogue{我}{会感到蛮无力的。}
	\dialoguesepline{咨询师}{(沉默)}
}
