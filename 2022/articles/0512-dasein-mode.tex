\chapter{行为模式与存在模式}

\ardate{2022-05-12}{mv9BGDGwMMn\_ZNL7hO1-lg}



我最近在刷正念冥想营的书单,其中有几本书提到大脑的运作存在着行为模式(Doing mode)和存在模式(Being mode)。行为模式适合解决实际问题,逐步缩短自己目前所处位置与目标之间的距离,将问题分割为片段,然后利用思想观察剖析,不断重新审视解决方案,判断它是否让你不断向目标靠拢。而存在模式则是以直观感受世界,以另一种视角看待苦恼,以极为不同的方式处理生活难题。

在刷书的时候,我逐渐回想起以前看过的一部美剧“西部世界”第三季剧终集里,机器人Dolores在自我牺牲前说的一段话:

\useimg{aimg/2022-0512-1.jpg}

当中的故事背景和剧情是:人类陷入彼此的战争,而人类Serac和他的兄弟创造了人工智能系统Rehoboam,Rehoboam通过算法来管控人类的工作和生活,并持续监测偏离了算法的个体。一旦个体偏离算法太远,就会被安排抹杀。在人工智能系统Rehoboam的算法管控下,人类获得了一定程度的和平。而机器人Dolores则通过自我牺牲得到了人工智能系统Rehoboam控制权,并将控制权交给了另一个人类Caleb来决定人类的命运。Caleb最后决定让人工智能系统Rehoboam自我销毁,将人类的自由归还给人类。

让我印象深刻的不是其中的剧情,而是机器人Dolores在自我牺牲前回想起的那些beauty\pozhehao{}包括她和她男友亲吻的时刻以及她去到小溪边画画时看见小溪对岸的一对母女在玩耍的画面。这是Dolores最后的记忆,同时也是获得人工智能系统Rehoboam控制权的密钥。

这让我想起这段时间的pandemic control。Pandemic control就像是人工智能系统Rehoboam的算法、程序,就像是行为模式\pozhehao{}逐步缩短自己目前所处位置(非零)与目标(零)之间的距离。在这个距离里,各种各样的事情都可以并已经发生。但始终不变的是,这依然是行为模式。

有一些事情是行为模式无法解决的,比如说情绪,而一直利用同一种固化的模式来处理无法解决的问题时,便会造成各种继发的问题。但这并不意味着存在模式就能提供解决问题的答案。存在模式只是能让个体客观地看待世界,而不是以个体希望它存在的方式、想要它存在的方式或者担心它可能成为的方式来看待世界。

我们大多数人都不想被感染,更不想因感染而死亡,但如果自己真的不幸感染,甚至是自己的重要他人因感染而死亡呢?如果这并不是自己能够解决的问题呢?

\blockquote{
	这一现象适用于人类的许多感受和情感,包括哀伤、忧虑和压力。例如,当我们感到不快乐时,就会自然而然地试图弄清这种情绪的原因,并找出解决这种哀伤情绪的方法。但是,紧张、哀伤和疲惫都是无法解决的“问题”。它们都是人所共有的情感,反映了精神和身体的状态。所以,我们只能感受,而无法解决它们。
	\citebook{正念禅修}
}

在情绪方面,有很多“问题”之所以会成为问题,很多程度上在于个体对待情绪的反应,比如说试图去压抑一时的愤怒而导致攻击性向内地产生长期的抑郁。我们看待外界的视角也同样受到我们的心理的影响,比如说把生老病死看作是必须要被解决的“问题”,而这种看法反而更容易成为问题本身。对于pandemic而言,我并不是评判control或not control是对还是错,而是control本身背后体现了人们是如何看待pandemic的,以及这种视角在多大程度上造成了多大的继发问题。

相比于身处存在模式去观察周围发生的变化,去欣赏日常生活里转瞬即逝的beauty,我见过也曾经和身边的人一样身处于焦虑和恐惧\pozhehao{}不仅仅是对死亡的焦虑和恐惧,还是对被pandemic control的焦虑和恐惧。甚至相比于被pandemic control,有的人宁愿选择死亡。我所处的环境的pandemic并不严重,甚至能够达成动态清零,但我所思考的是,这样的成就在多大程度牺牲了存在模式里的那些日常体验、那些转瞬即逝的beauty,比如说人与人之间的相聚,旅游的欢乐。

有时候这些日常体验、这些转瞬即逝的beauty甚至与pandemic control无关,比如说日出日落的天空,阳光落在皮肤上的温暖、落在眼皮上的明亮,半夜和喜欢的人走在街上的宁静。但能留意到、珍惜这些时时刻刻的人并不多,因为行动模式似乎一直在将人从存在模式往行动模式上拽\pozhehao{}正因为现在的生活乐趣不多、限制很多,所以才要更努力地、更投入成本地解决问题、解决现状、解决pandemic。就好像我们牺牲了现在的美好的一部分来换取未来的成功,那个并不是每个人都能看到的未来,甚至无从得知这一未来会有多“未来”。

正念冥想很强调的一点是待在当下,而不是将目光投注于过去或未来。当然这只是正念冥想的价值观,不代表每个人都会选择接受这样的价值观。但我想每个人的价值观在这一层面上更像是一个连续谱\pozhehao{}连续谱的一端是投注于过去或未来,另一端是呆在当下。而我们看待pandemic、看待自己和身边的重要他人的生命和生活甚至是死亡的态度有多大程度上更靠近过去或未来的一端,有多大程度上更靠近当下的一端?

\blockquote{
	我们必须再次强调,从思想上接受并不意味着放弃。并不意味着接受不可接受的事物。思想接受也并非懒惰的借口,也不是对你的生活、时间、固有天赋和能力无所作为的理由。(充满意义的工作,无论是否有报酬,都是获得幸福的必由之路。)正念是一种“接近感知”的过程,如果定期练习,它会自然而然地促进你不断进步。它能够使你平静地、不掺杂主观判断地、直接通过你的感知能力感受世界。它为你提供一种透视生活的感受。你可以感知哪些是重要的,哪些不重要。
	\citebook{正念禅修}
}

接受此时此刻的生老病死、接受现在的pandemic的存在并不意味着要放弃对生命的珍惜。恰恰相反,接受有的“问题”是自己无法解决的,这能够让自己更为平静地、更不掺杂主观判断地、更直接地感知和感受世界,去体验此时此刻的beauty,去体验那些在当下的、身边的转瞬即逝的事物和他人,以及自己。\par

\noindent\begin{minipage}{\linewidth}
	\center\quotefont
	``They created us.\\
	And they knew enough beauty to teach it to us.\\
	Maybe they can find it themselves.''
\end{minipage}


