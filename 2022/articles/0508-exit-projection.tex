\chapter{退群,投射}

\ardate{2022-05-08}{YD2fh125rBblwFZHBf5OaQ}




在昨天的读者群里,有个认识了三年的群友R先生退群了。然后我说:R先生“退了群,取关了公众号,还删了好友。感到伤心之余,我也在想,我和他之间的关系曾经是什么。(我)从来都触及不到(他)更深层的内心事物。比起他的离场,我感到更伤心的是,和他曾经的关系并没有多少深度、也并不是独一无二。”

在R先生退群后,通过和其他群友的交流里,我猜想我可能是触碰到了他对于体制的雷区。我会想去验证这个猜想,去想去加他微信,去问他我是否触碰到了他的雷区,他的雷区是什么,他所看重的事物具体是什么。但我又会想到,如果我这么做的话,这确实可以深入关系,而深入关系也确实是在和他的互动里我一直想要的。但我也会想到,这段关系真的重要吗?这段关系究竟是什么?

我感到很伤心,不只是因为R先生的离场,也不只是因为我们彼此的关系并没有多少深度、并不是独一无二。这并不足以解释自己伤心的程度。我回想起我和他的关系的起源:我和他相熟的时候是我和前任的关系结束之后,他在我眼中好像变成了一个偶尔的庇护所,比如说我能向他倾诉关于我控诉前任的所作所为的事情。在这个过程里,我好像也无意地将家的一小部分投射在了他身上,比如说稳定、安全和温暖。但现在回想起来,他并不稳定,并不安全,也并不温暖。他甚至还在之前的一次见面聊天里把他自己的焦虑投射给我,说我现在的状态是不行的,要改变……而我现在所感受到的伤心,我好像也在为那一小部分的家的投射的丧失而感到伤心\pozhehao{}在他身上,我不可能获得我所设想的、我所期待的稳定、安全和温暖的感觉了。

如果将投射到R先生身上的东西撤回,那和他的关系就真的只是“并没有多少深度、也并不是独一无二”。我会想起,我和其他一些朋友的关系也是如此\pozhehao{}并没有多少深度,也并非独一无二。所以R先生的离场也让我开始怀疑,人与人之间的关系的本质究竟是什么?既然任何一方都能够选择彻底消失,那关系的本质是什么?

我会想起昨天和前任的喝茶聊天。他说长期伴侣就是搭日子的,想对方在身边就在身边,想对方消失就消失,同样的,对方想他在身边他就在身边,对方想他消失他就消失。我问他他是如何让对方消失,他所说的方式是从对方的生活里消失。当他这么说时,我立即回想起他之前无故消失的那一年多甚至是两年的时间,我回应说:“你确实也这么做了”。既然每个人都能选择消失,就像之前我所认识过的男生里有一部分的人也会选择无故消失,那么关系的本质究竟是什么?我对人与人之间的关系感到很强烈的不确定性、不稳定感和不安全感。甚至在昨天和前任喝茶的那个当下,当他说到令对方消失的时候,我就已经感到很难过和很孤独,因为自己曾经就是那个被选择消失的人,而且并不只是被前任选择消失,也是被其他人选择消失。

在前任眼里,关系的本质更像是各取所需,比如说社会需求层面的各取所需。但在我眼里呢?我跟前任说,我觉得他在我眼中是特别的,不仅仅是因为在我眼中他是一个很自大的人,总是对任何事物和人都有着确定的答案,散发着一种确信感(当然后来他指出确信感也是我对他的投射物之一)。我觉得他在我眼中是特别的的原因还是因为我们彼此的过往经历,那些经历使得在我眼中的他与众不同,使得我们之间的关系与众不同。就像和R先生的过往经历也会让我认为他是特别的。

但这样的看待方式,是否只是一种活在对过去的回忆的方式?当我看向前任、看向R先生的时候,我是否只是看到了他们过去在自己眼中的形象,甚至是只看到了过去的我向他们投射的东西?比如说家、稳定、安全、温暖、独立、自由、喜欢深入交流、能理解他人的想法和感受,甚至三年前还会向前任投射一个父亲的形象。即使他们现在已经不是过去的样子,亦或者他们从一开始就不符合、就压根不是我向他们投射的东西,他们只是他们本身,如果是这样的话,我真的还要继续向他们投射这些东西吗?如果我将投射撤回的话,我会发现,和前任的关系、和R先生的关系并没有那么独一无二。独一无二的是经历,而不是关系。

那关系又是什么?对我而言,人与人之间的关系是彼此的投入和在场。而任何一方选择撤回或离场,都会让关系不复存在,就像是前任会让他的长期伴侣“该消失的时候就消失,该出现的时候就出现”,这也是我和他三年前在一起的那段时间里他重复过好几次的话。而我也会对此感到很伤心,因为我眼中的“关系”和他眼中的“关系”是那么的截然不同,彼此眼中的关于人与人之间的关系的人际世界也更截然不同。

当R先生离场后,我也开始撤回了我对他的投射,甚至只有在他离场后,我才能看清我眼中的他绝大多数只是来源于我自己的投射。就像如果在昨天喝茶时前任不把我对他的投射物提出来,我无法分清哪些是对方真实的部分,哪些是我自己的投射。只有通过对方的回应甚至是答案,我才能区分得出来,当然对方的离场也能做到这一点。而当把投射撤回后,我发现对方在自己内心的形象变得面目全非。自己所喜欢的、所在乎的、所仰视的、所崇拜的、所期待的、所渴望的、所吸引的人不复存在,从一开始就没有真实地存在过。一切都不复存在了,也许这“一切”从头到尾都只是我的投射物而已,都只是自己内心世界的投射。

但为什么我会向他们投射家、稳定、安全、温暖、独立、自由、喜欢深入交流、能理解他人的想法和感受,甚至是父亲的形象?为什么我要向他们投射我的内心世界?我回想起和R先生相熟的时候是我和前任分开的阶段,和前任相熟的时候是我处于大学毕业的生活变迁的阶段。也许在潜意识里,那时候的我和现在的我真的很需要那些投射物,很需要家、稳定、安全、温暖,很需要一个独立、自由、喜欢深入交流、能理解他人的想法和感受,甚至是父亲的形象的人。也许我很需要在现实世界里投射我的内心世界,因为这个现实世界并不足够,从来都不足够。这些东西、这样的人、这样的内心世界是我在现在的日常生活里、在过往的经历里、甚至是成长环境里都不曾拥有的。从来没过一个家、没有稳定、没有安全、没有温暖、没有独立、没有自由、没有父亲。那背后会有一份潜藏的渴望,渴望从他人身上获得自己所一直匮乏的。但对方给不了ta自己所没有的,R先生给予不了稳定、安全和温暖,前任也给予不了一个独立、自由、喜欢深入交流、能理解他人的想法和感受,甚至是父亲的形象。前任也更加给予不了关怀、给予不了连接,这是他所没有的。正如我也无法给予我自己我所没有的。

我开始能理解为什么前任总是在我提出我对他的猜想后说我有一个“很丰富的内心世界”,我的咨询师也会这么说。内心世界的丰富是因为现实世界的匮乏,而我好像也会无意识地向匮乏的现实世界里投射丰富的内心世界。但投射得越丰富,最终的幻灭也就来得越猛烈。因为匮乏的现实世界、匮乏的他人并不会因为自己的投射而变得丰富起来,他们依然是他们原本的样子,而前任也会很排斥、很拒绝我投射给他的东西,会说“我为什么要满足这些”。我的内心世界依然只是内心世界。将投射撤回后,将自己的内心世界从现实世界里撤回后,现实世界的面目全非或许从一开始就是它原本的样子,尤其是现实世界里的人。

当我把自己所投射的事物、把自己的内心世界从现实世界里的人身上撤回时,我好像也在把自己从现实世界里撤回,从一个令我感到幻灭的世界里撤回。我猜想这可能也是R先生、也是前任所做的事情。

