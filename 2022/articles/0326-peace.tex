\chapter{“Be peace with it”}

\ardate{2022-03-25}{1qWZTdfcAZ8Yupsc7WgoZQ}


当危机干预课程刚好讲到self-care时,我就发现自己发烧了。去看病说是急性肠胃炎,然后跟公司请假,在家里睡了一整天,然后还要防御着来自家里人的攻击。比如说他们总会给我下各种指令\pozhehao{}多喝水、多休息、不能洗澡、不能洗头、吃雪梨、吃XXX药、去量体温\pozhehao{}只要不服从他们的指令,他们就会说:“那你去看病干嘛,你吃药干嘛,还有什么用,还不如不去看病,还不如不吃药”,就差直接说:你还不如死了算了。感觉自己只是在被当作问题一样地被解决,以及被用于缓解他们的死亡焦虑。而且他们也从来没有在乎过我的情感和感受,因为这一过程在他们眼里压根与我无关,而只与他们眼中的儿子形象有关,那个像死物般任由摆布的傀儡,沦为客体的他人。

连照顾自己都很吃力的同时还要防御周围的人的攻击也是够心累的。

当退烧后开始上班时发现,工作很忙,自己忙到没有时间去思考些什么,而更像是个本能性驱动的机器。当课程继续讲到关于自我关怀时,讲师提到了“盈余”:

“我们常常说在过年的时候会祝别人年年有余,这个是希望你每一年都有多余出来的东西,可以享受,可以分享,你可以拥有这样多出来的东西。对我来说,心理的空间也是希望大家能够年年有余,日日有余,时时刻刻有余,这个有余其实是一个心理空间上的有余,有余地。留给谁呢?留给自己,或者是留给来访,或者是你心爱的人,重要的人,重要的事情。而除此之外,这个有余有的时候就是需要它余在那里,我们在过年的时候祝别人年年有余,不见得是这个多出来的事情一定要把它用掉,或者是赶快把它分出去,没有,这些多出来的事情,就是庆祝这个余地,庆祝这个多出来的,庆祝除了我吃饱,我们家人都吃饱,还有多出来的食物是我不需要去用它来生存的东西。”

原来自己在这一周里都一直都忙于应付身体、应付工作、应付家里人、应付各种各样的事物和人,几乎没有留给自己的空间、时间和精力,只是忙于生存。


\midnote{``Seven Days Walking'':\\http://music.163.com/album/82516126/?userid=112717718}

在这周,自己所在的城市开启了雨雾天气,近处总是在下雨,远处总是被雾气笼罩,怪不得抑郁症总是在春天复发。当置身于室外的雨雾时,我脑海里会响起一首或几首纯音乐\pozhehao{}《Seven Days Walking》。我发现每次脑海里自动响起这首纯音乐的时刻,都是在我的情绪很悲伤和低落的时候,就像是大脑的一个自我保护机制或自我安抚机制被启动了一样。当脑海里响着这首纯音乐时,远望外面的雨雾,我就会觉得周围的阴郁很美,悲伤也很美。而不会像以前那样因伤心而伤心(二级情感)并从而加重悲伤的程度。

脑海里这首纯音乐的出现,好像提醒着我,我一直有着内在空间,我和外界并不总是无距离的。只要我想,我就能从外界现实抽离出来,在自己的内在空间里呆着。例如可以在忙于应付身体、应付工作、应付身边的人和事物的时候通过冥想或一些小物品给自己制造一些抽离的道具、途径。这种内在空间的盈余似乎只需要自己在现实世界里停下来,并去“触碰”(远观)自己的想法和感受。

在上危机干预课程快结束时,我脑海突然有一种感觉\pozhehao{}读我写的东西的人是不是只是在看我还有没有活着,然后突然有一位读者在私聊我催更。好像那种直觉性的洞察力又回来了。

无论是那首纯音乐,还是那种直觉性的洞察力,好像都和一种be peace的状态有关,那既像是一种“触碰”,但又像是一种远观。

\blockquote{
	我也不是为了一个渺茫的目标而努力,而是为了脚下的每一步在努力,enjoy the view and see where it will lead。如果真的走到了尽头,be peace with it,然后再看看还能往哪里走。

	\blockquotesource{自己究竟是谁,内在世界和外在世界,探索 \& desire}{白色灯塔先生}{2022}
}

有位读者问:“名词的 peace 在这里怎么理解呢?或许是 be at peace 的笔误?” 我思索了下。我既不想用make peace with it\pozhehao{}不想强调make,也不想用be at peace with it\pozhehao{}不想强调at。就好像peace不是一种去make的东西\pozhehao{}不是去创造一件创造品,也不是去at的东西\pozhehao{}不是硬将自己投掷于某种状态当中。

那peace究竟是什么?我会想起一个短语“to the things themselves”。既不主动创造平静,也不主动将自己安放在平静当中。我记得在以前看的一本关于接纳承诺疗法的书里的一个正念练习就是将脑海里经过的念头和想法看作河流上的一片片树叶或山丘上的一个个士兵,就只是看着他们流过/经过就好。仅仅从远处看着平静的来临、停驻和离开。所以悖论之处在于,正是放弃对平静的追寻、改变和留恋,反而才能与平静同在(be peace)。

平静来临的时候,察觉到这一刻的到来,享受其中的平静感。当平静离开后,不去硬拽着平静感不放、不惋惜平静的消逝,而是接受事物原本的样子(to the things themselves)。当自己远观着平静感的流动时,自己似乎也流动于在其中,流动于不同的感觉当中。既是远观,也是“触碰”。

所以平静感并不是一种流动的情感,而是某种特定的情感\pozhehao{}一种很舒服、很平和的感觉,但并不是流动的。而我想表达的make peace with it,并不是对某个事物(it)追求一种平静感或将自己置身于一种平静感当中,而是对此“随波逐流”,接受事情原本的样子。

