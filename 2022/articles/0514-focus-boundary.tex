\chapter{独一无二的聚焦点,边界}

\ardate{2022-05-14}{2HuDugezcenR9EXa8HAAdA}




咨询前一分钟,我敲了敲咨询室的门。我记得之前有听到咨询室里传来声音,但现在敲完门后却没有人开门,我猜想咨询师会不会睡着了。过了一分钟后,咨询师开了门。

\midnote{(沉默了一会儿)}

\dialoguelist{咨询师}{
	\dialogue{我}{我还在适应这里的灯光、空调声、这里的环境。我会想到为什么咨询师能先进来,因为这样你就可以提前进来适应环境,而我好像每次进来咨询都需要适应一段时间。}
	\dialogue{咨询师}{你会对此有什么感受吗?咨询师能先进来,但你要在外面等。会感到不公平吗?}
	\dialogue{我}{倒是不会感到不公平。会想到这是设置的一部分,导致了彼此会有不同的体验,而且你也总是对设置很保守。\\
		其实我刚刚在想有一件事情可以提,但好像又不那么重要,可以不提,然后又想到可以提但又不想提本身会不会正是应该提出来的原因。\\
		我会留意到我的思维有点跳跃,这可能和我这周的状态有关。这周自己总是走神,比如说洗碗的时候自己走神到洗完碗才回过神来,然后在想自己洗过碗了吗?好像是洗过了,但我对此没什么记忆。}
	\dialogue{咨询师}{你走神的时候会走神到什么地方去吗?}
	\dialogue{我}{Em……会走神到关于将来的计划或者是对过去的事情。}
	\dialogue{咨询师}{除了洗碗之外呢?还会在做什么事情的时候走神?}
	\dialogue{我}{我不太记得了,走神的时候我是没有意识的,等意识回来的时候事情已经做完了。
		我想到可能和最近的情绪状态有关。就是上周末和前任约见面聊天后,我的关于曾经同居过的那个公寓的家的内心形象幻灭了,然后那时候自己就很崩溃,会感到很伤心和无力感。然后我的公众号读者群里有两个认识了蛮多年(一个认识了起码有三年,另一个有五年)的两个朋友退群、取关、删好友了。然后我开始处理自己的情绪时,我发现经过和前任聊完天后,我处理情绪的方式变得不只是看清自己内心的事物,比如说我对长期伴侣、对家投射的东西、我的期望、我的想象,当我看清属于自己的部分后,我会开始将这些东西曾经投射出去的东西收回来。因为这些是我内心的部分,而不是对方的,所以将自己的东西收回来后,再去看对方,就会觉得对方很平凡,独一无二的不是对方,独一无二的仅仅是经历而已。}
	\dialogue{咨询师}{虽然你说好像可以不提出来,但这当中好像依然有一些情感在。但你在一开始说你没有打算提出来。}
	\dialogue{我}{嗯。因为现在这个当下我并没有感受到悲伤和无力感,没有感到对前任、对曾经的关系,对家的内心形象的幻灭感到无力。不像是以前,以前的话我会刻意去回忆过去的事情或者是让自己刻意地沉浸在悲伤里。但这周当我让这些情绪自然地经过时,就像是下大雨,雨停了之后那些情绪也就不会固着在这里。所以我也没有想主动提出来,因为不想刻意地去沉浸在那种无力感和悲伤里,它们已经流过去,就不用去刻意追寻那些感觉。}
	\dialogue{咨询师}{嗯。但你好像也说到这周会经常走神。当你走神的时候,会不会是在回避那些感觉?}
	\dialogue{我}{Em……我不太确定。走神的时候我是无意识的,我意识不到那个当下我的感觉和想法是什么,更像是什么感觉和想法都没有。但如果那些悲伤和无力感涌过来的时候,我是有意识地感觉到它们的。}
	\dialogue{咨询师}{其实我会想起之前的几次咨询里,你会觉得我的一些技巧,你会觉得是套路。你会不会也觉得我不是独一无二的?}
}

我忍不住了笑了一会儿,内心想到:好僵硬的技术啊!强行将关系拉到此时此地的咨访关系。不过也不是第一次了,还是思考一下ta说的内容吧。

\dialoguelist{咨询师}{
	\dialogue{我}{Em……一定程度上会。因为我好像没有找到一些聚焦点,一些关于独一无二的聚焦点。就好像你没有表露太多的东西、特质能让我聚焦到那些东西上,没有让我觉得那些东西是独一无二的。比如说和一个朋友相处的时候,他的自我表露程度会很深,同时他也敢于向我表达他对我的愤怒。所以我会把他看作是独一无二的,因为他有一些我视为独一无二的特质。但你没有,你把自己的部分保护得很好。所以我的独一无二的视角聚焦不到你身上,而我也不会假设你有怎样的独一无二的特质就信以为真。}
	\dialogue{咨询师}{那你会是怎么理解的呢?我没有把我的部分呈现出来让你觉得独一无二。}
	\dialogue{我}{我会感觉这更像是你的边界、界限,一种被隔住的感觉,被局限感。}
	\dialogue{咨询师}{那你会感到失落吗?}
	\dialogue{我}{Em……我会想到可能每个人呈现的程度不一样,比如说我身边有的朋友就表露得很深,有的就表露得很浅。之前退群的其中一个朋友就是表露得很浅,我曾经有试过通过聊天的方式去了解更多,但都了解不到,所以后来我也就止步在表浅的关系上了。}
	\dialogue{咨询师}{我记得你在之前的咨询里会好奇我的其他来访会是怎样的,所以我会在想,你会不会也担心你没有被我看作是独一无二的?}
	\dialogue{我}{Em……倒是不会。可能因为我视我自己是独一无二的就足够了,我并不需要另一个人用这样的视角来看我。可能因为我没有把你视作那么的独一无二,所以我也不需要你以同样的视角来看我。但如果是很重要的人的话,我会想ta能这样看我,因为我会视对方为独一无二,就像前任一样。}
	\dialoguesepline{咨询师}{沉默}
	\dialogue{我}{我会想到我可能比较容易触碰到对方的个人议题或者是雷区或者是对方不想表露、不想触碰的部分。比如说之前退群的朋友里我可能是触碰到了他们对于审核、对于体制的个人议题。当我在之前的咨询里提到边界的时候,我突然想起你在咨询的很一开始的时候有提到界限是一个你要去处理的议题、一些你自己的部分,所以当我再提起边界的时候我会在想不知道那时候的你处理完你自己的部分了没。}
	\dialogue{咨询师}{你觉得这会影响到咨询吗?}
	\dialogue{我}{我会担心,我会担心你会不会影响到咨询。但这种担心好像并没有阻止我继续去这么做,继续去触碰那些对方的部分,那些局限。我好像总是会去试探、试图去跨域那些规则、边界,小到比如说闯红灯,我甚至会对自己说:“Rules are made to be broken”,大到比如说我会在读者群里会聊性话题,然后有的人就很不想被聊上,就会说“你聊性可以,但不要指向我。”后来在私下聊天的时候他才跟我说一些他自己的经历,然后我发现那更像是他自己的个人议题的部分。}
	\dialogue{咨询师}{我会想到最近几次咨询你都会提前一分钟敲门。}
	\dialogue{我}{嗯。这次你会准点了才开门。你是想问为什么我这么做吗?}
	\dialogue{咨询师}{嗯。}
	\dialogue{我}{因为之前我不敲门的时候,你不会准时开门,准时是指我的电子时间上的时间。你可能会迟了一分钟才开门。所以我觉得,一方面是我需要去促成开门这件事情的发生,另一方面我可以去承担控时的责任,至少是最开始的准时。}
	\dialogue{咨询师}{当你敲门了,我没有开门的时候你会有怎样的设想和感受吗?会感觉到被拒绝吗?}
	\dialogue{我}{我会想到你会不会睡着了。(咨询师笑了笑)被拒绝好像会有一点,但不是完全的拒绝,不是非黑即白的拒绝,而是你拒绝了那一分钟的互动。}
	\dialogue{咨询师}{咨询一开始的时候你会想提一些事情,但又没有提这件事。}
	\dialogue{我}{嗯。当我进来的那一刻会想提,但坐下来之后就不那么想了。可能是因为看见你的状态很平静,然后我也tune in了你的状态,或者是被你的状态无意识地影响到了我。而且你在每次咨询一开始都不会主动说什么。每次的状态都是很平静,而我好像需要用我自己的动力来驱动你,来推动一些我想聊的事情。这可能也是为什么我在最近几次咨询的一开始都会提我的观察和想法的原因,因为好像我需要从一些更小的动力开始,才能慢慢去谈动力更强烈的事情。就好像你的状态并不适合我一坐下来就谈很多需要很强烈的动力才能谈的事情。我在一开始好像不只是在适应环境,我好像也需要去适应你的状态。}
	\dialogue{咨询师}{那当你进来咨询室看到我的状态后,你会想到什么吗?}
	\dialogue{我}{我会想到,可能刚刚的事情没什么。}
	\dialogue{咨询师}{那如果我真的睡着了,一直没开门呢?}
	\dialogue{我}{那我会想可能是一些很大的变化发生了。就像刚刚那一分钟,那背后会有很庞大的事物的存在。我会有一点不满,但那种不满很快被巨大的好奇感所覆盖。}
	\dialogue{咨询师}{之前好像没有听过你提“巨大的好奇感”。}
	\dialogue{我}{因为这样的时刻不多,之前的话比如说在电梯里相遇之后,我会提出来。但这次没有马上提出来,因为你一开始平静的状态需要我用很大的动力才能驱动起来、才能激活你。}
	\dialogue{咨询师}{那你会感到累吗?好像需要你一个人去推动咨询。会有想到不想来咨询吗?}
	\dialogue{我}{Em……我此时此刻还没有太多的负面感受,可能我习惯了在人际关系里我总是那个推动关系进展的人,比如说见面。}
	\dialogue{咨询师}{你说“习惯了”。}
	\dialogue{我}{嗯。而且这种推动的努力好像也开始蔓延了开来,蔓延到了咨询前一分钟我需要去敲门。}
	\dialogue{咨询师}{那你会期望我在咨询一开始呈现一种怎样的状态吗?因为好像有一种期望在里面。}
	\dialogue{我}{嗯,会有期望,但这种期望好像找不到指向的对象。就只是期望而已。}
	\dialogue{咨询师}{就只是期望而已?为什么会这样呢?}
	\dialogue{我}{我会想到之前我和最近一个喜欢但拒绝了我的男生相处的时候,每次见面他总是会说我很低沉,他总是需要说一些话或做一些事情来启动我。那时候的我好像就是无欲无求的状态,因为我只是需要他的在场,他人在就足够了。也许咨询一开始的你也会是无欲无求的状态。如果我是他的话,我也会感到蛮累的,也会有失落。可能下次和他见面的时候我会开启些仪式感的东西,比如说拥抱上1、2分钟,看彼此、看我会不会触发到一些怎样的情感。但对你,我好像想不到什么仪式感的东西,因为我好像看不见那个独一无二的聚焦点。那个我喜欢的男生我能看得见他独一无二之处,那些点好像就成为了我能够和他接触的点。但你我找不到。}
	\dialogue{咨询师}{那你会对此感到失落吗?好像你找不到和我接触的点。}
	\dialogue{我}{我会有各种各样的感觉,会有安全感,因为不会触碰到太多的事物,也会有不安,感到不安全,也会有失落。但我想那更像是一种局限或边界或者是隔阂感,就是一种被隔住、被局限的感觉。}
	\dialogue{咨询师}{(笑了笑)我想我们可以在下次咨询里再继续讨论这一点。}
}
