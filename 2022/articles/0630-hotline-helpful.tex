\chapter{情境模型,心理热线真的有用吗}

\ardate{2022-06-30}{mBVt5rvI2YknOQBWV3SpdA}




最近在看《心理治疗大辩论・心理治疗有效因素的实证研究》在看的时候,我联想起最近半个月以来我对心理热线的一个怀疑:心理热线对来电者而言真的有用吗?

《心理治疗大辩论》中有提到一个情境模型,其目的是解释所有心理治疗的疗效。

\blockquote{
	\useimg{aimg/2022-0630-1.jpg}

	\citebook{心理治疗大辩论}
}

我会想到,心理热线的工作方式是持续几十分钟的一次性匿名通话,其功能和范围是舒缓情绪以及梳理目前困扰。倾听者在接热线前需要接受培训和考核后才能上线,这虽然能保证一定程度的专业性,但这与来电者对倾听者的信任和理解无关,与建立初次倾诉(甚至不能称得上是治疗)联结无关。

\blockquote{
	所有的治疗关系都有一些基本水平的信任。但是当注意力转向更受保护的内在体验时,就需要建立和发展更深的信任和依恋关系。

	\citebook{心理治疗大辩论}
}

但热线并不鼓励依恋关系的发展,甚至通过热线设置(比如说打给同一个倾听者的有限次数、不得在热线里建立热线外的联系等)来禁止来电者与倾听者产生依恋关系。

假设倾听者有一定的专业性,而来电者也能够在热线过程中建立对倾听者的信任和理解、建立一定程度的依恋关系,而这种带有信任和理解的依恋关系能够让来电者愿意敞开自己内心更受保护的自我体验(例如某些负面经历),那么接下来有三条路径能够产生疗效。

路径1是真实关系,即彼此之间的真实、亲密的情感关系。真实的关系建立在真诚和现实知觉的基础上,即通过未受移情及其他防御方式扭曲的知觉,在关系中真实、开放、诚实地表现自己的能力和意愿。热线能够尽可能地提供真实关系,但这也需要倾听者在处理自身移情和防御以及处理来电者的移情和防御方面有一定的专业性。

路径2是期望,当事人相信心理治疗会对他们有用。

\blockquote{
	唤起期望的关键不在于理论的科学有效性,而在于当事人能否接受对障碍的解释,以及这个解释是否与治疗行动一致。众所周知,心理障碍的原因难以确定,但这与唤起期望无关。因为,如果当事人相信对障碍的解释和投人治疗行动可以提高他们的生活质量,或帮助他们克服、解决他们的问题,期望就能够被唤起,并产生效果。

	……需要注意的是,如果没有特定的治疗方法,则无法就治疗目标和任务达成一致,也就失去了一个唤起期望的至关重要的成分。而有一种常见的情况是,一个人仅仅与一个能够共情倾听者讨论自已的问题,也能激活共同要素;虽然这种“治疗”(有时被称为“共同要素"治疗)有可能通过真实关系的路径产生效果,但这不足以充分激活潜在的治疗效果。弗兰克早在1961年就说明有效的治愈包含了“迷思”和“仪式”。换句话说,共同要素之一就是,以令人信服的方式系统地使用一套特定成分,并为当事人所接受。

	\citebook{心理治疗大辩论}
}

书中举的这个例子与心理热线很相似,而这也是我个人认为心理热线并没有多大疗效的原因。

当倾听者帮助来电者梳理目前困扰时,倾听者也是在间接甚至有时是直接给来电者一个解释\pozhehao{}为什么来电者的当下状况会是这样的。即使在最理想的情况下,来电者能接受这样的解释,但心理热线并不具备一套特定成分,而来电者也无从“接受”这样的特定成分,即缺乏一套旨在修复当事人缺陷的仪式,这一仪式本身具有特定成分,即路径3(特定成分):

\blockquote{
	治疗目标和任务达成一致后,当事人参与到治疗行动中。也就是说,当事人“接受”了治疗的特定成分。对许多人来说,这才是心理治疗起作用的部分。事实上,心理疗法与一般的心理治疗”之间存在区别,前者包含科学的特定成分,而后者没有。巴洛(2004)认为,心理疗法包含所有治疗都共有的成分,比如“治疗同盟、激发对改变的积极预期和重新振作”,但也包括重要且“特殊的心理治疗程序”,这些程序“特定性地针对当前人们所面临的心理病理”。也就是说,使心理治疗起效的是修复当事人缺陷的特定成分。

	\citebook{心理治疗大辩论}
}

比如说一个有抑郁情绪的人,他可能需要的不仅仅是一段真实关系,在这段真实关系里他能获得一套为什么他会抑郁的解释并接受这套解释,同时他也需要带有能够修复他的缺陷的特定成分的仪式,比如说在固定的治疗设置里修复他的不合理信念、自体感的不足、过往经历的创伤、不适用的防御等等。

不过,《心理治疗大辩论》里的(开放性)立场是:“尽管许多疗法都符合情境模型对心理治疗列出的条条路线的要求,但是这些路线不一定是治疗起效的唯一解释。”

我会想到,心理热线能做到的只有路径1(真实关系),并或许能够勉强地做到路径2(期望),但这两者只能促进更好的生活,而并不足以减轻症状(如图),即“不足以充分激活潜在的治疗效果”。这可能也是热线培训里有对来电者精神状况的粗略评估并在存在精神疾病症状时鼓励来电者寻求更专业的帮助的原因,因为心理热线极可能完全不具备任何治疗效果。

在我看来,打心理热线和找个朋友聊天在效果方面差不多\pozhehao{}提供真实的关系、对自身情况的解释以及对改变的期望。但很多来电者(包括有时作为来电者的我)会选择打热线而不是找朋友聊聊,是因为(主观现实当中的)现实生活里并没有这样的人在\pozhehao{}没有真实、亲密的情感关系,充斥着各种移情和自我防御方式的扭曲,无法给提供一套令人信服的解释,更不期望对方的情况会有所改善。

所以有时候,我会感觉心理热线更像是在填补当事人在人际关系上的缺失,比如说连接感、亲密感、安全感、真实感、信任、期许等等。我会不禁在想,我们所身处的人际关系网络甚至是这个社会环境究竟出了什么问题,为什么有那么多的人无法从日常人际关系里获得本该从人际关系里获得的东西,而需要通过一次性的匿名通话来获得这些东西?为什么我需要支持的时候难以从身边的人身上获得?

