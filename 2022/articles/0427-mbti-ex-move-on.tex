\chapter{科普课,前任,继续前行}

\ardate{2022-04-27}{hhH7-vpqPnKnNVd44tKRNw}




\blockquote{
	后来,随着我看见他的部分越来越多,我发现自己非常像他,并说:“我发现我越来越像你。但其实我还蛮不想成为像你一样的人的。”他说他在(三年前)遇到我的时候就看见了这一点。我问那一点是什么?他问我我觉得那时候的我处于怎样的状态?我说那时候的我害怕面对未来、想要找一个人来保护自己,来暂时不用面对那些需要面对的事物。他回答说:“那么那些东西的共性是什么?”我猜了几个回答,但没有猜中他想表达的。他说(那)是一个人想要做的是什么、人生的意义的那些问题。我意识到他在讲关于存在主义方面的议题。我回答说:“如果放在那个框架下的话,嗯,我在那时候确实看见了一点点苗头。”他回答道:“就是咯。”他说如果一个人在最终会变成一个完整的人的话,那些问题就是和那个完整的人有关的。我问:“你是想说,在我们那时候刚开始亲密关系时,你就看见了我将成为的那个完整的人的其中一部分吗?”他说:“当然了。”我继续问:“所以那时候你就能看见现在的我会变成这个样子吗?”他说:“嗯。我之前说你需要依赖他人,是因为你在其他人身上会拿走一些东西。”我意识到他想说的可能是我所内化的重要他人客体,然后我问他:“那些东西是什么?”他说,就是那些你像是我的东西。我笑了笑地点了头,然后问:“那可能更像是复制,因为你并没有因此而失去什么。”他说,不像是复制,而像是我拿走了一些属于我自己的东西。我想到,我确实会在每段亲密关系里内化不同的重要他人客体,将那些部分经过处理地成为自己的一部分。

	我也想到,前任的这一“角色”明明对曾经的我造成了那么大的伤害,但我却越来越像这样的他,甚至会对他感到更有吸引力,因为难得“遇到”一个和自己那么相似的人。无论是越来越像他,还是对他感到更有吸引力,我内心都既喜欢又排斥这两个部分。

	但我依然想和他有更深的关系,所以我说我在和他分手后的这两年半以来见过很多人,但都觉得(和他们的相处)蛮无聊,不过和他聊天不无聊。他说,我感到无聊是因为我和他人的互动方式以及我自己都没有怎么变化吧?我回答说:“不是的。我自己一直在变,和他人的互动方式也在变。我觉得和你聊天不无聊是因为我‘看见’了你背后的那个人,那个带着漠然、超然的态度对待身边的事物和人甚至是用这一态度对待你自己的那个你,也看到这一态度在你生活中的各方各面的延伸。……看见了这个人让我感觉我们之间的距离拉近了。”

	\blockquotesource{白色灯塔先生}{这就足够了吗?}{2022}
}

在昨天学的关于MBTI(人格类型)的科普课让我回想起年初和前任见面时的对话,而科普课其中一节的题目是「内倾的人是如何与外倾的人相处」,以下是一些笔记:


\blockquote{
	异性相吸\pozhehao{}一个人在人格类型上的某方面为弱势功能,而这个人会被另一个人在这一方面为强势功能的人所无意识地吸引。比如说内倾的人会被外倾的人所吸引,因为内倾的人当内倾为强势功能时,难以连接到甚至难以意识到作为弱势功能的外倾的一面。而当看见有另一个人能很好地连接自己外倾的一面的人(外倾的人)时,便会产生吸引。

	如果是觉得很吸引自己的部分,最终自己可以考虑、可以去思考这就是自己在自我发展的时候,很重要的一个关键点。对方没有改变,而是自己、自己的心态、自己的期待在这里产生了改变。如果真正希望有好的一个关系的话,最终是需要给自己一个转化、一个发展。

	\citebook{笔记节选 1}
}

前任从一开始就蛮吸引我的,现在依然如此\pozhehao{}那份自由和独立,以及一些其他的东西、一些我难以描述和明确的东西。如果说和前任分手后我“学”到最多的部分,可能就是理解他人了,这是自己刻意去提升的。因为曾经在一起的时候发现自己并不能理解到他,无论自己多么努力,都无法理解到他,而他也逐渐不想让我了解了。在分手后,自己则开始寻找各种原因去解释过去的变故\pozhehao{}特别是在心理学和心理咨询流派里找各种解释。当然,有很多事情并不可能找得到答案,可能也正因为这样,自己才会在寻找的过程中越来越擅长理解他人,唯独理解不了的就是前任。

\blockquote{
	爱情、情感能帮助我们跳出自己,放弃一些既定的观念,而相处的关键在于自己是否愿意稍微打开自己,是否能够不拘泥于自己就是一个怎样的人,试着走出舒适圈,和对方做更多的连接。

	\citebook{笔记节选 2}
}

我记得在和前任相处之前,自己的情感一直处于相对封闭的状态\pozhehao{}那时候自己能感受到的情感的深度和广度并没有现在那么庞大,而只是一些很有限的情感类型和情感深度,这可能也使得我在和前任相处的过程中难以理解他的情感的原因\pozhehao{}不过他也不会直接表露他的情感,只能靠我自己去猜。在相处的过程中,有一次前任说我并没有敞开内心地接受他,所以当我努力将自己的心墙卸下时,我发现自己突然能回想起童年时的很多记忆,以及那些记忆所带来的各种强烈的情感。在那之前,那些记忆都是我回想不起来的。但这依然不足够,那时候的自己依然不足以理解他,更不足以挽回关系\pozhehao{}虽然关系是个互动场。

\blockquote{
	从心理发展的角度来说,对于这样的吸引,最终自己会希望是自己也能发展这个部分,终究是自己需要发展是自己的这个部分。如果自己因为喜欢另一个人的这个部分而跟ta在一起,但自己终究没有发展的话,这就是很多时候发现在感情中:自己一开始最喜欢对方的什么什么,到了分手的时候反而会觉得对方的什么什么最讨厌。那个吸引的部分,需要自己去发展它、内化它。只有当自己也拥有这些特质后,才能和对方拥有一个很平起平坐的地位,才能真正能够包容彼此、欣赏彼此、尊重彼此的一个生活方式、一个态度。如果自己没有发展这些部分,终究只是想外求而自己又没有发展、没有变化的话,会发现在一段时间后,那些东西还是对方的,而不是自己的。但那些东西的存在,当它在自己的旁边但不能成为自己的一部分的时候,是非常碍眼的,从曾经吸引自己的部分变成了日后的痛处。

	\citebook{笔记节选 3}
}

过去的自己并没有发展这些部分,现在也依然没有发展完这些部分,比如说自由和独立以及一些我难以描述和明确的其他东西。我也说不清楚那些吸引我的东西是什么,而我又是朝着怎样的东西前行。但我隐约感觉到远处有一份东西是我想要抵达的、想要得到的、想要成为的。就像是以前的我很渴望能够理解他,但现在渴望的不是理解他,但那份渴望依然还在,依然渴望着些什么,一些与他有关但又与他无关的事物、一些属于我自己想要的事物。

那时候和前任相处的我确实能感觉到,他有着一些我所渴望的东西,而我又无法拥有那些东西时,我在他身边只会感到更加自卑、更加无力。在分手后,我就像是盲头苍蝇般到处寻找答案,在寻找一个没有答案的答案的过程中,我好像无意识地吸收了很多有关前任的特质,甚至没有意识到自己变得越来越像他。

但科普课里讲师说的内容让我看到了另一种可能性:并不是我从对方那学到什么我就一定要离开或者是我就一定会被对方所遗弃,而是两人可以因此而相处得更加融洽。两人的关系并不是必然会因为这样的内化而破裂消失,而是可以更为交融。

之前的我一直没有看到过这样的可能性,因为我的许多亲密经历里就只有被无故消失地对待。所以,我只能继续前行,继续走向自己所渴望的事物,一些与他有关但又与他无关的事物、一些属于我自己想要的事物。

