\chapter{“憎恨这个世界和这个世界上的所有人”}

\ardate{2022-03-07}{Yvn8t81373WYJDDS\_SIy3w}


\blockquote{
    刚在咨询椅上坐下,我开始说:“其实在上一次咨询,有一种感觉是我想无视但又不想抹去的,那就是在你知道我又有自杀意图的时候,你做了几分钟的评估。那时候我是知道你在做评估的,而我一直不想让自己表露出不情愿的样子,但当时我依然感觉到内心很不情愿。
    
    这种感觉可能也是我在那么多年里都没有跟身边的人自我表露过我有自杀意图的原因,因为我不想他人忙于阻止我,而不是了解我这个人。这就像是他人只看到了一个想自杀的人,只看到了自杀这个行为本身,而没有看到这个行为背后的那个人,那个人并不是只想自杀/自伤,TA还会有很多部分,TA还会热爱很多事物,比如说热爱着蓝天和白云。
    
    \blockquotesource{自我探索 | 10}{白色灯塔先生}{2022}
}

“我感觉你只是在忙于阻止我,而不是理解我。”这是我在某一次心理咨询里跟咨询师说的话,在那之前的上一次咨询里,我表达了自己重新出现的自杀意图\pozhehao{}那种意图更像是内心的一个声音\pozhehao{}而我也探索了这个声音背后究竟想说的是什么。但在我还没来得及表达我对这个声音的探索时,咨询师已经开始了TA一系列有关自杀风险评估的提问,比如说什么时候开始有这个念头、这个念头持续多久了、对此有什么计划或打算吗。我感觉到咨询师在看似平稳地说着话,但我也感觉TA只是忙于做评估。

昨天,我开始学关于自杀危机干预的课程,其中讲师说到了这段话:

\blockquote{
我们在做危机干预的时候,非常重要的是,我们不(是)好像答记者问一样有一个长串的问题,我们问完这些问题,来访者就安全,不是这样的,你问的每一个问题都应该是,和你理解来访者自杀危机,以及接下来要做安全干预的计划,要紧紧相扣的。……咨询师的心里面会感觉到惴惴不安,而正是这样的惴惴不安,这样的恐惧和焦虑,在新手咨询师当中,会更容易造成说我要问更多,我是不是漏问了什么,我再去问更多,我再去追问更多,那么在来访者的心中,会造成说自己真实的情绪体验,真实的经验被忽视,感觉到好像整个评估和咨询,是为咨询师的焦虑而服务。
}
在听完这段话后,我开始理解为什么那时候我的咨询师会忙于问各种问题、忙于做自杀风险评估,因为TA在那个当下很可能处于有意识或无意识的焦虑当中,并付诸行动。想到这里的时候,我对咨询师的责怪之情也减轻了很多,特别是我在接电话热线时也会遇到一些自杀风险来电,特别是既有自杀风险又有冲动性的来电。我记得那时候的我很慌张,一直在翻热线笔记里有关自杀干预的内容,不断问各种问题,但也只是沉浸于自己的焦虑当中,而完全没有留意到来电者在说些什么、在表达些什么情感。

不过,我也开始留意到,电话热线的培训里,并没有太多关于危机干预进程的内容,在培训完也难以做到“你问的每一个问题都应该是,和你理解来访者自杀危机,以及接下来要做安全干预的计划,要紧紧相扣的”,所以可能自己才会陷入焦虑当中,因为不知道自己在评估完风险后还能做些什么,不知道隔着电话另一头的我还能做些什么……(当然,那个电话热线的主要功能并不是自杀危机干预,也不建议有自杀危机的人打那个电话热线,但依然有不少有自杀危机的人打进来……)

其实,在回顾整个过程时,我会觉得自己蛮荒谬的,毕竟自己曾经也是曾经有过大概六年的自杀意图甚至是自杀计划的人。如果那时候的我“深处于”自己以前的状态去拨打电话热线,而电话里另一个头的人反而忙于缓解TA自己的焦虑并且忙于问各种问题、忙于评估,我估计会很恼怒。而那种恼怒感也在那次和我的咨询师的互动里存在着\pozhehao{}恼怒于“我感觉你只是在忙于阻止我,而不是理解我”。但这种恼怒感不单单指向着我的咨询师,也指向我自己\pozhehao{}接热线时的自己。

这种恼怒感\pozhehao{}无论是责怪他人也好,还是责怪自己也好\pozhehao{}的背后,都是在说同一句话:“你只是在忙于阻止我,而不是理解我”。所以有时候,我也会在想,我是否不能接受自己的有限性\pozhehao{}对理解他人的有限性,对理解自己的有限性,而不仅仅是接受热线的有限性。承认自己帮不到对方对我来说会有点难,因为很大程度上,我更难以承认我帮不到我自己,就像之前的那六年里的我并没有向他人求助,也没有找心理咨询求助。

是啊,我很难承认自己是无能的,难以承认连我自己都帮不到自己。因为这对我来说意味着,其他人也不可能帮到我,只有死路一条了,真正的“死”路。所以,比起我没有能力去救电话另一头的来电者,我更害怕的是,我没有能力救我自己。

就像是,如果我不去心智化他人、不去心智化我自己的话,就没有人能做到了。

我会想起周末的单休日和一个朋友见面时,我很疲惫,因为那天因为工作和排班的原因,我只睡了三个小时。

当和他在咖啡店里聊天时,我还会自己带出一些话题地聊,但后来在江边走时,我发现自己的精力越来越少,也就减少自己的精力消耗,更重要的是,减少心智化对方和心智化我自己。在江边时,他会经常看着我,但又不会说些什么。我觉得很失落,因为好像如果我不去心智化对方、不去心智化我自己的话,就不会有人心智化我了。就像身边的那个朋友不知道我现在的状态一样。然后他摸了摸我的头,我用手轻轻示意别摸。

后来继续走在江边时,我恢复了点精力,然后开始心智化我自己地对他说:“有时候我感觉活着也没有什么事情可做。当我回想起之前那些自己觉得很有意义的事情、觉得这个世界很丰富的时候,我发现其实丰富的、充满意义的更多是来源于自己。而一旦我不再将精力投注于这个世界、不再投注于身边的人的时候,一切都会变得毫无意义。而且也没有人会心智化我,只要我不去推动一些事情的发生,那么事情就不会发生了。就像是现在在走路一样,如果自己很累,不想往前走了,那么这条路就会停在这里。”他问我我想指的人是否就是他自己。我说:“是的,以及生活里的其他人都是这样。”

他说他很抱歉他没有像我这样的心智化能力,所以刚刚在江边时看着我的时候,他一直看不到我的状态是怎样的、不知道我脑海里的想法。他猜想我是否是因为上次见面的事情而决定我们彼此之后就要一直处于这样的相处状态。我说不是,只是因为这次我很累。而且我也说没事,肯定不是每个人有这样的心智化能力。然后我也(心智化自己地)解释了下刚刚我在江边的状态:“那时候的我感觉就像是介乎于睡着和清醒之间,有点像是飘走了。”他说他那时候也感觉我虽然现实里和他呆在一起,但灵魂已经飘走了。我回答道:“嗯。其实我并不太介意我自己身处于那个状态,那个状态就像是一个白板,倒是有不同的人会往这个白板里投射各种各样的事物,比如说你会猜我的状态是因为上次的见面。很多时候,我根本不在乎他人在乎些什么。”他说他之所以会这么猜,一部分原因是因为他在摸我的头的时候,我并不想被他摸头。我说是因为当我处于疲惫的状态时,我没有精力去心智化他人\pozhehao{}在心理层面不知道对方的意图,所以会在现实层面更加警惕他人的靠近和接触。

当他把手放在我头发上的时候,我的脖子和他的手腕触碰着,我感觉到温暖,同时也感觉到自己对温暖的抵触感,抵触于他人可能会伤害自己。这种出于自我保护的抵触感也让我更难以向他人求助,只能靠自己去去解决问题、去面对情感。而我之所以会变成现在这个样子,很大程度上也是因为,我曾经很奋力地试图拯救我自己,而且也成功做到了。虽然如此,我的内心深处依然很害怕他人的触碰,特别是当自己很疲惫或脆弱的时候,因为,毕竟是自己做到了,而不是他人做到了。

\blockquote{
根本没有人去帮助那时候的我,那种憎恨感,憎恨这个世界和这个世界上的所有人。如果我不去心智化他人、不去心智化我自己的话,就没有人能做到了;如果我不靠自己的力量去逃脱困境,去拯救我自己,就没有人会帮助我这么做了。这真是个令人感到孤独和绝望的世界。烂人到处都是,无能的人也到处都是。
}

我想,如果我面对着一个说完以上那段话的来电者的话,我会说:

\begin{compactitem}
\item 我会在想,你是否也会对自己感到愤怒?
\end{compactitem}

而另一部分的我会说:

\begin{compactitem}
\item 以前的我会是的,会对自己愤怒,但现在的我不会对自己愤怒了,我只会责怪他人!而且,凭什么我就不能责怪他人了!FUCK THEM ALL!
\item 他们会是谁?
\item 自己曾经身边的所有人。
\item 那现在呢?现在会有能理解你的人吗?
\item 会有,但理解的程度并不足够,依然不足够,一直不足够。
\item 好像你真正渴望的是有一个人能真正理解你,你一直渴望着。
\item 嗯,是的,是这样的。(从愤怒转为悲伤)为什么一直没有人能理解我!为什么总是我自己一个人去做所有的事情、去拯救我自己、去心智化自己和他人、去干着各种对方无能或不情愿去做的事情。
\item 这听起来确实会很累。
\item 是的,是的。我不想再坚持下去了,管他们的呢,他们与我无关。
\item 我也会在想,这会不会不仅仅与你和你朋友的互动有关,会不会也和你最近接热线有关?
\item 可能有关吧,心智化他人总是很累的。不过,我也会想尽可能地去心智化他人,无论是提高对方的心智化能力也好,还是给予对方一份倾听和陪伴也好,毕竟没有人应该落得和自己同样的境地。而且,如果是心智化身边的他人的话,说不定有一天自己身边会出现一个能足够理解自己的人呢。
\item 听起来,好像你的内心不止有充满愤怒和憎恨的部分,也有充满关怀和关爱的部分,而且这个充满关怀和关爱的部分也试图通过他人来间接地自我关怀和自我关爱着。
\item Maybe.
\end{compactitem}






% \dialoguelist{咨询师}{
% \dialogue{咨询师}{听起来,好像确实是这样的一个画面。一个更内在的你在内在的空间里,而另一个你则在相对外界的空间处理着外界的事情。我记得在之前的咨询里,你是一个协调者的角色,而这次又是……}
% \dialogue{我}{嗯,好像确实是两个不同的意象。现在更像是在内在空间里的自己和外界的自己。}
% \dialoguesepline{咨询师}{短暂的沉默}
% }
