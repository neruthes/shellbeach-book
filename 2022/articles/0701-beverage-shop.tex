\chapter{随笔 | 饮品店}

\ardate{2022-07-01}{YqMxbrG-iZFhY4\_RMvNgyQ}


“往左还是往右?”我和朋友站在居民区街道的三叉路口面前,左边是我们之前去过的饮品店,右边是我和一年半前离开了这座城市的男生在以前经常去的饮品店。“右边咯”,我朋友说。

这家店看上去和以前的样子变化不大,店门口面对着树下的夜晚街道,街道两旁停满了私家车。店门口摆了一些桌椅,桌椅正上方新增了挡雨棚和照明的灯带,店门前看上去明亮了不少,店里依然是亮黄和米白的装潢。

走入店内,我们发现店里没有人,然后从后面走进来了一个上了点年纪的阿姨,刚刚她正坐在店门前的座位上和一群人聊天。她走进吧台,问我们要点什么。我们纠结了一会儿,点好了饮品,在店里唯一一张双人桌座位坐了下来。我往窗外望去,刚刚店门前的那一桌人已经消失得无隐无踪。我看见店里的墙上挂着一副新的画,画很大,画里只有一片金色的叶子,画布是红色的,在我看来很奇怪,一点都不自然。

我记得一年半前的我就坐在这个座位上,而他就坐在我对面,就连我们最后一次见面也是在这里度过的。我记得那天晚上,我待到了晚上一点才走。我已经不太记得我们之间的聊天内容了,只记得他的语气总是很轻浮和欢快,很light,也很刻意营造一些开心的气氛,比如说“哇,这里有XXX耶~”,而有时候他也会说“也不知道自己哇来干嘛”。他的皮肤很白,摸起来很松软,就像是海绵一样。但与此相对的是,他总会以一种不在乎的态度保护着自己的自恋,比如说不在乎性、不在乎他人,并且努力试图让自己表现得不对任何事物和人投入太多的在乎,包括我们的每一次见面甚至是即将的分离。

我朋友在说着一些话,但我没有留意到他在说什么。我和这个朋友已经很久没见了,这次重新见面后,他给我留下的印象是:他的话题就像是他的眼神般飘忽,从来没有在任何一个地方呆着,总是会飘走到各种各样的地方去,比如说在家这边的工作状态、新买的相机的维修、家里人的入院、和亲戚的相处状态的改变、最近认识的男生的奇怪之处。但我并没有很想跟着他的脚步走,至少此时此刻不想。此时此刻的我的意识依然锚定在了这里,因为我想呆在这里。

如果一个人的话题、情感、意识甚至是存在本身总是在飘走,那么不就从未真正呆过任何地方吗?Never stay, never really there. 我开始说一些话来回应他,但我并不知道自己在说些什么,这更像是一些自动化的反应,比如说站立在地铁或公交上时保持着身体平衡;骑单车时控制着单车的方向和身体平衡;开车时不需要看表盘也感知着速度。我在自动化地回应着聊天,甚至打开着一些自发话题。和他一样,我也开始漂离于各种话题之间,但从未真正停驻于任何话题当中,从未真正停驻在这个现实世界里,我的意识还停驻在过去。

我记得那个男生以前会偶尔和店里的店员女生(已经不是现在在这里的店员)发起一些话题,比如说柠檬茶的茶底怎么样、哪款茶还没试过、她的老家是在哪里的,然后又会在话题渐入到一定深度的时候主动刹车,不再回应那个店员女生的某个提问。他每次想要与他人深入连接但又会在一定的深度那刹车的这一行为总会让我感到好奇:他过去究竟发生了些什么。但当我问他的时候,他总会说:没什么,只是突然不想聊了。也许我和他的关系也像是这样吧,当和他刚认识没多久,关系渐渐走近的时候,他就跟我说他打算(离开这座城市)回家。

喝着刚做好的热奶茶,我开始后悔没有叫店员少糖,因为现在喝到的每一口都是糖。我朋友问我平时会来这里吗,我说不会,一个人就不会想在(家)外面坐了,又无聊又孤独。同时我也回想起来刚刚我们还走在江边时,我朋友问我为什么总是来这边的江边,是因为还想念那个男生吗?我说不是,只是因为这个地方我比较熟悉,因为之前和那个男生在这里逛过很多次了,毕竟我也没有什么熟悉的地方了。

聊着聊着,我看了下时间,想到今天又快要结束了,就说:明天又要上班了。朋友说:他还全年无休呢,只要有客户找他他都要回复信息……然后我的意识又飘走了。

离开饮品店,走在外面的街道上,朋友说他觉得那家店的饮品做得很差,价格又贵看起来又做得很随便,我说那下次就别去的。不想喝的东西不再喝,不想去的地方不再去,不想见的人不再见。可惜的是,想见的人见不到,想触碰的人摸不着。

我的手机相册里还留着两次在那家饮品店里和他的合照。往回看那时候的照片,那时候的我还穿着有颜色的衣服,笑得还很自然很开心。好像在他离开之后,我就没有穿过有颜色的衣服了。

我记得临近最后几次见面的其中一次,我们坐在那家饮品店的门外,他问我:你觉得那个(店员)小姐姐在我走了之后会想起我吗?我说:你是想被想念吗?他说是的。


