\chapter{“依然做出了一个我所不认同的决定”}

\ardate{2022-07-22}{hl0jvCUdRVz\_AvRB\_wqZmw}




最近一个朋友和他现任的关系濒临断裂。他原来是在这座城市生活的,但当认识了他现任后就去了他现任所在的城市和他现任一起生活和工作。今年,他回来这座城市已经几个月了,而最近几天他又打算回去他现任那,试图挽回这段恋爱/室友/同事关系。

在这之前,我们有聊过几次,我问他为什么他既想要离开那个地方、离开他现任,但又想要回去那个地方、挽回他现任。他说他现任对他来说就像是个救生圈,没有了这个救生圈就像是在溺水。在我看来,他现任能给他提供一种自体感\pozhehao{}让他感到安全和稳定,而一旦失去了这份自体感,他的心理状态会变得更为抑郁,例如说他会回想起几年前的事情,认为自己之所以变成现在这样都是自己的错,二十几岁了还一事无成。

当昨天看见他的消息说他买了机票准备回去他现任那边的时候,我第一反应是气愤和难过,气愤和难过于我和他聊了好几次,但他依然做出了一个我所不认同的决定。不过,我也马上意识到我不可能去控制、去决定别人的选择,更不可能去改变他人的命运。我能做的也就只有支持他所做的决定和提供陪伴。

不过,事后我还会感到羡慕,羡慕他还有一个能给他提供自体感的人,虽然这个人在我看来并不是一个值得成为伴侣的人。我会想起一周多前有好感的那个男生,前几天想跟那个男生约见面时,他已经没有再回复我了,所以最近几天自己一直蛮难过的,意识也总是在走神,除了必要的事情外,不想去做任何事情,只想发呆。不过这种难过的感觉也开始逐渐退去。无论是一个多破烂的救生圈,好歹也是个救生圈,而我没有一个救生圈(前任)也已经三年了,让自己沉下去已经三年了,越沉越深。我是多么渴望能有一个救生圈,渴望一个能给予我自体感的重要他人。如果是三年前和前任的关系濒临破裂的那个我的话,我也会这么做,也会试图挽回那个也许能够让自己不那么支离破碎的人。

回想起和那个朋友的关系,其实一直都有羡慕的部分在,甚至还有嫉妒。在他认识他现任之前,他和他前任在这座城市相处得并不开心,然后他就找到了他的现任。三年前,他和他的现任刚开始熟悉起来的时间段恰好是我和我的前任的关系开始破裂的时间段。那时候我蛮喜欢去他曾经工作过的一个艺术室的小花园里找他聊聊天、看看书,缓解一下那时候的我在前任公寓同居时的各种糟糕的情绪,算是除了前任公寓外的另一个我还能呆着的地方。但后来随着他和他现任逐渐熟悉,他决定离开这座城市去他现任那边生活和工作。在他离开后,我便一个人去他曾经工作过的那个小花园附近逛,直到后来那个小花园随着艺术室的搬迁而彻底消失。再后来,连那个小花园附近的一间我经常喜欢去的风景还可以的便利店也关店了,一种自己赖以停驻的空间被连根拔起的感觉。在那之后,我便不再去那个小花园和便利店都彻底消失的地方了。

对我来说,那时候的我不仅少了一个能听我倾诉、能陪伴自己的人,我还看着这个人去了一个更遥远、更幸福的地方。除了羡慕之余,那时候的我还会感到一种被遗弃的愤怒感,甚至希望他能够永远死在他现任那边。

想起三年前我和前任的关系破裂的那段时间,几乎没有任何人在自己身边,这可能也是因为大学毕业后我回来这座城市时还没有认识到多少朋友。我记得那时候唯一给自己提供支持的,只有我自己了\pozhehao{}那时候一个有双相障碍的男生(后来成为了朋友)问我要不要去他家陪他几天,我也就去了他家过了几天的夜,通过陪伴他的过程中,我好像也间接地获得了陪伴。

而现在那个朋友又为了同一个人(他现任)而准备离开这座城市。虽然现在的我已经在这座城市里认识了一些朋友,不再是以前那么的孤独、无助、绝望、时时刻刻都想去自杀,但最近和他见了几次面后他又决定回去他现任那的时候,三年前对他的那份憎恨感好像又活了过来。



