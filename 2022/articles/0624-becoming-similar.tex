\chapter{“又再一次开始变得越来越像前任”}

\ardate{2022-06-24}{PKlvhEm6licGZUGs6InkAw}

% \dialoguelist{咨询师}{
% 	\dialogue{咨询师}{是啊。}
% 	\dialoguesepline{咨询师}{(沉默)}
% }



\midnote{\href{https://mp.weixin.qq.com/s/IbVoy4c0eDO6zmD7B5LLbQ}{哀悼,并继续前行}}

\midnote{\href{https://mp.weixin.qq.com/s/YD2fh125rBblwFZHBf5OaQ}{退群,投射}}


在一段时间前,我逐渐意识到自己又再一次开始变得越来越像前任。在那次喝茶聊天后,我自认为对他的理想化形象开始破灭、丧失,并且也想要去探索自己究竟想要成为一个怎样的人,而不再从他人身上去寻找我所想要成为的那个自我。但有点讽刺的是,后来的自己又渐渐变得更为像他。任何人和事物在他眼里都是一样的、同质的,并没有什么是独一无二的。同时他也很少会主动地去开启些什么,只有我问了他才会回应说:那找天聊聊。

我会在想这个过程是怎么发生的。在和前任喝茶聊天后,我意识到我并不想成为他的模样,那种看待事物和他人的同质化视角很灰暗、很色彩匮乏,就像是人生的任何乐趣都褪色了。但当我面临朋友的主动分离以及内心关于前任的旧形象的破灭和丧失时,我却开始发展出一套适用于自己的撤回方式:在察觉清楚自己的情绪以及看清楚情绪背后属于自己的部分后,从关系中撤回。

之前的我更像是在投射和内摄\pozhehao{}向他投射和内摄了某些特质,比如说独立、自由、喜欢深入交流、能理解他人的想法和感受,并内摄了回来。但后来当他说他并不是我所设想的这样,这更像是我的投射时,我不仅仅不再向他投射些什么,我也不再向任何事物和人主动地投注任何精力,不再主动向这个现实世界投注任何精力\pozhehao{}这一点或许正是又一次开始变得越来越像他的过程。

\blockquote{
	要让个体真正地达到理解、接纳和处理ta之前所无法应对的事件或关系,依靠的并不是矫正、修正,并不是一些来源于外在并试图让个体内化的改变,并不是内化后的矫正、修正和改变,而是内化后的叠加\pozhehao{}过去的那些令人感到痛苦的、骇人的事情和人已经发生了,已经在个体的内心内化为一个客体恒存了下来,但新加进去的东西(客体)足以平衡、足以冲淡这些令人感到痛苦的、骇人的客体。
	\blockquotesource{白色灯塔先生}{随笔 |}{2022}
}

不过,在写完昨晚的随笔后,我发现其中的同样也适用于自己。(内心的另一部分在说:不然呢?肯定是自己写的东西才最适用于自己呀。)我意识到自己并不只是单纯地变得越来越像前任的样子,并不只是单纯地内化了他的客体,而更像是在原有的自我的基础上叠加了新的自我\pozhehao{}一个更像是现在的我眼中的他的自我。

一直以来,我之所以会无意识内化他的形象,甚至会无意识内化很多曾经亲近但最终一如既往地消失了的人的形象,或许是因为对我而言,我一直没有一个好的客体(父母亲),所以在和他人相处的过程中,我也在不断地内化对方的形象,试图在自己的内心建立起那个在现实世界里早已丧失掉的客体(他人)。

而在叠加了那个现在的我眼中的他的那部分自我后,那部分自我开始与其他的自我(特别是在乎他人的那部分自我)发生冲突。之前的我会有一种自我消散的感觉\pozhehao{}就像是之前写的短篇故事“咨询室”\pozhehao{}外部世界在崩塌,内心的自我也逐渐随之崩塌。而我也选择“放任”这个过程的发生,没有想去紧紧地抓住什么意义、什么在乎。我知道有的人并不喜欢这样的对任何事物和人都满不在乎的自我,而我也没有选择为了讨好对方而刻意把自己塑造回以前的自我的模样。一方面是因为那时候的我早已没有像以前那么在乎他人了,另一方面是因为,如果自己真的那么做了,那只会像是在演戏,在我看来只会更加毫无意义、更为虚无。

现在的我会觉得这个转变的过程蛮必要的。因为如果我选择用抗拒、回避、否定等方式去排斥这部分新的自我的话,那这部分的自我只能一直被分裂或压抑,甚至有可能一直投射在前任甚至是其他人身上。但可能因为自己现在有在练冥想,所以是带着觉察、接纳和自我关怀去对待这个转变过程的发生\pozhehao{}在觉察到自己的状态开始变得和以前截然不同时,容许这个过程如其所是地发生,去探索和尝试如果自己真的完全沉入到这种状态里究竟会是怎样,同时温柔地对待自己,在一些感到耗竭的人际互动里及时撤回、在自己的状态不足以应对人际关系时允许自己不去处理人际关系。

现在的我会感觉自己有更多的灵活性,因为那部分满不在乎、不向这个世界投注任何精力的自我似乎不再和其他部分的自我产生冲突,而是能和其他部分的自我融洽地共存。我可以选择不在乎,也可以选择在乎,并时刻觉察到自己时刻都有进入在乎或不在乎的状态的自由\pozhehao{}想在乎的时候就真正地在乎,不想在乎的时候就真正地撤回。同时,我也会感到这样的自己更为真实、舒服、自在。

我会想到格式塔治疗里的“极性”概念。每个人内心都充满着各种矛盾的两极,对于最近的我而言,这两极就是在乎与不在乎。现在的我感觉,之前的自己并没有在这两极之间有多少自由\pozhehao{}当我在乎时,我看上去是真的在乎的;但当我不在乎时,我依然要表现得我还是有点在乎。但真正重要的是,我是否能在这两极上走到尽头\pozhehao{}是否真的能走到我很在乎很在乎另一个人的那个尽头,以及我丝毫不在乎对方、不在乎任何事物和人的另一个尽头。在尽可能走完两极的尽头后,我才能安然地停驻在任何一个我此时此刻感到舒服的位置,才有一份能够如此行事的自由,而不是像之前会担心万一自己真的太在乎对方或者是太不在乎对方而且还被对方看穿的话该怎么办。

