\chapter{自我探索 | 17}

\ardate{2022-01-17}{A5yRUXcaIJJhsxW3s0Q-Iw}


上个周末,我去了另一个城市F,去找一个之前没有见过面的人约个饭聊聊天。

在一起走去吃饭的店的路上时,走在一片安静的街区里,我突然感到很宁静。那种宁静感唤起了我过去的回忆,我回想起还在读大三时,当时的我也去了另一个城市Z,在经过城市Z里的一片街区时,那是我第一次感受到这种宁静感。在这两个城市的这两片街区里,街边都有不少在居民楼一楼开的路边咖啡店,偏雪白色的地面砖,干净的路面,间隔得十分整齐的树木,一列又一列的树荫。

距离上次感受到这种宁静感,已经是四年前的事情了。当我第一次身处于城市Z的那片街区,我的大脑开始自主地产生各种幻想,幻想着这条充满树荫的街道无尽地延伸,延伸至视野的尽头。我感觉这里更像是一个梦境,一个安静得与世隔绝的梦境、一个比现实世界更为宁静的梦境,而我想在这个梦境里一直停驻于此。

在大学毕业后,有好几次,我有考虑去城市Z找回那片自己曾经路过的街区,但我并没有这么做。在以前,还在读大学的我总想着之后如果还去城市Z玩时,说不定就能再次经过那片街区。如果有这样的机会的话,我会立即在路边停下来,到其中一家咖啡店坐着喝杯咖啡,停驻于此。但在那之后,我就再也没去过城市Z了,因为那时候我所认识的一个生活在城市Z的曾经的相识,早已不再相识。

但即使如此,如果我真的很想再回到城市Z的那片街区,即使是独自一人,我也会选择去。但我并没有这么做,所以我在想,我真正感到恐惧和悲伤的似乎不只是人来人往,还是:如果不只是人,连记忆里的场景也在现实世界里不复存在了的话,那好像任何事物、任何人都能消失不见。Nothing and no one can stay forever. I can't stay forever.

如果真的再次回到了那个地方,我可能就要面对那份幻灭,那份知道自己无法停驻于任何人、停驻于任何地方的幻灭,以及那份幻灭背后的无意义感和虚无感\pozhehao{}我又能停驻于什么呢。

\tristarsepline

但现在,我可以“随心所欲”地去城市F的那片街区。这一方面是因为城市F并不远,另一方面是,我不需要触碰到过去的回忆,也能找回以前曾经感受到的那份宁静感。“不需要触碰到过去的回忆”也会让我想到,这可能也是我在毕业两年半后再也没有尝试回去大学校园走走的原因\pozhehao{}因为过去的场景似乎有着许多会让自己感到更为脆弱的部分。我会很畏惧:如果我所看到的现状和我所记得的样子相差很大怎么办?这好像会进一步证实了我更为畏惧的想法:过去的一切早已回不去了,过去的事物和人早已消失得无隐无踪,抓也抓不住的事物和人。

也许选择活下去的代价之一,就是要选择放弃\pozhehao{}放弃那些自己所无法控制的事物和人,以及最终有一天,也要放弃自己。

\useimg{aimg/2022-0117-1.jpg}

\useimg{aimg/2022-0117-2.jpg}
