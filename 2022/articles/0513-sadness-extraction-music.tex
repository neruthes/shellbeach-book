\chapter{郁闷,抽离,音乐}

\ardate{2022-05-13}{IURj0izaiGFB3ebwLCkBGw}




今天是周五,我的心情很郁闷。郁闷这一描述更多是各种情绪的叠加和堆积甚至是交杂而成。过了今天我就能休息了,但我并没有感到开心,相反的,我感到郁闷。我想到周末和咨询师要聊些什么,回想起这周的经历,想到周初时持续了两三天的自杀意图,想到自己经历了家的内心形象的幻灭后自己还活着还真神奇,想到自己开始有一种像是《哈利波特》里校长办公室的冥想盘一样每当脑海里飘过自杀意图就将它抽离出脑海的能力,想到今天隔壁城市的朋友和我倾诉他的情绪、休息和工作的事情,仅仅从文字就能感受到他的绝望感和被困感,这也是我在接热线时总能够从有自杀危机的来电者身上感受到的,当然自己也不例外。

在走去办公室旁的洗手间时,我在走廊那停驻了一会儿,看向窗外,我看见了对面楼的产妇医院病房里的窗边站着一个穿着病号服的女性,她也看向窗外,不过我是从很高处看向她,她则是看向更低处的位置。我会好奇她的生活是怎样的,就像我好奇那个在隔壁城市的朋友的生活是怎样的,我也好奇那些我接过热线的那些来电者后来都怎样了。

周初的时候,当发现自己的自杀意图开始疯狂涌现,不停地去抽离的同时又不断自动涌现(抽离思绪并不总是一个可行的方式)时,我有考虑过联系我的咨询师,因为那难得会是我“需要”而不是“想要”心理咨询的时刻。但我又会在想咨询师说不定就会对此“大做文章”。同时我也会有点焦虑于周末的心理咨询。我意识到我看待咨询师的视角更像是督导师\pozhehao{}对我在这一周的生活过得怎么样的人生督导师。我也意识到这种移情更像是来源于我自己的:是我的一部分在充当着督导自己的生活的督导师,那个部分总是在judge我自己,有时候即使是写作时也不例外。

但这周我感受到的无力感的时刻还蛮多的,这可能也是我有点焦虑于周末的心理咨询的原因,害怕将这份无力感呈现给咨询师,同时也害怕将这份无力感呈现给自己。这周即使是一些小事也能触发到我的无力感,比如说下半身被暴雨淋湿了和听隔壁城市的朋友倾诉的时候。那种无力感在说:我没有办法挽回家的内心形象的丧失,什么都挽回不了,什么都做不了,我什么都改变不了。然后思绪就会停下来,情感开始喷涌,就像是开始下一场很大很大的雨。过了一阵子,自己的思绪会飘到其他地方去,比如说去想接下来干些什么。当我意识到自己的思绪从情绪那飘走时,我就知道这次的“雨”消散了。而在这一周里,这样的“雨”下了很多很多次,充满着悲伤和无力感的“雨”。那个隔壁城市的朋友说他也陷入了无法掌控情感的状态,我问他:“那就别掌控?”他说那就崩溃吗?那时候我思考了一段时间,我不知道应该怎么回应他,好像我怎么想都想不到一个对他而言有用或满意的回应,而我也说不出什么能安慰到他的话。当我意识到自己的思绪时,我跟他表达:“好像我给不到一个会让你满意的回答”。然后他回答:嗯,就比较绝望。

后来我想到,嗯,现在的我的状态也有点绝望,但没有太绝望,因为我从一开始就没有抱有多少希望,没有希望自己能修复、能找到那个曾经的家,没有希望自己能让这场充满无力感和悲伤的“雨”停下来,没有希望前任能表露出多少欲求,我甚至没有希望自己能活下去。但似乎恰恰是因为自己没有了希望,很多事情才在某种程度上“达成”了\pozhehao{}家的内心形象的丧失没有击垮自己、前任表露出了他对任何人和事物的无欲求、充满无力感和悲伤的“雨”时下时停,以及自己还活着。

最近还发现的一项新技能是,每当自己的脑海里“无缘无故地”响起一首音乐时,我就能根据那是哪一首音乐来判断此时此刻自己的心情是什么,比如说有的音乐是关于悲伤的、有的音乐是关于离别的、有关音乐是关于(死亡)丧失的,就像是大脑会自动播放曲子来安抚自己的情绪。当这些音乐响起的时候,是我唯一不需要觉察思绪和情感的时候,因为我的思绪和情感会自然而然地随着音乐流动,脑海里的一切都化为了音乐,就像是跟随着一只舞曲在跳舞,不需要做任何事情。

\noindent\begin{minipage}{\linewidth}
	\center

	\noindent\fbox{\begin{minipage}{0.9\linewidth}
			\center
			\textbf{关于悲伤}\\
			专辑:Seven Days Walking\\
			By Ludovico Einaudi
		\end{minipage}}
	\vspace{8pt}

	\noindent\fbox{\begin{minipage}{0.9\linewidth}
			\center
			\textbf{关于离别}\\
			歌曲:Say Goodbye To The Sea\\
			By Trembling Blue Stars
		\end{minipage}}
	\vspace{8pt}

	\noindent\fbox{\begin{minipage}{0.9\linewidth}
			\center
			\textbf{关于(死亡)丧失}\\
			纯音乐:A Gleam in the distance\\
			By 岩崎琢
		\end{minipage}}


\end{minipage}






