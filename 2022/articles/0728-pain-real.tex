\chapter{“被对方更为痛苦、更为真实的一面所吸引”}

\ardate{2022-07-28}{YjJbvtrxia1x3jM28qGXrQ}







前几天和一个新认识的男生面基了。在第二次见面时,我问他:我会给你留下怎么样的第一印象吗?他说的其中一个第一印象是“真实”\pozhehao{}我会给他一种真实的感觉,好像我没有在向他隐藏着什么。我回应说:但我依然有一种不足够的感觉。他问我:你会有一个怎样的目标吗?我说没有,不是一个目标,而是在于当我试图去展现、去看见自己更为真实的一面的时候,自己内心的一部分依然在抗拒我,依然隐藏着一些不为我知的秘密,阻止着我去看见一些更为深入的内心事物。所以即使我已经很努力试图成为一个更为真实的人,但这对我来说依然感到不足够。

今晚我和一个朋友约饭时,聊到在上一次聊天里,他跟我表露说,他就是这么无趣,不会有什么变化,他还是之前的那个样子。我说,当我回想起你上次这么说的时候,我会在想:你真的就只能这样了,好像我就只能靠我自己了,那我又能做些什么呢?我的内在好像一直以来都在变化,而我好像能凭借我的变化来促成我们之间的互动的变化,能以一种不一样的方式来与你互动,并从中产生出一些新的东西来自我满足我对新鲜感的渴望。现在的我好像不再期望对方能有多有趣、能有多丰富,而是我能够通过让我自己变得有趣、变得丰富的方式来获得彼此之间的互动变得有趣和丰富的喜悦感。

他问我为什么我还想去见他,因为在他看来,我并不享受这个(和他见面的)过程。当时我说我不知道为什么,也想不到为什么。和他见完面后,在地铁上的我听着音乐,闭上眼睛用心去看:自己之所以还想继续去见他,与他是否有趣无关,而是在于他和我的聊天、互动过程(包括今晚)中,他会很敢于去面对、去向我表露他自己内心更为脆弱或柔软的一面、内心感到不安、烦躁、焦虑的一面、更为真实的一面。当我看到他的这一面时,我会感到一种喜欢的感觉。

后来我在想:为什么我会被对方更为痛苦、更为真实的一面所吸引?因为我好像也在做着同样的事情\pozhehao{}试图将自己展现得更为真实,试图看见自己内心更为真实的一面。那为什么自己要那么努力地想要不断达成这一点?因为在我的成长环境里,大人们总是在隐藏着什么\pozhehao{}向彼此隐藏着情感,甚至也向他们自己隐藏着情感,自欺欺人地说:我没有感受到愤怒、我没有感受到憎恨。从来没有人会去正视无论是他人还是自己内心的事物,只是停留在事实层面的信息交流:吃饭了吗?作业做完了吗?考试成绩多少?

但事实上,事情不并只是停留在事实层面,但往往内心的事物都被遮盖了、隐藏了,向彼此之间隐藏了,向自己也隐藏了,从而刻意刻画出一副异常和谐的画面:好像我们之间的关系不是早已支离破碎一样\pozhehao{}那怕任何一点冲突,都足以让这么多支离破碎的关系彻底毁灭。但正是因为一开始的不愿意、不选择正视他人和自己内心的情感,才使得彼此的人际关系落到现在这般形同虚设的现状。好像唯有共同维持着表象,这样的关系才能够继续下去,除此之外,便没有其他出路了。没有人愿意戳破这些关系背后的痛苦、折磨和憎恨。而我至今也不愿意戳破,因为小时候的经历让我意识到,试图去戳破这些禁忌只会招惹更多被打被骂的痛苦体验,他们并不值得我向他们投注任何精力。

我想起在某一次心理咨询里,咨询师说:当你想要去拥抱他人时,好像你更想要的是他人来拥抱你,而你通过拥抱他人,从中间接地拥抱了你自己。这句话似乎依然适用于自己会被对方更为痛苦、更为真实的一面所吸引的情况\pozhehao{}我真正渴望的不是看见对方更为痛苦、更为真实的一面,而是希望自己那更为真实、更为痛苦的一面能够被对方看见、能够被对方接纳、能够被对方拥抱。因为在我的成长环境里,即使是前任还是之后认识的男生,都没有从对方身上获得过这样的看见、接纳和拥抱。

所以在现在的人际关系里,这种喜欢的感觉似乎依然引导着我去获得一些在过往成长经历里一直缺乏的东西,那份对更深层的真实和痛苦的接纳和拥抱的体验。


