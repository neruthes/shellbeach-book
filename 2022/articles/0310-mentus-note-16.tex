\chapter{白色灯塔先生的情感笔记 | 16}

\ardate{2022-03-10}{U227LMQJVlxzF1Qrxy4P2Q}




\subsection*{那片孤独、无意义和黑暗的黑夜感}

咨询师继续问道:“在那片黑暗里的感受是怎样的?”我说我感觉到自己无欲无求,就像是之前在很长一段时间的抑郁里,除了只有在一两周的时间里我能感觉到强烈的悲伤外,接下来几个月的时间里我看可能会什么都感受不到。这种无欲无求的感觉就像是我在之前的咨询里不知道我们是要干些什么,我不知道我在咨询里要干些什么的迷茫感。但在那种无欲无求的状态下,我能感觉到自己内心深处暗藏着一种强烈的悲伤,悲伤于没有人能看见我,没有人能听见我的哭声。我像是躺在了床上,抱着被子缩成一团地在黑暗里哭泣,没有人能看见我,没有人能听见我的哭声……

咨询师问:“这会让你想到什么场景吗?”我说我能回想起我从前任公寓搬走的那天晚上,我一个人背着两个背包拖着五袋行李走在公寓小区路上。那天晚上我一直想等前任回来公寓,想和他聊一聊我们之间的事情,哪怕是见最后一面也好。但他一直没有回来,他说他还在忙。直到最后一刻,我依然幻想着前任会回来公寓,见自己最后一面。到了零点后,我决定自己终究还是要离开,便一个人拖着行李走在路上,还被公寓小区保安拦了下来,说住户搬家需要搬出证明。我打电话给前任,然后把手机递给保安听电话,在确定了我不是住户不需要搬出证明后,我继续走在半夜的街道上。当时的我很累\pozhehao{}身体很累,但心里更累\pozhehao{}但我不能停下来,我不能突然倒在路边一动不动,我不能躺在地上缩成一团地哭泣起来,因为我还要把行李搬到另一个“住处”。后来我走到路边,打了个的士离开了。

在那之后的一年多,我不敢一个人晚上逗留在街上,因为我会感觉到黑夜里的那股黑暗会侵蚀自己的内心,让我“重回”那个一个人拖着一大堆行李从前任公寓搬走的场景,会让我感到自己越来越难以动弹,只想倒在地上缩成一团地哭泣。

\citebook{自我探索 | 3}



\subsection*{能被看见和理解的开心}

在微信上,他问我,和他聊天很开心吗。我说蛮开心的,因为他有一种能把话题聊深的能力,同时也愿意去理解和看见他人的感受和想法,不像是在我那公众号群聊里的我自己的感受总是不被看见一样。他说我不用太苛求自己的感受要被看见这一点,而且群里的人也没有义务做到这一点。我说是的,正是因为没有义务还愿意去那么做才显得这样的人(包括他)更为宝贵。

\citebook{随笔 | 国庆,白}



\subsection*{自知自己的存在方式之独特所带来的孤独感}

在说完这一切后,我问咨询师:“你会觉得我是个奇怪的来访者吗?” TA说TA认为我所用的方式(构建空间来获得安全感)会比较有趣,然后问我为什么会这么问TA。我说,因为每当自己的自我表露程度有点深的时候,我都会好奇对方是否能relate(关联)到我的存在方式。我似乎一直知道自己和身边的人很一样,所以我会想知道其他人会不会也有类似的、独特的存在方式,而relate则是能够连接人与人之间的存在孤独的桥梁,比如说,当我丧失了一个重要的他人,但对于对方而言,我们丧失的并不是同一个人,但可能对方也在曾经丧失过对对方而言重要的他人,所以对方能够将TA的经历与我的经历相relate(关联)。Relate能够一定程度地消融那种存在孤独所带来的孤独感,那种知道自己的存在方式是独一无二的所带来的孤独感。

\citebook{随笔 | 反感有人用“正常人”一词来描述我}



\subsection*{对有人用“正常人”一词来描述我而感到反感}

这可能也是为什么我会那么反感有人用“正常人”一词来描述我的原因,因为这时候的我会立即强烈地感受到:噢,原来对方并不能relate我的存在方式,并不能和我的存在产生连接。原来在对方眼里,我一直都没有和他们建立起联系。

\citebook{随笔 | 反感有人用“正常人”一词来描述我}



\subsection*{Feeling Good}

在国庆的最后一天,我又去走了走江边。那天中午正是台风来临前夕的好天气:头顶的阳光晒得猛烈,眯着眼睛的我感受着远处吹来的大风,远望着那不断波动又看似静止的蔚蓝江面。虽然我能感受到阳光照在我的双臂和脖子上的热量,但那份热量又不断被持续吹来的大风所吹散\pozhehao{}阳光炙热但依然温暖。听着人们和我一同走在江边的说话声,仰望着天空中被吹散得毫无形状可言的云,我突然想起来一首歌《Feeling Good》\footnote{Nina Simone}里的歌词:

\blockquote{
Birds flying high you know how I feel\\
Sun in the sky you know how I feel\\
Reeds driftin on by you know how I feel\\
It's a new dawn\\
It's a new day\\
It's a new life\\
For me\\
And I'm feeling good

Fish in the sea you know how I feel\\
River running free you know how I feel\\
Blossom on the tree you know how I feel\\
It's a new dawn\\
It's a new day\\
It's a new life\\
For me\\
And I'm feeling good

Dragonfly out in the sun you know what I mean, don't you know\\
Butterflies all havin fun you know what I mean\\
Sleep in peace when day is done\\
That's what I mean\\
And this old world is a new world\\
And a bold world\\
For me

Stars when you shine you know how I feel\\
Scent of the pine you know how I feel\\
Oh freedom is mine \\
And I know how I feel\\
It's a new dawn\\
It's a new day\\
It's a new life\\
For me\\
And I'm feeling good
}

\citebook{随笔 | 江边,海边}



\subsection*{因喜欢之人无法理解自己的情感而感到伤感}

梦醒了之后,我把梦境的前半段内容告诉了那个曾经离开了这座城市的男生,他说:“哈哈哈哈哈哈 typical me”。那时候的我比在梦境里的失落还要感到伤心,因为他似乎无法理解我的情感。

\citebook{梦境 | 去了那个曾经离开了这座城市的男生的城市}



\subsection*{对离别的倒计时所感到的伤感}

每一次的见面似乎都在提醒着我,我们的见面机会又少了一次,就像是时钟的摆动,每一次滴答滴答都预示着离尽头越来越近、越来越近。

\citebook{梦境 | 去了那个曾经离开了这座城市的男生的城市}



\subsection*{喜欢之人离别后的伤感}

然后我也在想,如果我真的等到了那个退房的人离开呢?我可能就会在房间里安顿下来,真真正正地独自一人在一间没有他的房间里,就像现在的我一样,真正地独自一人在这座没有他的城市里。

\citebook{梦境 | 去了那个曾经离开了这座城市的男生的城市}



\subsection*{因对方试图理解我的经历和故事的努力而感到有所感动}

我向和我一起做这个课程作业的同学表露了关于前任无故消失的事情。虽然在整个过程下来,那位同学只是在试图厘清事情的来龙去脉而没有太关注我的情感,但我依然被对方那试图理解我的经历和故事的努力而有所感动,并认为那位同学在后续反馈部分的一些观点还是捕捉得蛮准确的。

\citebook{随笔 | “‘回忆在,人就在。’ 我可能这辈子只顾着回忆了。”}



\subsection*{曾经重要他人的无故消失给自己留下的对亲密关系的不安全感}

我并不打算重新恢复感情生活,而只是希望知道前任彻底消失的原因是什么,因为即使是几个月前恢复联系后,前任也依旧会无缘无故地消失上好几个月(当然对他而言,理由恐怕依然是几个月前再次见面时他所说的“工作忙”)。这种突然的、彻底的消失在我对于亲密关系的感觉上留下了一种不安全感

“为什么一个人能消失得无影无踪?”这种不信任感一直潜伏于之后我每次想要开始一段新的亲密关系的尝试中。这种隐隐约约的感觉一直在告诉我,亲密关系里的对方是不安全的、不牢固的、会随时无故消失的、会随时无故离场的。

……让我感到困扰的更为核心的问题是一些关于存在主义的议题,比如说为什么一个人能消失得无影无踪,就像毛姆所著小说《刀锋》里男主人公在战场上目睹战友在自己面前突然被杀后突然意识到了人的存在之渺小和虚无。“为什么一个人能消失得无影无踪”这个提问后来也一直在我试图开始新的亲密关系里若隐若现,那种不安全感无可否认地影响着我对未来可能存在的亲密关系的判断和态度。……正如那位同学所说,TA认为我对前任已经没有太多留恋,但前任留给我的无故消失的这个问题可能会长期存在于我的回忆乃至生活里。

\citebook{随笔 | “‘回忆在,人就在。’ 我可能这辈子只顾着回忆了。”}



\subsection*{对前任无故消失的答案的渴望,以及对答案背后可能存在的平静的渴望}

在看完那位同学写的反馈,我想到了电影Nomadland里的一句台词:“'What's remembered lives.' I maybe spent too much of my life just remembering.”(“‘回忆在,人就在。’ 我可能这辈子只顾着回忆了。”)我想我可能也spent too much of my life试图去找到前任无故消失的答案了。在寻找的道路上,我收获了许多知识和技能,甚至心智能力也成长了不少,还习得了理解他人的内心世界以及安慰他人的情感的能力。但在内心深处,我似乎依然渴望着得到前任无故消失的答案,依然在spending too much of my life trying to find the ANSWER that would give me some explanations and peace。

\citebook{随笔 | “‘回忆在,人就在。’ 我可能这辈子只顾着回忆了。”}



\subsection*{自己的吼叫背后的愤怒与畏惧}

我回想起当时自己的吼叫背后的愤怒,是来源于从小学到初中对被霸凌经历的反击。那时候的我一直被霸凌,无论怎么反抗都没用,直到后来我释放了内心一直被压抑的那股愤怒,在被霸凌时将自己转换成一种丧失理智只为摧毁对方的状态。那种状态让初中的霸凌者不再霸凌我,甚至还开始有点尊重我,有可能是因为他在我身上看见了他自己的影子吧。在过了那么多年后,那样的被霸凌的经历给我的感受依然存在着,那份不断被挑衅(水枪喷出来的水的不断逼近)所激起的愤怒,以及那种极度想将愤怒宣泄出去(吼叫)的欲望。

当时自己的吼叫还让我想起了我父亲。我和我父亲的声音很相似,一种我并不想要拥有的相似。相比于小时候会打骂我的母亲,我父亲没有打过我一次,但会呵斥甚至是吼叫,而我父亲的吼叫和愤怒的表情让我感到他比我母亲更为可怕,就像是他想杀了我。所以即使是现在,我也一直和我父亲保持着比我母亲更疏远的距离。在长大后,我的嗓音变得越来越像是我父亲的声音,就好像自己变成了小时候那个自己所畏惧的野兽。

\citebook{随笔 | 水枪,吼叫}

\subsection*{对具有足够的涵容能力的父亲形象的羡慕}

我不禁有点羡慕那个男孩能有这样的一个父亲形象,因为我从来没有这样的父亲形象,我也没有涵容过我的情感的人,包括那份长期被霸凌所遗留下来的愤怒。可能因为这样,我才会轻易地把那份愤怒释放出来吧,就好像我依然是那个时刻需要警惕身边的欺凌者的小学生/初中生,就好像那个男孩被我所吼叫后站在原地哭泣时的那般弱小。

\citebook{随笔 | 水枪,吼叫}

\subsection*{对方愿意将精力投注于自己身上所带来的抱持感}

在我倾诉的过程中,对方一直试图理清我的困扰的来龙去脉,并为我假设各种情况来试图解决问题,试图为我找到一条情感上的出路。虽然对方并没有直接“触碰”到我的情感,但我依然被TA的努力有所感动,感动之处在于另一个人愿意将TA的精力投注于自己身上所带来的抱持感。

\citebook{“提供倾听”的缘由}

\subsection*{从来没有成为过他人生活里重要的一部分的可有可无感}

我从来没有成为过他人生活里重要的一部分,至少说我从来没有感受过成为他人生活里重要的一部分的那种感觉,那种被重视、被珍重的感觉。我一直感觉自己是可有可无的,而距离那种被视为他人生活里重要的一部分的那种感觉最接近的一次,就是第一次在前任公寓过夜后的第二天早上他把公寓钥匙给了我的时候。

我记得这种感觉起源于我还在读小学的时候。我刚开始有记忆的时候是在3岁之前还住在外婆家的时候,后来因为幼儿园开学便搬回父母家住。那时候在外婆家住的原因是因为我有脑积水,所以总是在哭闹。父母暂时无力应对我的情况,所以把我送去了外婆家生活。父母在我读小学的时候告诉我我曾经有脑积水这件事,并说到当时看病的医生建议他们生多一个孩子,因为在长大后的我会是智力低下的。我觉得那个医生的言下之意是建议我的父母把我给遗弃了,并让他们再生多一个正常人。还在读小学的我在听到这段故事后,突然觉得一切都说得通了:怪不得自己那么蠢、做作业那么慢、每晚的作业都做得那么晚。

由于小学时期的周一到周五几乎都会因作业问题而被打,在得知了自己曾经有脑积水以及父母曾经考虑过将我遗弃并多生个正常的孩子的故事后,那时候的我在想:为什么自己要被生下来?为什么自己要被诞生?为什么我要遭受这一切?那时候我带着这些问题去问我妈,我妈说:“因为我和你爸想要一个孩子,所以你就被生下来了。”   那时候的我突然意识到了自己的“偶然”性:我只是偶然地出现在了这个世界上,父母所爱的是他们的孩子,那个他们选择不遗弃的孩子,但那个孩子并不是我。如果他们再多生一个小孩,他们依然会用同样的方式来对待TA。

现在的他们依然会偶尔给我“塞“一些我并不想要的东西,因为他们并不知道也没有想过去在给我“塞东西”之前问一问我我真正想要的是什么,而是他们认为他们主观世界里的那个孩子需要些什么。他们爱的是他们的孩子,但那个孩子并不是我。当然,这也不能说他们遗弃了我,因为他们从未拥有过我,一个人无法遗弃TA所从未拥有过的事物。

当我意识到自己的偶然性后,那种感到自己可有可无的感觉开始逐渐萌芽,并在之后的人际关系里不同的人的无故消失行为的不断重演中逐渐浮现出来。

……当人际关系(特别是亲密关系)的对方无故消失后,我还会有一种被悬置的感觉,就像是一个母亲在商场的角落里扔下她的孩子,并叫TA呆在原地别动,然后就欢然地去逛商场,在逛完商场后再将孩子捡回来。那个孩子就像是一件衣服,随时都可以被扔下,随时可以再被捡回来。我从来没有成为过他人生活里重要的一部分,正如在亲密关系里的对方总是可以找另一个人或独自过着与我无关的生活,而我并不是唯一的、并不是不可被替代的。

\citebook{自我探索 | 4}

\subsection*{对对方可能无故消失的不安全感,被无故消失地对待的被散席感}

在一段关系深入的过程中,我会害怕对方的无故消失,所以每当我自认为察觉到一些不详的迹象或感到一些不安的感觉时,我会将两人关系的距离拉远,将自己保护在安全的范围内。另一方面,我和在我的生活中曾经出现过或仍在身边的许多人(特别是在亲密关系方面)相处时,对方都有过无故消失的行为,比如说:(1)和初恋分开后初恋消失了两年;(2)和前任分开后前任消失了一年多才重新出现,然后又消失了几个月;(3)和一个持续频繁见面半年的男生在一年前也无故消失了;(4)年初认识的那个曾经还在这座城市的男生有一段时间以“除非我主动在这里和你聊天,否则你发给我的消息我都会选择已读不回”这一方式消失了一个月;以及(5)一个之前试图一起深入关系但我却突然退缩的男生也消失了一年时间。

在我的人际关系(特别是更为亲密的关系)里,无故消失似乎总是一个若隐若现的主题。

在人际关系(特别是更为亲密的关系)里,对方的无故消失所带给我的感觉,就像是我在一个漆黑的电影院影厅里,电影放映完了,周围的人都走了,只有我还在座位上\pozhehao{}那种散席的感觉。

\citebook{自我探索 | 4}

\subsection*{对前任的爱与恨相交织的平静感}

在终于能够看见那份恨意背后的爱意后,我似乎不再执着于要找到前任无故消失的原因了,因为我好像找到了那份为了找到前任无故消失的原因背后我真正想要找到的平静。现在的这份平静是由恨意和爱意交织而成,而我不会选择其他的方式来遗弃两者当中的任何一部分。这份平静来源于:I  wouldn't have it any other way.

以前的我自认为我是想要找到前任无故消失的原因,但我真正想要找到的是在得到前任无故消失这一原因后我所能找到的平静,但不是一种将恨意抹去的平静。因此,我所找到的平静是一种由爱意和恨意两者相互交织、相互平衡的平静,以及那份能够与恨意相互平衡的爱意。

\citebook{自我探索 | 4}

\subsection*{玩具被扔掉的可惜感,对依恋的遗弃感}

我记得小时候我一直在买玩具,最后直到上小学时我已经有两箱玩具,都是自己很喜欢的玩具。但我妈说上了小学就不能再玩玩具了,所以就把那两箱玩具都扔了。我难过了很长一段时间,所以后来我也把对父母的依恋也顺手扔了。

\citebook{文字分享 2021-10-27}

\subsection*{想要亲近和远离同一个人的撕裂感}

但我之所以又打消了和他约见面的念头,是因为我感觉自己在被自我的两个部分所撕裂着。一个部分是爱着他的自我,另一个部分是恨着他的自我。爱着他的那一部分自我一直很想和他见上一面,但恨着他的那一部分自我却一直很抵触和他见面,想要尽可能地远离他。回想起来,这两者的撕裂感在年初和他见上一面的临近几天前一直都有感觉到,不过那时候的我并没有察觉到这两个部分的自我,而是想着:“既然能见一面,为什么不见呢?难道要让一个这么难得的机会消失吗?” 不过年初时候的我还处于抑郁的心境以及无意义与虚无的心境里,所以没有能察觉到自己的细微的情感象征和意义可能也是一件对我而言蛮正常的事情。

\citebook{随笔 | “这可能就是对我而言更为真实的关系吧”}

\subsection*{只有自己一个人在推动着彼此关系进展的疲惫感}

然后我就这一点去问了他,他回复说“并没有(操控)”。但他的回答里也加上了一句“见不见的都行,随意”,我回想起这一点正是曾经和他处于亲密关系时我十分反感的一点:好像每一次的见面都需要我很努力地去达成,好像只有我一个人是在推动着彼此的关系的深入和修复,而很多时候,当我发现自己“推不动”时,我心理上真的感觉很累很累,一种对人际关系(特别是亲密关系)的无力感。

\citebook{随笔 | “这可能就是对我而言更为真实的关系吧”}

\subsection*{更为真实的关系的情绪交杂感}

我自我感觉了一下,我好像能从他的那句回答里看见他的自大,而他的自大似乎是他最为吸引我的一部分,但这一部分也是一直让我倍感受伤的一部分,特别是他那自大的一部分选择了去追求一个客体而不是追求我这个“他人”时。

但另一方面,我喜欢他那自恋的一部分,但我并不爱他那自恋的一部分,因为他那自恋的一部分无一例外地让我倍感受伤。我爱的是他那温柔的、愿意为了我而在场的那一部分。对他的各种情感似乎都对应着他的不同部分,比如说爱着他那温柔的、愿意为了我而在场的那一部分;喜欢着他那自大得目中无我的一部分;恨着他那无故消失的那一部分以及他那自大得目中无我的一部分。

我想这可能就是对我而言更为真实的关系吧\pozhehao{}对一段更为真实的关系的对方总是怀有着各种各样的情感,而每一个不同的情感总是对应着对方的不同部分。

\citebook{随笔 | “这可能就是对我而言更为真实的关系吧”}

\subsection*{人际关系的无意义感}

但另一方面,我依然会感觉到孤独,因为我并没有得到在人际关系方面我所渴望的抱持感和意义感,而很多时候只是停留在了每一段人际关系的表浅之处。我想要在纸张上写下丰富且深刻的内容,但我并没有或者说不能够总是做到这一点。这种对人际关系的深度和广度的渴望依然深藏在我的内心,只不过能满足这种渴望的人并没有多少。这似乎也是我对亲密关系的渴望\pozhehao{}我依然渴望着能深深地去爱另一个也能给予我爱意的人,但还没有一个这样的人。

\citebook{随笔 | “我是否需要心理咨询?”,难以感觉到人际关系的意义感}

\subsection*{对金钱的无力感}

我自我感觉了一下那种对金钱的无力感的来源。那种感觉来源于在读小学的时候,我很喜欢玩网游,但网游里的大多数装备或赛车都需要充钱才能获得,但我父母并不会给我生活费,更不想让我把钱花在电脑游戏上,最好是连游戏都别打,做个他们眼中的“好学生”。甚至我连玩电脑游戏的时间都是每天放学飞奔回家玩上一个小时,然后趁我妈回家之前在主机板上喷水散热,以防被她发现。当然有好几次我还没来得及关电脑她就踏入家门了,然后就是被打。

由于我没有生活费,所以我要省钱才能去买点券,然后再将点券冲进网游离。我开始从我妈给我的早餐费里省钱,比如说那天可能就不买面包吃了,或者买一个更便宜的面包,剩下的钱自己存着。当然有时候外婆会背着我妈偷偷给我零花钱。所以每次去放学路上的买点券的小店时,我都是带着一大堆一块、五毛、五块去的,凑齐了二十块或者是五十块的点券就拿回家冲。

所以每当我缺钱时,我的大脑便只会想到怎样才能省钱、怎样才能活下来、怎样才能用省下来的钱干自己想干的事情。我想这种成长经历带给我的影响,那种自我消融的无力感依然会在每次我需要在意余额的时候涌入心头。

\citebook{自我探索 | 6}

\subsection*{对分离的恐惧感,对分离后孤独一人的绝望感}

我感觉我好像变成了另一个人\pozhehao{}我很想去尽快地和他见面,很想去“抓紧”他,不想他再离开这座城市,虽然我知道他待在这里的时间总是不长。

我在想为什么我那么想要去尽快和他见面,为什么那么想去“抓紧”他?这是因为我很渴望他,很喜欢他吗?但这种情感的强度根本不属于喜欢之情,而是恐惧。我会很恐惧他的离开,正如之前他每次来这座城市时我都会很想有更多见面的时间,甚至会在脑海里倒计时着我和他之间还能聚在一起的时间。这种患得患失的感觉在每次我和他见面的时候都会充斥于我的脑海里,我很害怕失去对方,即使我从未拥有过对方。

这种恐惧的感觉似乎不只是恐惧于他的离开,更是恐惧于在他离开后我会身处的境地\pozhehao{}在他离开后,我便会真正孤独一人地在这座城市里了,没有任何值得我活下去的事物。但当我察觉到这一个想法时,这对于以前的我来说确实是真的、十分恐怖的、让我想要极力逃避的,但这对于现在的我而言却并不是真实的,因为现在的我还有我的爱好、我在上的课程、我喜欢做的事情、我想要达成的事情。

在之前和他的每次见面时,我所身处的状态是:我没有任何值得活下去的事物,每天几乎都只是像是一副行尸走肉般地“仅仅延续着”。所以以前每次他来到这座城市时,我都会想要极力地抓住唯一值得我活下去的那个人。无论那个人是他,还是前任,我都会紧紧抓住。这就像是在一片黑暗里的我总是在等待着一颗流星的划过。当那颗流星划过的时候,周围的一切都被照亮了,我的生活也就充满着各种各样的色彩,直到那颗流星消失后,周围的一切又消失在了一片黑暗当中,我再次发现自己又孤独一人地在这里,不知道下一次流星划过的时候会是什么时候。

\citebook{自我探索 | 6}

\subsection*{自己并不属于任何地方的流浪感}

其实从小到大,我都有那种感觉\pozhehao{}自己并不属于任何地方的感觉。无论是从小学、初中、高中、大学还是公司。我记得我还在读小学时,中午我会被安排去学校附近的一间负责托管的幼儿园去午休。在吃午饭的时候,我想去和班上那些成绩很好的同学坐在一桌,但是他们并不欢迎我,甚至会赶我走。所以我走了,我去了学习成绩和我差不多甚至更差的那一桌人,但那一桌人对我也并不友善。他们喜欢在吃饭时玩一些口头的游戏,而在游戏里我总是那个输家。直到后来,每次吃午饭我都是一个人独占一张桌子,因为没有人想和我坐同一桌。这时候幼儿园的女主管(是一个应该已经退休的年轻阿婆)会夸我吃饭吃得真快,不像其他桌的那些人一样只顾着聊天不吃饭,导致午休时间到了他们还没吃完饭。

这种被排挤的经历从那时开始就在我内心埋藏下了一种我不属于任何地方的感觉,直到后来我身边的环境里已经没有霸凌者或排挤我的人之后,我依然自我封闭着,因为敞开心怀便意味着随时都会受伤。所以,无论是小时候的环境,还是我从小时候的环境里延续下来的自我保护策略,这两者都共同导致了我从小到大都有一种我不属于任何地方的感觉。而当我想到我要独自一人踏入那个高档酒店时,这种感觉猛然涌上心头。这种感觉似乎在提醒着我,我不属于那些成绩好的人,也不属于那些成绩差的人,我哪里都不属于,谁也不属于,现在也依然如此。

\citebook{自我探索 | 6}

\subsection*{依然无法用聊天来拉近彼此之间的距离的无力感}

相较于上一次的找不到话题,这次在和鬃积雨云吃早餐时的聊天里,我依然感觉到我“触碰”不到他。我本以为自己在看了不少相关的书籍、上了一些课程甚至有一些练习经验后,我会做得更好,让彼此的聊天更加愉快且深入。但我最终能够做到的只是将日常的话题往深处引,而且在对方不想表露太多时,我自己能够多做一些自我表露来打开话题,“引出”对方的好奇心。即使我学习了那么多东西,我依然会感觉到一种无力感\pozhehao{}对于想要亲近的人,我却无法通过简单的聊天来拉近距离的无力感。

\citebook{随笔 | 鬃积雨云2 ;自我探索 | 7}

\subsection*{对他人的不在乎感}

在每次见面里,我似乎都能感觉到他是在乎我的,所以我也很在乎他。但在我对这种在乎的珍重的同时,这似乎也反衬了在我的日常生活里几乎没有任何人在乎过我想要的是什么。

在最近几周的社交里,人际交往的无意义感越来越强烈,因为在人际交往里的对方并不在乎我想要的是什么\pozhehao{}我愿意腾出时间去和他们社交,但他们并不在乎我想要的是什么,而只是顾着去干他们自己想干的事情。我更像是一个配件,一个附属品,一个可有可无的、多余的孤儿。所以我之后也不再打算花费自己的时间去满足他们的欲求,他们的欲求与我无关。如果他们并不在乎我想要的是什么,那我为什么还要在乎他们想要的是什么。这似乎也体现在了我在酒店办理入住时我对那个前台人员的态度:如果他不在意我想要的是什么,那我为什么要在意他们的酒店规定。

\citebook{随笔 | 鬃积雨云2 ;自我探索 | 7}

\subsection*{想要抓住生活里转瞬即逝的在乎我的他人的渴望}

“在乎”对我而言真的是一个非常重要的事物,这也是为什么我在和那个同学一起做课程作业练习时感觉自己不再是一个人\pozhehao{}因为TA是在乎我的,而我也很愿意因此而去在乎TA。我对“在乎”的珍重同时也意味着我在成长的历程中几乎从来没有被“在乎”过。

在这之前,在遇到前任和鬃积雨云之前,其实我早已习惯了这样的生活\pozhehao{}没有人在乎我想要的是什么,我已经可以做到尽可能地安然无恙地承受着那份孤独感地生活下去。但他们的出现总是让我一次又一次地意识到(或无意识地感觉到),除了他们之外,我的生活里似乎就不会再有任何其他在乎我的人了。他们的出现扰动了我身边的这片自己一直以来不被“在乎”的平静的孤独感。这就像是自己早已习惯了黑夜,但在灯光(再次)出现的时候,我才意识到原来自己一直以来都身处于黑暗。而我想要紧握黑夜里的那唯一的转瞬即逝的灯光,似乎也就不足为奇了。因为在那灯光消失后,我周围的一切又会回归到黑暗当中,我又会再次孤独一人,不被看到的、不被珍重的、不被在乎。

\citebook{随笔 | 鬃积雨云2 ;自我探索 | 7}

\subsection*{试图用自己的生命力来冲破困境的垂死感}

由于那个“男孩”的感知力,你突然意识到自己一直以来都是支离破碎的,而你不知道应该如何将支离破碎的自己拼凑回去,不知道如何让自己变得完整。同时,也没有人跟那个“男孩”说:“一切都会好起来的”,更重要的是,也没有人会对你说:“一切都会好起来的”。你不知道这一切是否会变得更好,你不知道支离破碎的你在将来是否还能够变得更为完整。孤独一人的处境让你感觉到,在自己试图追求幸福和变得更好的路上,无论再多的努力都只不过是徒然。

……好像如果你不继续前行,你就会被困于原地。但望向前方,你又看不见希望,你看不见那无尽的悲伤背后是否还存在着你所期望的事物:是否还有幸福,是否还有他人会来告诉你:“一切都会好起来的”。那个“小男孩”已经处于支离破碎的极点,所以他只能推开周围所有可能亲近的人,而你只能牵着那个“小男孩”孤独一人地继续前行,既承受着那个“小男孩”内心的脆弱,又承担着只能继续往前走的责任。这一切都让你感到不断前进的徒然,你甚至已经开始逐渐丧失前进的动力,只能匍匐地前进着。

……你害怕当自己完全丧失了前进的动力后,自己只能选择死亡。所以你宁愿选择自我欺骗也不想去面对走投无路后的唯一“出路”。同时,你认为自己迟早都会选择这一条“出路”,因为你内心的“光明”终究会被耗尽,而你也终究会消亡,那时候的你只能被永远地困在原地。

……你开始意识到自己已经逐渐被困在原地了,被stuck在了一个只能等待自己的生命的自然结束的境地。但相比于自己的躯壳的消逝,你内在的那个最纯真的自我早已消亡。在匍匐前进的过程中,你感觉到自己的动力在不断流逝,你的自我在不断消亡。

\citebook{自我探索 | 8 ;回顾 | 随笔 | Crawling in Vain(徒然地匍匐前进)}

