\chapter{梦境,情感自我,理智自我}

\ardate{2022-04-30}{y29ga2eIeJXLcikL4vzVww}



\blockquote{
	半夜的我做了一个梦。梦里的我在自己房间里,我坐在桌子旁,桌子的对角位还坐着咨询师。我们在做心理咨询,而我在讲述着同样的议题\pozhehao{}被遗弃。说着说着,我发现自己的身体控制不住地想哭,也逐渐地哭了出来,但自己的言语上一直在说“没事”,而且在说:“很奇怪,我不知道为什么自己会想哭,这些明明都是之前已经谈过很多次的事情了,都是些日常的事情,我不知道不知道不知道不知道……”我卡壳了,卡在了无法用言语表达自己想要表达的内容和情感。咨询师问:“你会想起什么具体的事情吗?”我说我想不起来了。然后咨询师试图伸手去拿桌面上的笔,但够不着,我把桌上的笔递给了ta,然后ta开始记一些笔记,并表露出好奇的眼神和表情。
	然后我就醒了过来。

	\blockquotesource{白色灯塔先生}{梦,迷宫,指南针}{2022}
}

在写下这个梦境时,我并没有想去释梦,因为那时候的我觉得梦境的意思还蛮清晰的\pozhehao{}对自己的身体反应的不理解。但后来当有个朋友在评论里和我讨论时梦境里的内容时,我才开始尝试释梦。

梦里的咨询师代表的次人格可能是情感的自我,而我自己所代表的次人格则是理智的自我,笔代表的是情感得以理智化表达的工具。而且由于梦里的我是作为理智自我,而不是情感自我,所以在梦里的我认同着理智自我这一身份\pozhehao{}即我看待自己的模样就是理智自我。但通过(我)把笔递给情感自我,理智自我好像在作出让步,将话语权让给情感自我,因为我知道自己的身体反应以及这种反应背后的情感是(理智)自我无法理解的,甚至是让我感到奇怪的。

然后那个朋友回复说:“情感的自我还有点陌生,不太习惯”,我想到:Maybe.

昨天读书时读到了这段话:

\blockquote{
	然而,倾诉本身并不一定会带来内省。有些抑郁症患者会向别人倾诉,他们很容易以消极和强迫性的方式谈论自己,这很快就会把倾听者吓跑了。他们可能会透露太多关于自己的信息,以及太私人的细节,不适合社交语境或与倾听者的亲密程度。他们可能会向新认识的人倾诉,跟人家说自己的婚姻有多不幸或者自己的孩子有多不听话,而这是非常令人讨厌的。他们的倾诉中充满了沮丧,与他们交谈之后,倾听者会感到害怕。

	因此,他们很难交到新朋友,这只会让他们越发觉得自己愚蠢、丑陋或不会社交。他们的自我暴露(self-revelation)不是为了寻求真相,而是摆脱沮丧情绪。

	人格障碍者也常以弄巧成拙的方式倾诉。自恋者只喜欢谈论自己,仅把别人看作一群仰慕者,或一面可以打扮自己的镜子。在交流的大部分时间里,他们很少或根本不允许别人说出自己的观点。他们就像被卡住的双向无线电,只能发送,无法接收,所以不可能从自我关注的世界之外获得知识。

	\blockquotesource{Oliver James}{原生家庭生存指南}{2002}
}

\blockquote{
	读完后的我开始怀疑,自己通过写公众号所进行的自我表露是否只是为了摆脱情绪,而不是为了寻求真相,不是为了进行自我认知探索呢?或者说,我所写的东西里有多少是为了摆脱情绪,而在摆脱情绪后我又是否有进行自我认知探索?其实我自知这个问题的答案,我本应更加自信地回答道“不是这样的”才对。但我做不到这样回答,也更加没有这样回答的自信。

	和何同学一样,我并不喜欢自我表露,但最后还是找到了自己所擅长的事物来进行自我表露,来与外界产生联系。何同学一直做视频的目的或许是想要打破自身的孤独,但我只是因为自身的痛苦太难以承受,才想要把这种情绪写出来。我也无意想让读到这些文字的人感受到类似的悲伤或剧烈的反感。在写东西的大多数时候,我只是在对虚空说话。最终有多少人听,或者压根没有人在听,我没太在意。但无论如何我都忍受不了的是\pozhehao{}一个人也没有,而我只是在一间漆黑的房间里永远地自言自语,永远都逃不出去,就像是小时候总会梦到的一个漆黑的无尽楼梯间,无论怎么往上走或者往下走都走不到尽头,只是在那漆黑的虚无中一直环绕着、一直徘徊着。

	后来我带着这个问题(“自己通过写公众号所进行的自我表露是否只是为了摆脱情绪,而不是为了寻求真相\pozhehao{}不是为了进行自我认知探索”)去问了几个微信好友,收获到的其中一个回答是:“你写,别人看,每个人都有自己的需求。别人怎么看,重要吗?”对此我想说的是,这与别人无关,而与自己有关。如果我在误以为进行自我探索的过程中,什么都没有找到,而只是单纯地宣泄了自己的情绪,那对我自身而言,恐怕帮助不大\pozhehao{}除了让自己的心情好点之外,我还会在未来继续犯下同样的错误、开始同样的循环。但如果我真的在宣泄情绪的同时,找到了些自身的pattern,甚至突破了这一pattern,进入了下一个新的阶段,那么这样的历程才是值得去embark的吧。

	\blockquotesource{白色灯塔先生}{随笔 | “老师好我叫何同学”}{2021}
}

我会想起一年前的自己就是那个只会把笔仅仅地握在手上的情感自我,但后来发现自己无论写得再多也只是一直在宣泄情绪,而自己的情绪状态并没有多大改善,也走不出困境。所以当情感自我书写完内容后,理智自我就上线了,开始对写下来的内容进行加工\pozhehao{}对自己进行心智化。那时候是情感自我把“笔”递给理智自我,但现在好像是理智自我开始反之把“笔”递给情绪自我,因为理智自我有无法理解情绪的部分,而情感自我也有无法加工情感内容的部分。也许将来有哪一天这两个自我不需要刻意地递“笔”,而递“笔”的过程能够自然而然地发生、自然而然地流动。

我还会想起昨晚和朋友约完晚饭走在路上聊天时,他搞不明白他自己为什么既想要拒绝对方,但又做不到拒绝对方。我提议说:“可以试着用自己的内心去看事物,而不是用头脑去看”。他说他从来没有试过用这种方式去看。那时候的我想到:嗯,也许这并不是每个人都习惯的方式,并不是每个人都会把理智自我手中紧握着的“笔”递给情感自我\pozhehao{}特别是当自己身处于困扰和不理解的情感的当下,并不是每个人都能从自己的内心看事物。就像用头脑看事物一样,用内心去看事物也是需要刻意练习和培养的,就像是情感自我得以书写情感内容本身也是如此\pozhehao{}正如我花了好几年才一步步地更擅长通过写作识别自己的情感,并写下来。


