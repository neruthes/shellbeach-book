\chapter{焦虑、匮乏与无力,依赖与依靠}

\ardate{2022-05-21}{TfPDHupPA5mfD-2LxNKctA}



前天睡觉前,躺在床上的我在知乎刷到了一个关于某从业道路的经验分享视频,然后自己继续在这方面刷了更多的经验分享视频。越刷自己越焦虑,但又越想继续刷下去,我的大脑一直在思考自己应该怎么面对考研、备考班、读完研后的继续培训、入行,从业之路的发展。更重要的是,这每一步都需要用到钱,那钱怎么来?自己需要换工作才能赚到更多的钱,然后想到自己一直以来都很穷,家里人并没有太多资源支持自己在这条路走下去,然后想到自己真的想走这条路吗?自己并不是那么喜欢助人,至少在接电话热线方面发现自己并不那么热衷于助人,只是因为自己在这方面有擅长之处,而且自己并没有任何其他擅长之处了。所以我真的需要往这条路走吗?如果不走的话,我就没有其他选择、没有其他擅长之处了,而在发展我擅长能力的这条路上,又会遇到重重困难,需要面对很强烈的不确定性和对金钱的匮乏感。好像身边的人面对工作、生活和未来的不确定性时都处理得很好,比如说前任、很久没见最近才见上面的男生看似都处理得很好。好像身边的任何人都处理得很好,为什么我就处理不好,为什么我总是会那么焦虑?只是一点点的改变都会焦虑。而且为什么大家都不会把这部分摊出来说,好像这部分、这样的焦虑就不存在似的。

后来,我意识到自己正在处于焦虑当中,而且大脑正处于行动模式,但无论怎么刷视频、怎么思考都解决不了问题,然后开始感到很累、很头痛、很焦虑,但就是睡不着,开始失眠。然后我开始通过冥想来进入存在模式。在冥想的过程中,我能感觉到自己的焦虑情绪程度逐渐降低,但降低到一定程度就不再下降了。那种焦虑感依然还在,但焦虑的程度还能允许我入睡。

睡醒后,第二天的我一整天都处于焦虑当中,上午坐公交时焦虑、等看病时焦虑、看完病坐公交回去时焦虑,然后午饭没吃睡了一个下午地试图逃避现实世界、躲在被窝里、躲在梦境里。睡醒后,头疼、头晕、想吐,然后开始后悔自己不应该不吃午饭的,因为不吃午饭睡醒后的状态更加焦虑了。我会想起认知行为理论的ABC模型\pozhehao{}A是现实事件,B是我对现实事件的解释,C是解释所产生的情绪。我可以通过冥想让自己和情绪保持一定的距离,也可以通过改变B(对现实事件的解释)来改变情绪。但当下最重要的是保持规律的饮食和作息,让自己的精神和身体状态足以应对焦虑,而不是一旦不开心就跳过一餐不吃,从而让自己的精神和身体状态更加糟糕。

吃完晚饭后,我跟家里人提起了这件事,知道了他们对此是表示支持的,但也只是尽他们所能地支持,毕竟他们手上的资源也并不多。

我想起最近我接的一个热线里,来电者问我说,我会觉得焦虑症真的能通过药物或心理咨询来根治吗?我说在我的理解里,药物只是降低焦虑情绪的程度,足以让人有一定的空间去应对焦虑情绪,而心理咨询可以提供一个工具箱,每当焦虑情绪出现的时候,就可以用工具箱里的工具来应对。

我想到,好像自己也不知不觉在利用自己的工具箱来应对、来涵容之前自己所无法涵容的情绪。对金钱的匮乏感,对身边没有一个能依靠的人的无力感,对未来、对身边事物的变动的不确定性的焦虑,当这些强烈的情感交杂、融合在一起的时候,就足以将自己淹没了。

这些感觉让我想到小学时期的自己,那时候的自己手上没有零花钱,身边没有任何一个能依靠人,自己也无法控制周围发生的任何事情,更无法控制未来。每天的生活就是,起床,上学,上课,偶尔被霸凌,放学回家,写作业,总是被骂、被打,睡觉,然后又轮回一次,不知道这样的日子什么时候才能到头,无力改变现状、无力保护自己、无力改变未来,想买自己想要的东西也总是没钱,没有朋友,一直都是一个人在重复了无数次的日子里仅仅存在着,已经不期望未来、不期望现在、不期望他人、不期望身边的事物、不期望手上能有钱、能有资源,放弃自己所渴望的一切,放弃自己想买的东西、放弃和他人的连接和依靠、放弃对未来的期望、放弃对改变现状的努力,最终,放弃自己。

在今天吃晚饭的时候,我跟之前喜欢的那个有夫之夫的男生表露了自己的对身边没有一个能依靠的人的无力感以及对未来、对身边事物的变动的不确定性的焦虑。这一表露对我而言蛮有意义的,因为在之前和他的互动里,我好像一直都在扮演着一个知道得更多、更有能力、力量和动力的人,但事实上我感觉之前的我只是在他面前扮演着某个全能角色,因为我不敢在自己喜欢的人面前表露出自己脆弱的一面、无力的一面、焦虑的一面。我会害怕对方会因此而回避、鄙视、拒绝我,就像前任曾经拒绝过我的脆弱的一面、悲伤的一面、无力的一面、焦虑的一面那样。

那时候的我刚大学毕业,正陷于找工作的焦虑、选择自己未来的方向的焦虑,但前任只是跟我说:我可以怎么做、怎么做,但那时候的我根本无力去解决问题,而他也没有从我的情感下手,而只是聚焦于如何解决问题。当他发现我没有动力去应对生活的变迁时,他就撒手不理了,去忙他自己的事业了,还说“(在事业上)不可能等你一辈子”。那时候的我甚至不知道那种情绪是焦虑,而只是感受到一系列的躯体症状\pozhehao{}手脚无力、胸闷、过度换气、疲惫、心脏剧烈疼痛、肚子绞痛,只有晕过去在被窝里睡一觉才能缓解过来。

在和那个男生表露完我脆弱的一面后,我意识到自己并不是需要一个能够依赖的人,而是一个能够依靠的人\pozhehao{}能够涵容我自己难以涵容的情绪、能够让我感到自己没那么孤独的人。而之前的我一直以来都没有一个这样的人,无论是父母也好、前任也好、之前见的那个很久没见的男生也好。好像只有今晚吃饭的那个有夫之夫的男生做到了这一点。

我并不像前任所说的那样“你只是想找个能依赖的人”。不!是依靠,不是依赖。这两者是有很大区别的。依靠不代表对方就有解决办法和答案,但那个人能给予自己一个休息的地方、一份理解、陪伴和安慰。依赖则是一个人依附到另一个人身上,放弃自己地和对方融合在一起,将对方的生活、能力、特质等任何事物都融合成为自己的一部分,以此来回避焦虑。在三年前和前任的相处里,我确实是通过依赖着他来防御对生活变迁、对未来的焦虑,但现在我发现自己能够、想要的是一个我能够依靠的人\pozhehao{}对方并不需要拥有解决办法和答案,而我也不需要放弃自我地和对方融合在一起。


