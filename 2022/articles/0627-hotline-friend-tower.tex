\chapter{热线,朋友,塔底}

\ardate{2022-06-27}{nXU-mrZXq\_6rEc3nOpGiyA}

上周周末,我去了隔壁城市和一个很久没见的朋友见了一面,一起去了之前没去过的公园,看了看风景,还一起约了披萨和咖啡店。

晚上是最后一次的热线团体督导,督导师在最后留出了一部分时间去处理我们这个持续了几个月的团体督导小组的分离。当每个同学开始分享各自在接热线过程的自身经历时,我分享说:XXX(其中一个同学)(的分享)让我想到我也有这个过程。一开始的我很想要去帮助来电者,但我越来越发现这免除了他们需要去承担的那份责任,而自己也越来越耗竭。我开始变得很漠视人性,变得不在乎任何事物和人,而这个部分我现在也还在处理。后来在明确了热线服务范围和功能之后,我就变得越来越无为了,而自己的情感也没有那么的耗竭。这种无为的态度也延伸到了其他地方,比如说之前我很想要成为一个心理咨询师,但现在我会更加无为,不会在理智上跟自己说一定要达成某个目标,而是自己每踏出的一步都会问自己:这是我想要的吗?我此时内心的感受是怎样的?

晚上睡着后,我梦到了自己和许多同学在大操场上,我去黑板上看班级分配,看见我并没有呆在原来的班级里,而是被分到了一个只有八个人的小班里。其他同学也是从大班里被分出来的,所以当我和其他同学聚在一起排队时,我们都觉得这样的安排很奇怪,而且我们彼此之间并不认识。我们的新老师是那个热线的督导师(在梦里的我无法回想起清醒时的记忆,所以也认不出他来),他把我们带到了一个新教室,让我们填新的测试题。我发现题目都是一些开放性的问答题,是关于人生道路选择的,比如说自己擅长些什么、对这个擅长的东西有怎样的感觉。我写了不少内容,写的时候老师会巡我们的桌子,看我们写了些什么,但不会说任何话。我把题目写完了,准备交的时候发现我想补充一些内容,然后示意老师我要再多一点时间,他点了点头。我便找我的一个女同学(她也是热线同学之一)要题目的原PDF文档。这时候我就醒了。

醒来后,想到自己要去上班就感到很难受。躺在床上,脑海里浮现了那天中午和那个朋友吃披萨时的画面:我们在一家暗红色装潢的披萨店,我面对着窗外炎热的夏天\pozhehao{}明亮的午后阳光和树荫,面前是那个朋友。不知道为什么,我会感觉那一刻很温馨,想到如果能有机会和他谈恋爱的话估计会感觉很舒服,但又想到即使是作为朋友关系偶尔聚一次也同样能让自己感受到这种舒服的感觉。

然后我想到我们一起在咖啡店里喝咖啡的画面:我们靠在窗边座位上,后背斜靠着窗户,窗户外同样是明亮的午后阳光和树荫。当时之所以进了这家咖啡店是因为他上次记得这里有家店,同时也因为突然下起了太阳雨而急着要找个地方躲雨。后来外面的雨停了,天放晴了,我们继续靠在玻璃上聊天。他聊到他很容易忘记过去发生的事情,包括那些事情所带来的情感。然后提到了这可能和他小时候的经历有关,忘记情感是他处理负面经历的一种方式。我跟他说:如果是这样的话,那这一生会显得很短暂耶,因为如果没有什么特别的情绪能留下来的话,那每天都会过得很快吧。他说是的,就是觉得每天都差不多,过得很快。聊到后面,我分享了在一周前我开始用的一个记录每天心情的app「MOODA」,我说这可以让我记下过去的一些特别的情感。后来他也想去下载试试。

他在备考考研,然后说我也可以去试试看,我说我还在考虑当中。他说我之前不是对心理咨询蛮有热情的吗,我说那时候的我更多是因为突然找到了自己所擅长的东西而感到兴奋,同时也是为了摆脱之前的毫无意义、虚无的生活。但后来我发现,擅长一件事情并不意味着做这件事情就会开心、就有意义感。我发现我做这样的事情,比如说接热线,我确实擅长捕捉对方的情感,也越来越擅长给解释,这些解释在对方看来也越来越有说服力、越来越适用。但我在挂完电话之后,我并没有感到有太多的开心,也没有多少意义感。如果做一件事情仅仅是因为我擅长而不是因为我想这么做、我是开心的、我是能感受到意义感的话,那我会感觉自己就像是个工具人。然后他整个上半身往斜倾地用力笑了笑。

然后他问我为什么我会擅长但又不喜欢,不是因为喜欢才变得擅长的吗?我说不是,我之所以擅长可能是因为小时候经常被我妈打,所以那时候我需要逼自己锻炼地能够时刻监控她的情绪变化,特别是当她从一个正常的状态神经质般地变成了一个随时会骂人打人的状态时,我要做到及时躲开、逃跑。

然后我想到那天下午和他去了一个之前没去过的公园,那个公园里有一个电视塔。走到电视塔的塔底的时候,我会联想起在读大学时玩的一个叫《破坏者》的单机游戏。游戏里的场景是二战的巴黎,而最后一幕的剧情就是要上埃菲尔铁塔。那时候我知道剧情快要结束了,因为好像编剧一直在将这个标志性建筑留到最后才让玩家进去,而之前的我有试过想在剧情初期就上塔看看,但发现只能爬上去……后来,在构建脑海里的世界时,我也将这个部分加了进去\pozhehao{}想象埃菲尔铁塔前有一片公园,塔底的壮阔,走去电梯上塔的整个过程,塔顶的整层公寓。

下午在那个公园的塔底那,当我想起游戏《破坏者》的画面以及以前在在脑海里构建的世界的画面时,那时候的我感觉到了一种转变的感觉,好像我突然回想起了一些过去早就忘记的事物,回想起了曾经的那个读大学时的我的其中一部分是怎样的。同时也会有一种崭新的感觉,好像一切从那一刻开始都能变得截然不同。

在通勤的路上,我想到我确实可以和那个朋友一样去备考考研,但我内心感受了下,这种想法似乎只是为了摆脱现在日复一日的重复生活。这真的是我想要的吗?我会对此感到开心吗?这会带来意义感吗?好像很多提问都无法得到一个确切的答案。但我能确信的是,我喜欢和那个朋友在披萨店面对面的感觉,我也喜欢和他一起走到塔底时我所感受到的崭新感。

\useimg{aimg/2022-0627-1.jpg}

\useimg{aimg/2022-0627-2.jpg}

\useimg{aimg/2022-0627-3.jpg}
