\chapter{自我探索 | 16}

\ardate{2022-01-16}{-ip2xxVGZ6\_XNu\_wFuM8aQ}


进入咨询室的我感受了一下自己在当下最想要谈的内容,然后说起了这周的热线模拟考核:“第一个个案模拟是一个重新踏入职场的女性。她之前在家待业了两年,现在在准备各种面试和相关的PPT。她最近觉得自己很累。在倾听的过程中,她陆续表达了一些符合抑郁症状的标准,所以我就直接问她是否有自杀意图。她说她自从初中就有这样的想法了,但没有这样的计划。当模拟结束后,(扮演个案的)考官还安慰了我一句:‘这个个案确实蛮困难的。’当时我在想的是:并没有那么困难,for I have seen worse and I have been worse。我从小学就开始有自杀意图,而且之前还差点实施了自杀计划。

我有进了一个热线模拟的备考群,群里有的同学在模拟热线后会感到很焦虑和不安,会有很多需要去处理的情感,需要找其他同学去倾诉。但我好像并不需要倾诉,因为我的个人经历似乎让我更觉得这样的模拟只是一件小事。(当然我没有在热线模拟里向对方表达这是一件小事,因为这太不尊重人了。)

……以前我还会想到,如果我没有过去的那些经历的话,如果我能像身边的正常人一样,那该有多好。但我也不会想放弃那些与他人差异很大的经历,因为那是属于我自己的部分。

我会想到,这周的冥想体验里是以慈悲心为主题,冥想的引导语里有:‘原谅他人,原谅自己’。当我听到这样的引导语时我会很生气、很愤怒,因为那种原谅更像是在抹去自己的痛苦和悲伤,抹去属于自己的部分。如果我真的抹去了我自己的那份悲痛,then who am I?AND WHAT AM I? 我并不想抹去自己的那些充满悲痛的部分,因为那些都是我的一部分。

我也会想到,现在这些悲痛好像转化为了另一些东西,但我还说不出来那是怎样的东西。但它们不是悲痛了,而是一些属于我自己的东西。”

咨询师说TA的脑海里出现了一副画面:好像在伤痕上长出了一朵花。听到这里的我真诚地微笑了笑。

\tristarsepline

后来我继续说到:“其实我一直感觉自己是一个独特的异类,一直有一种自己不属于任何地方,也和身边的人格格不入的感觉。无论我身处于何处,我都和身边的人差异很大。但就像刚刚说的,我并不想去抹去那些差异的部分、那些属于我自己的部分,因为我自己能看见他人所难以看见的事物。”

咨询师问:“你能想到一些具体的例子吗?”我回忆了一阵子,并说:“很多记忆都已经很模糊了,但我记得有一次我还在读大学时,我将自己的一些内容写在推文里然后发到朋友圈,但评论里有几个同学(那些同学还是住在我宿舍附近)评论说:‘不要想那么多’、‘开心一点’之类的话。这让我感到自己更加不被理解了。所以在那之后我就将自己的文字藏得更深,或者说不会主动将这些更深处的文字转到朋友圈,因为不想再叠加那种不被理解的感觉。”

咨询师问:“你还记得当时你在推文里写了些怎样的内容吗?”我回忆了一下,说:“大概是一些关于自己不合群、与身边的人格格不入的事情。读大学前我会幻想自己一旦离开了原来的城市,去了另一个地方,我可以重新开始。我可以变得更合群、更懂得社交、不会像之前那么的孤独。但上了大学后,我发现事情并不会奇迹般地变好,我依然和以前一样地孤独一人,甚至比之前还要更加糟糕。但当这些感受没有被他人看见时,我会感到很失落、很伤心,但后来我也开始慢慢习惯这个部分了,人与人之间充满差异的部分。

我会想到之前的一次咨询里,我向你发出把我的公众号内容分享给你的邀请。那时候我是想让你看见我在文字里的自我,但那时候你很谨慎,并说接下来可能需要好几次的咨询来好好讨论后,才能决定是否应该将我的公众号内容分享给你。所以即使是同一个事物\pozhehao{}分享公众号内容的邀请\pozhehao{}在彼此的眼中也看到了不同的东西:我看到了我在文字里的自我,而你看到的则是能对此进行工作的素材。不过,每个人都有自己所看重的事物,我和你也有彼此所看重的不同的事物。这并没有谁对谁错,而我也愿意接受彼此的差异。

我会回想起,在咨询的一开始,你有在好几次咨询里问我,你没有办法很大程度地理解我这一点会让我感受到怎样的感觉。我会在想那时候你是否是出于担心或是焦虑?我想那更多是属于你的部分,而不是我的部分,因为我这一生都身处于不被理解的环境里。”

\tristarsepline

当我将打算继续讲下去时,咨询师打断了我,并开始表达:“我在咨询的一开始确实感觉很难跟上你,因为你带入咨询的很多内容好像都是经过你的写作的内容,所以会很高度抽象化。但现在你所表达的内容就很容易让我跟上。”

我回忆了一下,回应道:“嗯。我也会在想,为什么我写作的内容会那么高度逻辑化和理论化。我身边有的人也会在语言和文字的沟通上用一些高度逻辑化和理论化的术语、一些属于他自己的事物、他人很难懂的理论体系来表达一些浅显的内容。有一次我跟他聊天时,我就跟他说,他好像是在用这种给人际关系“建模”的方式来回避与人相处的未知性,好像只要呆在自己的理论里,就能保留着属于自己的那份安全感,而不敢去面对属于他人的未知。”我思考了一下,继续说:“我会想到,我的写作内容之所以会那么高度逻辑化和理论化,可能是因为我需要去面对那些我周围环境里的未知,比如说那些虚无和无意义的部分。一旦能够命名那些未知的部分,那些未知好像就变成了已知,我好像也就能从这些逻辑和理论里找到一份属于自己的安全感。但当然,并不是每个人都会拥有和我相似的经历,所以每个人所构建的对于如何理解他们各自的周遭世界的理论和框架都很不一样。

不过不是每个人都会将理论和框架搭建得那么高(无论是通过写作还是其他创造性方式),或者说不会总是站在最高的那一层去和其他人进行沟通。对于我来说,自从开始学咨询后,我找到了属于我自己的桥梁,能够连接到他人所构建的‘建筑物’的桥梁,那就是情感。即使我们(我和咨询师)之间有很大的差异,但我们的失落感、悲伤等情感都是十分相似的。当然我也知道我无法触碰到有的人的情感,无法和有的人建立起桥梁。我肯定也不可能和所有人都能够建立起联系。有的人甚至连TA自己的情感都不清楚。”

\tristarsepline

离开咨询室后,我感觉自己很轻松。在表达了自己不被理解的情感和想法的这些部分后,我感觉自己更加轻盈了,那种不被理解的感觉没有像以前那么的沉重。我想到,那种不被理解的感觉所真正渴望的并不是真正地、完全地被理解,而是渴望向他人去表达自己不被理解这一部分。仅仅是表达本身就已经足够了。这也像是痛苦:我并不是要去消解痛苦或原谅他人、原谅自己,而只是需要将这份痛苦表达出来。这就足够了。

