\chapter{这就足够了吗?}

\ardate{2022-02-04}{TagtHJ1O-usBAZiIgeEqkw}




在过年这几天,我和前任见上了一面,上一次约见面聊天已经是去年一月的事情了。

他选择了个面对面的聊天方式,泡了红茶,并问我最近怎么样了。我说是指哪方面怎么样,他说:“各方面。谁知道你最近怎么样,毕竟我们有……三年没有联系了吧。”在他思考时间的时候,我在等待着他的回答,想知道他是否真的记得距离我们分手到现在过去了多久,是否真的还在意那段时间。我回答道:“两年半。”

我说我还是干着以前的工作,还是和家里人住,不过大半年前开始学习新的东西。他说:“那蛮好的,起码你没有对现在的生活有所抱怨。”我问:“我之前会在抱怨些什么吗?”他说:“不是,是今年的大环境,大家坐下来后都会开始抱怨。”我回应说:“Em……我也会有想抱怨的事情,不过我想那都是旅程的一部分。”

\tristarsepline

聊完日常后,我有点介意是否要直入话题,并问道:“我会想问上次跟你说的那件事,那件对曾经的亲密关系造成破坏的事情。我会想知道为什么那时候还身处于彼此的亲密关系时,你会作出那样的事?”他说他一直知道他在那方面的事情,他知道自己的想法、自己的感受以及这个过程本身。我问他:“你不会想在我们确定亲密关系时就提前告诉我吗?”他说不会,以及很多事情都不会提前主动地说。我问:“那你不担心你所在乎的人会因此而受到伤害,并且离开你吗?”他说他不担心,有的人会看见他的那个方面,然后选择走近或远离,最后大家都会停在各自觉得舒适的位置,况且,(他)最多就换一个人。听到他的回答时,我的牙齿在敲响、我的下巴在颤抖,我从中意识到我在有意识地压抑着自己所感受到的强烈的愤怒、悲伤和被遗弃感,但我依然把这些情绪暂时悬置到一边,因为它们会妨碍我更好地理解他。

他说不是每个人都会将这些的事情摊出来说,如果拿到私人领域去说的话,他会愿意去回应,但如果拿到公共领域的话,那就是另一回事了。我回答说:“当然不是每个人都会摊出来说,毕竟说出来需要一定的勇气去面对曾经发生过的事情。”

\tristarsepline

我继续问他:“那你对我们这次聊天有什么期望吗?”他说他什么期望也没有。我说他选择了和我见面,这是经过他的选择的。他问我我是想问关于主动性方面吗?我说也可以这么说。他说他区分主动性和非主动性之间的区别是他是否能意识到这个过程的发生。有的人会无意识地做些事情,但有的人能意识到自己在其中参与的过程。我说:“所以你能意识到在这次见面里你主动参与的部分?”他说:“当然了。”

\tristarsepline

随着聊天的深入,我开始了解到他对待他人和事物的态度都是:在事情发生后再回顾自己身体的想法和感受,只需要看见这些想法和感受就足够了。这让我想到了冥想,以漠然、超然的态度看待自己的想法和情感。我问他:“那你不会觉得孤独吗?”他说其他人可能会觉得他孤独,但他自己认为这是他的本性的一部分。我开始意识到这像是他的一种自我防御、自我保护的方式,而这一方式对他而言很可能非常适用,然后说:“我想这种方式能给你带来很大的平静感吧。”他点了点头,说:“也许吧。”我继续说:“那么,事情好像都只是发生在你身上,而不是你选择让这些事情发生。”他回答道:“是的。”

我问他:“为什么你没有在我们还处于亲密关系时就把这一面表露出来?”他说我那时候还没有能力看见他的这一面,而只是看见我想要看见的那一面。我回答说:“这似乎就像是之前所说的,你不会主动将这些事物呈现出来,而是等他人看到这一面后,再漠然、超然地去看事物的发展,看对方是选择远离还是靠近,亦或是离开。”他的回应是:“嗯”。

自从坐下来聊天,他的手机便一直播着古典纯音乐歌单,我说能把音乐关了吗,他说是因为音乐的节奏和话题很不搭吗?我说何止是不搭,话题越来越沉重,但音乐却越来越欢愉。他说等等这首歌就过去了,或者你也可以离开这里。我望向窗外远处的漆黑街道,说:“最终一切都会过去呢。”

\tristarsepline

后来,随着我看见他的部分越来越多,我发现自己非常像他,并说:“我发现我越来越像你。但其实我还蛮不想成为像你一样的人的。”他说他在(三年前)遇到我的时候就看见了这一点。我问那一点是什么?他问我我觉得那时候的我处于怎样的状态?我说那时候的我害怕面对未来、想要找一个人来保护自己,来暂时不用面对那些需要面对的事物。他回答说:“那么那些东西的共性是什么?”我猜了几个回答,但没有猜中他想表达的。他说(那)是一个人想要做的是什么、人生的意义的那些问题。我意识到他在讲关于存在主义方面的议题。我回答说:“如果放在那个框架下的话,嗯,我在那时候确实看见了一点点苗头。”他回答道:“就是咯。”他说如果一个人在最终会变成一个完整的人的话,那些问题就是和那个完整的人有关的。我问:“你是想说,在我们那时候刚开始亲密关系时,你就看见了我将成为的那个完整的人的其中一部分吗?”他说:“当然了。”我继续问:“所以那时候你就能看见现在的我会变成这个样子吗?”他说:“嗯。我之前说你需要依赖他人,是因为你在其他人身上会拿走一些东西。”我意识到他想说的可能是我所内化的重要他人客体,然后我问他:“那些东西是什么?”他说,就是那些你像是我的东西。我笑了笑地点了头,然后问:“那可能更像是复制,因为你并没有因此而失去什么。”他说,不像是复制,而像是我拿走了一些属于我自己的东西。我想到,我确实会在每段亲密关系里内化不同的重要他人客体,将那些部分经过处理地成为自己的一部分。

我也想到,前任的这一“角色”明明对曾经的我造成了那么大的伤害,但我却越来越像这样的他,甚至会对他感到更有吸引力,因为难得“遇到”一个和自己那么相似的人。无论是越来越像他,还是对他感到更有吸引力,我内心都既喜欢又排斥这两个部分。

但我依然想和他有更深的关系,所以我说我在和他分手后的这两年半以来见过很多人,但都觉得(和他们的相处)蛮无聊,不过和他聊天不无聊。他说,我感到无聊是因为我和他人的互动方式以及我自己都没有怎么变化吧?我回答说:“不是的。我自己一直在变,和他人的互动方式也在变。我觉得和你聊天不无聊是因为我‘看见’了你背后的那个人,那个带着漠然、超然的态度对待身边的事物和人甚至是用这一态度对待你自己的那个你,也看到这一态度在你生活中的各方各面的延伸。”他有点不相信地说,难道不无聊不是因为我没有预料到这次的聊天(的内容和进展)吗?我说:“不是的,我在来之前就设想到现在这一步。我之前的那些隐隐约约的想法和感觉都在这次的聊天里慢慢地展开了。但我没有设想到的是你背后的那个带有着漠然、超然的态度的那个人。看见了这个人让我感觉我们之间的距离拉近了。”

他问我:“这就足够了吗?”我说:“我会想知道我们之后的关系可以变成怎样?我会对未知充满好奇。”他说我们之间的关系已经没有其他可能性了。两人关系的顶峰就是共同生活,之后就会走向分离和结束,所以还不如在“无聊”之前就停在那里。我说:“如果你已经知道了两人的关系会走向的阶段,那不会让自己难以呆在当下吗,既然你都知道事情会怎么变化?”他说他能呆在当下的状态。

\tristarsepline

之后我试过几次试图突破他的自我保护,想要问出他会想要些什么,但每次都问不出来。我感觉他把自己有所欲求的部分隐藏地很深,而且他还不断问我:“这就足够了吗?”同时,我也感觉他在用这个提问促使我走得更深,或结束这次的聊天。我说我脑海里有一幅画面:一个观察者的画面。你在这副身体的背后一直观察着事情的走势。他说我如果这么觉得就这么觉得。

一方面,我想找到那个他还渴望着我甚至只是对事物和他人有所欲求的那部分自我,但另一方面,我并不想破坏他的自我保护所可能营造出来的平静感。如果他以这种方式获得了平静,为什么我还要为了自己的私欲而破坏那份平静感?

\tristarsepline

不过我逐渐意识到他的很多话语都是经过处理和包装的。我说:“我发现你在表达你自己的内容的时候都没有即兴发挥或及时化的东西,好像这些话早就对自己或对他人说过很多遍的。”他说每个人都有准备的痕迹,有的人多,有的人少。我说:“那你不会觉得这不真实吗?”他说:“每个人本来就在生活里扮演着许多不同的角色,最终这些角色加起来的就是那个人。”我说:“好像你没有办法分清真实和虚假?”他说(这)本来就没有真实和虚假之分。

我在想,我不知道他在我面前展现的这个“角色”到底会在多少人面前展现出来,以及除了这个“角色”之外,他的其他的角色又会是怎样的。

\tristarsepline

最后离开前,我和他拥抱了,并跟他说:“好像在身体触碰时,我感觉你也只是在观察着这个过程的发生,没有真正参与进来。”我问他对此有什么感觉?他说身体接触对他来说“也就这样”,真正的接触仅仅是通过聊天就足够了。

\tristarsepline

离开后,我想到了欧文·亚隆所著的《存在主义心理治疗》里的那个“上帝视角”。我会认为他有很强的心智化能力,并利用那个“上帝视角”将自己保护得很好。但我并不相信他有所欲求的那部分自我就消失了,因为在后来我发现他眼角有泪水时,他说他只是困了,但我想他会不会是因为疲倦而放低了防御。不过我感到很强烈的一点是,我并不想成为像他一样的人\pozhehao{}我不想只是看着事物和他人仅仅在自己的身边发展、流过,我想主动地参与到周围的事物和他人当中,我想更丰富地活着。

我回想起一开始和他见面的初衷是为了拉近和他的距离,而在这次的见面里也确实做到了这一点,但我依然有一种并不足够的感觉,距离还不够近。
