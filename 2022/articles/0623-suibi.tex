\chapter{随笔 |}

\ardate{2022-06-23}{zsw-djia-tVh1Vdow59rXQ}

% \dialoguelist{咨询师}{
% 	\dialogue{咨询师}{是啊。}
% 	\dialoguesepline{咨询师}{(沉默)}
% }



最近一个朋友分手了,然后在读者群里时不时会发一张带有悲伤气息的截图。那些截图是来源于一个B站的up主,而那个up主近期的投稿视频里大多是全黑的背景,中间是一个彩虹的图案,然后在彩虹上是一些令人伤感的文字,搭配上音乐。

这让我想起当我和初恋分手时,那时候的我疯狂地在社交软件上写了几十条关于两人两处的经历的动态,每条动态大概写了一两百字。那时候的我不知道自己是在借此宣泄情绪,只是在不断地写、不断地写,甚至像是在无意识地不断写。后来这种习惯也延续到了写公众号上。

我会想到,当面临挫折或困境时,或者更准确而言是一些让个体产生压力、感到悲痛、丧失等难以承受但还不至于击垮ta的情绪经历,这样的情绪体验能够激发个体运用ta的潜能去处理内在的情绪和外在的现实。

不仅如此,当个体能够发掘并发展自己的潜能去应对周遭环境和内心世界时,ta也就能与内心更为悲伤、更加充满恐惧、更无助的部分相连接,而不至于分裂、压抑掉这些部分。正因为个体能和这些更为内在、更为脆弱的部分连接在一切,ta也因此能和他人的这些更为内在、更为脆弱的部分连接在一起,能够和另一个个体分担对方的悲伤和脆弱而不至于被对方的情绪所击垮。而这种人与人之间的更为内在的连接便能够缓解作为一个个体所无法避免会感受到的孤独感。

我在接热线的过程中遇到不少人说他们无法向身边的人表达他们的情绪,特别是那些难过的情绪。他们身边要么没有一个能倾听他们的,要么身边的人会用各种方式防御掉这些的情绪,所以他们才会选择打热线来和一个陌生人聊一些ta自认为不敢/不能向ta所处的环境里表达的情绪。这时候的我总会感到一种孤独感,就好像我们生活在一个人与人之间充满连接和互动的社会,但许多人之间并没有真正地连接和互动,甚至是每天都会见上一面的人更是如此\pozhehao{}家人、亲戚、邻居、同事、朋友。没有深入的交流,没有情感的共鸣。我们看似很快乐、很满足、充满陪伴,但事实上,deep down,许多人都有一种难以言说的孤独感,因为自己的存在无法完全展示在他人面前。我们需要去隐藏、需要去伪装、需要去进入佯装模式、需要去扮演特定的角色,需要放弃一部分自我地去让自己得以嵌入与他人的那早已固化的互动模式里。将充满和谐、开心、表象的东西留给别人,将难受、痛苦、悲痛、绝望、更深层的东西留给自己。

因此,有不少人觉得自己的情绪和经历是很独一无二的,认为任何人都无法理解自己那独一无二的情绪和经历,无法理解自己的难过、悲痛、丧失感、幻灭感、绝望,甚至是自杀意图。对于这些人而言,也许他们的世界里从来都没有过人与人之间的情感连接,从来都没有过充满真实的连接感的经历和体验。

我会想到一个词:structural weakness。当一个不重视情感、不允许情感的存在和表露的社会,一个已经高度固化的structure,在这个structure里必定存在着薄弱之处\pozhehao{}那些看似普普通通却出现了重大问题的个体。在每一个structural weakness里,都能看见那个个体所身处的环境的许多共同之处。正如突然变得高度僵化的魔都马上就出现了许多structural weakness.有的structural weakness甚至是直接崩塌掉了,连补救的可能性都不存在。

所以一个structure并不是要有多完美,而是要有足够的灵活性。它能够去忍受、去直面、去涵容(contain)那些structural weakness,正如同个体能够去忍受、去直面、去涵容那些让ta产生压力、感到悲痛、丧失等难以承受但还不至于击垮ta的情绪经历。在个体与structure的互动里,任何一方的崩塌都会导致另一方的崩塌。个体的崩塌导致结构薄弱之处的崩塌,结构薄弱之处的崩塌进一步引发更多个体的崩塌。

不过,这样的互动同样也不应该是卷入和共生\pozhehao{}因为这两者都意味着个体被迫或选择放弃一部分的自我才能够融入/嵌入这样的环境。一个个体不应该试图为另一个个体承担一些对方本应承担的责任,也不应该让自己卷入一些丧失自我的互动当中。否则,这种丧失自我感会逐渐让个体的自我与现实世界脱节,使人难以/无法触碰到自己内心的感受,因为内心的感受早已变得过于痛苦与绝望,以至于难以承受和直面。

有足够的灵活性去涵容(contain)自己、涵容他人,尤其是那些令人难受、痛苦的部分,而不至于在涵容的过程中丧失任何的自我。当一个人试图去涵容自己之前所无法涵容的情感和他人时,ta能感觉到自己的内心更为丰富、内心世界更为宽广,同时也能看见内在的自我变得更加多面,而不是丧失掉任何一面的自我存在,那些个体经过对外界客体(他人)的投射和内摄而内化并保留下来的自我。

我会想起心理咨询里有一个词:corrective emotional experience(矫正性/修正性情绪经历). APA(美国心理学会)对此的其中一个定义是:an experience through which one comes to understand an event or relationship in a different or unexpected way that results in an emotional coming to terms with it. (这是一种经历,通过这种经历,个体以一种不同的或出乎意料的方式来理解一个事件或关系,从而在情感层面上接受它。)这个定义强调的是个体能够理解(understand)并且能够接纳和处理(coming to terms)某个事件或人际关系。但我会认为,要让个体真正地达到理解、接纳和处理ta之前所无法应对的事件或关系,依靠的并不是矫正、修正,并不是一些来源于外在并试图让个体内化的改变,并不是内化后的矫正、修正和改变,而是内化后的叠加\pozhehao{}过去的那些令人感到痛苦的、骇人的事情和人已经发生了,已经在个体的内心内化为一个客体恒存了下来,但新加进去的东西(客体)足以平衡、足以冲淡这些令人感到痛苦的、骇人的客体。

正如有自杀意图、计划甚至是尝试的人很少会在余生完全放下自杀的念头,生活之所以能够继续下去,或者更应该说他们之所以选择让生活继续下去,更可能是因为那些令ta难以承受、难以直面的痛苦和绝望并不再是生活的全部。

过去的那些东西还在,但那并不是全部。正如内心的情绪和外界的苦难,they can't be gotten rid of, they are just a bit more bearable.
