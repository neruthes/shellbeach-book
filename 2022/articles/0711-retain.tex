\chapter{“害怕自己的无力留不住我想要留住的人”}

\ardate{2022-07-11}{gm7Aq3F\_mNVmuxB4jSlRYA}




周末我见了一个有好感的男生。

在我的眼中,他算不上很有趣,但他会有趣的那一面,比如说他会去分析、去总结事物背后的规律,有自己的一套看待事物和周围环境的规律和逻辑。不过他所看到的世界就像是一个很大的困境,而他似乎也难以摆脱他所描述的这个主观世界。我有跟他提到这一点,然后他说他那只是片面的观点,而他能吸收一些新的观点。

所以在听完他是如何看待他眼中的这个周围环境后,我提到我了解到的人咨询流派里有一些理论,那些理论似乎和他所提及的周围环境背后的规律和逻辑有关。说完之后,他也能听得进去,会认为这和他所学到的知识、理论有相似之处。

在和他见完面后,我会感到有点孤独,因为缺乏连接\pozhehao{}至少不是我所期望的那种程度的连接。当然这可能是我的期望比较高,但当我回顾为什么我没有我所期望的那种连接时,我发现他很少会把精力投注于他人、投注于我,而是把绝大多数的精力投注于一些事物的规律和逻辑上,比如说个体店的没落和连锁店的兴起与人流之间的关系。而他很少会问我一些内心的事物,比如说很少问:你对此的感受、你看重些什么、你是怎么想的,而更多是在谈他自己是怎么看、怎么想的。

当我跟他提出这一点后,他说他的内心事物会向他面前的人有所描述。我猜想这可能是他表达他在乎面前的这个人的方式\pozhehao{}会把内心如何看待周围环境的方式、他的世界观表露予面前的这个人。不过我不确定这种猜想是否是因为我在乎他并想要他也在乎我,所以才把他在乎我的这种期望和设想投射到他身上。

他说他从一个在他看来很普通的本科,通过他自己的努力竞争到了现在这个在他看来更好的研究院。而他打算将来去读博,甚至考虑出国继续发展他的学业和学术。

我们走在路上的时候,他问我:一个月两万的工资你觉得足够吗?他说按照他现在的水平,他大可以就这样毕业,然后他家里人能给他安排一份一个月两万的工作。当时我马上就惯常性地回避掉了\pozhehao{}将焦点转移回对方身上,回答说:“在你看来好像这并不足够。”但其实我在那一刻真的很想冲口而出地说:不够!而我并没有在当下和事后告诉他这一点。

当我事后回想起来时,我会想到,这不是两万一个月还是几万、几十万一个月的问题,而是这些金钱本身能够给我带来安全感和力量感,而我这一生都没有过多少安全感和力量感。如果我真的能够一个月有两万或几万、几十万的话,好像我就足以立足于这个世界、足以应对这个世界了。

同时我会想到为什么我在那个当下要回避掉自身的部分,因为我会感到一种自卑感:我在经济能力、学术能力、学习能力上都比不上对方。同时我也不想将自己向他认同,因为我和他并不像。

他说他更多是受到家庭的影响才想要在学术、学业的道路上翻山越岭。他用的一个比喻性场景是:他在翻过一座又一座的山,想要看见山后的那片大海。但走在他之前的人在翻过一座又一座的山后都没有看见那片大海,所以他也不知道那片大海具体是怎样的。有的人在翻山越岭时失败了,躺平了,甚至是猝死了。但他依然会继续翻山越岭,因为如果不这么做的话,他更加不可能看见那片大海。翻山越岭的过程在他看来指的是一次又一次考上更好的学校,通过竞争挤掉一群又一群的人,在学术上取得一个又一个的成就。

当他说他更多是因为受到了家庭的影响才想要翻山越岭时,我会蛮羡慕他的。我会羡慕他的家人对他的期望,以及这份期望背后可能存在的那份他的家人对他的看重。当然我也不确定这份看重是否只是我一厢情愿地投射,因为我蛮希望有人能看重我。

我会想到,在我的家庭、成长环境里,直到现在,我父母都不对我有任何期望。每年我的生日就像是赎罪日,我的诞生只给他人带来了麻烦和痛苦。每当生日我想要去做一些自己想做的事情时,我妈会说:这不是你的节日,这是我的节日,是我把你生了下来,是我经历了分娩的痛苦和折磨。这一直让我有一种亏欠感\pozhehao{}好像我亏欠了他们、亏欠了很多人很多东西,同时也会有一种无价值感\pozhehao{}我的存在只是一种痛苦和折磨的象征物,仅此而已,除此之外便毫无价值。

回想起学生时期,那时候我有被老师所期望、有被父母所期望考一个好成绩。但这让我感到很沉重和无力,以至于后来小升初时我直接自我放弃了,因为我并不觉得我有这样的能力去达到那个要求。我找不到力量,力量都耗竭了。所以那时候的我仅仅只是在延续着自己的存在,想着只要熬过去这段学生时期就好了,如果能熬得过去的话……现在回想起来,那时候的我所想要、所期望得到的期望是一种能够在我没有力量的时候给予我一份支持、不带要求的无条件期望。这种期望并不是要求我一定要达到某个多高的分数或排名,而是希望我能朝着我所想要的方向发展。即使我不往任何一个方向发展、不往他们所想要我发展的方向发展,或者是我达不到怎样的高度,他们也不会因此而指责我、谩骂我,不会像我母亲在我小时候说:“你就是懒”、“生个叉烧都好过生你”。

我没有从父母、家人、亲戚、老师、任何人身上得到过这样的支持\pozhehao{}“你可以去做任何你想要做的事情,无论成绩、结果如何,如果你需要支持的话,可以来找我”之类的话。这也像是之前所学的人本主义课程里提到的无条件接纳\pozhehao{}之所以接纳对方,并不是因为对方满足了自己怎样的要求、期许,而是无论对方是怎样的,都能够无条件接纳对方本来的样子、原本的存在,而不是向对方投射一些自己的欲望,不是将自己本想要成为的样子投射到对方身上,将对方塑造、扭曲为自己的理想自我。

晚上,我做了一个梦。梦里的我在大学宿舍,今天下午3点要考大学最后一门科目\pozhehao{}考研政治。这门科目让我感到很恐惧,因为我并没有复习,也没有背书,更不知道这个考试是不是开卷考,甚至连正式的考试通知都没有,只是同学之间默认下午3点就会去考试。而我焦虑得没有吃午饭。快到下午3点的时候,宿舍的同学说体育这门考试不用考了,再也不用理了。我想到这蛮好的,只不过这是体育而不是考研政治。过了下午3点后,我们依然没有收到考试通知,宿舍里的人也不知道要不要去考。然后宿舍里的一群人就打算去市区打火锅,吃点东西。我就跟着他们一起坐公交,离开了在城郊的学校。

公交上全部都是我们宿舍的学生,坐了超过一半的座位。在前往市区的公交上,我依然很焦虑:万一考试通知突然传下来了,我们是不是要往回跑。但我什么都没有复习、没有准备。我和他们在车上聊着天,在不同的群体里到处聊聊。但后来我还是决定一个人坐在一个窗边的座位去看书,我在那本书的最后一章找到了应该如何着手做毕业设计的内容。在看的时候,我的焦虑下降了,而我也醒了过来。

当回想起这个梦,无论是考研政治还是毕业设计,这两者都让我极度焦虑、极度恐惧。我在读大四时对此的应对方法一直都是回避,甚至是逃离学校(去实习),直到最后一刻才回学校做毕业设计。

在梦里我似乎是在一个人多的氛围下,我开始有力量去应对这样的焦虑和恐惧。但是这并不是一些针对个人的支持,而只是一种氛围。而且梦里的我所求助的工具是书本(一个人的资源),而不是找个同学去聊聊、去了解。梦里的我依然是靠自己一个人的努力去应对焦虑和恐惧。

当回想起梦里的那种恐惧感,其实我在和那个有好感的男生相处时也会感受到这种恐惧感。和他相处时,我想到,如果我和他真的在一起了,万一他之后的学术学业需要他去另一个城市发展,万一他需要去国外读博\pozhehao{}不仅仅不在同一个城市,甚至不在同一个国家的话,我并没有足够的经济能力去跟上去,我没有能力去追上那个我想要共处的人。这并不只是一种“我比不上对方”、“我没有足够的能力”的恐惧,而是一种更深层的恐惧\pozhehao{}我害怕丧失他,害怕被他所遗弃,害怕自己的无力留不住我想要留住的人。

因为这是我在过往\pozhehao{}无论是初恋还是前任身上\pozhehao{}很强烈地体验到的,也是我在童年和父母、老师、亲戚等一些我本期望他们能够留在自己身边,能够一直给予我支持,甚至只是带来支持的人的关系里,我都感受到了一种强烈的无力感\pozhehao{}好像是因为自己的能力不够,他们才消失了、离开了、不再看重我了。

