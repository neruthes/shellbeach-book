\chapter{事件,镜子,解释,幻灭和崩塌}

\ardate{2022-06-12}{P2krRyzIVDEvyfjzQiMaEg}



对于最近发生的一个事件,我会感到有点难以描述,有被称之为“唐山打人案”的。

这个事件的有趣之处在于,它就像是一面镜子,每个人都其中看见自己有意识或无意识想看见的东西:有人看见了暴力、有的人看见了女性、有的人看见了性别对立、女权、非文明。正因为这个事件在一开始并没有一个固定的解释或唯一的答案,所以似乎每个人都有对此进行解释的自由\pozhehao{}每个人都是作者,一个对这个事件的作者。

我会在想,为什么不同的人想要去解释这个事件?因为这个事件本身好像会对人们的社会观带来很大的冲击\pozhehao{}为什么在现代社会里会发生这样的事情?这种冲击似乎刺激了人们想去防御更深层的事物的崩塌,去合理化、去解释这件事情为什么会是这样的,这件事情是合理的、解释得通的、明了的、确定的,更重要的是\pozhehao{}可控的。而一旦找到了某种解释\pozhehao{}比如说这就是暴力、这就是女性、这就是性别对立、非文明,这个解释便能够成为一个锚点,提供了人们能够宣泄愤怒和攻击性的方向和目标。

另一个有趣之处在于,正是因为每个人看见的东西都不一样,因此不仅仅是事件本身,而是间接目睹了这个事件的人与人之间也产生了冲突,就像是一个声音通过不同的洞穴里传来了不同的“回声”,而有的“回声”则容忍不下与自己不同的其他“回声”的存在。

而那些人甚至并不只是网络世界上的网民,有的更是朋友圈、生活里亲眼能看见的他人,那些距离自己更近、太近的他人。而如果这样的事件不是发生在千里之外,而是发生在自己很边甚至是自己身上呢?身边的他人会如何反应?会是旁观者甚至是施暴者吗?那些和自己身处于同一个社会的距离自己很近的他人原来并不和自己拥有着同一种社会观,而这种社会观的差异甚至会让自己感到恼火和愤怒甚至是恐惧。

在一个看似稳定的社会里,暗藏着各自千差万别的社会观,而不同的人的内心那各自冲突的社会观由这次的事件中镜映了出来,并在相互冲突的过程中幻灭了彼此对同一个社会共同体的美好幻想\pozhehao{}每个人眼中的那个我们赖以生存的社会并不一样。对同一个社会共同体的幻灭和崩塌能够激发个体的各种情绪,例如不安全感、愤怒、恐惧,而这些丰富的情绪开始沿着这个事件的传播而扩散了出去\pozhehao{}人们并不只是传播着事件本身,更是传播着每个人对此所产生的情感\pozhehao{}将对同一个社会共同体的幻灭和崩塌传播了出去。

