\chapter{“凭什么我要在乎他人?他人关我屁事?”}

\ardate{2022-03-09}{KsJdR6OF3SFuxa\_FjkxFKg}



\dialoguelist{August}{
	\dialogue{冬裔}{但说实话,你好几次说现在还单着的人大概很糟糕了。}
	\dialogue{August}{是因为遇到几次单身的人都很糟糕。}
	\dialogue{冬裔}{抱抱你~运气不太好。}
	\dialogue{August}{我最近也不想面基了,休息下。}
	\dialogue{冬裔}{前段时间面得很多吗?}
	\dialogue{August}{好像不少,我忘记见过几个了。除了糟糕的记忆外,好像没有什么特别记忆。}
	\dialogue{冬裔}{那都做了些什么呢?}
	\dialogue{August}{喝咖啡,到处走走,也有have sex的。}
	\dialogue{冬裔}{在面基前你会期待什么?}
	\dialogue{August}{期待能和对方有些深入的聊天吧。}
	\dialogue{冬裔}{那最后没能深入是为什么呢?}
	\dialogue{August}{因为好像对方没有什么自我探索,(对方)也没有什么兴趣探索他人。}
	\dialogue{冬裔}{具体是什么表现呢?}
	\dialogue{August}{(比如说)话题飘忽,很少情感和经历的描述,更多聊日常有的没的。还有人觉得他们自己很受欢迎。Male of the Year.}
	\dialogue{冬裔}{有没有可能是不熟悉所以不愿深聊呢?}
	\dialogue{August}{嗯,可能是。}
	\dialogue{冬裔}{像我就还比较慢热的。如果和不熟悉的人一开始就聊一些很深的话题,我可能会有防御心态。}
	\dialogue{August}{其实当你把焦点放在他人而不是放在我身上的时候,我会感觉你没有看见我的情感,而只是在乎他人。(我)会有一种更加被孤立的感觉。}
	\dialogue{冬裔}{啊对不起,让你有这种不好的感受了。但我不是更在乎他人噢,我可能是有些不自觉的去代入到你的视角然后去猜测他人不愿深聊的原因。抱歉抱歉。}
	\dialogue{August}{在自己心情不好的时候,真的不会有太多精力去在乎他人。所以当你这么说的时候,其实我没有太‘听得进’你说的关于他人的话。}
	\dialogue{冬裔}{那怎么做可以让你好过一些呢?}
	\dialogue{August}{可能过段时间吧。}
	\dialogue{冬裔}{用时间的推移来淡化现在不佳的心情吗?唔,那……}
	\dialogue{August}{嗯,算是。再聊呗~}
}

我记得一年前的我也对去年认识的一个现在不在这座城市生活的男生干过这事:在对方还处于很难受的情绪时,和对方去讨论关于心智化他人的事情,比如说“对方可能会是……;对方可能想的是……”。那时候,当我发现他不会心智化他人时,我就将其归因为他不愿意共情他人、共情能力差、在共情方面有个人议题。现在想起来,那时候刚开始学心理学的自己还真是稚(you)嫩(zhi)。

在这次和冬裔的聊天里,我似乎处于和一年前那个男生身处的同一情况,或者说能感觉到他当时可能会有的感受\pozhehao{}感觉自己的情感没有被看见,觉得对方只是在乎他人,感觉自己更加被孤立。在和冬裔聊天的那个当下,其实我更想说的是:“凭什么我要在乎他人?他人关我屁事?为什么你要关心一个我不喜欢的他人。为什么就不多关心下我?”

不过,在和冬裔聊天的那个当下,我也意识到这一共情失败可以用理论来解释,而且在理论层面我知道这是怎么一回事,但当时的我的情感依然很难受,我需要照顾好自己,而不是和冬裔说关于理论的事情。

\blockquote{
	我逐渐在接热线的过程中发现一些共同的过程\pozhehao{}containment(涵容) and holding(护持),这两者也是如果个体要发展心智化能力,那么其依恋对象需要具备的两个关键点。

	大多数来电的人都带着(强烈的)情感困扰甚至是自杀意图/计划。一开始,我处于试图contain(涵容)对方情绪的阶段,包括去镜映对方的情感和重述对方的话,试图跟上对方的脚步并和对方呆在同一个心理境地。涵容指的是去contain(包含)对方的(负面)情感,但这并不完全是安抚,也更加不是要说些什么安慰人的话(比如说“一切都会变好的”、“乐观一点”、“你别不开心”)。和对方呆在同一个心理境地,耐心地等待对方的情绪宣泄的结束,直到对方的精力开始慢慢恢复过来后,就可以开始护持着对方去探索(心智化)TA自己和他人。

	我有试过在一开始就和对方去探索(心智化)TA周围的处境和他人,但我发现对方有大量的情绪需要宣泄,而在宣泄之前,处于强烈情感下的对方难以运用心智化的能力。所以只能先涵容,后护持,而不能拽着对方去干对方在那个当下不想且无力去干的事情。等强烈的情绪舒缓后,对方的心智化能力也就开始慢慢恢复过来,才能够开始探索周围的处境、TA自己和身边的他人。

	然后是holding(护持)阶段,我会试着和对方一起探索周围的处境和、TA自己和身边的他人,试图去看一些固化的思维模式、负向信念等事物,一些将对方固化了在原地难以动弹或局限了对方的事物。如果幸运的话,有的人在看见那些将自己所固化的事物后,会试图去挪开那些固化物,试图给自己的生活作出改变,思考接下来要作出的生活决策。

	\blockquotesource{电话热线,Spirit Guide,涵容和护持}{白色灯塔先生}{2022}
}

冬裔说的:“我可能是有些不自觉的去代入到你的视角然后去猜测他人不愿深聊的原因”,在我眼中,可能更像是护持阶段所做的事\pozhehao{}和对方一起探索周围的处境和身边的他人。但在这之前,倾听者需要先涵容对方的情感,特别是当对方处于强烈的情感而暂时丧失或不愿意去心智化他人和自己的能力时。

我想起在一开始接热线时,我也会急于和对方探索这件事究竟是怎么一回事,从而忽略了对方的情感,比如说(虚构):

\dialoguelistthin{对方}{
	\dialogue{我}{那你会想到,这究竟是怎么一回事吗?}
	\dialogue{对方}{我现在脑子很乱,什么都想不到,我很难受。}
}

我意识到对方的情绪状态还不足以让对方启动心智化能力,所以先去涵容对方的情感,让对方知道我在跟随着TA,并愿意和TA呆在原地。如果没有精力去探索,那就一起“躺平”。

\dialoguelistthin{对方}{
	\dialogue{我}{好像你现在真的很难受,什么都想不到,内心乱成一团。}
	\dialogue{对方}{是的,这种感觉真的很难受,就像是……(对方继续宣泄情绪)}
}

在接热线的过程中,我蛮认同的一个观点是:如果对方不愿意去做自己想对方做的事情\pozhehao{}无论是自杀风险评估也好,详细说自杀计划也好,探索也好,甚至不想继续聊下去了\pozhehao{}那就好好倾听呗,不然还能干嘛。

有时候当自己手头上的工具越来越多,或者说越容易看见对方的人格“裂缝”在哪的时候,我一直都会一种欲望:撬开那个裂缝,去看裂缝背后的是什么。但这种求知欲也好,好奇也好,并不一定是以对方为中心,甚至不是对方想要的,更不用说可能给对方造成心理伤害。或者是极力要在他人身上保全自己的全能感,比如说一些父母对待孩子的态度:我就要“帮”到你,如果“帮”不到你,我就是无能的。

我并不知道冬裔的助人动机是什么,但在和冬裔聊天的过程中,当他试图“帮”我去看那些他人的时候,我除了有一种自己的情感不被看到的感觉之外,我还有一种感觉就是:他好像只是为了他自己而做这些事情,而并不是为了我。

另一方面,我也会想到这样的互动方式是否会和母婴互动有关,比如说有的养育者在婴儿哭泣的时候,并不是去涵容婴儿的难受的情感,而是去拿各种各样的玩具或其他东西来试图分散婴儿的注意力。后者可能能做到让婴儿停止哭泣,但婴儿的情绪也因此而完全没有被看见,因为养育者这么做的原因更多是为了缓解TA自己对婴儿哭声的焦虑,而不是为了理解和安抚婴儿的情绪波动。在这样的互动模式下,婴儿长大后或许能够心智化他人(比如说知道对方在试图引起自己的注意力),但难以心智化自己(比如说难以识别和命名自己内心的不同情感)。






% \dialoguelist{咨询师}{
% \dialogue{咨询师}{听起来,好像确实是这样的一个画面。一个更内在的你在内在的空间里,而另一个你则在相对外界的空间处理着外界的事情。我记得在之前的咨询里,你是一个协调者的角色,而这次又是……}
% \dialogue{我}{嗯,好像确实是两个不同的意象。现在更像是在内在空间里的自己和外界的自己。}
% \dialoguesepline{咨询师}{短暂的沉默}
% }
