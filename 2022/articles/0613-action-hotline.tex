\chapter{无为和有为,热线}

\ardate{2022-06-13}{U1CXsmY39rX9E2KKIOSBrQ}


\midnote{以下来源于课程作业(经修订)}

反省自己的生活、工作等经历,你是很有为的还是很无为的做法?什么时候有为?什么时候无为?留意因为怕事而不敢行动,并不是真正的无为,是不能为,不敢为。无为是可以出手而不出手。

我会想到现在我接心理热线已经有大概23个小时。一开始的我是很有为的,很努力地推动整个倾诉的进程,比如说主动去问对方、问来电者很多问题:这件事情、这个过程是怎么发生的?你现在的感受是怎么样?你觉得你可以做点什么吗?你会想到什么应对方法吗?你过去是怎么应对这样的情况的?这些也是我在心理热线培训里学到的用于应对不同情况的“工具箱”。

但时间长了,我会感觉自己越来越累,自己接热线的状态越来越耗竭\pozhehao{}越来越不想去共情、不想去理解对方、不想去问任何事情、不想说任何话,甚至连自己在接热线和日常生活的状态都变得十分漠视人性\pozhehao{}我完全不在乎你的情绪、你的存在、你的死活、你的困境、你的挣扎,你的任何所说所做与我无关,而我也不需要为你做任何事情。

当我察觉到自己的接热线状态越来越不对劲后,我发现问题并不在于“工具箱”的多或少,也不在于我的回应或提问时机是否恰当,而在于我好像太有为了。当我很有为时,对方似乎有意识或无意识地知道ta可以越来越不去做某些事情,比如说可以不去推动倾诉过程的发生\pozhehao{}如果我不问些什么,来电者可能就不打算说任何话,就只是持续地沉默,特别是一些处于抑郁情绪的来电者。

因此,我开始调整自己的工作状态\pozhehao{}去确定我自己感到舒服的边界和明确热线的功能和范围\pozhehao{}舒缓情绪和梳理困扰。在自己的热线工作边界里,我会尽力地有为,比如说尽量地通过共情对方来舒缓情绪。而在热线工作边界外,我在大多数情况下是无为的,比如说如果来电者想要去找解决办法甚至是向接电员讨要解决办法时,我会在明确边界的情况下陪伴来电者去找ta想要的答案,而不是走在ta前面甚至是代替ta去找答案。

另一方面,我发现当自己明确了有为和无为之间的边界(即哪些是我应该做的,哪些是我可做可不做的)之后,倾诉(即使是听朋友倾诉)的过程变得更加顺其自然,。如果对方缺乏继续倾诉的动力而沉默过多、过久时,我会问对方“你还会想继续说点什么吗?”如果对方有所回应,那倾诉的动力也就出现了;如果对方说“没有什么想说了”,我会继续问对方“那你觉得我们这次就聊到这里,你觉得ok吗?”热线倾诉的过程也得以更顺其自然地结束,而不再是像一开始自己接热线时:明明主动打热线的是来电者,来电者却在倾诉过程中越来越将ta自己本该承担的动力和责任推给、投射给接电员,而我则逼着自己去扮演那个像是全能照顾者角色。

