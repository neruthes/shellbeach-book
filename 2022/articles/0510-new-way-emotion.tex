\chapter{新的情绪处理方式,内心形象的幻灭}

\ardate{2022-05-10}{Q5mXIj3QWQ5zQPrZrHzQWg}





在上一次心理咨询,我担心自己处理情绪的方式是否会开始固化。但在周末和前任喝茶聊天后,在处理读者群里两人的退群和被删好友所激发的我的情绪时,我开始发现自己处理情绪的方式有所改变:不仅仅停留于看见那些属于自己的部分、那些自己的投射物本身,而是进一步将自己内心的事物从他人那、从现实世界里分离出来、撤回。所以当这样回想时,现在的我不会担心自己处理情绪的方式会变得固化,因为这样的方式时刻都在改变着,随着自己的生活经历和自我不断改变着。

不过,我又在开始担心新的事情:如果以这样的方式去处理情绪的话,这是否意味着我可以面对任何人的丧失、任何家的丧失?如果是的话,我真的愿意接受这样的事情吗?如果一份关系、一个家对我而言真的意义重大,而在探索完属于自己的部分后我依然认为那就是我所珍重的事物的话,我会想要付出行动去挽回。

同时,新的情绪处理方式也让我意识到,我并不是需要挽回每一段关系。在一开始我发现他们退群并删了我好友时,我会很想去挽回关系。当我很想见诸行动时,我问自己的另一部分:“为什么我想要挽回这段关系”,然后那个部分回答:“因为如果连普通朋友关系都挽回不了,那我还怎么挽回和重要他人的关系”。当听到自己的另一部分这么说时,我突然意识到:噢,原来我想要挽回普通朋友关系是因为自己恐惧于丧失和重要他人的关系。然后我开始想到和前任的关系:我和前任恢复联系了,恢复联系这个结果有属于他的部分,但我也相信这当中也有属于我自己的部分,而在那属于我自己的部分里,或许在某种程度上也可以说是我“挽回”了关系。

在最近几天经历了家的内心形象的幻灭后,我开始在思考我能为自己做些什么,以及我能为自己身边的外界做些什么。之前的我是很直观地去体验,比如说去体验做某件事情是不是我真的想做的,如果感觉不想做的话就不做。但现在的我像是站在一个更高的视角去看待自己和外界,去看我能给予自己些什么和能给予外界些什么。不过我不太确定这样的方式是否是一种隔离,隔离掉我可能会直观体验到情绪。因为最近每当想起家的内心形象的幻灭时,我都会感到很悲伤,想到自己好不容易在搬离前任公寓后在脑海里建立起了一个家的内心形象,然而这个内心形象终究还是丧失掉了,就和现实世界里的家一样。当自己很悲伤时,我会变得很困,说话也没有力气,思维也没有力气,只想睡过去。所以最近的睡眠时间也比平时多了几个小时。如果只是普通的悲伤,那并不会带来疲惫感,但当我感受到疲惫感时,我也意识到这份悲伤比其他类型的悲伤要更为沉重和深入。

当有一天躺在床上因悲伤而带来的疲惫感而准备入睡时,我往回看自己和前任分手那几天的聊天记录。我会看见双方交流之间的不同频\pozhehao{}那时候的我在疯狂地宣泄情感(悲伤和愤怒),但那时候的他一直在用理智化的方式来回应。

不过,在看着聊天记录时,当回想起上周末和前任喝茶聊的内容时,我会觉得那时候的我对那个家赋予了太多的意义和价值(比如说港湾、重要他人、依靠、安慰、陪伴),而在前任眼中那些意义和价值甚至是“家”本身根本不存在。那时候的我并没有去跟他核实我赋予那个我眼中的家的意义和价值,而是认为他辜负了我对一个家、一个重要他人、一个伴侣、一段亲密关系的期待和要求。我眼中的家在对方眼里甚至从一开始就不存在,再多的意义和价值只不过是我的投射物,只是我的“丰富的内心世界”的一部分。否则对方也不会无故消失。

不过这似乎依旧是自己以前的一贯做法:为对方的无故消失、所做作为找解释。但属于对方的部分的猜想只能通过跟对方核实才能够证实这样的猜想是否符合现实,而我能决定的,只有属于我自己的部分。

写到这里时,我去问前任:那个曾经同居的公寓对他而言意味着什么?他回应说:住所,只是一个能承载功能和满足需求的地方,没有什么特殊性和独一无二的,什么人、什么物都一样。

