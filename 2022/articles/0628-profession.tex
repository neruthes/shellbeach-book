\chapter{擅长但不喜欢,不被理解,兜圈}

\ardate{2022-06-28}{aoOH5lkZ-1tQICOUFkNIXg}



昨天的情绪很糟糕,然后我在群聊里问了问:“你们会遇到一些事情是自己擅长,但并不总是喜欢做的吗?会选择把这样的事情往职业方向发展吗?”

有的回应是扯到了自己对前端开发的态度和看法;其他的回应则是:工作和生活分开呗,兴趣能作为工作总是少数人的幸运,何况还有久了以后兴趣变得无趣的可能性;工作是为了生活,兴趣是为了娱乐;如果预期会有巨大的收入飞跃,转行的尝试就还算值得,外部诱惑不够大时,留在舒适区几乎是必然的选择;这样看下来你只是思考是否要从一个擅长的领域到另一个擅长的领域?这其中和兴趣没有什么关系,将赚不到钱的兴趣放在工作之外去做。

在这个过程中,我一直感到自己不被理解,他们只是在说兴趣、工作和生活之间的事情,他们所说的都与我无关,也并没有在充分了解我的情况下给回应,而是马上就奔涌出了属于他们自己而与我无关的看法和态度。

所以后来我去打了热线。一开始我会不知道该怎么开始,我就说我今天的状态很累,既是身理的累也是心理的累。但我也发现在我诉说的过程中,我的语速越来越快、越来越兴奋,根本不是一种累的状态,而更像是自己想说的话、想表达的内容没有地方说而压在了内心的时候,会让我感觉到累,需要耗费精力去抑制自己想要表达的冲动,因为再多的表达只会听到来自他人的不被理解的回应。

在热线里,我说我昨天去见了一个朋友,他有打算考研,所以我又重新有了往心理学专业考研的念头。但我说我有在学习心理咨询方面的知识,在学习的过程中会有实践的环节,但在实践的过程中我并没有感受到多少的开心、享受和意义感,给我的感受更多是:我擅长做这方面的事情,而且在日常生活里我也擅长去捕捉对方的言语用词、语气、肢体语言,去捕捉这些表象背后可能暗藏着些怎样的事物,但我并没有在这个过程中感觉到自己是开心的、是有意义感。所以整个过程下来,我越来越感觉自己是个工具人,我只是在做我擅长的事情、在实现我的自我功能。

我之所以擅长,可能是因为小时候的我需要去监控我妈的情绪,去时刻监控她什么时候从一个正常人突然间就打人骂人,好让自己躲开、逃跑。这些并不是一些好的经历、让我感到开心和快乐的经历,反而是一些痛苦的经历。所以在我做我擅长的事情时,我并没有感觉到有多少开心和意义感。

所以我说,在那段实践的过程中,我有一段时间经历了耗竭,就是我不想去在乎任何事物和人了,退回到了自我封闭的状态。接电员问我会想去找身边的朋友吗,比如说昨天见面的那个朋友。我说我有一个群聊,群聊里有我的一些朋友或群友,我会跟他们说我的状况,但是没有得到多少理解,而且那些回应都只让我感觉到自己更加不被理解,比如说工作和兴趣应该分开、工作是为了维持生活之类的话。当感觉到自己不被理解之余,我又会想到,那我要他们干嘛?只是一群停留在表面的、吃喝玩乐,但并不理解更深层的事物的朋友吗?

不过我也说,昨天见面的那个朋友确实能够相比与其他朋友更理解我,但我不想想去打扰他,因为真正能更理解我的人已经很少了,所以这段关系对我来说更为珍贵。如果我真的表露了自己这部分的话,我会担心他会不会无法足够理解我。如果是的话,可能会显得好像我很能理解他,但他没那么能够理解我,我会担心这会给关系带来破损,比如说他可能会感觉他自己的能力不足,而我可能会感觉到两人之间的理解对方的能力的差异感。而我也不想因为自己的擅长之处而让对方感受到ta自己的不足、感到一些不好的体验。

不过我也会想到,如果我真的那么小心翼翼、真的那么停留于表面的话,那这段关系好像并没有那么的真实、那么的有韧性,而是会显得很脆弱,好像我只是想要维持一个表象\pozhehao{}不带强烈的目的性地去约见面聊生活,而不是像热线那样我带着一个很明确的目的说我就是要来聊这件事的。

接电员说我会想得很周全,会想到方方面面。我说是的,有时候这种猜想甚至会自动地浮现出来,这背后是一种对关系的不安全感。然后接电员问我之前的那种自我封闭的状态是否就是因为这些猜想的存在才会那样的。我说不完全是,这种猜想会固化那种自我封闭的状态,但那种自我封闭更多是一种止损的行为,就是我已经没有力气去在乎任何事物和人了。

接电员后来联想到了一个密室的画面,我在密室里找到了一把钥匙,那把钥匙代表着考研,但我会担心如果拿了那把钥匙开门我只会去到另一个密室。我说“密室”这个词让我想到一个很漆黑,甚至是很密闭、很缺氧的环境。我说其实我在平台上陆陆续续买了不少课程,一开始我会很沉浸于学习新课程的快乐里,会觉得很开心、会有满足感,但后来这种开心越来越少了,而更像是不断的重复。那时候我意识到,好像我需要去找其他新的东西去做了,就像是我在一个缺氧的环境里,我要不断地找氧气罐,一瓶用完了就要马上去找下一瓶,但这个过程就很让人窒息。

然后我想到为什么我会处于一个缺氧的环境,因为对我来说我这一生并没有什么乐趣可言,有的只是一些转瞬即逝的快乐,而不是一些持续的、稳固的开心的源泉,好像我一直都只是在找一些东西来缓解生活里的无趣、无意义感、虚无感。而如果是考研的话,我好像已经能设想到这只会让我自己去到一个更不开心、更糟糕的境地。

我说,对我来说意义感好像就在于我和他人的连接,比如说和一个很久没见的朋友或者是自己喜欢的人见面的时候、重逢的时候的那种连接感就能带来意义感,而如果我能从身边的朋友得到这一点的话,为什么我还要去学咨询?好像并没有这个必要。接电员说是啊,这是一个助人的工作。我说,嗯,我助人,那谁来助我?当我感觉到很耗竭的时候,身边并没有足够的支持。

在热线的中途,接电员有提到说我有考虑过去读一些偏学术方向的心理学专业吗,这样就不用和人有太直接的接触,而又能发挥自己的擅长之处。我说之前确实没有考虑过,我之后会考虑看看。然后我跟接电员说,听起来好像是一个避开自己的雷区,去找一条折中的道路,尽可能地自我满足。

在聊完热线后,我会感觉自己的心情并没有减轻多少。当我回想到为什么的时候,我会想到,好像彼此的聊天没有足够地走深,没有往一开始已有的深度继续走深,好像对方并没有捕捉到太多能让我走深的切入点(例如密室),而我只是一直在热线的过程中一直在抒发一些已有的情绪。我会想到,走深是一件蛮难得的能力,就像是半年前上的人本主义课程,看到案例演示视频里的咨询师是如何捕捉对方的线索,如何将对方言而未说的内容以巧妙的方式呈现出来。

回顾自己在不断倾诉的整个热线过程时,我会觉得自己是在兜圈,只是描述了困境的不同角度,但一直都还处于困境当中。如果我真的停下来呢,如果我什么话也不说呢?我会感觉蛮绝望的。就像是小时候总是梦到的无尽漆黑楼梯间,如果我不再不断往上走或往下走的话,好像那片漆黑,那片无趣、无意义、虚无就会开始侵入自己的内心。

小时候我很害怕下楼梯间,因为家楼下的那条楼梯间总是没有灯泡。后来我想到的一个应对方法是:在楼梯间里,我会逼自己呆在那里,和那片黑暗融为一体,后来自己也就慢慢习惯了褪入到黑暗当中,就像是从一个猎物转变为一个狩猎者。

也许我依然在一个转变的阶段,依然在将一份又一份本不属于自己的部分,甚至是自己之前所恐惧的部分不断内化、转化,直到自己变得面目全非。

