\chapter{“理想中的自己”就是现在的样子}

\ardate{2022-02-16}{oLOBiRgxy2VdVduT6v\_aNQ}



\midnote{以下来源于我的本周课程作业(经修订)}

\blockquote{关于人的成长,人本-存在主义先驱维克多·弗兰克尔曾谈到过“一个人朝向理想中的自我去行进时,虽然终其一生都不可能完美实现,但他或她也必将会一次次遇见更让自己满意的自己”。你想要成为的理想中的自己一个怎样的存在呢?如果你愿意,请试着描述自己的理想存在。如果这样的描述比较难,你也可以以某个角色为路径进行探索,比如谈一谈作为一个咨询师(培训师、工程师、演员、作家、创业者等等),5年、10年、30年之后的理想自我分别是怎样的存在。}

当第一眼看到这个题目时,我内心会有一种不舒服的感觉。在细读题目后,我发现那种不舒服的感觉来源于一种排斥感\pozhehao{}对题目的人生观感到排斥。“理想中的自己”以及“5年、10年、30年之后的理想自我”的表述会让我感到一种强烈的感觉:当事人并没有好好地呆在当下,而是在把目光投向未来\pozhehao{}正因为把目光投向未来才更难以呆在当下。

我会想到,如果在咨询室里的咨询师开始思考一些关于未来的事情(比如说待会儿晚饭吃什么)或过去的事情(比如说上一个咨询记录要补充些什么)的话,那么咨询师就已经没有和来访者呆在当下了。同时,我会想到这种人生观会影响的可能并不只是处于咨询室里的咨询师,还是咨询室外的咨询师。如果在咨询室外的咨询师时刻把目光投向未来,投向那个理想自我,那么当面对来访者时,又如何促使来访者愿意呆在咨询的当下,而不是沉浸于过去或担忧着未来?毕竟连咨询师自己都难以做到在日常生活里呆在当下\pozhehao{}更多地身处于一种“我—它”关系,而不是“我—你”关系。

所以,对于我而言,“理想中的自己”就是现在的样子。我会想起以前的我会以影视作品里的某些人物(例如日漫《黑执事》里的夏尔)或某个作家(例如Virginia Woolf)为模范,直到学咨询后我发现这些模范都比不上属于自己内心的经历、特质等事物,而且这些事物是经过自己的努力地吸纳、转化和创造而来的,是完完全全属于我自己的东西。以某个角色或他人作为模范,似乎会让我越来越偏离属于我自己的样子,局限自我的可能性。

所以,我试图描述一下我的“理想自我”(也是现在的自我)的存在:在处于非强烈情感的状态下有足够知识和能力去心智化自己和心智化他人,而当自己心智化能力受到影响时能够察觉到自己的个人议题;在乎和尊重他人的感受和想法,并能够看见感受和想法背后的那个人本身;有一定的呆在当下的耐力;对表象背后的事物抱有着好奇心(特别是会引起自己的负面情绪的事物);能够接受自身的有限性和他人的有限性;无论是在何时面临学习、职业发展、自我成长、与他人的交集、自身的生命等道路的尽头,在回顾这一路上的历程时,都能对自己说:这段历程本身已经足够丰盛、足够精彩。

同时我也在作业区里看见有同学的5年、10年、30年之后的计划(早已远超于描述理想自我的范畴)很宏伟很宏伟。当我下意识地把自己和对方进行比较,并开始感到自卑感时,同样有另一种感觉冒了出来,那种感觉好像在说:那并不是我,那不完全是我想要的,我想要的是在这个当下成为我自己。

我也会想到最近在读的《格式塔治疗实录》里皮尔斯说过的一段话:

\blockquote{一旦我们意识到我们行为的结构,在自我提升的例子中就是上位狗和下位狗之间的分裂,如果我们通过倾听,能够理解发生了什么,那么我们就能让两个争斗的小丑达成和解,然后我们意识到我们不能刻意地为我们自已和其他人带来改变。这是非常有力的一点:很多人终其一生想要实现他们应该是什么这一概念,而不是实现他们自己。这种自我实现(self-actualizing)和自我意象(self-image)的实现之间的差异是非常重要的。大多数人只是为自己的意象而活。有些人有自我,大多数人是空的,因为他们忙自己投射成这个或那个。这仍是理想的诅咒,即你不能做自己。}

我不想去投射一个理想自我的自我意象,我只想在现在这个此时此刻、这个当下成为我自己,这个我一直以来、每时每刻都在用心培养的自我。

