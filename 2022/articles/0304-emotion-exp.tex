\chapter{(情感)体验的工具性价值和存在性价值 }

\ardate{2022-03-04}{wqLCKYy\_NsYPKJYNEUBjvQ}


\dialoguelistthin{Neruthes}{
    \dialogue{Neruthes}{我没有多少需要他人参与的情绪,甚至缺乏自身情绪变化。平常的生活中容易有的情绪状态只有兴奋和无聊。所以我现在好像还不能完全理解 contain 和 hold 的具体细节。兴奋是一种奖励机制,驱使我探索新知;无聊是一种保护机制,防止我在不重要的事上钻牛角尖。\\
        ……例如,某些事没做成,「遗憾」只会短暂停留,一分钟之内就会进入「兴奋」状态来做总结复盘、改进计划、blah blah。\\
        由「理性的自我」去harness情绪机制,实现意志与生理现象的和谐共处。情绪不是洪水猛兽,只要相处得当,就可以harness the power of emotions for the better living。}
}

在Neruthes眼里,情感似乎只是一种机制,一种能够 harness its power for the better living 的东西。

然后今晚我在《心智化临床实践》里读到了这段话:

\blockquote{
一个世纪之前, Darwin逐渐形成了这样一个理念,认为情绪是具有工具性价值的,这个理念可以具体体现为愤怒和恐惧在战斗—逃跑反应中的适应性价值。与其他许多人一起,Damasio通过不断细化的方式,继续深化了这一理念,例如,记录了直觉是如何适应性地掌控个体行为的。因此,有关直觉的科学开始涌现。不过,我们对于情绪心智化重要性的强调,远远超出了它所具有的工具性价值。情绪并不仅仅会在知觉的基础上促进适应性的反应,通过它们所具有的目的性结构,情绪还可以对知觉和判断进行组织。此外,情绪还具有存在性的价值。斯多葛派学者曾经为了增进自身的宁静而重新思索他们的情绪判断,Grayling则雄辩地指出了他们所犯的错误:
尽管这种(斯多葛式的)教育是为了帮助人们去更好地忍受世事的变化无常,而且其灵感精髓是哲学史上最为敏锐和思辨的一种,但是它却遗漏了一个非常重要的方面。那就是,如果一个人吝惜于自己的情绪为了避免伤痛而限制去爱,为了逃避实现它们的代价而去压抑自己的欲望和需求那么这个人将度过自已成长受阻的、压抑的,且平淡无味的一生。这就等同于为了减少那些生存时令人震撼的特质\pozhehao{}愉悦、狂喜、丰富和多彩,以及与之相匹配的痛苦、 沮丧、灾难和悲伤\pozhehao{}就选择部分死亡一样。要拥抱生命,去体验和接纳它,带着能量和欣赏的姿态融入它,当然也就等于在邀请所有类似情况下出现的痛苦和麻烦。但是,完全避免痛苦所带来的代价将是巨大的:这种代价就是去践踏人类能够在地球上存在的不到一千个月的时间,让人从未真正地活过。
}

在昨天的电话热线里,一个来电者向我分享了一些TA生活里的美好瞬间,而我也对TA的分享表示了感激。这让我感受到人与人之间的真实相遇,moments in life。我在那个当下所感受到的开心,仿佛和对方共同身处于那个场景里,享受着那个瞬间所带来的开心。如果工具性价值的角度来看待这份开心的话,那最多只能说是:这一情感拉近了彼此的距离,nothing more。但如果不只是从工具性价值的角度,而是从存在性价值的角度去看,这份情感体验本身就像是彼此的内心里留下了一道印记,it's something without equal。

我在最近上的释梦课程里也会看到相似的处境。课程中的讲师会对大多数的梦进行释梦,但有一些梦并不需要释梦,例如“灵性”梦\pozhehao{}“灵性”梦邀请梦者向更高或更深层的潜能敞开,向内寻找答案,给梦者提供直接的体验,而这些体验不会被当做隐喻理解。

这会让我想到在读大三时的那个塔的梦:我走到塔的平台,看着日落的天空和远处的沙滩,然后有个小男孩在邀请我去平台另一头的游乐设施玩,而他那走到半路的父母还在等着他过去。那份情感体验本身并不需要被解读、被释梦,而只是to the things themselves(让事物如其所是)\pozhehao{}这也是我在去年年底学习的人本主义课程里,讲师讲到的现象学家的一句口号。讲师通过一杯茶来示范现象学家的视角:

\blockquote{现象学会让我尽可能地描述这杯茶的各种特征,比如说气味、温度等。两个人对同一杯茶会有不同的体验,两人的体验都是主观的。如果将这些主观元素消除掉,只是客观地谈论这杯茶,那就只会剩下客观因素和对茶的特质的解释,但这不会给你带来现象之香味。通过留白或悬置所有理论和解释,抽象的假设以及关于什么是真正的茶的情感联想,这样我们才能更好的接近这杯茶的香气浓郁的现象。“让事物如其所是”,让茶或品茶者为它/他/她自己说话(speak for itself),而不掺杂一层层地诠释和解释。如果想要了解一杯茶,那就必须自己喝一口茶,让其成为不可替代的体验,让体验为其发声。任何解释都代替不了体验。当对真相诠释时,现象学家避免使用“对”与“错”。他们仍然认识到有时候,至少在某种程度上,由个人解释的含义与其他人甚至是大多数人分享的含义存在很大差异。现象学家看来,人的行为和神经学过程最终都不能离开人的背景和处境来理解。我们的体验离不开我们是谁以及我们所身处的情境。}

这可能也是为什么现象学方法之一就是描述体验(The Description of Experience)的原因,描述事件、经历、回忆、信念、感受、观点等体验。描述体验这一过程本身似乎已经在给体验赋予其存在价值,即体验这一存在本身是有价值的,其价值不仅仅是作为 a mean to an end。就像是人这一存在本身是有价值的,其价值不仅仅在于被利用为工具人。

所以有时候我也会在想,另一个人对我而言意味着什么?是否只是 merely the sum of my experience?我对对方的感知本身只是我的一种体验,而对方在我内心世界里的形象也只不过是the sum of my experience of that person。

如果是的话,对于活在没有了存在性价值而只有工具性价值的体验里的有的人而言,无论是他人也好,情感也好,都只是不同的体验而已,都只是个工具、机制、a mean to an end,那么“对自己的体验”何尝不也只有工具性价值\pozhehao{}活着只是为了利用这副身体和这份意识去达成某个目的,仅此而已,I will die when I die。

