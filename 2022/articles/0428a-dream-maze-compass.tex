\chapter{梦,迷宫,指南针}

\ardate{2022-04-28}{2QfrYRBd1uFK68-UWp8S8g}


半夜的我做了一个梦。梦里的我在自己房间里,我坐在桌子旁,桌子的对角位还坐着咨询师。我们在做心理咨询,而我在讲述着同样的议题\pozhehao{}被遗弃。说着说着,我发现自己的身体控制不住地想哭,也逐渐地哭了出来,但自己的言语上一直在说“没事”,而且在说:“很奇怪,我不知道为什么自己会想哭,这些明明都是之前已经谈过很多次的事情了,都是些日常的事情,我不知道不知道不知道不知道……”我卡壳了,卡在了无法用言语表达自己想要表达的内容和情感。咨询师问:“你会想起什么具体的事情吗?”我说我想不起来了。然后咨询师试图伸手去拿桌面上的笔,但够不着,我把桌上的笔递给了ta,然后ta开始记一些笔记,并表露出好奇的眼神和表情。

然后我就醒了过来。

醒了过来后,我想到,这就像是自己在写作,即使写完一次又一次,事情并不会因此而变得更好,过去也不会因此而有所改变,自己依然感到很悲伤和难过和奇怪,奇怪于如此熟悉的诉说也依然会感到悲伤和难过。

然后,我在微信上跟一个之前无故消失但最近我又想约见面的男生说,我还是不打算和他再次见面了,因为我无法原谅他的无故消失,而且还是什么解释都没有的无故消失。当自己想要对他的无故消失要一个解释时,他还用再次消失来威胁我。我依然很憎恨他,不只是憎恨他的无故消失,更是憎恨他曾经让我抱有了一个没有希望的希望抱有了一段时间,然后他连同那个希望也一起无故消失了。我知道即使我说了这些话,事情并不会有所改变,但我并不想在再次见到他的时候我还要假装什么事情都没发生,更不想在他面前一直自我压抑着对他的憎恨。

把信息发给他后,事情并没有什么改变,我的心情也没有什么改变,但自己依然说了出来,因为如果不说的话,有些东西(比如说憎恨)就只能一直固化在内心深处。

睡不着的我开始在看一些关于心理咨询的科普知识:

\blockquote{
	我们都渴望走出内心的迷宫,很多人忙于寻找指南针,却忘记了迷宫本是我们自己所造。我们成长的环境也常常要求我们管控自己的情绪,用理智来解决问题。但是心理咨询却告诉我们:在心理困境面前,我们真正需要的不是否认、压抑、消灭情绪,而是在安全的环境里发现并理解情绪在诉说什么,从其中探索自己的真实体验和需求。

	……学习心理学确实对人们的困扰很有帮助。但是很多时候我们发现“道理我都懂,但就是做不到呀!”人的困扰大多是困于“感受”而不是困于知识。在处理情绪和意义这些事上,很多时候自己是自己的敌人,我们几乎永远不能做到客观。
}

我发现,其实自己一直以来都很无力、一直以来都改变不了什么\pozhehao{}我改变不了对方、改变不了彼此的关系、改变不了过去发生的事情、改变不了任何的现状。但自己依然会去做很多无法改变现实世界的事情,比如说现在向那些曾经无故消失的人表达我对他们的憎恨、曾经不断地看心理学知识试图给过去的关系破裂和无故消失找个解释。自己所做许多事情都是“情感用事”\pozhehao{}仅仅出于情感本身而做的,而这些事情并不能让我更贴近真相、更贴近答案。

而且这也很可能并不会有一个真相、一个答案。再多的“指南针”也无法走出内心的迷宫,因为只要自己还忙于寻找各种各样的“指南针”,自己便不需要去面对丧失\pozhehao{}对曾经的依恋关系和依恋对象的丧失、对过去的人事物的丧失。只要还在寻找“指南针”,自己便还一直呆在迷宫里,毕竟要是离开了迷宫,自己又能去哪呢?There's nothing left and no one left. 无家之人。

不过我也会在想,为什么自学心理学对我而言是有用的?可能因为在之前的那段自学过程中,我不仅仅是在为过去的关系破裂和无故消失寻找那个唯一的答案、唯一的真相,并不只是在寻找知识、寻找道理、寻找解释,我还在寻找自己到底是谁、自己渴望的是什么、自己能往哪里去,以及在这样的过程中一次又一次地写下同样的往事、同样的情感、同样的事情。

一次又一次地写下同样的东西并不会改变些什么,也不会让与之伴随的情感强度有所弱化,nothing will make it feel better。但我想,也许那些情感本身就是“唯一的真相、唯一的答案”,一个能让自己敢于卸下充满保护和已知的迷宫的“指南针”。

也许在过去的那段寻找答案的历程中,我一直在寻找着某些自己早已拥有的答案。在这段历程里不断内化着不同的情感,那些对自己而言曾经无比陌生的情感,将它们成为自己的一部分,将迷宫成为自己的一部分。
