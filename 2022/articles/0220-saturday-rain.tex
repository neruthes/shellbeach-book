\chapter{随笔 | 周六的“雨水”}

\ardate{2022-02-20}{uET13fU7c77S6CQDu5QL8g}

在周六的“雨水”里,我在离开咨询室后见了一个男生,这次是四次见面。(这说法有点像是第四次咨询的感觉)

在这周的工作日,其实我并不想约他见面,但又想约他见面\pozhehao{}在理智上不想见面,因为我好像看不见更多关于他内在的事物了;在情感上想见面,因为有一种想要亲近对方的感觉。而我也在咨询室里跟咨询师说了这一点,我说当我意识到自己好像看不见更多他的内在时,我试着去感受自己内心的感受\pozhehao{}闭上眼睛不去看,而是去感受,跟随内心的感觉。所以后来我还是约他见面了。

\tristarsepline

我们在公交车总站的一间书店里约看书。我在之前的见面里有提到说我在大四实习时一直想象哪天能到这家书店坐在窗边看书。然后我问他:“你之所以会选择来这家书店,是因为我之前提过我有想象在这里看书的样子吗?”他说是的,然后问我现在什么感受。我看着外面的大雨天和店里的摆设,我说:“在我之前来的时候,这里是下午,阳光很好,这里(我用手指着店内的一片空地)摆着一个堆满书甚至看似随时要倒下来但还没倒的书桌,书柜里的书都挤得难以拿出任何一本。但现在,今天刚好是阴雨天,也许之后(再来)会有阳光。不过这里更像是一个卖书的超市,而不像是个书店。这里的书都摆放得很整齐、很工整,就像是超市里的产品。”他说:“所以这和你的想象有差距?”我说:“何止是差距……”他说:“刚刚在坐下来的时候我感觉到了你的情感好像有点微妙”。我沉默了一会儿,然后说:“其实我觉得现在并不足够,看书并不足够。我之前也有和朋友约看书,结果是大家除了看书就没了,没有多少聊天。”他说:“所以你想要的是聊天?”我回应了声嗯。他说:“那好像约看书和你想要聊天两者是不一致的。”我说:“是啊。”

不过在回顾的时候,我会想到:喝茶本身也不是单纯的喝茶,约过夜本身也不是单纯的过夜,去心理咨询也不是单纯的心理咨询本身,那为什么约看书会变成了单纯的看书、约喝东西本身变成了单纯的喝东西、约看电影本身变成了单纯的看电影、约sex变成了单纯的have sex?本是为了让更多事情的发生和展开而提供框架的活动,却变成了框架本身,而内容则空缺了出来。

但我想,匮乏的不是活动框架本身,而是参与活动的人。

\tristarsepline

离开书店后,我们去了一个几乎没有顾客的商场,找了个booth在里面聊天,聊了下他的感情生活现状。临近最后时,我说:“好像无论你做怎样的选择,这背后都会有痛苦的部分在。”他认同了这一点。

当和他散场后,我在回顾一天下来发生的事情时,我才有点怀疑和他在booth里聊天时,我是否“用力过度”了。让对方了解到无论任何选择都会带来痛苦,pushing the bitter truth in front of him。不过考虑到这是跟随着对方的脚步一步步地走深的,而他也愿意这么走,我也就没有那么担心了。

\tristarsepline

当回顾到见面前我想到的用心去感受,而不是用眼睛去看时,我发现好像当我愿意用心去感受,甚至敞开内心的感受时,我能看见更多无法单纯用眼睛看见的对方的事物。这可能是因为当自己更敞开自我时,对方也愿意更敞开他自己吧。

我也会联想到在心理咨询里,有的流派并不在意过去发生的事情,而在意发生在此时此刻的事情。我想,在我和他的当下里,一直有新事物的出现。而我在心理咨询里保持咨访关系活力的方式,似乎也是我如何在咨询外的人际关系里保持关系活力的方式\pozhehao{}去看和感受当下发生的事物,关于我的事物,关于对方的事物,关于彼此关系本身的事物。

\blockquote{
    我会警惕彼此的关系越来越固化,就像是这个房间被一块又一块的石头挡住,直到空间越来越小,能聊的东西越来越少。但如果我能依靠自己的能力向你指出那里有一块石头,说不定就能把那个石头挪开,看见更多的部分、更多的东西。……因为当我在一段关系里,从同一个人身上看见不同的东西时,就像是一起去一个新的地方。但这个地方不是外界的地方,而是内在的地方\pozhehao{}我的内在、你的内在、彼此关系的内在。而且,我也不想换完一个又一个的人,在看完一个人的内在后又换,看完又换,看完又换。如果是这样的话,我好像也没有真正呆在什么关系里。但如果我能将一些东西、一些石头指出来,而那个石头又能被挪开的话,那每次见面我都看见焕然一新的对方。当然我不是去冲破对方的个人议题,不是去撞开对方的墙,而是向对方指出这一点,看对方、看彼此能做些什么。

    \blockquotesource{自我探索 | 20\pozhehao{}“内在的丰盛来源于外界的匮乏”}{白色灯塔先生}{2022}
}

