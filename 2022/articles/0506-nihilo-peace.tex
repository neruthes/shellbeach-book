\chapter{虚无的平静,不被选择}

\ardate{2022-05-06}{4uhiCBw2CUEk3y-304B19g}







当我往回看昨天自己写的文章时,我会感到很奇怪,奇怪于为什么那时候的自己会有这样的感受,因为现在的我、此时此刻的我并没有昨天文章里的那些感受。在昨天,起初那种感受是孤独,然后当陷入孤独后,孤独背后的是虚无。而现在的我好像开始陷入虚无的下一层,但我还不确定虚无的下一层会是什么,也许是什么也没有。

今天上班日的我没有任何感受,没有感受到假期时所感受到的孤独,没有感受到悲伤,没有感受到丧失,没有感受到解离,没有感受到“恍若隔世”,也没有感受到痛苦。就是什么感受也没有。这让我回想起在大学时期,每当自己有一段很沉重的长期抑郁期时,当抑郁的情绪达到顶峰后,就会有一段短暂的轻盈期。那段时间的轻盈感便是什么感觉也没有,似乎一切都会好起来,所有的负向情感都消失得无隐无踪。不过那段轻盈期并不会持续太久,也许现在的心境也是如此。

我会想起前一段时间和一个炮友聊天时,他说他还蛮期待他自己变老之后的样子。我问他为什么,他说到那时候他自己的欲望就不会像现在那么强了。我回答说,噢,原来你是渴望那种无欲无求的状态。他说是的。所以当我往回看昨天写的文章时,我会留意到,如果自己真的陷入了那种虚无的状态里,那自己本应没有任何感觉,就像现在一样,而不是回顾过去地去哀悼乐趣的丧失以及那些曾经享受乐趣的自我的丧失,更不会远望将来地去担心将来的路应该怎么走、应该朝怎样的方向走。

现在的自己的目光很“短浅”,目光从过去和未来收回到了现在,收得越来越窄,就像阳光下的放大镜将光亮聚焦在一个点上,此时此刻的那个点。

\midnote{\href{https://mp.weixin.qq.com/s/oLOBiRgxy2VdVduT6v\_aNQ}{《“理想中的自己”就是现在的样子》}}

我还会想起之前的某次课程作业里,我在作业区里看见有的同学列出了ta在接下来的5年、10年、30年的宏伟计划,而我并不是那样的人,我是那种更愿意去享受当下的人,而不是为了某个遥远的计划、宏伟的目标而往前走的人。但现在的我的目光似乎比那时候的自己还要窄,我不打算将期望、希望甚至是救赎放在未来的考研计划、读研、学业、毕业甚至是未来的工作上。如果此时此刻的我并不开心、并不满足的话,那我大可以换一份工作、换一个城市生活、换一个地方居住。有很多事情是我在当下就能去改变的,而不是势必要在未来树立一个远大的目标才能使自己从这个我所厌倦的现在脱离出来,而不是把改变留在永远停留在未来的未来。

然后这份平静并没有持续多久……

下午,我打算一个之前没约成的男生约晚饭,他说还有另一个人想和他约晚饭,然后说临近下班时再看看情况。在临近下班时,我问他我们今晚还约晚饭吗?他说不约了,他还是和那个男生约晚饭,因为是昨天就答应了对方的。那时候的我突然很伤心,很想哭。然后自己的心智化马上就“上线”了:为什么自己会感到那么伤心?因为他选择了别人而没有选择我。好像我几乎没有被他人所选择的经历,都是别人选择了另一个人而不是选择我。前任选择了现在的长期伴侣,在前任后自己喜欢的几个男生要么选择了事业,要么选择了他们的现任,要么选择了回家。没有一个人选择我。

不过最让我难过的是前任选择了他现在的长期伴侣,而我压根不知道那个人到底是怎样的。我感到很嫉妒,嫉妒那个人能在前任心目中所处的位置。而我也意识到这个提问我也憋了很久,从过年到现在一直憋着没有跟前任提这个提问,于是便在微信上问他:“能好奇地问一下你的长期伴侣会是个怎样的人吗?”当然后来他的回复里也没有给任何确切的信息或答案。不过我也想到,他是否回答是一回事,我问不问是另一个回事,至少我做了属于我的部分。

看完前任的含糊回复后,我感到更加伤心了,我更加嫉妒那个所谓的“长期伴侣”、前任眼中的他的长期伴侣。但我也意识到,这并不完全与前任有关,而且也是和我所遇到过的喜欢的男生有关,他们都没有选择我。这些经历是属于我自己的部分,属于我自己的“易感性”。而且,没有人选择我,这并不是something I can fix。I can't fix this. 我无能为力。我想到,我真的是羡慕前任心目中的长期伴侣那个人本身吗?还是我羡慕的更多是那个人所身处的位置。嗯,我更多的是羡慕那个人的位置,而不是那个人本身。如果是这样的话,那前任的位置呢,好像也不一定需要是前任。就像在大街上看见情侣牵着手,或者是聚餐时看着一对情侣的样子时,我也会羡慕,羡慕对方被另一个人视为唯一、被另一人坚定地选择。

后来前任说“有空来喝茶”,那时候我心情好了一点,而我也马上意识到自己的心情之所以会变好是因为我好像能感觉到我在他心目中还是有一定的位置的。好像“位置”、“地位”对我而言还蛮重要的,我想要别人选择我,想要别人视我为重要的、视我为唯一。

在做关于安抚悲伤的冥想时,我顺着这种悲伤的感觉回溯,突然想起小时候大概在读小学前的自己曾经问过我妈我是从哪里来的,她说我是从垃圾桶里捡回来的。那时候我突然觉得事情说得通了,一切都明了了\pozhehao{}怪不得他们只能看见他们眼中的“孩子”,而看不见我,他们从来都看不见我,因为我的存在只是一个偶然,我并不没有被选择诞生,我并没有被任何人选择过。

所以当初恋和前任选择和我在一起的时候,我会感到很惊奇和幸福,惊奇于自己居然会被选择,因为在内心深处,我并不觉得自己会被选择。然后当亲密关系破裂后,以及在这几年认识到的喜欢的男生的关系里,当他们选择了其他人或其他事物而不是选择我的时候,我也就越来越自我证实了这一认知:我是不会被他人选择的,没有人会选择我。

因为这比起一次又一次地去经历不被选择要好,要没那么残酷、没那么痛苦。

