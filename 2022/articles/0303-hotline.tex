\chapter{电话热线,Spirit Guide,涵容和护持}

\ardate{2022-03-03}{uCiXY7Jv\_nVxccQ3X5B\_Aw}

昨天,随着接电话热线的时间变多,我开始逐渐知道自己应该干点什么。(没错,之前在接热线的大多数时候我并不知道自己应该往哪个方向走、不知道应该干些什么。)

当然,在接热线前的培训里的标准话术有:“聆听”、“提供情感陪伴和支持”、“帮助舒缓情绪”、“梳理目前困境”、blah blah blah… 但抛开话术,在之前接热线时,我脑海里的声音更多是:I don't know what I can do and there's not much I can do. 我不能帮对方去处理对方的生活问题,不能当对方在不同的生活决定中犹豫时给任何具体的建议,更不可能跨过电话的另一头去阻止想要冲动自杀的对方。所以之前,我会把自己看作是一尊佛像,有不同的人前来拜佛,每个人都带着他们自己的情感困扰或生活问题甚至是自杀意图或计划,然后我就在此倾听和陪伴,仅此而已。每个人都在拜佛的这段短暂时间里倾诉着自己内心充满痛苦或鸡毛蒜皮的事情,也会在拜完佛后继续踏上人生的历程,wherever it'll lead, to this life or the next. 有的人甚至会拿同一个提问一直拜佛,就像是一个重复梦。

不过昨天,我觉得自己在热线里的自我形象开始不再像是一尊佛像。如果要说这个形象开始变成了什么,我会想起之前玩的一个游戏《柯娜:精神之桥》里的柯娜\pozhehao{}Spirit Guide(灵魂向导)。在游戏剧情里,Spirit Guide的任务是解救那些被困在过往悲痛、愤怒和憎恨里的无法超度的灵魂,有的灵魂甚至不想被超度。我想,在接热线的我的存在里,有着越来越多的guide的成分。

我逐渐在接热线的过程中发现一些共同的过程\pozhehao{}containment(涵容)and holding(护持),这两者也是如果个体要发展心智化能力,那么其依恋对象需要具备的两个关键点。

大多数来电的人都带着(强烈的)情感困扰甚至是自杀意图/计划。一开始,我处于试图contain(涵容)对方情绪的阶段,包括去镜映对方的情感和重述对方的话,试图跟上对方的脚步并和对方呆在同一个心理境地。涵容指的是去contain(包含)对方的(负面)情感,但这并不完全是安抚,也更加不是要说些什么安慰人的话(比如说“一切都会变好的”、“乐观一点”、“你别不开心”)。和对方呆在同一个心理境地,耐心地等待对方的情绪宣泄的结束,直到对方的精力开始慢慢恢复过来后,就可以开始护持着对方去探索(心智化)TA自己和他人。

我有试过在一开始就和对方去探索(心智化)TA周围的处境和他人,但我发现对方有大量的情绪需要宣泄,而在宣泄之前,处于强烈情感下的对方难以运用心智化的能力。所以只能先涵容,后护持,而不能拽着对方去干对方在那个当下不想且无力去干的事情。等强烈的情绪舒缓后,对方的心智化能力也就开始慢慢恢复过来,才能够开始探索周围的处境、TA自己和身边的他人。

然后是holding(护持)阶段,我会试着和对方一起探索周围的处境和、TA自己和身边的他人,试图去看一些固化的思维模式、负向信念等事物,一些将对方固化了在原地难以动弹或局限了对方的事物。如果幸运的话,有的人在看见那些将自己所固化的事物后,会试图去挪开那些固化物,试图给自己的生活作出改变,思考接下来要作出的生活决策。

不过,并不是所有来电的人都能进入到护持阶段,比如说一些有自杀意图和计划的人,他们往往会被太多太多固化的认知偏差、负向信念等事物所完全固化在了原地,深陷于强烈的情绪当中,以至于在他们的眼中,死亡是为数不多的“出路”之一。所以可能在整个热线时间里,我都只能试图去涵容对方的情绪,并试着去detoxify(去毒)\pozhehao{}通过言语或非言语信息来弱化对方的情感强度,比如说将“我真的活不下去了”转为“好像活着是一件很困难的事情”。

我想我之所以会想到那个Spirit Guide的形象,是因为我在接热线的过程中更多试着去comfort and free other spirits,而且我也有自己可以如何去guide的大致思路。当然,之前的我会在想,也许来电的人只是因为遇到了生活的一些小事呢?也许对方并不需要自己的guide呢?不过,在自己直至现在所接过的热线里,没有任何一个来电的人遇到的事情仅仅是一件小事,那背后总是有各种错综复杂的情感和议题,各种将对方所固化的事物。至于对方是否想去探索,还是想要我仅仅停留在我不断试图去涵容对方的情感、和对方呆在原地的阶段,那就完全由对方自己决定了。

我也会想到,在日常生活里的我并不想花精力和时间去涵容他人的情绪,而只是在自己感兴趣的范围和方向里去探索对方的内心世界。这似乎也是以前的我一直对待我自己的方式:总是急于去找各种解决办法,而且也总是能很快地找到(一定程度的熟能生巧),但在这个过程中并没有很好地照顾甚至并没有尊重自己的情绪、涵容自己的情绪\pozhehao{}比如说通过自我对话的方式来识别和命名出一些情绪,让自己仅仅停留在感受情感本身。之前的我对待自己的态度就像是在对方情绪很糟糕的情况下还试图拽着对方去心智化自己和他人。

我想之前的我之所以会有这样的态度,是因为在过去二十多年里,并没有一个能涵容我的情绪的人。而如果我不逼自己去心智化我自己和他人,不逼自己去寻找困境里的出路的话,那就真的只能被困在原地。(事实上我也确实被困了好几年。)就像是在小学上学路上的马路边,总有一些人在人行道上等待着车流主动停下来,但并没有任何一辆车会选择礼让行人,所以我总是那个第一个走向车流的人,because no one else would do it for me. 就像是在那个无尽楼梯间的梦境,如果不靠自己的勇气走到尽头,那就只能一次又一次地被困在那片无尽的黑暗里。
