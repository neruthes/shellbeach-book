\chapter{“对这个现实世界的彻底绝望恐怕也是必经之路”}

\ardate{2022-07-15}{Ay8xoWVBQGSbGv-OR9698g}



惦记了一周想约周末见面的男生。周初的时候问他周末还见个面吗,他说还不确定周末有没有空。然后邻近周末我又问了一次,但这次他没回复了。我立即感到很伤心,然后突然想起这周其中一个梦境里的一幕,梦里的我一直在说:没用的,没用的,没用的,没用的……

在那个梦境里,我和他在学校的一栋教学楼里。我们在四、五层楼的高度,本来打算一起做实验,但发现缺了些东西。我们站在教室外的走廊,走廊护墙边站着很多学生和一个老师,他们正在拿绳子拉着些什么。我靠着护墙往下看,看到下面也聚集了不少同学,绳子挂着一个篮子,篮子里放着实验用品,然后他们在齐心协力地将实验用品拉上来我们所在的楼层。等实验用品上来了,我和他就终于能去做实验了。看着他们很卖力的样子,我一直在说:没用的,没用的,没用的,没用的……然后我便醒了过来。

这周还有另一个梦境。梦里的我从咨询室附近的地铁站刚出站,我就看见咨询师牵着一个小朋友的手在走。我猜那个小朋友是隔壁咨询室的,因为最近的几次咨询我都留意到隔壁咨询室来了个小朋友以及隔壁咨询室里多了个沙盘。我看着咨询师牵着小朋友的手在走,我下意识回避了他们,因为不想在咨询室外碰到咨询师。等他们走远了,我看了看手环,发现距离咨询开始只剩最后十分钟了,可我连早餐都还没吃。我去看了看面店,但想到用时太久,然后打算去便利店买个包子就马上赶去咨询室。赶到便利店门口的我便醒了过来。

这两个梦境我一直记在备忘录里,当时并不知道这可能会是什么意思。过了段时间,我才能多猜一点这些梦境的大致意思,但其中的象征依然很模糊。

在第一个梦境里,和那个男生一起做实验可能意味着关系的深入,而即使有很多人(学生和老师)齐心协力地促成这个过程的发生时,我依然认为这样的努力是没有用的,那份无力感。在第二个梦境里,看着咨询师牵着小朋友的时候,我感觉那个小朋友是自己的一部分,而咨询师将自己内心的那个小朋友的部分牵走了。没有了内心的小朋友的部分的我变得惊慌失措,不知道应该怎么解决早餐、怎么赶得上咨询时间,就像是这一周的我总是处于惊慌失措的心情\pozhehao{}不知道他可能会怎么想什么、他可能会怎么做。

“没用的,没用的,没用的,没用的……”我内心一直响着这个声音。顺着这个声音,我能回想起蛮多事情\pozhehao{}和前任的相处、去初恋一次又一次曾经出现过的地方,甚至是小学几块钱几块钱地储钱买网游的充值点券的事情。好像过去的很多事情,都会给我一种无力感,自己是无法改变些什么的,没有用的。

而第二个梦境启发了我一点:那个男生在我看来会相似于一个能够提供自体客体经验的咨询师。和他上周见面后,在这一周里,我会有一种时有时无的力量感\pozhehao{}好像我也能去做我想尝试去做的事情,因为在我眼中的他做到了很多他想去达成的事情。这种力量感很可能是我向他所投射的,但或许我正是需要一个这样的人\pozhehao{}能让我愿意将这份力量感投射出去并内摄回来,不断通过自体客体经验来内化这份力量感。但问题是,他真的能够持续提供这样的自体客体经验吗?他真的能够持续“给予”我这份力量感吗?我并不确定。

但他好像依然将我内心的那个更为弱小、更为脆弱的小朋友的部分牵走了,而我也因此变得惊慌失措,不知道接下来的我应该怎么办,不知道我是否还能从他身上继续获得这份力量感,不知道我是否还能利用这份我渴望得到的力量感去达成我想尝试去做却一直犹豫不决的事情。

我想起昨天和一个朋友聊天时,他问我:如果不能解决问题,只是更好地看清自己的话,那心理咨询有什么用?那一刻的我能想到不同流派对于当事人的改变有不同的理论道路,也有不同的元分析能证明心理咨询是有用的。但那时候我的回应是:是啊,如果不能解决问题,心理咨询又有什么用?

心理咨询有什么用?自我探索有什么用?写得再多又有什么用?还不是无法改变现实,无法改变他人,无法改变对方的决定。对这个现实世界的彻底绝望恐怕也是必经之路。



