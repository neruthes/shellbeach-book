\chapter{热线,无法被共情、理解,要建议}

\ardate{2022-04-18}{h5Typ46u5nZYMHKXM4iHNg}




在接热线时,有的来电者会反馈说我无法共情、无法理解到ta,这时候的我会说:“也许人与人之间的差异真的很大,而我也不是总能共情到、理解到每一位来电者。”当我无法做到让对方感受到被共情、被理解,那我能做到只有真诚和陪伴\pozhehao{}坦白自己的有限性,如果对方依然愿意继续聊下去,那就继续聊;如果对方想换个人聊,good luck。而且不同的来电者在面对不被共情、不被理解的感受时的反应也各不相同:有的人会直接挂电话,有的人会表达自己不被共情、不被理解后留有一定的空间后才说要结束通话了。

我想这背后可能也体现了每个人(不仅仅是来电者)在面对不被共情、不被理解的感受背后所暗藏的过往经历和应对模式,我自己也不例外。我记得在心理咨询的前十几次里,我的咨询师经常问我当我知道ta难以理解到我的时候,我会有怎样的情感和想法。那时候我说:我从小到大几乎都没有过被理解的经历,所以这对我而言还蛮平常,因此我也不会对此有太强烈的情感,只是有一种我很熟悉的失落感和孤独感。当咨询师在后续的好几次咨询里还陆续如此提问了好几次后,我跟ta说:“我感觉这更像是属于你自己的焦虑、属于你自己的部分,而不是属于我的。”然后ta就再也没提过这一点了。

但我依然会有一种感觉:似乎有的来电者会把“被共情”、“被理解”当作是理所当然,就好像这是轻而易举就能达成的,自然而然地、随时随刻都能get到ta们的一切的。而当对方感受到不被共情、不被理解时,就会拒绝、撤回、离开、消失,就像是个小朋友在发脾气\pozhehao{}运用更低级别的防御,甚至是带有攻击性的报复\pozhehao{}例如直接挂电话。

不过,当我把案例放在团体督导里讨论时,督导师说:“XXX(我的姓名)的问题在于他太聪明了……太早提出这样的解释了”,“大可不必怀疑自己的能力”。这时候我也更加确信我并不是无法共情、无法理解对方,而是其他方面的问题。

虽然当面对彼此的“联盟”断裂或濒临断裂时,有关课程里教授的技巧是:暂停、倒带、探索、反思\pozhehao{}暂停当下的聊天进程,和对方倒回到“联盟”开始出现断裂的时候,一起去探索那一刻发生了什么,并反思怎样在未来能做得更好。但这一技巧并不适用于所有人,比如说直接挂电话的人、还无法探索过去发生了什么的人(当下没有足够的心智化能力或不愿心智化的人)以及“联盟”关系的强度或许并不足以应对危机。不过,在有机会的情况下,我会指出,“当你感觉到不被理解的时候,好像你就沉默了,而没有继续去解释或帮助我去理解”,试图让来电者看见这一不被理解的互动场中ta所作出的“贡献”。

不过,比起疏离端,我更常遇到的是依赖端。

“你能给一点建议吗?”“如果是你你会怎么办?”“你可以随便聊聊你会怎么办。”比起在热线里,我在日常生活倒是很少遇到有人会这样问我,除了一个工作上的同事\pozhehao{}那个同事总是喜欢按照别人的做法行事,怕犯错、怕被骂。

我想起之前的热线培训里规定接电员不能直接给建议,而在刚开始接热线的时候我会直白地跟对方说我给不了对方任何适用ta自己的建议。但随着接热线的时间越来越长,我开始摸索出了适用于自己的应对方式。

什么是“建议”?“建议”只是一个词,而不同的人会往这个词里加入很多个人的含义和目的,比如说:“我建议你考个好成绩”、“我建议你做好你的工作”、“我建议你去死”。对我来说,“建议”更像是一个空间,正如“自我表露”也是一个空间,而不是一道围墙或一个固定的、凝固的东西。在这个空间里,每个来电者都会将不同的东西带进来,比如说有的来电者可能带入了ta们在某些处境里的迷茫、困惑甚至是悲伤、绝望和无望,我会先去探索、去共情这些带入场内的隐而未说的内容。而我也能带着属于我自己的东西进入这个空间里,试试看彼此是否能够相遇,但前提是双方都愿意进入到这个空间里。(以前的我就不愿意。)

如果我有联想到关于自己经历的话,我会在不作不必要的自我暴露的情况下表露这样的经历。如果没有这样的经历的话,那大可以直接表露:“我想不到自己有经历过与你相似的经历,也许这样经历并不寻常。如果是我经历了这些事情的话,我也会感到手足无措、无力、迷茫。”

对于那些自我表露的关于我的个人经历或许会和来电者的经历很不一样,但也有可能有相似之处。有时候我会说完自己的经历后就沉默下来,等待对方的回应。如果对方没有回应,我会问“不知道现在的你会有怎样的感受或想法”。有时候,如果说着说着自己就已经发现这和来电者的经历很不贴近,我会说“但这只是我自己的经历,也许对你来说你会更想找到属于自己的答案”。

有的来电者在听完我的经历后会想到一些属于他们自己的办法、他们身边的某些资源。有的来电者会指出彼此的经历的不同,然后去思考自己能够如何面对这些不同之处,而不是那个庞大困境本身。有的来电者会感到自己并不是一个人经历那一切,而是有另一个人会有相似的体验。而这也是我想通过自我表露向对方传达的:“你并不是一个人”。但我不可能直接说“我能理解你”,因为这只是空话,而真正的理解需要内容,需要你懂我懂你、我懂你懂我\pozhehao{}心智理论层面的相遇。

就正如危机里有“危”险也有“机”遇。

有时候,我会认为,选择挂电话、离场、无故消失\pozhehao{}不仅仅是指在热线里,也是指在生活里\pozhehao{}总是那个easiest way out。而我会想起美剧《男巫》里的一句台词:“To choose the easiest way out is to barely exist.”


