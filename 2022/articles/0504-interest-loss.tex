\chapter{假期,乐趣丧失}

\ardate{2022-05-04}{HtcgkdpcWTMkot\_wlDM9jg}







五一假期,我休息了五天,上一次有那么长的假期还是过年的时候。在这之前的我每当有假期来临的时候都总要想方设法去应对,而不是去享受,因为每次假期自己都没有什么好干、没有什么想干,这次的五一假期也不例外。

第一天假期是心理咨询以及和朋友约自习,随后的两天便是生物钟的完全丧失规律以及丧失对此的控制。在那两天里,有时候我连续看了十几个小时的电影,有时候跳过吃任何一餐,完全不知道当时是早上还是下午还是晚上的几点,也想不起来最近的日子甚至是当天是怎么过的,也不记得自己是怎么进去到这样的状态里的。然后,第三天的我一口气睡了一整天\pozhehao{}睡了17.5个小时,从清晨睡到凌晨。

以前的我在面对假期时会找各种方式回避孤独,比如说找朋友聚或找地方逛或上网课,但这次,我就让自己完全沉浸于无序当中\pozhehao{}没有任何计划,也不去回避些什么,想干嘛就干嘛。我想我可能是对逃避感到厌倦了。无论自己怎么逃,孤独依然在自己身旁,孤独总会在那里。如果真的沉进去了孤独里,那又如何?

在这几天里,我觉得活着没有任何好干的事情、没有任何想干的事情,不想与他人产生联系,也不想与自己产生联系,不想去任何地方,也不想见任何人,只觉得现在的任何一刻都很无聊,未来只意味着更多的无聊,没有任何值得期盼、值得享受的事物,感觉活着就像是一杯白开水:无论自己喝还是不喝,它依然会在那里,如果喝的话,就和没喝没什么区别,如果不喝的话,也和喝没有任何区别,如果选择把水倒掉,那更加与喝或不喝没有区别\pozhehao{}活着和死掉并没有任何区别。无论喝与不喝,抑或是倒掉,自己都不会有任何感受。

在刷着社交软件里的动态的时候,我发现自己之所以会这么做的原因,并不是因为社交软件里的某个人有多有趣或帅气或有肌肉,而是我想通过那个人去看他眼中的他的生活是怎样的,他眼中的“那杯水”是怎样的,至少那杯水不像是我的那杯白开水。

我今天上的网课里恰好讲到了解离症,同时今天看的美剧“月光骑士”里也有关于人格解离的情节。当中的主人公Mark在小时候面对创伤时分离出了另一个人格Steve,而之后的人格转换都发现在Mark濒临情绪崩溃时,Steve便会一脸茫然地出现\pozhehao{}一种防御机制,使得自己免于精神崩溃。不过解离症状有三级\pozhehao{}初级解离、次级解离和三级解离,多种人格障碍属于最严重的那一级:三级解离。而我发现初级解离和次级解离倒是能很好地解释自己过往的一些行为,比如说读小学时自己几乎每时每刻都会幻想自己是某个电视剧剧本里的一个新主角,并不断构思后续的剧情来回避那时候自己无法逃离的生活,以及之前在自杀意图达到顶峰后出现的解离症状\pozhehao{}就像是看着自己生活着,但自己并没有完全活着的感觉;就像是隔着屏幕在看电视剧里的自己在生活、在扮演着在生活。

后续的课程里也讲到了解离和创伤的因果关系、成年期危害健康行为与孩童期受虐史的关联系数、孩童期受虐经验与成年期自杀企图风险的关联系数、创伤年龄与自伤年龄的关联系数、孩童期创伤与不同的人格障碍的关联系数。在看完这些数据和因果关系后,再回想起自己的经历,我越来越觉得自己和他人格外不同,从小就没有和他人相似过,就像是个异类,即使是现在也是如此。其他人可能会趁五一假期出去玩或在城市里找个景点游玩或去发现些新的好吃好喝的店或风景好的景点去探索、去发掘。但对我而言,我并没有这样的欲望,我什么欲望都没有,就像是自己的灵魂已经在过往的无数经历里破碎得什么也不剩,觉得自己并不像是一个人,而像是一个鬼魂,一个并没有和现实世界里的事物和他人有任何互动的鬼魂。

这也是有时候的我会很“嗜睡”的原因:醒过来后发现自己并不想呆在这个现实世界里,因为和这个现实世界并没有任何联系,并没有和任何人或事物有任何联系,因此便继续回到梦境里“生活”。

课程里的讲师重复用一个词来形容解离症状:“恍若隔世”。我会想起,随着这一年的pandemic control的加剧,许多有这方面经历的人也开始有“恍若隔世”的体验,比如说他们可能会发现对于自己曾经感兴趣的事物现在自己不再对此感兴趣了\pozhehao{}乐趣丧失。

我想到,我也曾经经历过乐趣丧失,还是不止一次的乐趣丧失。每次的乐趣丧失,自己都会失去一些乐趣,比如说对读小说的乐趣、对摄影的乐趣、对闲逛的乐趣、对风景的乐趣、对课程的乐趣、对故地的乐趣、对美食的乐趣、对性的乐趣。每次的乐趣丧失后,自己都觉得没有什么好干的了、没有什么想干的了。自己会有恍若隔世的感觉,但也意识到并不是这个世界变了,而是自己变了\pozhehao{}自己不再是以前的那个自己了。而每次的乐趣丧失,自己所丧失的不仅仅是那份乐趣,还是丧失了那个曾经享受于这份乐趣的自我。就像是自己的一部分又一部分不断地丧失于虚无、消逝于虚无,而自己也无法、也无力改变甚至是减缓这一过程,直到自己完全丧失掉所有的乐趣。而在这个不断丧失乐趣的过程中,自己也会好奇:究竟要丧失多少乐趣,自己才会完全丧失掉自己。

如果自己真的完全丧失掉了任何欲望,这样的我如果继续前行下去的话,就像是在沙漠里遗失了指南针,只能朝着对曾经的方向的仅有的记忆前行\pozhehao{}仅有的对曾经的欲望的记忆前行。Without that, all is lost,迷失在时间和空间当中。

比起“控制灵魂对自由的渴望”,也许一部分的人早在pandemic前、早在很久很久之前就已失去了对自由的渴望,甚至早已失去了对任何事物和他人的渴望,丧失了渴望本身。

因此,有时候我看见一些关于提高生育率或促进消费的字眼的时候会感到十分荒谬:对于丧失了对生育的欲望、对消费的欲望、对生活的欲望、对未来的欲望的人而言,他们为什么要去做任何事情?我为什么要去做那些事情?如果只是为了减少代价、摆脱痛苦的话,结束生命难道不是最直接、最一劳永逸的方式和出路吗?

