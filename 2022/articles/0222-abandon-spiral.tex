\chapter{被遗弃感,无尽楼梯间,脆弱与全能}

\ardate{2022-02-22}{CG6K6SslxcUyUCffKbi8bw}


\dialoguelistthin{3em}{
    \dialogue{我}{小艺小艺,今天天气}
    \dialogue{小\nolinebreak{}艺}{今天4度到7度……最美不是下雨天,是和你躲雨的屋檐}
    \dialogue{我}{是啊}
}

\blockquote{
    听着外面的雨声,我想起之前和心仪男生在晚上的雨中停驻于一家服装店门口躲雨的场景。那天晚上的雨也是和今天那么大,两人的鞋子都湿透了,而且我们一起撑着的同一把伞也不够大。之前,在这间店还没装修好时,他还在好奇这间店到底会是间怎样的店,他甚至在想这间店会不会是喜茶。结果,这只是间服装店,一间能让行人在门口那凹进去的三角空间里多余的服装店。
    我想起以前只要有空就会想约他出来见面,但现在不能了,早已有很长一段时间都不能了。无聊使我被迫面对我自身的匮乏\pozhehao{}人际关系的匮乏、亲密关系的匮乏、事业的匮乏、兴趣爱好的匮乏等。而且,我发现越是我在乎的人,彼此便越可能会失去联系。所以之前很长一段时间的我才会逼迫自己不去在乎任何人,除了那一两个自己真正喜欢的人。但最终还是几乎和这零零星星的几个人彼此断绝了联系,不知道这是否也是我自己的一种“僵硬的行为模式”呢。
    \blockquotesource{随笔 | 长假,收拾房间,无聊,匮乏,焦虑,羡慕交集}{白色灯塔先生}{2022}
}

晚上准备躺下床时,小艺AI的回应语让我回想起了一年前的事情。虽然之前也有过同样的回复语,但我想可能是因为昨晚窗外的雨一直打在雨棚上,滴滴滴地响了一整天,才让我更容易想起以前的事情吧。

与落叶\footnote{2021 年年初认识的一个男生,他在年中时离开了我所在的城市,回家发展了。在我的印象里,他最常说的一句话是:“我要回家”,让我想到了落叶终究还是选择了归根。}的相识已经是去年一月的事情,而与最近认识的一个男生的相识是过年时候的事情。两者在时间上的相似点是,前者在过年前的几周,后者在过年时;而共同点则是,和两人的相识都发生在我和前任的见面相近的时间点。这会让我想到,我和他们的关系亲近是否和我与前任的见面有所关连?也许和前任的见面进一步激发了我内心的脆弱感,而这种脆弱感才得以让另一个人走进我的生活?亦或者是,和前任的见面让我更加确信我不想成为像他的样子,所以我会更渴望能与另一个人建立更亲密的联系?还是说我可能无意识地在利用另一个人来防御自己想要更亲近前任的渴望?

早上睡醒时,想到要周末才能见到最近认识的那个男生,就觉得很想哭。但同时我也留意到奇怪之处:为什么自己会想哭?我明明和那个男生才见了四次……我感觉那种想哭的感觉背后,由于晚上的AI回应语而让我产生了联想,将我对落叶的部分情感迁移到了最近认识的那个男生身上。去年落叶离开这座城市给我带来的情感冲击好像又涌现了起来,或者应该说最近认识的那个男生激发了去年落叶离开这座城市时给我带来的余波。

在通勤路上,我依然感到很难受。在巴士还在等交通灯时,我想到可以利用这一短暂但能够独处的时间。我闭上双眼,感受那种想哭的感觉,试图回溯这种感觉的道路,试图回忆过去有关这种感觉的场景\pozhehao{}我能回想起去年和落叶相处的画面;回想起前任的无故消失、初恋的无故消失,一次又一次重回故地想要追寻曾经的身影、过去的回忆;回想到小时候被打时坐在地上哭;回想到小时候被赶出家门时边拍打着家门铁门,边坐在台阶上哭。

被遗弃意味着死亡、孤独、无意义、虚无、不值得活下去、一个人在黑暗里,一个人在无尽的漆黑楼梯间里,无论往上走还是往下走都走不到尽头,永远在黑暗里徘徊,there's nothing left and no one left, there's no way out.

然后不知道为什么,我突然睁开了双眼,好像自己从过去很遥远的地方突然被拉回了现实,被拉回了现在的此时此刻。我内心在想:噢,原来是这样的,那种被遗弃感\pozhehao{}被家里人“遗弃”、被初恋“遗弃”、被前任“遗弃”、被落叶“遗弃”,然后现在和最近认识的那个男生的相处好像也激发了我的那份被遗弃感。这种“遗弃”也并不是客观世界里的遗弃,而是主观世界里感受到的“遗弃”。当能在理智上看见那种感觉是一种被遗弃感后,我反而不再感觉到那种感觉了,反倒是,我感觉自己的心脏有一种被勒住的感觉\pozhehao{}躯体上的感觉,而不是心理上的。可能是心理上的情感太难以承受,转为了由躯体来承受吧。

我想,每次和前任见面被激发的那种被遗弃感,好像与我会和另一个人建立更亲密的关系有关,那背后的渴望和恐惧。

我会突然回想起前几天和 Neruthes 的对话:

\dialoguelistthin{5em}{
    \dialogue{Neruthes}{虽然可能有些过度解读,但我在最近的文章中隐约体会到了一种「在亲密关系中得到救赎」的倾向。}
    \dialogue{我}{应该没有亲密关系里没有救赎。只有自己才能救自己。}
    \dialogue{Neruthes}{读出了「有了它之后就可以…」这样的想法。}
    \dialogue{我}{比如说可以什么?}
    \dialogue{Neruthes}{「缺乏理解/支持/(此处填写一些其他要素)的生活状态,可能只有以建立和保持亲密关系的途径才能足够充分地摆脱」。}
}

“救赎”、“充分摆脱”,这都让我想到了小时候的那个梦境,一个人在无尽的漆黑楼梯间里,无论往上走还是往下走都走不到尽头,永远在黑暗里徘徊。

\blockquote{
在和父母逛商场时,我突然想去上洗手间。跟着洗手间的标志走,最后发现洗手间的标志指向一扇靠墙的花店正中央的防火门。

打开那扇防火门,来到了一个黑漆漆的无尽楼梯\pozhehao{}上不见顶,下不见底,无论往上望还是往下看都是漆黑一片,只有通过从远处渗透过来的微光才能看清楚周围的楼梯和墙壁。我身后的防火门被锁上了,怎么都打不开。有时,我尝试往上走,有时则往下走,希望能找到出路,但无论往上走还是往下走都走不到尽头。在走了一段路后,我开始往反方向走,回到一开始的那扇防火门,想再试一试还能不能打开,但那扇防火门依然锁得紧紧的。我转身去看楼梯间里是否还有其他门,然后再回头看时,防火门便消失了,只剩下一面漆黑的水泥墙,就像周围的所有水泥墙一样。

在几次幸运的梦境时,自己能够重新打开防火门,离开那既漆黑又空无一人的无尽楼梯间,回到热闹的商场里。但大多数情况是被永远困在无尽楼梯间里。

同一个梦境出现得多了,便慢慢出现了变化。有好几次梦境,自己往下走时碰巧能够走到无尽楼梯间的底部,那是一片空空的水泥地,周围被水泥墙所包围,四处依然是漆黑一片,但我能感觉到墙的另一边有着另一片空间的存在,也有着另一种生物的存在。

由于无尽楼梯间的底部并没有出路,我转身往上走,走了一段路后再往下走,这时楼梯间似乎又恢复到了走不到尽头的状态。但有时,当我再次往下走时,我依然能走到尽头,而且这时,无尽楼梯间底部的水泥墙上出现了一扇防火门。但这扇防火门大多数情况下也还是锁住的。

在经历了无数次同样的梦境后,我终于遇到了能打开防火门的情况。打开那扇防火门,我来到了一片长满杂草的停车场。我走到建筑的屋檐外,抬头看着夜空里那白得死寂的月亮,但依然能感觉到自由的清新。在夜光下的停车场刮着冷风,我能看见前面的其中一个草丛里有东西在动,使得整个草丛都在沙沙作响。草丛前有个垃圾桶。我试图无视草丛里的东西,想要往前走,走出这片停车场。但有好几次,我都无法向前迈步,即使我知道身后的是无尽楼梯间。不过还好,我没有一次是胆小得重新打开防火门回到那个不见天日的楼梯间,况且我也不太确定那扇防火门是否还能打得开,亦或者那扇防火门是否还存在。

在尝试了好几次梦境后,我终于有勇气向前迈步,把目光死死地盯着前方,无视草丛里的沙沙作响。但只要我忍不住地往草丛那看过去时,我便又会惊醒过来。直到有一次梦境,我闭上了眼睛,我能听到周围在刮的风和沙沙作响的草丛,但我依然假装什么事情都没有地走了过去。当我睁开眼时,周围的一切\pozhehao{}地面、建筑、草丛、天空\pozhehao{}都变成了白色。我回头往草丛那看过去,唯一不是白色的便是草丛的轮廓\pozhehao{}一片黑影的轮廓。我试图走向那片黑影,想去看草丛里的是什么,这时我便醒了过来,但这次醒来时,我感受到的不再是紧张和恐慌,而是十分舒缓平静。那便是我最后一次梦到那个梦境。

\blockquotesource{自我探索 | 1}{白色灯塔先生}{2021}
}

在这个重复了无数次的梦境的最后,并没有另一个人将自己带出那个漆黑的无尽楼梯间,也没有因为我找到了另一个人而离开了那个楼梯间,最终还是要靠自己一个人走出去,还是要靠自己一个人走过草丛里可能存在的未知怪物。最终还是要靠自己一个人去“救赎”自己,去“充分摆脱”那个漆黑的无尽楼梯间,一个人去面对小时候的糟糕的事情,面对和初恋的分手,和前任的分手,面对生活的变迁。总是一个人面对任何事情、所有事情。也许我不想再一个人了。

然后在公众号里看到一个留言:“你那么全能,活该单身……拒绝(脆弱)的同时意味着亲手毁掉了无论是与自己的还是与他人的关系……要么被完全孤立,要么被投射理想化直至破灭。”

这让我想到了两个词:全能感和脆弱感。每个人看似总是徘徊在这两者之间,要么全能,要么脆弱。然后我脑海里有一个画面:一个人架起了一个结界。如果这个结界太大,会将周围的人都弹开;如果太小,会让所有的人都随意进来;如果太弱,那么就容易被击碎;如果太强,那么任何外界的事物和人都难以冲破这个结界。结界越大,则越弱;越小,则越强\pozhehao{}一个人需要保护的内心事物(甚至可以说是雷区、个人议题等)越少,则越不费力;需要保护的内心事物越多,则越需要时刻戒备。

在上次见那个最近认识的男生时,我问他为什么会从我身上感觉到一种脆弱感,他说是因为我不怎么笑,但我又会向他表露一些过去的经历(比如说校园霸凌),所以会觉得我背后有很多故事。

所以,如果我架起了一个结界的话,那么这个结界会足够小(例如能表露很多内心的事物),足以腾出让他人停驻的空间,但同时也足以保护自己(不怎么笑)。在承认自己的全能(结界)的同时,也承认自己的脆弱(缩小结界后腾出的内心空间)。在将自己的结界缩小\pozhehao{}敞开自己的同时,会让对方看见自己脆弱的一面\pozhehao{}腾出的内心空间,同时这个腾出的内心空间的背后也是一种更全能、更强的结界。那么全能,也那么脆弱,但并不是只有全能,也不是只有脆弱,而是两者融合在了一起,融合成了自己内心的空间。

我想,正是因为自己足够全能,才能承认自己足够脆弱吧。反之,承认自己足够脆弱的同时,也因此变得足够全能。
