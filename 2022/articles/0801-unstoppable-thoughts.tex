\chapter{“期待这些一连串、停不下来的想法和念头到底什么时候才能消失”}

\ardate{2022-08-01}{vzkTSqpJRGGMeoYLG3iPVQ}



周末去和一个男生面基时,我说起我和前任的经历。

那时候我刚大学毕业,工作还没找到,所以一整天在他公寓里。那时候我就没有多少安全感,对生活的不确定,对未来的不确定,而他是那种工作狂,一年下来几乎不会有多少休息日,从早上7点忙到晚上11点才回到公寓。这种不安全感会让我去猜测很多事情,猜测他会不会在外面有其他人、会发生一些其他的关系。后来我确实发现了一些他身体出轨的证据,但我没有想拿这个证据去质问他,因为当时的关系就已经够崩的了,同时我也没有多少对这段关系的安全感\pozhehao{}如果把他出轨的事情说了出来关系可能就直接崩掉了。但这好像也验证了我的不安全感:他确实会去出轨,而不是他所说的工作忙,而是他在外面有他自己的生活,而这个生活里是没有我的。后来这种加剧的不安全感也加速地恶化这段关系\pozhehao{}在两人的沟通里产生更多的误会、矛盾和冲突,最后这段关系便结束了。

当我和那个男生分享完这个经历之后,他说他也经历过类似的事情。他和他对象之前是处于异地的关系,他对象每年都有一段时间要回公司总部(不在这座城市),后来他发现他对象每次回公司总部的时候都会和他对象的前任住在一起。当他发现他们两个人还住在一起的时候,他提出分手,但是他对象很极力地挽留,甚至为了他而转职、搬到这个城市和他一起生活。

然后他说,一个男性(特别是男性)总会出轨,这是避免不了的,重要的是要将它(出轨)摊出来谈。那时候我就感到不太开心,回到家后我就直接睡过去了。在回去的路上,我感觉他的回应激发了自己的被遗弃感,因为在路上我脑海里一直响着一连串停不下来的声音:他就是会出轨的,他就是不会一直在这里的,他不会一直为了你而停驻,没有人会那么看重你,没有人会觉得你值得和他在一起,没有人会愿意为你投入那么多的时间和精力,没有人会等你一辈子,每个人都会离开的,每个人都会离你而去。

我能意识到这些话是不符合我现在的主观现实的,毕竟现在我身边还有一些朋友在。那些话是在和前任以及在那之后的一些男生的关系断裂时,以前的我会一直用这些话来安慰自己,跟自己说:事情就是这样的,而不是因为自己做错了什么,他们就是会离开我、遗弃我。所以到现在,好像这些话依然会自然而然地不断滚动下去。也许这是一种自动化地尝试自我安慰的方式吧。但这样的话并不会让自己感到更加舒服、释怀,而是会让自己感到更加难过。

在最开始,我会期待这些一连串、停不下来的想法和念头到底什么时候才能消失,就像是自杀意图一样\pozhehao{}从一点点的不开心到对待世界的绝望的态度再到“我想要自杀”、“只要自杀了就能怎么怎么样”。后来这种一连串、停不下来的自杀念头便消失了,不再像是火车开在轨道上一直往前开。而且它的消失也很奇怪,是不知不觉消失的,而不是有意识地让它消失就消失的。所以我可能也期待这样的被遗弃的想法和念头最终能够消失。

但之后我发现,我能够以另一种视角去看待这样的被遗弃的想法和念头,就像是冥想一样,以一种共情、接纳、不评判、让它如其所是地存在的态度来看待这样的想法和念头,比如说那些自己一直、终将被遗弃的想法和念头可能是出于安慰自己的初衷,只是这样的想法和念头让自己的情绪更加不好。与其等待这一连串的、停不下来的想法和念头的消失,倒不如去描述、去看、去探索这样的想法和念头原本的样子,接受它们现在就已经在这里了,也许它们并不会马上就离开。与其希望去回避、去拒绝、去抹去它,倒不如保持中立地留有一定距离。

在过去那段时间里,我确实被不同的人所遗弃了,而也许我也确实很需要这样的自我安慰,去告诉自己这并不是自己的错。


