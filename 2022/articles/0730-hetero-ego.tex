\chapter{“怎么面对自己内心的那个异我的部分”}

\ardate{2022-07-30}{Ulk206JAugGEs3XNqfVdWg}


\dialoguelist{咨询师}{
	\dialogue{我}{我昨晚只睡了40多分钟。昨晚感到很难过,半夜哭了一场后就睡不着了,然后上了一下网课,在床上躺一躺就过来咨询了。}
	\dialogue{咨询师}{我留意到你好像在过去有两次咨询都是在咨询前没有怎么睡过觉就来了。}
	\dialogue{我}{嗯,我已经分不清究竟是因为难过所以才睡不着,还是因为睡不着才难过,因为在学习抑郁症方面的知识时,知道这两者是会共同影响的。}
	\dialogue{咨询师}{你会怎么理解这样一个过程吗?}
	\dialogue{我}{我记得我昨晚是悲伤和愤怒,然后在这之前的情绪是被遗弃感。前几天回去了另一个城市的朋友会激发起我的被遗弃感,因为三年前当我和前任的关系变得糟糕的时候,他离开了这座城市而去和他的现任相处了。同时前几天我也梦到了之前有好感的那个有对象的男生,这同样也激发了我的被遗弃感。\\
		然后昨晚我在修一张日落的风景图,我修完之后觉得那张图蛮好看的,但是之后我会感到很悲伤。因为我回想起在大一大二时,我有想买一部单反,但是那时候父母的意见是说:我不是专业人士,我应该去多学构图方面的知识,而且我的钱是来源于他们的生活费,所以他们并不认可我去拿这些钱买一部那么贵的单反。后来我就真的听了他们的话没有再去买了。现在想起来会感到很悲伤,悲伤于那时候的自己放弃了在这个爱好上的可能性,如果我跟随着这个爱好走的话,我或许能够抵达的地方。同时我也感到很愤怒,愤怒于凭什么他们就这么说了,以及凭什么他们这么说我就真的这么做了。}
	\dialogue{咨询师}{那你会想对此做些什么吗?}
	\dialogue{我}{我会想接下来买一部更加贵的、摄影功能更好的手机来拍照,去尝试补偿大一大二时的自己。但和那个朋友的相处,我好像就没有太想做一些什么。}
	\dialogue{咨询师}{“没有太想做一些什么”,会是什么意思?}
	\dialogue{我}{我昨天才和那个朋友见完面。见面的时候他问我最近怎么样,我说还好,然后他就开始说他自己的事情,说了25分钟,差不多半节咨询。然后在和他相处的过程当中,我会觉得他一直在说他自己的事情。我也想说我自己的事情,但是当他一开始就说了那么久他自己的事情之后,我就不再想说我自己的事情了。\\
		而且在和他的相处里,我更像是我在实现某种自我功能,比如说给他一些回应或者是帮他分析一下他所说的话。他昨天的状态和我之前见他好太多,是因为他现在有一个救生圈了\pozhehao{}一个他现在正在相处的男生,然后他问我:他是应该选那个旧的救生圈(前任)还是新的救生圈(备胎)好。我没有太想回答他,因为好像他并没有朝着我想要他走的方向走。但另一方面我也意识到这种想法好像更加是一种操控,而我并不想去操控他的人生道路,因为这终究还是要他自己一个人走的。但如果不带任何目的性的话,那我根本就不会作出任何回应,因为我每一次的回应都是朝着一个特定的方向去回应的。所以当他说完他自己的事情之后,我就没有太想回应他,或者是在他的话语里找一些突破点,只是听着而已。同时我也会感觉到一种被淹没的感觉,就是他的话语像是洪水一样涌了过来,一直涌了25分钟,然后我跟他说他说了25分钟,然后他又继续说下去了,后来我没有看时间,不知道他又说了多久。在听他的说话的时候,我甚至想将自己的意识放在其他地方,比如说路上的风景、周围的声音、自己的呼吸上,我想去逃离,但又逃不掉,总是时刻能意识到他在说着一些什么,还是能记得住他说的话。有一种难以逃脱的感觉。所以我感觉和他的互动里,我只是在实现着某种自我功能。而且在他说了一大堆话之后,他会说:其实在和你聊完后,我感觉自己的事情要理清了不少。\\
		我会想起昨晚冥想时的一句引导语:“这是你的空间,这是你的时间”,但在和那个朋友相处的时,我并不觉得这里面有我的空间、我的时间,而完全是他的空间、他的时间。所以我也不敢在他面前表露关于我自己的事情,因为我没有感觉到一种安全感。就像是,如果我表露了我自己的内容、去占据这个时间和空间的话,我会担心他马上就把他自己的内容盖过去或是把我挤出去、排出去这个空间和时间。}
	\dialogue{咨询师}{怪不得你说好像你只是在实现着某种自我功能。}
	\dialogue{我}{嗯。而且我最近会在走神,就是有一段时间我会不知道自己在哪里、在干嘛,然后当我突然意识到这一点时,我会在想我是怎么来到这里的。\\
		我意识到这和被遗弃感有关。每当我感受到那种被遗弃感时,首先我会感到很悲伤、很难过,然后我就开始走神。当我走神的时候,我会不知道自己走神到了什么地方。当我回过神的那一刻,我也不知道这个过程是怎么发生的。就像是我的元意识\pozhehao{}对自我意识的意识\pozhehao{}下线了。这会给我带来一种失控的感觉,因为自从去年十二月到现在,我一直有在做冥想,在这个过程里我一直能够时刻的自我觉察,知道自己此时此刻在想什么、在感受到什么情感。但当我走神时,那个自我监控的自我就像是下线了,我不再知道自己在想什么、在感受到什么。这会给我一种对控制的丧失感,我失去了对自己的自我意识的控制、对我的想法和情感的控制。}
	\dialogue{咨询师}{这种走神通常会持续多久?}
	\dialogue{我}{几分钟到半个小时都有。}
	\dialogue{咨询师}{那通常会发生在什么地方?}
	\dialogue{我}{通勤路上会有,坐在办公桌前会有,一个人去江边散步的时候也会有。}
	\dialogue{咨询师}{那你会怎么理解这种走神吗?}
	\dialogue{我}{我会想到那种被遗弃感在过往一直能够激发起我不同的反应。比如说在最早之前,这种被遗弃感会让我想去自杀,后来一段时间,这种被遗弃感让我困得想直接睡过去,然后现在它给我引发的反应是走神。这些好像都是一种想要逃离的反应,而且都是下意识的、不受控制的反应。所以我会感到很奇怪。我在意识层面会对自己说:这种被遗弃感我已经很熟悉了,我这一辈子都在感受着这种被遗弃感;但我的身体好像没有办法去承受这种感觉,它依然会通过一些下意识的反应来回避掉这种被遗弃感。就像是,我的意识知道这是怎么一回事,但我的身体好像依然有它自己的想法,依然想逃。}
	\dialogue{咨询师}{听起来,好像你的身体需要这种走神。}
	\dialogue{我}{嗯,也许是吧。其实现在说起来,我好像没有感觉到那么的糟糕,因为好像最坏的情况也糟糕不到哪里去。\\
		但我会对此感到恐惧。其实在之前我都有一种虐猫的倾向。无论是前任还是朋友家的猫,我都会对它们有一种冲动,想要去虐待他们、伤害他们,特别是当我想要它们回应我但它们不理我的时候。虽然这样的虐猫行为并没有给那些猫带来实质性的伤害,不是他们断了腿或者是受伤了,但当我处于那种状态时,我会控制不住我自己\pozhehao{}我能看见我在做什么,我能感受到当肾上腺素突增时,我的心脏会突然跳得很重,我的头脑会发胀,我能感觉到自己的血管在膨胀,自己的指尖甚至都能感觉到血管扩张时血液流动带来的刺痛感,但我就是没有办法停下来,比如说将猫抛起来或扔出去。我停不下来,控制不住这副身体。\\
		这会给我一种异我的感觉\pozhehao{}好像这并不是我,而是我被其他人(至少是自我的另一个部分)所操控着去做各种事情,而我无法控制自己的身体。向自己身边更弱小的生命施加虐待更像是小时候我妈对我做的事情,而当这个部分在我自己的身体上重现时,我会感到很排斥、厌恶和恐惧。\\
		后来我找到了一个办法\pozhehao{}通过元意识来时刻监控自己的情绪和想法,将那个想要虐猫的自我进行监控、划分界限。但这前提是我的元意识的那部分自我是上线的。但如果我的元意识下线了,而万一我进入了这种想要虐猫的状态,我又会作出怎样的事情呢?我觉得真正难以接受的,不是我害怕会对这样的小生命带来怎么样的结果,也不是害怕那个猫的主人会有怎样的反应。我真的难以接受的是,如果我真的作出了这样的事情,那我要怎么面对这样的自己,怎么面对自己内心的那个异我的部分。那个部分是小时候的施虐者在我身上遗留下来的部分,而我唯一能做的好像就只有去接纳它,去给它设定界限,而没有办法抹去它。而我也不可能将这个部分释放出来,不可能将我自己的父母杀掉,因为毕竟还有这个社会、法律和道德的约束。我好像能做就只有等待他们的自然死亡,只有等到他们的人不在了,这种愤怒和想要复仇的欲望才会消失,因为到那时候就没有能够复仇的对象了。}
	\dialogue{咨询师}{听起来,好像这种做法只是你什么都不做,然后等他们死掉。也许我们能在之后继续谈这个部分。}
}
