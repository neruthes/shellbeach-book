\chapter{板子,无故消失,等待和期待}

\ardate{2022-03-13}{x\_-xw6IraRZYKI4dxR5tNw}

\dialoguelist{咨询师}{
\dialogue{我}{我又想不到应该说点什么了。}
\dialogue{咨询师}{没关系。或者你现在会有什么感受或想法的出现吗,总会有点东西出现。}
\dialogue{我}{Em……其实进咨询室之前的一两分钟,我会感到有点焦虑,感觉自己的心跳会加速。}
\dialogue{咨询师}{会是在焦虑?}
\dialogue{我}{一方面是焦虑于在咨询室里会想不到说些什么,另一方面是……我好像还没想到。}
\dialogue{咨询师}{我想起你之前会带一些东西进来,比如说纸条或 iPad,这些东西会不会是你之前用来缓解在咨询室里不知道说点什么的焦虑?}
\dialogue{我}{嗯,对于那时候的我会是的,因为那时候的我不知道如果在咨询室里不知道说点什么会是一种怎样的体验。但现在的我还好,因为之前已经尝试过了。但我依然会感觉到不舒服,因为……比如说如果我和一个朋友约出来聊天,然后我对朋友说:‘我不知道这次约出来我们能聊点什么’,其实我的本意是想让对方主动地说点什么,来跨过彼此之间的空隙、隔阂,但你却会‘对此进行工作’。}
\dialogue{咨询师}{我脑海里有一个画面,好像你总是放下板子的那个人,放下的板子连接着另一头在小船上的人。}
\dialogue{我}{如果是这样的话,好像有时候我放不下这个板子。上周周末我和朋友逛江边的时候,我那时候只睡了三个多小时就出门了,很困很累,所以那时候我并没有多少精力去心智化我自己和他人。我没有精力去放下这个板子。而我朋友只会在江边看着我不说话,看不透我内心的状态和想法。而这次,我好像心情上希望自己能在咨询里说些什么,但我理智上好像做不到,说不出些什么。就好像在和朋友走在江边的时候,只要我不去心智化我自己和对方,这条路就在这里停下来了、不动了,只有自己踏出脚步,这条路才会继续下去。}
\dialogue{咨询师}{那放下这个板子对你来说意味着什么呢?}
\dialogue{我}{意味着……可以跨过两个小船之间的深渊般的虚无。就像是我放不下板子,期望对方能放下板子,而你却会说:‘我们一起来看看这个深渊般的虚无里有些什么,一起凝视深渊。’但这并不是我的本意耶。我是想对方说点什么,而不是想让这个深渊越扩越大。}
\dialogue{咨询师}{好像在你看来,这个板子意味着与他人的连接。}
\dialogue{我}{嗯。这种连接能给我一种安全感。}
\dialogue{咨询师}{噢?安全感。你能想到一些怎样的经历吗?}
\dialogue{我}{我会想起,大概一年半前,我认识了一个男生,和他相处得还蛮频繁的,几乎每周周末都会去他家过夜,也会一起去一些地方走走逛逛。但在相处了半年后,他就无故消失了,直到现在也见不上面,他一直说他很忙。那时候的交流都只是停留在表浅的层面,比如说聊聊日常、吃吃喝喝这些,并没有什么深入的聊天。而当他消失之后,我会感到孤独感、被遗弃感、对人际关系的不安全感。我甚至不知道我们彼此的关系是什么,不知道他究竟是个怎样的人。如果我能和对方有深入的话题的话,起码我会有一份对人际关系的安全感,知道对方不会轻易地无故消失。}
\dialogue{咨询师}{我会回想起你之前分享过一个你和你母亲一起逛街的经历,你走在大街上的时候发现你母亲突然不见了,结果发现她是去看衣服去了。那时候你在原地很不知所措。}
\dialogue{我}{Em……和现在走在江边的画面会有相似之处,但也有不同。以前小时候的我更像是在等待,而现在是一种期待。}
\dialogue{咨询师}{噢?等待和期待之间对你来说会有怎样的不同吗?}
\dialogue{我}{Em……等待就像是我什么也做不了,只能困在原地。期待更像是我希望对方能做点什么。期待的时候不会像等待的时候的那种不知所措的感觉、孤独感。}
\dialogue{咨询师}{好像等待的你在关闭着自己,而期待的你好像将自己敞开着。}
\dialogue{我}{Em……其实当你说到敞开和封闭的时候,好像确实是这样。小时候的我只能通过封闭自己来自我保护。但现在我能敞开自己,看着周围的事物和他人在流动,同时期待对方能做点什么。}
\dialogue{咨询师}{好像现在的你也不害怕和他人经历一些事情,其中可能会有失落或被攻击,但你好像不怕了。}
\dialogue{我}{嗯,因为现在的自己不需要通过自我封闭来保护自己了。现在我能保护好自己,而且也知道如何反击。\\其实我好像感觉到你在咨询的过程中是有用力推动着咨询的进行的,比如说你会联想起不同的画面,然后将这些画面带到我面前。我想你可能是担心现在的我陷入到了以前的境地,但你好像也看到了这两个画面之间的不同点,然后将不同点敞开在我面前,而我再继续澄清其中的不同之处。}
\dialogue{咨询师}{这对你来说会意味着什么吗?好像我也在推动着这次咨询的进展。}
\dialogue{我}{意味着……就好像我们一起走在江边,当我走不动的时候,你会替我多走几步,然后指给我看你走的方向,我再看看你走的方向,如果我觉得那个方向确实有点东西能看的话,我又会继续走多几步。感觉我不再是一个人推动着关系的进行。}
\dialogue{咨询师}{不再是一个人,好像对你来说蛮重要的?}
\dialogue{我}{嗯,我会有一种安全感,就像是买股票,我看见对方也有投入的时候,我也就没那么担心投入了。因为不然对方突然消失的话,对对方而言也不会有什么损失。}
\dialogue{咨询师}{关于突然消失,你还能想到些什么吗?}
\dialogue{我}{我会想到以前的亲密关系里,以及一些较为亲密的男生的相处里,对方总是无故消失,而我需要去面对很强大的不确定性,不确定彼此的关系究竟是什么,不确定彼此的未来,不确定对方究竟是个怎样的人。但现在我认识的人里,几乎不会有人无故消失了,这种情况改善了很多。}
\dialogue{咨询师}{为什么现在的情况会有所改善呢?这其中是……?}
\dialogue{我}{我想可能是因为自己学咨询后,自己的存在改变了。当我看不见对方内在的事物的时候、当对方不愿意敞开自己的时候,我会很回避,因为我感到不安全。这种不安全感好像也保护了我不再经历对方无故消失的经历。而且自己喜欢的人的特质也变了,比如说喜欢对方的心智化能力、在乎他人的感受和想法,而这些特质也让我更能够与对方建立起深入的联系。}
\dialogue{咨询师}{我会想起小时候你被你母亲落在街上的画面,好像你已经不是小时候的你了。}
\dialogue{我}{嗯。其实现在当你提起的时候,我发现那时候我妈的行为也是一种无故消失!现在起码对方不会无故消失了,现在我的朋友会在原地等着我。虽然我期望对方能替我多走几步,但即使期望落空,对方也不至于无故离开,而是在原地等着我。}
\dialogue{咨询师}{那你对此会有怎样的感受吗?}
\dialogue{我}{Em……我会觉得蛮开心的,因为,我不用再经历以前那样的无故消失的破事,也不用在对方无故消失后面对那强烈的不确定感,我甚至不确定对方究竟是个怎样的人。}
\dialogue{咨询师}{其实你说到这里的时候,我眼眶已经有点湿润了。}
}

我看着咨询师开心地笑了笑。

离开咨询室后,我想到:“人与人之间的连接好像就是通过心智化所搭建起来的。而心智化自己和他人总是很累很累。”
