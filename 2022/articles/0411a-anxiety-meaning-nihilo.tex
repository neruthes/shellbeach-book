\chapter{僵局,焦虑,意义,无}

\ardate{2022-04-11}{-MkqzUalzLWuIOKGYbjymw}





\blockquote{
    听过一句话:成为自己总是从失望开始,然后到绝望的时候才发生。

    上面那句是基于文中的“无”联想到的,有如道家思想中的“无用之用方为大用”。个人粗浅的理解,之所以感到“无力”,有可能是还想在某个点上“用力”,好比大家都很积极地在做有用的事、偏爱向阳而生的鲜花,而那些阴暗的角落、被视作绝望之境的地方往往会被忽略,其实也是能绽放出美丽的花朵。绝望,未尝不是一个选择。
}


一位微信好友在上面那篇文章里留下了这段留言,这段留言会让我联想起皮尔斯的五层次神经症模型里的僵局层。

\blockquote{
如果我们丢下角色,则到达第三层或者僵局层(impasse layer)。我们可能体验到我们的两部分卡在冲突中:想要完成未完成事件的健康的一面,以及想要避免痛苦的另一面。我们中的很多人想要避免体验僵局,因为我们想要避免为我们卡住的状态承担责任。我们倾向于否认存在性恐惧或焦虑,而这二者是当我们意识到自己的自由和局限时不可避免地体验到的。僵局的特征是感觉被卡住、混乱和焦虑,经常体验为非常不舒服。

在僵局层后面,并且支持僵局层的是死层(death layer)或内爆层(implosive layer)。这一层是对立力量的崩塌。我们把自己拼凑起来,我们收紧肌肉,我们向内爆。我们相信如果我们向外爆的话,那么我们就活不下去,或不再被爱。

如果个体真的面对僵局层的卡住和混乱的存在性焦虑,和内爆层死亡的感觉待在一起,那么最终他在外爆层(explosive layer)重获新生。在这一层,这个人是真实的,他可以体验和表达自己的真实情感。他外爆进入哀悼、愤怒、喜悦、大笑或高潮。他采取行动,他活着。“据我看,这是变得真实的必要一步。主要有四种类型的外爆:外爆为喜悦,外爆为哀悼,外爆为高潮,外爆为愤怒。有时候,外爆非常温和,这取决于向内爆层投射了多少能量。”

\citebook{格式塔咨询与治疗技术(第三版)}
}

\blockquote{
皮尔斯从一个更宽广的存在主义立场出发来看待焦虑,把它看作个体真实活力潜能的标志,看作寻求个人意义必要的第一步。比如,一个开始意识到之前“固化”或不真实存在方式的个体,当他到达转折点(僵局),可以回顾早先的决定并改变行为的时候,经常感到焦虑、混乱或绝望。他感到焦虑是因为现在他必须再次体验在原始事件中曾压抑或想要避免的感受,因为他意识到他有与已经成为习惯的行为不一样的选择。他面对自己的存在性自由,这是他到目前为止以固化或严苛的行为方式回避的。皮尔斯鼓励来访者面对、探索和找到焦虑的个人意义,而不是把它当成一个待治疗或压抑的症状……

\citebook{格式塔咨询与治疗技术(第三版)}
}

僵局层里充斥着焦虑。当事人既“想要完成未完成事件的健康的一面”,又“想要避免痛苦的另一面”,这一股对立的力量维持着僵局的存在。如果这一对立的力量塌下后,便是死层或内爆层\pozhehao{}攻击性、动力转向自身。如果当事人能承受得住僵局层的焦虑感和内爆层的自我攻击的话,能与这样的感觉呆在一起并幸存下来的话,那么TA最终能在外爆层获得新生。

这也会让我想到改变的悖论。

\blockquote{
在精妙的存在悖论里,改变发生于一个人面对和充分成为她已经是的样子,而非她试着变成她不是的样子时。格式塔治疗的目的不是改变本身。治疗师的工作是帮助来访者发展她自己的觉察,充分地与她现在的样子接触。改变只能发生于来访者放弃\pozhehao{}至少是在那一时刻成为她想要成为的样子而尽可能充分地体验她是谁时。

……皮尔斯用以下关于改变的评论,为他与埃莉的工作做总结:

“我们都关心改变的想法,大多数人通过做规划的方式来靠近它。他们想要改变。“我应该像这样”,如此等等。发生的真实情况是刻意改变永远,永远,永远不会起作用。一旦你说,“我想要改变”做一个规划一一个对抗你改变的力量就被制造出来。改变是自己发生的。如果你深入你自己的状态,如果你接受那里存在的东西,那么改变会自动地发生。这是改变的悖论。”

\citebook{格式塔咨询与治疗技术(第三版)}
}

正是当自己放弃改变,放弃成为自己应该成为的样子,放弃僵局层里对抗的力量,放弃对自己的期待和期望,放弃自己的未来后,当事人才能进入下一层\pozhehao{}死层/内爆层以及外爆层。

从学生时期开始,我就期望自己能成为一个很厉害的人,无论是学术还是工作还是经济能力上。在爱情方面,我也期望自己能有一个能够有所依靠的他人,能有一个温馨的家,能养宠物。当然,这些期望从未实现过,他们一直都呆在了永远在未来的未来。

在前几天开始的关于应对焦虑的体验营里,课程内容里有说到,有的人给自己设立一个人设,而当现实情况难以满足这个人设时就会产生焦虑,包括习得性无助本身也能看成是给自己设定了一个无助的人设。我会想到,之前的我(甚至现在的我)也会给自己设定一个很全能的人设\pozhehao{}一个很厉害的人,无论是哪方面很厉害都行。但现实情况是,我并不是一个很全能、很厉害的人,我只是个普通人。

课程内容里说到,如果向内探索的话,可以去探索这个人设对自己而言的意义是什么,为什么会出现这样一个人设。(“鼓励来访者面对、探索和找到焦虑的个人意义,而不是把它当成一个待治疗或压抑的症状”)我会想到,自己之所以会给自己设定一个很全能、很厉害的人设,是因为自己在过去的童年时期和学生时期的经历都是身处于无力感里\pozhehao{}无力做出自己的选择、无力决定自己的生活,甚至连想自杀都缺乏这样的勇气。我不想再回到那个像是监狱般的日复一日的生活,所以我需要自己变得全能、变得厉害、变得独立、变得不需要依靠任何人也能活下去,并生活在自己想要的生活里。

这样的一个希望能够自我拯救的人设对我而言意味着生存、意味着摆脱无力感、摆脱过去糟糕的经历。

在近期的一次心理咨询里,我跟咨询师说我不知道自己是怎么变成现在这个样子的(我也不知道要如何具体描述“现在这个样子”是怎样),我不知道这个过程是怎么发生的。(“改变是自己发生的。如果你深入你自己的状态,如果你接受那里存在的东西,那么改变会自动地发生。”)这让我想到,改变更像是副产物,而主要的过程反倒是:一步又一步地深入自己的内心世界,一次又一次地接受更深处的内心事物的存在\pozhehao{}特别是那些之前难以接受和发现的事物。而在不断往内心深处走的过程中,身边的事物和他人以及自己也随之而变。

当穿过那些对抗的力量,穿过自我攻击,穿过难以承受的情感,穿过过去被遗忘的沉重记忆后,最终抵到了一个“无”的地方\pozhehao{}什么情感、什么想法都没有的地方。

\blockquote{
尽管没有觉察到任何特别的东西,但这个人处于警觉状态,对所有的可能性开放。他的兴趣可能去向任何方向,此处亦或他方。他在平衡中,处于中间。他就任其自然。场还没有分化,图形和背景还是一体。

……在东方宗教中“无”(nothingness)意味着没有什么是实在的:只有过程、发生、纯粹的存在。现象学和存在主义哲学家也探索过“无”的概念,无来源于一种存在性恐惧,这种恐惧是因为意识到每个个体都是孤独的,终极意义并不存在而产生的。很多西方的普通人害怕并避免无的体验。存在主义思想认为,否认焦虑、死亡和无的现实的人,生活得不真实,皮尔斯受到这种思想的影响,提出面对存在性空可能是找到个人真实的手段。

皮尔斯不回避空,而是鼓励人们进入它并了解它。在他的第一本书中他用了一整章来教读者倾听其内在的寂静\pozhehao{}一种类似冥想的练习。他相信内在的寂静可以帮助一个人与其存在的更深的、直觉的层面接触。二十多年后,皮尔斯仍然用诗化的语言谈论空:“我们发现当我们接受、进人这种无、这个空的时候,沙漠开始开花。虚幻的空开始活跃,被充满。荒芜的空开始变成盈空。”

\citebook{格式塔咨询与治疗技术(第三版)}
}

“对所有的可能性开放。他的兴趣可能去向任何方向,此处亦或他方。他在平衡中,处于中间。他就任其自然。场还没有分化,图形和背景还是一体。”大概一周前,自己喜欢的那个有夫之夫的男生对我说:“这应该就是我俩最大的不同,你会跟随自己的感觉和欲望走,而我是有预设的界限,所以当我感觉到你前进的方向是我无法和你相伴前行的时候,我就产生了保持距离的想法。”

这句话提醒了我,跟随着自己的感觉和欲望走在很长一段时间里都是我的“生活之道”,以至于我把这种生活方式看作理所当然的。但对方并不是这样,对方会设立界限,对方有比自己更值得珍重的事物和人,对方并不像自己这么一无所有,nothing to lose。不过我也相信他作出了他真正想要的决定。

跟随着自己的感觉和欲望的前提是要能够时刻察觉到自己内心不断变动的感觉和欲望,总是在“正念”着,总是投入新的情感体验循环,又很快地会从循环里离开,回归到休憩状态,回归到平静里,回归到什么也没有的“无”里。

有悲伤、有气愤、有失落、有迷茫、有孤独、有无力、有沮丧、有虚无、有抑郁、有难过、有被遗弃感,当然也有短暂的开心和温暖,and then, we will begin again。
